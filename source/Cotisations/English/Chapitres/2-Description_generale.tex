
This chapter aims at providing an overview of the architecture of contributions and payroll taxes, as well as the Social security scheme they fund. More detailed information on the determination of tax bases and computation methods of taxes can be found in the following sections. However, the reader should keep in mind that this guide is only meant to provide an insight on the French contributions and payroll taxes scheme and the way it works: the particular rules of the general scheme and the case of professions with particular status will not be presented\footnote{For exhaustive information on the computation method of tax bases an contributions and payroll taxes rates, please refer to tax law and social law textbooks, in particular \citet{lefebvre2013fiscal}, \citet{lefebvre2013paie} and \citet{lefebvre2013social}.}.

%%%%%%%%%%%%%%%%%%%%%%%%%%%%%%%%%%%%%%%%%%%%%%%%%%%%%%%%%%%%%%%%%%%%%
%%%%%%%%%%

\section{Historical review}

In France, the law regarding Social security has begun to exist in the end of the 19\up{th} century. A first law has been voted in 1898 to protect workers against work accidents consequences: it insured to employees financial compensations when the employer's responsibility could be established. This law paved the way for the future development of social protection measures: the first old-age insurance for farming and industrial workers has been created in 1910, whereas a more comprehensive social insurance scheme against the sickness, the maternity and the old-age risks in the commercial and industrial sectors has been set up in 1930\citep{SecuriteSociale1988}.

However, before any law has been enforced at the national level, some private initiatives emerged from some employers or workers in several professions. Employees in public companies (national company for gas and electricity, Public transportation company, etc.), self-employed workers, farming workers, civil servants, etc. benefited from a partial insurance against social risks. In particular, regarding pensions, the first actual schemes have been implemented in the 19\up{th} century, but the affiliation was conditional to the worker's occupation. Such special schemes existed for employees at the \textit{Banque de France} (1806), for actors at the \textit{Com\'edie Fran\c{c}aise} (1812), for civil servants (1853), for railways employees (1855) as well as for miners (1894) \citep{ipp_retraites}. Some of these schemes were supplemented with a sickness insurance. These different schemes have been maintained even after the implementation of a more comprehensive system covering most of the population -- active or not -- as independent schemes, and allow us to understand the current shape of the social protection system in France.

Still, most of the French Social security scheme's features have been set up after World War II. The first comprehensive scheme covering all the main social risks has been implemented after the Liberation, in October 1945. Initially, it was intended to cover the workers outside the farming sector, but it has been progressively extended to the whole population. The objectives of this system, as inscribed in the law, were to protect workers and their families against risks that could induce partial or total inability to work, that is to say sickness, maternity, old-age or accidents. An entitlement to partial compensation for additional spending related to the presence of children in the family was also laid down in the law. While the general scheme was extending, other autonomous structures emerged: a scheme for self-employed workers not coming under the farming scheme (itself particular) has been created in 1948; other schemes intended to cover other professional categories developed thereafter, and still exist today (for instance, the insurance general agents, dental surgeons and other liberal professions are insured under a specific scheme).  

The Social security scheme has then been generalized without achieving its initial unifying goal. However, despite the multiplicity of structures, the social protection system as a whole is mainly based on an insurance logic: all the schemes are mostly funded by social security contributions paid by workers in order to finance benefits granted in case the covered risk realizes. Another key feature of the French system is that the non-active population (mainly children and housewives at the time the system has been thought) is covered as assignees of the workers. The French social security scheme is classified as Bismarckian, in keeping with the one implemented the Chancellor Bismarck at the end of the 19\up{th} century in Germany; it differs from the Beveridgian model existing in the United Kingdom, where the main part of the Social security spending is funded by taxes transfered to the general Budget. This distinction, though often underlined, is somewhat simplistic: as unemployment was raising, the need to have a social protection structure covering the whole population whatever its situation with regard to the labour market became more crucial, leading to some changes in the original shape of the social insurance in France. Thus, for some contributions, the link between contributions paid and benefits received is thin: it is the case, for example, for the sickness insurance, as the treatments' reimbursements from the Social security scheme are defined without taking into consideration the amount of contributions paid. On the other hand, faced with the decline in payroll and then in contributions collected, the Social security scheme had to be funded through alternative sources not only tied to earnings: for this purpose, the CSG and the CRDS have been created as flat-rate taxes also based on capital and substitution incomes. Although CSG and CRDS appear on the payslip as social security contributions, they can also be considered direct taxes, as their payment does not open any specific social entitlement.


Contributions partially tied to benefits, funding more and more resting upon taxes: those are, in a nutshell, the main two characteristics defining the current French Social security scheme, and which make it a hybrid structure.


%%%%%%%%%%%%%%%%%%%%%%%%%%%%%%%%%%%%%%%%%%%%%%%%%%%%%%%%%%%%%%%%%%%%%%%%%%%%%%%%%%
%%%%%%%%

\section{Social security scheme architecture}


The French Social security scheme is characterized by a great variety of different schemes, as a result of its progressive extension. A fist distinction should be drawn between \og base \fg~ schemes, which, whatever the scheme in question, cover four great social risks: sickness, old-age, work accidents-professional diseases and family, and the \og complementary \fg~ schemes. The latter are not Social security schemes strictly speaking: yet, since it is mandatory for private-sector employees to contribute to a complementary pension scheme, this type of  schemes developed until representing more than 4~\% of GDP in 2012. Similarly, the unemployment insurance, though often wrongly considered a branch of the Social security scheme, is actually an autonomous structure. However, these definitions are purely institutional: from an economic point of view, we should therefore consider all these schemes as insurances covering the population against social risks.


\subsection{The different Social security schemes}

\subsubsection{The Social security schemes strictly speaking}

\subsubsection*{The general scheme}

The general scheme, covering the biggest part of private sector employees, and part of the civil servants, represents, for its financial weight as well as for the number of insured persons, the main Social security scheme. Since the 1967 ordinance, it is divided into autonomous branches. These four branches are administered by Social security funds: National sickness insurance fund for wage-earners (\textit{Caisse nationale d'assurance maladie des travailleurs salari\'es} (CNAM-TS)), National family allowances fund (\textit{Caisse nationale des allocations familiales} (CNAF)) and National old-age insurance fund (\textit{Caisse nationale d'assurance vieillesse} (CNAV)), which have the status of private-law organizations with public service assignment. The Central agency of social security organizations (\textit{Agence centrale des organismes de s\'ecurit\'e sociale} (Acoss)) is responsible for the collection of social security contributions on behalf of the different funds of the general Social security scheme. This collection task is sometimes considered a Social security scheme in its own right. Relations between the different Social security funds and the State are ruled by the Social security code (SSC).


\subsubsection*{The schemes specific to the public sector}

To insure its civil servants, the three public services are affiliated to specific Social security schemes, different from the general one. Regarding the sickness insurance, the mutual health insurances for civil servants (which receive part of the sickness contributions paid on civil servants base pay) are in charge of reimbursing the spending, whereas the different administrations are responsible for the payment of cash benefits (in the case of wage preservation). This organization however, only applies for tenured civil servants in the State civil service (SCS): health benefits for tenured civil servants in hospital and territorial civil services are managed by local health insurance funds (\textit{Caisse primaire d'assurance maladie}, CPAM)\footnote{Or by regional health insurance funds, in the case of Ile-de-France.}, as for private-sector employees, and then come under the general scheme. Nonetheless, in some departments, agreements have been reached between the CPAM and mutual health insurances for hospital and territorial civil servants to delegate to them the management of health benefits. 

Family allowances contributions are managed by the CNAF as in the general scheme. The base old-age insurance of tenured civil servants in the State civil service is run by the Pension Service of the State, created as an independent service and linked to the General directorate of government finance (\textit{Direction g\'en\'erale des finances publiques}, DGFIP) in 2009. On the other hand, the National pension fund for civil servants in local authorities (\textit{Caisse nationale des retraites des agents des collectivit\'es locales}, CNARCL) transfers the base pension benefits to agents in the hospital and territorial civil services, and collects the corresponding contributions. Finally, the disability insurance is managed by the Temporary disability benefits fund for civil servants in local authorities (\textit{Fonds d'allocation temporaire d'invalidit\'e des agents des collectivit\'es locales}, FATIACL) for tenured civil servants in hospital and territorial civil services, whereas it falls within the jurisdiction of the Pension Service of the State for civil servants of the State civil service.

Contractual workers come under the general scheme for the four branches of the Social security scheme.

\subsubsection*{The non-farming self-employed scheme}

Self-employed workers are part of a category bringing together non-farming non wage-earning workers: they are affiliated to their own scheme since 1948\footnote{Law \no 48-101 of 01/17/1948}. It is a specific scheme but offering the same guarantees. It consist of a mandatory maternity-sickness insurance, a mandatory base old-age insurance, mandatory and optional complementary pension schemes and contingency funds, including or not a disability-death insurance, and an optional insurance for work accidents and unemployment risk. Self-employed workers also benefit from the same family allowances, but are required to pay a personal contribution, potentially supplemented with the family contribution they pay as employers.

The designation "self-employed workers" is defined in opposition to wage-earners: it brings together the persons who do not usually employ workers, as part of their activity, apart from their spouses, their minor children, or apprentices. Are also considered self-employed the persons who did not employ a worker more than 75 days or 500 hours during the calendar year.\footnote{\textit{Memento pratique social 2009, ed. Lefebvre}.}. We can decompose this category into two subcategories: craftsmen, traders and manufacturers on the one hand, and liberal professionals on the other hand, these two entities being subjected to different rules regarding social security contributions and benefits.\newline

\textbf{Craftsmen, traders and manufacturers} \newline

According to the definition of the INSEE (\textit{Institut national de la statistique et des \'etudes \'economiques}, i.e. the National Institute of Statistics and Economic Studies), craftsmen correspond to active people creating value for an economic capital as managers of their own enterprise, but working alone or employing only few employees, in a field where the handmade aspect is important, outside farming\footnote{INSEE website.}.
This definition highlights three important criteria: \emph{(i)} the active person must be manager, associate or family worker not employed; \emph{(ii)} he must not employ more than 9 persons; \emph{(iii)} he must have a handcraft activity, that is to say coming under the field of manufacture, maintenance, building, repairing, transport, or cosmetic care. 

The category of industrial and commercial activities differs to the extent that the activity is not mainly handcrafted, and that it does not require a very high cultural capital. It brings together activities such as trade, food service, administrative services, education services, health, social action and some other services; provided that the intellectual and artistic aspect do not prevail in these activities\footnote{INSEE website.}. \newline

Regarding social protection, craftsmen, traders and manufacturers are gathered in a unique category falling under the Social self-employed scheme (\textit{R\'egime social des ind\'ependants}, RSI). Are affiliated:
\begin{itemize}
\item Craftsmen registered to an official directory of occupations (\textit{r\'epertoire des m\'etiers}) or carrying out an activity coming under the craft category by decree;
\item Traders and manufacturers registered to an official commercial and industrial directory (\textit{registre du commerce et de l'industrie}), carrying out an activity coming under the commercial or industrial categories by decree or paying the business tax as traders;
\item Partners or managers of a limited liability enterprise with a unique associate (\textit{entreprise unipersonnelle \`a responsabilit\'e limit\'ee}, EURL), partners in a collective company (\textit{soci\'et\'e en nom collectif}, SNC), controlling directors of a limited liability company (\textit{soci\'et\'e \`a responsabilit\'e limit\'ee}, SARL), manager member of a controlling directors board or associate carrying out a non wage-earning activity within a SARL, members of a joint-venture, associates in liberal partnerships with stock-holders or in liberal limited partnerships, members of a \textit{de facto} partnership with a craft, industrial or commercial activity.
\end{itemize}

However, equal or minor managers of a SARL, or SA and SAS managers come under the general scheme as wage-earners. Similarly, as soon as the activity implies a subordinate relationship, as in some medical occupations for instance, workers come under the general scheme. The affiliation can also be decided by a law, as for models, artist-authors, etc.

Texts ruling the old-age insurance define the type of workers falling under the non-farming self-employed scheme. The scope of the maternity-sickness insurance of the non-farming self-employed scheme is then determined in reference to the old-age insurance, subject to some exceptions.


% Elle correspond aux entrepreneurs individuels et EIRL (entrepreneurs individuels \`a responsabilit\'e limit\'ee), aux g\'erants et associ\'es de SNC (Soci\'et\'e en nom collectif) et EURL, et aux g\'erant majoritaire de SARL (Soci\'et\'e \`a Responsabilit\'e Limit\'ee).

Self-employed workers must join the RSI. It protects them against the sickness and old-age risks, and also includes a complementary pension scheme. It makes it a major representative with regard to social security contributions and benefits.
The RSI is administered by insured people's representatives, and its objectives and assignments are defined in the Management and Objectives Collective Agreement (MOCA, \textit{Convention d'Objectifs et de Gestion}) signed with the State, and renewed on a regular basis (the one in effect covers the 2012-2015 period). The RSI is managed at the regional level through a network of 28 health insurance funds for craftsmen, traders and manufacturers, and authorized organizations to which the RSI delegates the management of the sickness-maternity-daily sickness leave benefits.

Some branches of the social protection are managed by other organizations, as for the family branch for traders and manufacturers, who must turn to the Urssaf. However, we observe a convergence trend in terms of rights and reference organizations, as shown by the harmonization in January 2013 of the complementary pension and death insurance schemes for craftsmen and for traders, in terms of rate, tax base, retirement age and reversion age.\newline


\textbf{The liberal professions}\newline

According to the INSEE definition, the liberal professions bring together individuals carrying out an activity regulated by the possession of a diploma, and the compliance with some ethical rules. This regulation is, in general, carried out by an ordinal organization, as for doctors, pharmacists, lawyers, notary, professional accountants, architects, etc.

It can also include other legal or technical occupations less strictly regulated but still requiring a specific diploma. Liberal professional being, by definition, not employed, the doctors and pharmacists working as employees cannot be classified as liberal professionals. \newline
%Regarding mixed activities (both self-employed and wage-earning), the convention used states that the liberal activity takes priority over the wage-earning activity, whatever the working time and income linked to each activity.\newline

The management of the social protection is more complex in the case of liberal professionals, as they have not a unique representative: though they depend on the RSI for the sickness-maternity-daily sickness leave benefits, they fall under the Urssaf for the family branch, and under the National old-age insurance fund for liberal professions (\textit{Caisse nationale d'assurance vieillesse pour les professions lib\'erales}, CNAVPL) for their base pension, and under different professional sections managed by the CNAVPL for the complementary pension. Liberal professionals with several occupations are required to subscribe to only one professional section\footnote{Art. R 643-3 of the SSC.}.\newline

\texttt{Legislative references:} RSI website; \textit{M\'emento pratique social 2012}, ed. Francis Lef\`ebvre. ; Art. L 133-6, L 133-6-1 to L 133-6-6 of the SSC \newline


\textbf{Special affiliation situations} \newline

\underline{Co-workers liberal professionals} \newline

Liberal co-workers working with a professional of the same occupation must both affiliate to the non-farming self-employed scheme.\newline

\underline{Multiple job holding workers}\newline

This paragraph applies both for all categories of self-employed workers.

\begin{itemize}
\item In case of coexisting wage-earning and non-farming self-employed activities, workers contribute simultaneously to both schemes. As part of the old-age insurance, in addition to the contribution directly paid on gross earnings, they pay a personal contribution to the non-farming self-employed scheme, as for the family contribution. For the sickness insurance, though they contribute to both schemes, they only depend on one scheme for the payment of their benefits -- the one corresponding to their main activity. Therefore, persons working both as employees and non farming self-employed must pay all the contributions due to the RSI, except the contribution funding daily sickness leave benefits, and all the general scheme contributions on their gross earnings. However, they are not affected by the payment of the minimal contribution for the sickness branch.

The wage-earning activity is considered the main activity if the person works, during the reference year, at least 1200 hours as employee or equivalent, and if he earns an income equal or greater than what he earned as part of his self-employed activity.


\item In case of coexisting non-farming self-employed and farming activities, workers contribute on the totality of their earnings in compliance with the effective rules in the scheme corresponding to their main activity.
If one activity is permanent and the other seasonal, it is the permanent activity which prevails. If both activities are permanent, the main activity is determined according to the criteria of the working time and the income yielded: the activity is classified as main activity if the income yielded (the definition of income used is the CSG tax base) is higher than the income yielded by the other activity, and if the worker dedicates more time to it, according to his self-declaration. If the two criteria do not match, the income criteria is prioritized.
\end{itemize}
 
\vspace{0.5cm}

\underline{Co-worker spouses}\newline

Regarding the craftsmen, traders and manufacturers spouses, they can be classified under three status, if they participate in the activity on a regular basis: the spouse is either employed, associate or co-worker spouse. The definition of a co-worker spouse implies that he carries out a regular activity within the craft, commercial or industrial enterprise of his spouse, that he does not earn a wage, and that he is not classified as associate. For the sickness insurance, he benefits from the status of assignee, and can receive cash benefits. Regarding old-age insurance, the co-worker spouse of the craftsman is affiliated independently, and is then required to pay contributions for the funding of the base and complementary pensions schemes, and for the disability-death insurance: the tax base is equal either to one third of the SSC or to the professional earnings of the head of the enterprise, within the limit of the minimal contribution. The co-worker spouse of the trader can be insured as part of the old-age insurance on a voluntary basis.


\texttt{Legislative references:} Art. L 613-3, art. L 613-4 of SSC; art. L 171-3, 1\^{st} paragraph of the SSC; art. R171-3 paragraph 2 of the SSC.



\subsubsection*{Other schemes}

Other Social security schemes are designed to cover some professions in a particular way. Thus, the scheme run by the Farming social mutual fund (\textit{Mutuelle sociale agricole}, MSA) insures all the farming wage-earners and farmers. Besides, some public establishments, companies or occupations have their own schemes, called special schemes or equivalent special schemes. Each scheme has its own contributions and benefits definition rules, which will not be presented in this \emph{IPP legislative guide}\footnote{The official website \texttt{www.securite-sociale.fr} contains additional informations on other French Social security schemes, and especially an institutional organization chart at the following address: \texttt{http://www.securite-sociale.fr/Organigramme-institutionnel-de-la-Securite-sociale}.}.


\subsubsection*{The local scheme in Alsace-Moselle}

The Alsace-Moselle territory \footnote{L'Alsace-Moselle corresponds to a set of three departments (le Bas-Rhin (67), le Haut-Rhin (68) et la Moselle (57)) which has been re-annexed  to Germany between 1871 and 1919 after the Franco-Prussian war in 1870-1871.} has its own Social security scheme, said local scheme (LS). The LS is self-managed at the local level since 1955 by employees trade unions representatives. This scheme is actually a complement to the general Social security scheme, which guarantees additional health, maternity, disability and death benefits to its recipients. 

The affiliation to the LS is mandatory for:
\begin{itemize}
\item employees carrying out their activity in at least one of the three departments belonging to the Alsace-Moselle territory, wherever the company's head office is located;
\item employees of a company established in the Alsace-Moselle territory and carrying out an itinerant activity in other departments.
\end{itemize}

Besides, recipients of old-age benefits (retirement or early retirement pension) as well as recipients of unemployment benefits can be granted additional benefits from the LS, provided they have been affiliated to the LS as employees for a certain number of years. In case of affiliation, a contribution is paid on pensions.

The employer of a worker coming under the LR is required to pay an additional contribution to the local scheme (deducted from gross earnings, as it is an employees' additional contribution). The employee must show his payslip displaying the additional health contribution to his Local health insurance fund for the entitlements to the LS to be effective. \newline

\texttt{Legislative references:} {art. L325-1 of the SSC}

\subsubsection{Complementary pension schemes}

The initial organization of the Social security scheme has been supplemented with complementary pension schemes aiming at providing more benefits to old-age insurance affiliated, whose advantages were rather low.

In the private-sector, the main two structures are the Arrco (\textit{Association pour le r\'egime de retraite compl\'ementaire des salari\'es}, Association for the employees' complementary pension scheme), and the Agirc (\textit{Association g\'en\'erale des institutions de retraite des cadres}, General association for executive workers' pension scheme), set up by inter-professional agreements reached at the national level. These schemes work as pay-as-you-go systems, funded by contributions paid by employers and employees. These contributions, optional at first, became mandatory in 1970. However, these complementary pension schemes do not fall under the Social security scheme under the supervision of the State, but are private though mandatory schemes, managed by employers' and employees' representatives. Their resources are drawn from contributions, and their spending are however counted up as public administrations spending in national accounts, even if the State has no direct control over these organizations.

The public sector has its own complementary pension scheme, the additional pension scheme of the public sector (\textit{Retraite additionnelle de la fonction publique}, RAFP). It is a point system allowing civil servants (tenured ones only) from the three civil services to acquire a complement to their base pension. The affiliation to the RAFP became mandatory in 2005.

Non-permanent civil servants contribute to a specific complementary pension scheme, the Complementary pension institution for non-permanent public agents (\textit{Institution de retraite compl\'ementaire des non-titulaires de la fonction publique}, IRCANTEC). The affiliation to this unique scheme, created in 1971 from the merging of two preexisting complementary pensions schemes, became mandatory in 1973.

Finally, professions with specific base schemes (farm workers and farmers, self-employed, railway workers, etc.) have also their own complementary pension schemes.


\subsubsection{The unemployment insurance scheme}

A new entity has been created in 1958 to cover the unemployment risk, following an insurance logic rather than an assistance one, as it has been the case until this date. This new insurance scheme, run by the National inter-professional union for employment in the business and industrial sectors (\textit{Union nationale interprofessionnelle pour l'emploi dans l'industrie et le commerce}, Unedic), has merged with the National job-center (\textit{Agence nationale pour l'emploi}, ANPE) in 2009, and gave birth to \emph{P\^ole Emploi}. This organization's assignment is to transfer benefits to people unemployed and to provide them with placing, advice and training services.

This scheme grants benefits only to unemployed people previously employed in the private-sector. Contractual civil servants, when their contract comes to an end, can request the unemployment insurance benefits (\textit{Allocation de retour \`a l'emploi}, ARE), determined according to the same rules as the ones which prevail for private-sector employees, to their formerly employing administration.

\subsubsection{Overview}

The following table provides an overview or the main social insurance schemes for employees existing today in France, and the population covered with their benefits.

\begin{fig}[14cm]{Architecture of Social security and other social insurance schemes in France.}
{\graphique[1.5]{Architecture_SS.pdf}}
{\footnotesize \textsc{Note:}~The self-employed, farming workers, students, parliamentarians schemes as well as special schemes are not reported here.}
\end{fig}

\begin{fig}[14cm]{Architecture of the social protection scheme for non farming self-employed workers in France.}
{\graphique{orga_institutionnelle_inde.pdf}}
{}
\end{fig}

It should be noticed that the social protection architecture in France does not stem from any economic necessity, but from an institutional history, from political and social stakes. If we consider that unemployment is a risk people should be insured against at the society level, just as for the disability risk, there is -- \emph{a priori} -- no reason to institutionally distinguish the unemployment insurance from the four other Social security branches.
 
In return, we could imagine the health insurance and the family branch, as they exist today, being institutionally separated from old-age insurance, work accidents and unemployment branches, as the entitlements acquired do not depend on the amount of contributions paid by the insured worker, but on the \og seriousness \fg~ with which the risk came true. For the family branch, it is indeed the number of children and their age which determines the monthly amount of family allowances. But can having children be considered the realization of a risk? Family allowances are more a response to a horizontal redistribution objective aiming at smoothing out differences in standards of living between households with and without children, like the family quotient in the income tax computation, than a motive for social insurance. Thus, to separate the family branch of the Social security scheme would not be an \emph{economic} non sense. Besides, we could also imagine the health contributions being independent from earnings as the health benefits are; however, people with higher earnings contribute more (in absolute terms) to the CNAM-TS.

Moreover, the complementary pension schemes have been implemented to overcome the insufficient benefits paid by the base pension scheme already existing\footnote{As the word \og complementary \fg indicates.}. Initially optional, the subscription to one of these schemes became mandatory. We could imagine a given individual receiving his pension from the same fund. It is worth noting that, as they are not mandatory, the supplementary healthcare organizations are not considered on the same level than complementary pension schemes. However, the trend towards an increase of the complement to be paid on a certain number of medical treatments, and the effort of the legislator to encourage the generalization of the supplementary healthcare contracts\footnote{Implementation of a means-tested aid to the purchase of a supplementary healthcare organization, obligation for employers to make available for all their employees a minimal collective healthcare insurance (coming into force on January 1\up{st} 2016, etc.} allow us to imagine a harmonization of the institutional status of the supplementary healthcare organization with the complementary pension schemes.

A last point should be underlined: the partition of the French population into sub-populations falling under different schemes according to the professional situation of the \og household head \fg: private-sector wage-earner, tenured civil servants, self-employed, employee of a company belonging to a special scheme, student, etc. To each of these situations corresponds a different scheme, determining its own contributions rates and insurance benefits. If they reflect the progressive construction of the French Social security scheme, these differences between schemes are at the root of some administrative difficulties for persons going from one scheme to another, and arouse suspicion on inequality of treatment. Although it would be very complex to put into practice, nothing, from an economic perspective, prevents us from imagining a merging of all the different schemes specific to different professional categories.


\begin{fig}[12cm]{Economic cutting up of the French Social security scheme.}
{\graphique[1.5]{Architecture_eco.pdf}}
{\footnotesize \textsc{Note:}~The self-employed, farming workers, students, parliamentarians schemes, as well as special schemes are not reporter here.}
\end{fig}


\subsection{The funding of the schemes: contributions and payroll taxes}

\subsubsection{The different contributions and payroll taxes}

Today, a private-sector employee receives a payslip which, apart from mentioning the net earnings and the net taxable earnings, includes also about twenty lines corresponding to as many different contributions or taxes, based on gross earnings. The standard payslip of a public-sector civil servant, if it is relatively more simple to understand, includes a line for each of the contributions and payroll taxes -- around ten -- that apply to civil servants' earnings. Though they are not all meant to fund the social insurance schemes, this \emph{IPP Legislative Guide} provides a detailed presentation of all of them.

The different contributions and payroll taxes fall into three categories:
\begin{itemize}
\item \textbf{Social security contributions broadly speaking.} They bring together contributions funding the four branches of the Social security scheme (sickness-maternity-disability-death, old-age, family, work accidents). Although the unemployment insurance scheme is formally separated from Social security scheme, contributions intended to fund it are, in practice, regarded as social security contributions, as the contributions designed to fund the complementary pension schemes AGIRC-ARRCO;
\item \textbf{Flat-rate income taxes, CSG and CRDS}: also paid on substitution and capital income and not directly meant to fund a specific social protection scheme, they differ from social security contributions strictly speaking. Even though they finance different funds of the Social security scheme, they do  not open any specific social entitlement;
\item \textbf{Taxes on earnings not subjected to social security contributions}: these taxes of a particular type have developed since the end of the 1990's, to tax some elements of earnings not subjected to the CSG. Often created in a \emph{ad hoc} way to reduce the social and tax advantage some elements of earnings benefit from, these flat-rate income tax are still more favorable in terms of rate than the base contributions; 
\item \textbf{Payroll taxes (PT)}. They bring together taxes meant to fund apprenticeship, continuing training, housing building, as well as the wage tax. \\
\end{itemize}

These four big categories are complemented with "penalty" taxes firms must pay if their workforce does not meet some criteria:
\begin{itemize}
\item \textbf{Fee for the respect of parity on the workplace}: Companies (or companies part of a group) of at least 50 employees are subjected to a fee when they did not reach an agreement regarding professional egality;
\item \textbf{Fee to favor the senior employment}: due between January 1\up{st}, 2010 and September 30\up{th}, 2013 (date when the generation contract (\textit{contrat de g\'en\'eration}) came into force), this tax aimed at penalizing companies of more than 50 employees which did not reach an agreement regarding the employment of workers over 55 years-old;
\item \textbf{Fee to favor young employment}: the implementation of the generation contract in 2013\footnote{The generation contract is a measure aiming at encouraging integration in the labor market of young people under 26 years-old without penalizing older employees. Technically, it plans to condition financial aid to the hiring of young people under permanent contracts, and, at the same time, to preserve earnings of employees above 55 years-old.} introduces a financial penalty for companies of 300 employees and more (or part of a group of 300 employees and more) which have not finalized and estimated a plan or agreement at the company level and in compliance with the objectives of the generation contract. This fee replaces the 1~\% senior fee;
\item \textbf{Painfulness fee}: Companies of 50 employees or more (or part of a group of 50 employees or more) whose at least 50~\% of the workforce is subjected to some professional risks must reach an agreement or draw up an action plan at the company or group level laying down some objectives to achieve in terms of painfulness prevention. Companies not having finalized such an agreement or plan are liable for the payment of a fee;
\item \textbf{Fixed-term contract fee}: companies employing workers on a fixed-term contract must pay a fee equivalent to 1~\% of gross earnings of each employee on a fixed-term contract;
\item \textbf{Fee for insufficient employment of disabled workers}: companies have some obligations regarding the employment of disabled workers. Each company (private or public) must have at least 6~\% of its workforce having an administratively recognized handicap. If the company does not comply with the rule, the employer is liable for the payment or an annual contribution proportional to the number of disabled workers he should have hired to meet the obligation. This contribution, paid to the Association of management of the fund for the integration of disabled persons (\textit{l'Association de gestion du fonds pour l'insertion des personnes handicap\'ees}, Agefiph), is used to finance actions helping at the professional integration of disabled persons.  \\
\end{itemize}

We can draw a distinction among contributions and payroll taxes, according to their contributory feature, that it to say according to whether or not they grant to employees any specific social entitlement. The OECD (\citet{oecd201quatre}, Appendix four, \S 35) makes the contributory character the main criteria to distinguish between the social security contributions and the payroll taxes. In practice, the link between contributions paid and benefits received can be weak for some social security contributions (especially for sickness insurance and family allowances contributions). Nonetheless, it is indeed the payment of the social security contributions that allows to open some social rights, whereas the payroll taxes do not determine the entitlement to any social benefits.

%%% V\'erifier \`a partir de l\`a
The CSG and CRDS classification arouses some debate: the OECD considers them direct taxes, whereas the Court of Justice of the European Union (CJEU) classifies them as social security contributions on the grounds that their receipts are assigned to the funding of the Social security scheme\footnote{The Court of Justice of the European Union replaced the Court of Justice of the European Communities, which decided on this topic in 2000 (CJEC, February 15\up{th}, 2000, aff. C-169/98, Commission c./France).}. The French domestic law is more ambiguous, provided that the measures regarding these contributions fall under both the General tax code (CTC) and the Social security code (SSC). On the other hand, members of the Constitutional Council assimilate the CSG and CRDS to flat-rate income taxes\footnote{In 2000, the Lionel Jospin government proposed that the CSG became a progressive tax, but the bill has been rejected on the grounds that, as the CSG is a tax, a non-proportional tax scale should take into consideration the income of all the members of the fiscal household of the individual to estimate to what extent he could contribute.}, whereas the Supreme court of appeal regards them as a social security contributions\footnote{French Supreme court of appeal, i.e. \textit{Court de Cassation}, 2\up{nd} Civil chamber, March 8\up{th} 2005, \no 03-30.700.}.


\subsubsection{The circuit of contributions and payroll taxes}

Contributions and payroll taxes are paid by the employer. Most of them (social security contributions, CSG and CRDS) are collected by the URSSAF (\textit{Unions de recouvrement des cotisations de s\'ecurit\'e sociale et d'allocations familiales}, the organizations in charge of collecting social security contributions and family allowances). These private organizations constitute a network spread across the national territory, and their main assignment is to recover the contributions designed to fund the general Social security scheme. The URSSAF are also in charge of collecting other contributions meant to fund other social protection schemes, in particular the complementary pension schemes (AGIRC and ARRCO), the National fund for housing aid (\textit{Fonds national d'aide au logement}, FNAL), the Solidarity fund for old-age (\textit{Fonds de solidarit\'e vieillesse}, FSV), and the authorities in charge of the management of urban transportation (\textit{Autorit\'es organisatrices des transports urbains}, organized at the municipality or group of municipalities level). Finally, since January 1\up{st} 2011, unemployment contributions, formerly collected by the Associations for the employment in the business and industrial sectors (\textit{Associations pour l'emploi dans l'industrie et le commerce}, ASSEDIC)\footnote{Created in 1958, the ASSEDIC merged with the ANPE in 2008 into Pôle Emploi.}, are also collected by the URSSAF.

The contributions are then transfered by  the URSSAF to the different structures of the social protection scheme. In the case of the general scheme, it is the Central agency of Social security organizations (\textit{Agence centrale des organismes de S\'ecurit\'e social}, Acoss) which manages the distribution of the contributions to the different branches.


Some contributions and taxes, called the \og payroll taxes \fg (PT), are, on the contrary, considered taxes and collected by the Treasury department. Some of them, such as the wage tax, go directly to the general State budget, while others (especially the apprenticeship tax) are assigned to the funding of specific spending (professional training in the case of the apprenticeship tax).



\subsection{Guidance data}

\subsubsection{Receipts and spending}

Macroeconomic data available each year in the annual Social security Finance act (SSFA) allow to measure the weight of the Social security organization and its different schemes. The general scheme has the highest financial weight, as its spending represent more than 16~\% of the GDP (in 2012). It is also the general scheme which, in absolute terms, has the highest deficit (~0.70~\% of GDP in 2012). All in all, the receipts of all the different Social security schemes, including the complementary schemes, represented 550 billion euros in 2012, equivalent to more than 27~\% of GDP. 


\begin{fig}[14cm]{The different Social security schemes: spending in 2012 (in \% of GDP)}
{\graphique{Securite_sociale_depenses.pdf}}
{\footnotesize \textsc{Note:}~Social security scheme accounts in France in 2012.}
\end{fig}

\begin{fig}[14cm]{The different Social security  schemes: receipts in 2012 (in \% of GDP)}
{\graphique{Securite_sociale_recettes.pdf}}
{\footnotesize \textsc{Note:}~Social security scheme accounts in France in 2012.}
\end{fig}

The decomposition of the spending of the base schemes highlights the prevailing weight of the old-age insurance and sickness, maternity, disability and death insurance branches. They respectively represent 45~\% and 39~\% of spending, against 11~\% for the sum of the work accidents-professional diseases and family  branches.


\begin{fig}[14cm]{Spending of all the base schemes in 2012, by branch (in billion euros)}
{\graphique{Depenses_branches.pdf}}
{\footnotesize \textsc{Source:}~\textit{Les chiffres-clefs de la loi de financement de la S\'ecurit\'e sociale 2012} (The key figures of the 2012 Social security finance act). \\ MMID stands for the sickness-maternity-disability-death branch (\textit{branche maladie-maternit\'e-invalidit\'e-d\'ec\`es.}}
\end{fig}

Social security contributions represent 64~\% of base schemes revenues in 2012; the CSG contributes to the funding of the Social security scheme by 16~\%.

\begin{fig}[14cm]{Revenues of all the base schemes in 2012, by type (in billion euros)}
{\graphique{Recettes_types.pdf}}
{\footnotesize \textsc{Source:}~\textit{Les chiffres-clefs de la loi de financement de la S\'ecurit\'e sociale 2012} (The key figures of the 2012 Social security finance act).}
\end{fig}

\subsubsection{Contributions and payroll taxes and the labor cost in France}

In 1976, the sum of aggregated employers' and employees' contributions rates ($t_e$ + $t_r$) was equal, on average, to 53~\% for the lowest 50~\% of the earnings distribution, and to 49~\% for the top 50~\%. In 2009, these rates are respectively equal to 67 and 78~\%, and fall into two categories:
\begin{itemize}
\item 21 to 22 points of employees' contributions rate;
\item 45 to 56 points of employers' contributions rate. \\
\end{itemize}


\begin{table}[h!]
\begin{center}
\centering
\caption {Aggregated rate of employers' and employees' contributions in 2014.}
\vspace{0.2cm}
\small
\begin{tabular}{|l||c|c|c|c|}
\cline{2-5}
\multicolumn{1}{c|}{} & \multicolumn{2}{|c|}{Employees' contribution rate}  & \multicolumn{2}{c|}{Employers' contribution rate} \\
\hline
\multicolumn{1}{|c|}{} {Monthly gross earnings} & \multicolumn{1}{|c|} {Non-executive} & \multicolumn{1}{|c|} {Executive} & \multicolumn{1}{|c|} {Non-executive}  & \multicolumn{1}{|c|} {Executive}  \\
\hline
\hline
1 445 \euro~(Smic)   & 21.4~\% & 21.6~\% & 22.7~\% & 22.9~\% \\
2 000 \euro  & 21.4~\% & 21.6~\% & 41.9~\% & 42.2~\% \\
4 000 \euro  & 19.3~\% & 21.1~\% & 48.5~\% & 48.9~\% \\
8 000  \euro & 15.5~\% & 20.3~\% & 48.1~\% & 48.8~\% \\
\hline
\end{tabular}
\end{center}
\footnotesize \textsc{Source:}~\TAXIPP model. \\ \textsc{Note:}~Rates have been computed for fictional private-sector employees working full-time, in a company of more than 20 employees not subjected to the wage tax and to the local Alsace-Moselle scheme.
\end{table}

\begin{table}[h!]
\begin{center}
\centering
\caption {Gross earnings, labor cost and net earnings in 2014.}
\vspace{0.2cm}
\small
\begin{tabular}{|l||c|c|c|c|}
\cline{2-5}
\multicolumn{1}{c|}{} & \multicolumn{2}{|c|}{Net earnings}  & \multicolumn{2}{c|}{Labor cost} \\
\hline
\multicolumn{1}{|c|}{} {Monthly gross earnings} & \multicolumn{1}{|c|} {Non-executive} & \multicolumn{1}{|c|} {Executive} & \multicolumn{1}{|c|} {Non-executive}  & \multicolumn{1}{|c|} {Executive}  \\
\hline
\hline
1 445 \euro~(Smic)   & 1 128 \euro & 1 126 \euro & 1 663 \euro & 1 666 \euro \\
2 000 \euro  & 1 562 \euro & 1 559 \euro & 2 687 \euro & 2 692 \euro \\
4 000 \euro  & 3 209 \euro &  3 135 \euro & 5 797 \euro & 5 812 \euro \\
8 000  \euro & 6 722 \euro & 6 335 \euro & 11 567 \euro & 11 617 \euro \\
\hline
\end{tabular}
\end{center}
\footnotesize \textsc{Source:}~\TAXIPP model. \\ \textsc{Note:} Earnings variables have been computed for fictional employees working full-time in a company of more than 20 employees subjected to the wage tax and not to the local Alsace-Moselle scheme.
\end{table}

These rates lead to two statements:
\begin{enumerate}
\item There is today an important gap between the cost incurred by the employer to employ a worker and the net earnings received by employees. For the top 50~\% of earnings distribution, the earnings tax wedge is about 50~\% (equal to the ratio $\frac{(1-t_e)}{(1+t_r)}$), meaning that the employee directly receives only half what he costs to his employer;
\item Contributions and payroll taxes based on earnings became redistributive, whereas they were regressive until the mid-1990's.
\end{enumerate}


\begin{fig}[14cm]{Evolution of the effective contribution rate between 1976 and 2009 - average by decile of earnings.}
{\graphique{Tx cotis par decile.png}}
{\textsc{Source:} \textit{D\'eclaration Annuelle de Donn\'ees Sociales (DADS), Fichier Panel 2010} (Annual data compiled by employers containing the wage declaration and other employees-related informations). \\ \textsc{Note:} Contributions have been simulated with the \TAXIPP model on private-sector employees having worked full-time the whole calendar year.}
\end{fig}

Contributions and payroll taxes which are not computed from a unique rate applied to the whole tax base have a regressive tax scale, as the marginal rate \textit{decreases} with the level of earnings. As the weight of social security contributions as a proportion of gross earnings increased, this feature has been subjected to controversy. To make higher earnings contribute more to fund the social protection scheme, the ceiling have been lifted on some contributions: it is the case for the sickness-maternity-disability-death insurance, whose contribution was based only on the part of earnings under the social security ceiling before the 1980's, and which is paid on all the earnings since 1980. Similar changes have been observed for old-age insurance, work accidents-professional diseases and family allowances contributions in the 1980's and 1990's. Introducing uncapped flat-rate taxes -- the CSG and CRDS -- public authorities followed the same trend. This trend towards uncapped contributions and payroll taxes explains the convergence of the average tax rates by deciles between the mid-1970's and the mid-1990's.

However, removing the ceiling on contributions tax base is not sufficient to fully explain the overall redistributive feature of contributions and payroll taxes today, as higher earnings (especially those above 4 SSC, representing 1~\% of private-sector earnings today) are still subjected to a lower marginal tax rate. The actual decrease in the average contribution rate for the lower deciles of earnings stems in the fact that policies introducing reductions in employers' contributions on low earnings have been implemented. To fight against a persistent mass unemployment affecting especially less skilled workers, French governments, starting in 1991, have granted reductions in employers' contributions (see chapter \ref{Exonerations}), leading to an average contribution rate inferior for low earnings than for high earnings, reversing the situation that prevailed before the 1990's.


%%%%%%%%%%%%%%%%%%%%%%%%%%%%%%%%%%%%%%%%%%%%%%%%%%%%%%%%%%%%%%%%%%%%%%%%%%%%%%%%%%%%
%%%%% Section 3: concepts importants

\section{Key concepts}

\subsection{Contributions tax scale}

As in most of the OECD countries, contributions and payroll taxes are due starting from the first euro of earnings (\citet{oecd2011}, \emph{Special features: Trends in personal income tax and employee social security contribution schedules}, p. 59). Some contributions and taxes, such as the flat-rate taxes CSG and CRDS, or the health insurance contribution, are based on the totality of earnings with a unique rate. But most of them have a marginal tax rate \textit{diminishing} with earnings, as there exists an upper bound above which a reduced or zero rate is applied. Besides, some social security contributions are specific to executive workers (such as the contributions collected by Agirc).


\subsection{The Social security ceiling (SSC)}

The Social security ceiling (SSC) is an amount set by an order published in the Official Journal. It is the basis for the computation of most of contributions and payroll taxes, as the different tax brackets are defined as multiples of the SSC; thus, the tax base must be compared to this parameter to determine the effective rate to apply and the amount of the contribution.\newline

\begin{table}[h!]
\caption{Definition of the gross earnings brackets as a function of the Social security ceiling.}
\small
\begin{center}
\begin{tabular}{|l|c|c|}
\hline
Portion of gross earnings & Executives & Non-executives \\ \hline
1\up{st} bracket & below 1 SSC & below 1 SSC \\
2\up{nd} bracket & between 1 and 4 SSC & between 1 and 3 SSC \\
3\up{rd} bracket & between 4 and 8 SSC & between 3 and 4 SSC \\
4\up{th} bracket & above 8 SSC & above 4 SSC  \\
\hline
\end{tabular}
\end{center}
\end{table}

In 2014, the monthly amount of the SSC was set to 3129 euros (\texttt{Order of November 7\up{th}, 2013, OJ \no 0268 of November 19\up{th}, 2013}). Since 1997, it is revalued each year on January 1\up{st} according to the evolution of full-time earnings, the hourly SSC amount being of the same order of magnitude than the average hourly wage in France. Between 1982 and 1997, The reassessment occurred both on January 1\up{st} and July 1\up{st}. Between 1962 and 1982, the reassessment occurred as for recent years, only on January 1\up{st}. \\

\begin{table}[h!]
\caption{Values of Social security ceilings in 2014.}
\small
\begin{center}
\begin{tabular}{|l|c|c|}
\hline
Ceiling & Monthly value & Annual value \\
\hline
1 SSC   & 3 129 \euro   & 37 548 \euro  \\
3 SSC   & 9 387 \euro   & 112 644 \euro \\
4 SSC   & 12 516 \euro  & 150 192 \euro \\
8 SSC   & 25 032 \euro  & 300 384 \euro \\
\hline
\end{tabular}
\end{center}
\end{table}

It is important to keep in mind that individuals earning a salary equivalent to four SSC represented in 2012 less than 2~\% of full-time employees working the whole year. As a consequence, though measures regarding brackets of earnings can have a lot of consequences in terms of money saved for the Social security accounts, they affect only a small part of French wage-earners.\newline

\texttt{Legislative references:} {Article D242-17 of the SSC}.\\
%
%\textbf{P�riodicit� du plafond, r�duction et neutralisation}
%
%En fonction de la p�riodicit� de la paie du salari�, l'employeur doit se r�f�rer � la valeur annuelle, trimestrielle, mensuelle, hebdomadaire, journali�re voire horaire du PSS pour appliquer les taux de chaque tranche et calculer les pr�l�vements sociaux. Dans la tr�s grande majorit� des cas, la p�riode de paie correspond au mois civil, et c'est donc le plafond \textit{mensuel} qui est utilis�. Les professions ayant des r�mun�rations � p�riodicit� irr�guli�res peuvent se voir appliquer un plafond hebdomadaire, journalier ou m�me horaire. Des r�gles particuli�res s'appliquent pour d�terminer le plafond en cas de temps partiel, d'employeurs multiples ou d'absences du salari�. \\


\subsection{Employer's rate, employee's rate and concepts related to earnings}

In the public debate regarding social security contributions, different concepts of earnings are used, which need to be defined. Private-sector employees can generally bargain over their \emph{gross} earnings, and similarly, civil servants receive a \emph{gross} base pay (\textit{Traitement indiciaire brut}) varying with their rank and their position in the hierarchy. However, what is directly earned by workers is lower than gross earnings. On the contrary, what is paid by the employer to employ a worker is higher than gross earnings.

Although it is always the employer which is required to compute and transfer the contributions to the collection organizations, the contributions and payroll taxes can be either formally the employers' or employees' responsibility. The statutory liability of some contributions is entirely borne by employers (as payroll taxes), whereas the statutory liability of other contributions and taxes (such as the CSG and CRDS) is entirely borne by employees. Finally, the liability of most of the social security contributions narrowly defined (contributions for the four branches of the Social security scheme, for the unemployment insurance and complementary pension schemes) is borne both by employers and employees, with an employers' part and an employees' part, based on the same tax base.
 
Employees' contributions and flat-rate income tax are deducted from gross earnings \textit{w} to compute the net earnings \textit{n}, whereas the labor cost \textit{z} (also called supergross earnings) is obtained by adding the employers' contributions and payroll taxes to the gross earnings. Denoting $t_r(w)$ (respectively $t_e(w)$) the average employers' (resp. employees') contributions rate for a gross earnings at $w$, we have, by definition:
\begin{eqnarray}
z=(1+t_r(w))w \\
n=(1-t_e(w))w
\end{eqnarray}

Net earnings can however be defined several ways, as:
\begin{itemize}
\item \textbf{Taxable earnings}; in this case, $t_e(w)$ corresponds to the sum of employees' contributions and deductible CSG;
\item \textbf{Net earnings}; $t_e(w)$ is then defined as the ratio of the sum of employees' contributions, CSG and CRDS over gross earnings;
\item \textbf{Net-net earnings}, or \textbf{disposable income}; this concept corresponds to the earnings left after deduction of contributions, payroll taxes and income tax (IT). In that case, $t_e(w)$ is more complex to defined as the IT does not only depend on earnings from work, and also varies with the income of other members of the tax household.
\end{itemize}

\begin{fig}[14cm]{Links between the concepts of earnings.}
{\graphique[0.9]{Sal_superbrut_brut_net.pdf}}
{\footnotesize \textsc{Source:}~ \TAXIPP model. \\ \textsc{Note:}~The computation of the variables has been done from a fictional individual working full-time in the private sector and paid 2000 euros as his monthly gross earnings. For the IT computation, we assumed the individual lived alone and was not receiving any  other income in addition to his earnings. For the computation of contributions and payroll taxes, we assumed the individual was working in a company of 20 employees not subjected to the wage tax. The value of the contribution rate is the one effective in 2014.}
\end{fig}

When there is both an employers' and an employees' contribution part, the distribution of the contribution varies. The employers' rate is generally higher than the employees' rate; this situation is not the result of any legislative rule but of collective bargaining between employers' and employees' representatives. Indeed, apart from the evolution of the overall rate, the negotiations between employers' and employees representatives focused on the way the employers' and employees' rate were delimited.


\subsubsection*{Computation example: the Arrco contribution}
Let a non-executive worker earn 4000\euro ~ as his monthly gross earnings. Thus, we have 1 SSC < \textit{w} < 3 SSC. The overall amount of the Arrco contribution, for the complementary pension scheme, is calculated as follows: 
\begin{table}[h!]
\begin{center}
\small
\begin{tabular}{|>{\centering\arraybackslash}m{6cm}||>{\centering\arraybackslash}m{6cm}| |>{\centering\arraybackslash}m{6cm}|}
\hline
Bracket of earnings                      & Overall contributions (employee + employer) \\
\hline
\hline
Bracket 1 (earnings below 1 SSC) & $1~SSC \times 0.0763 = 238.74 $ ~\euro \\
Bracket 2 (earnings between 1 and 3 SSC)    & $(w -1~SSC) \times 0,2013 = 175.33 $ ~\euro \\
\hline
\emph{Total}                            & $ 414.07 $~\euro \\
\hline
\end{tabular}
\end{center}
\end{table}

The effective average Arrco contribution rate for this employee is then equal to $ \frac{414.07}{4000}$, equivalent to 10.35~\%.







