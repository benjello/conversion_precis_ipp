\ifx\isEmbedded\undefined
\input{../../Style/ipp-separate}
\setcounter{chapter}{0}
\chapter{\label{Synthese}Synth�se}
\else \fi


Ce \textit{Guide l�gislatif IPP} d�crit la l�gislation des prestations sociales en France, depuis 1945 jusqu'en 2012. Le premier objectif est de pr�senter un panorama complet de la l�gislation afin de pouvoir simuler l'impact redistributif de ces prestations ou leurs possibles modifications. Ce document sert donc de base aux param�tres qui sont utilis�s dans le mod�le de micro-simulation de l'IPP, \TAXIPP.
Pour autant, ce document ne se limite pas � une suite de formules et de param�tre et vise � pr�senter au lecteur l'architecture du syst�me et, dans la mesure du possible, les d�bats qui ont pr�sid� � son �volution.


\ifx\isEmbedded\undefined
\newpage
\bibliography{../../Biblio/biblio-rapport}
\end{document}
\else \fi
