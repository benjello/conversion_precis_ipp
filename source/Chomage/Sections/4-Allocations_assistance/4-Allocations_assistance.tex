\ifx\isEmbedded\undefined
\input{../../Style/ipp-separate}
\setcounter{chapter}{0}
\chapter{\label{Chapitre3}Le r\'egime de solidarit\'e}
\else \fi


%%%%%%%%%%%%%%%%%%%%%%%%%%%%%%%%%%%%%%%%%%%%%%%%%%%%%%%%%%%%%%%%%%%%%%%%%%%%%%%%%%%%%%%%%%

Le r\'egime de solidarit\'e du syst\`eme d'indemnisation ch\^omage vient prendre le relais du r\'egime d'assurance pour des populations priv\'ees d'emploi et qui n'ont pas ou plus de droit aux allocations d'assurance. Les allocations de solidarit\'e r\'epondent donc \`a une logique diff\'erente puisqu'elles n'imposent pas de condition sur la carri\`ere de l'individu et sa participation au financement du syst\`eme: elles ne sont donc pas contributives mais ont pour objectif de fournir un revenu d\'ecent \`a ceux qui n'occupent pas d'emploi.

\section{\label{section:sec1}L'Allocation de solidarit\'e sp\'ecifique}



L'allocation de solidarit\'e sp\'ecifique s'adresse aux ch\^omeurs ayant \'epuis\'e tous leurs droits aux allocations d'assurances, aux ch\^omeurs \^ag\'es de 50 ans et plus et \`a certaines cat\'egories de travailleurs particuli\`eres (artistes non salari\'es, ouvriers dockers occasionnels...etc).

 

\subsection{D\'etermination des r\`egles d'\'eligibilit\'e}

 


L'ASS est vers\'ee aux ch\^omeurs sous conditions de ressources et d'historique d'activit\'e. Les crit\`eres \`a remplir sont les suivants:
\begin{itemize}
\item \textbf{Etre demandeur d'emploi}, ce qui implique donc d'\^etre apte au travail et de rechercher activement un travail;
\item Avoir \'epuis\'e ses droits \`a l'ARE ou \`a la r\'emun\'eration de fin de formation (RFF) ou bien d\'ecider de toucher l'ASS \`a la place de l'ARE car son montant est sup\'erieur;
\item \textbf{Avoir travaill\'e au moins 5 ans (\`a temps plein ou partiel) dans les 10 derni\`eres ann\'ees} pr\'ec\'edant la fin du contrat de travail ayant donn\'e lieu au versement de l'ARE. Si l'activit\'e a \'et\'e interrompue pour \'elever un enfant, une ann\'ee par enfant est d\'eduite des cinq ans, dans la limite de 3 ans. Les p\'eriodes d'activit\'e sont prises en compte quel que soit le type de contrat de travail, et incluent les p\'eriodes assimil\'ees \`a du travail telles que le service national ou la formation professionnelle;
\item \textbf{Avoir des ressources mensuelles inf\'erieures \`a un certain plafond}: en 2015, ce plafond correspond \`a 1 137,50\euro~ pour une personne seule, et 1 787,50\euro~ pour une personne vivant en couple. 
\end{itemize}

\texttt{R\'ef\'erences l\'egislatives:}{ Art. R.5423-1 \`a R.5423-6, L.5423-1 \`a L.5423-6  du Code du Travail}

 

\subsection{Mode de calcul des ressources prises en compte}


Les ressources prises en compte sont celles de l'individu et de la personne avec qui il vit en couple et qui sont soumises \`a l'imp\^ot sur le revenu (y compris l'ASS), et qui ont \'et\'e touch\'ees pendant les 12 mois pr\'ec\'edant le mois de la demande de l'individu. La valeur mensuelle s'obtient en divisant le total des ressources per\c cues sur ces douze mois par 12. Sont donc exclues de la comptabilisation les allocations d'assurance ch\^omage pr\'ec\'edemment per\c cues, les prestations familiales, les allocations de logement, les majoration de l'ASS, la prime forfaitaire mensuelle de retour \`a l'emploi, la pension alimentaire ou prestation compensatoire \'eventuellement vers\'ee \`a un(e) ex-\'epoux(se). Enfin, ne sont pas non plus pris en compte les revenus d'activit\'e per\c cus au cours des 12 mois pr\'ec\'edant la demande si leur perception a cess\'e \`a la date de la demande et s'ils n'ont pas donn\'e lieu \`a un revenu de substitution. Si un revenu de substitution est touch\'e, un abattement de 30\% est appliqu\'e sur la moyenne des ressources auxquelles ce revenu se substitue.

\texttt{R\'ef\'erences l\'egislatives}{Art. R5423-2 du Code du Travail}

 

\subsection{Calcul}

 


Le montant de l'ASS est forfaitaire, et revis\'e chaque ann\'ee\footnote{La prochaine revalorisation est pr\'evue pour avril 2016}: en 2015, il est \'egal \`a 16,25\euro~ par jour, avec, une majoration de 7,07\euro~ par jour pour les 55 ans et plus t\'emoigant de 20 ans d'activit\'e, et les 57,5 ans et plus t\'emoignant de 10 ans d'activit\'e. Depuis le 1\up{er} janvier 2004, cette majoration n'existe plus mais elle continue \`a \^etre vers\'ee \`a ceux qui en b\'en\'eficiaient avant cette date.\\
Le montant mensuel est calcul\'e comme suit:

%\begin{table}[h!]\label{tab:4.2}
%\begin{center}
%\centering 
%\caption{Calcul mensuel de l'allocation de solidarit\'e sp\'ecifique.}
%\vspace{0.2cm}
%\small
%\begin{tabular}{|>{\centering\arraybackslash}m{2.5cm}|| >{\centering\arraybackslash}m{2cm} >{\centering\arraybackslash}m{2.5cm} | >{\centering\arraybackslash}m{2cm} >{\centering\arraybackslash}m{2.5cm}|}
%\hline
%\multicolumn{2}{c}{Personne seule} \\
%\hline
%Revenus inf\'erieures \`a 650,00\euro~~  & Revenus entre 650,00 et 1 137,50\euro~~  \\
%\hline
%Montant journalier (16,25\euro) * nombre de jour du mois & Diff\'erence entre le plafond (1 137,50\euro) et le montant des revenus (vers\'ee si sup\'erieure \`a 16,25\euro) \\
%\hline
%\multicolumn{2}{c}{Personne vivant en couple} \\
%\hline
%Revenus inf\'erieures \`a 1 300,00\euro~~  & Revenus entre 1 300,00 et 1 787,50\euro~~  \\
%\hline
%Montant journalier (16,25\euro) * nombre de jour du mois & Diff\'erence entre le plafond (1 787,50\euro) et le montant des revenus (vers\'ee si sup\'erieure \`a 16,25\euro) \\
%\hline
%\end{tabular}
%\end{center}
%\end{table}

\begin{tab}
[15cm]                                                                           %  Argument 1 [8 cm] : largeur du tableau (mettre 15 cm comme taille maximale)
{
Calcul mensuel de l'allocation de solidarit\'e sp\'ecifique\label{tab:ASS} % Argument 2 : titre du tableau
}
{
\begin{tabular}{ll} \toprule                                                  % Argument 3 : contenu du tableau
\multicolumn{2}{c}{Personne seule} \\
\multicolumn{1}{c}{Revenus inf\'erieurs \`a 650,00\euro~}  & \multicolumn{1}{c}{Revenus entre 650,00 et 1 137,50\euro~}   \\
\midrule
Montant journalier (16,25\euro) * nombre de jour du mois & Diff\'erence entre le plafond (1 137,50\euro)\\
	&  et le montant des revenus (vers\'ee si sup\'erieure \`a 16,25\euro)   \\
\midrule
\midrule
\multicolumn{2}{c}{Personne vivant en couple} \\
\multicolumn{1}{c}{Revenus inf\'erieurs \`a 1 300,00\euro~}  & \multicolumn{1}{c}{Revenus entre 1 300,00 et 1 787,50\euro~}    \\
\midrule
Montant journalier (16,25\euro) * nombre de jour du mois & Diff\'erence entre le plafond (1 787,50\euro) \\
	&  et le montant des revenus (vers\'ee si sup\'erieure \`a 16,25\euro) \\
\\
\bottomrule
\end{tabular}
}
{}                                                                              % Argument 4 : notes du tableau (vide)
\end{tab}


\texttt{R\'ef\'erences l\'egislatives}{: Art. L.5423-6 du Code du Travail; D\'ecret n°2014-1719 du 30 d\'ecembre 2014 revalorisant l'allocation temporaire d'attente, l'allocation de solidarit\'e sp\'ecifique, l'allocation \'equivalent retraite et l'allocation transitoire de solidarit\'e}

 

\subsection{Mode et dur\'ee de versement}

 

La demande d'ASS est directement adress\'ee par P\^ole emploi au ch\^omeur lorsqu'il arrive en fin de droits, de m\^eme que les demandes de renouvellement, par p\'eriode de 6 mois. Toutefois, le paiement peut \^etre interrompu pour les raisons suivantes:
\begin{itemize}
\item Si le b\'en\'eficiaire cesse de remplir un des crit\`eres d'\'eligibilit\'e (ressources d\'epassant le plafond, absence de recherche d'emploi, etc.);
\item En cas de formation r\'emun\'er\'ee ou de reprise d'activit\'e non  cumulable avec l'ASS;
\item Si le b\'en\'eficiaire per\c coit des indemnit\'es journali\`ere pour maladie, maternit\'e ou accident du travail;
\item Par d\'ecision du pr\'efet ou suite \`a une radiation;
\item Si le b\'en\'eficiaire per\c coit l'allocation de pr\'esence parentale ou  le compl\'ement de libre choix d'activit\'e (CLCA) ou l'allocation journali\`ere d'accompagnement d'une personne en fin de vie;
\item Si le b\'en\'eficiaire remplit les conditions pour b\'en\'eficier d'une retraite \`a taux plein, ou  lorsqu'il atteint l'\^age l\'egal limite d'activit\'e.
\end{itemize}


\texttt{R\'ef\'erences l\'egislatives}{:Art. R.5423-8 du Code du Travail}

 

\subsection{Cumul de l'ASS avec une activit\'e r\'emun\'er\'ee}

 

L'ASS peut \^etre per\c cue en cas de reprise d'une activit\'e r\'emun\'er\'ee. Les modalit\'es du cumul d\'ependent du volume horaire de l'activit\'e, du montant des revenus et de la dur\'ee des droits restants \`a l'ASS:
\begin{itemize}
\item Activit\'e salari\'ee inf\'erieure \`a 78h par mois: l'ASS peut \^etre cumul\'ee avec le revenu pendant 12 mois maximum d\`es la reprise d'activit\'e.
	\begin{itemize}
	\item Revenus mensuels bruts inf\'erieurs \`a 728,76\euro~: du 1\up{er} au 6\up{\`eme} mois, l'ASS est vers\'ee en int\'egralit\'e. Du 7\up{\`eme} au 12\up{\`eme} mois, du montant total de l'ASS est d\'eduit le montant journalier multipli\'e par un nombre de jours \'equivalent \`a 40\% de la r\'emun\'eration brute divis\'ee par 15,90.
	\item Revenus mensuels bruts \'egaux ou sup\'erieurs \`a 728,76\euro~: du 1\up{er} au 6\up{\`eme} mois, du montant total de l'ASS est d\'eduit le montant journalier multipli\'e par un nombre de jours \'equivalent \`a 40\% de la partie de la r\'emun\'eration brute au-dessus de 728,76\euro~ divis\'ee par 16,25. Du 7\up{\`eme} au 12\up{\`eme} mois, du montant total de l'ASS est d\'eduit le montant journalier multipli\'e par un nombre de jours \'equivalent \`a 40\% de la r\'emun\'eration brute divis\'ee par 16,25.
	\end{itemize}
\item Activit\'e sup\'erieure \`a 78h par mois et activit\'e non salari\'ee: en cas d'activit\'e non salari\'ee, le travailleur peut toucher l'ASS en int\'egralit\'e pendant les 3 premiers mois d'activit\'e. Du 4\up{\`eme} au 12\up{\`eme} mois, du montant total de l'ASS est d\'eduit le montant du revenu mensuel, mais une prime forfaitaire mensuelle de 150\euro~ est vers\'ee. Si, au terme des 12 mois, le volume horaire total de l'activit\'e ne d\'epasse pas 750 heures, le versement de l'ASS se poursuit jusqu'\`a ce que cette limite soit atteinte.
\end{itemize}

Prenons l'exemple d'une personne dont le revenu brut mensuel est de 1 000\euro: au cours des 6 premiers mois, elle touchera l'ASS dont est d\'eduit l'\'equivalent de 7 jours non indemnis\'es, car:
\begin{equation}
\begin{split}
1 000 - 728,76~=~271,24  \\
0.4 * (271,24 / 16,25)~=~6.68
\end{split}
\end{equation}
Du 7\up{\`eme} au 12\up{\`eme} mois, elle touchera l'ASS dont est d\'eduit l'\'equivalent de 25 jours non indemnis\'es, car:
\begin{equation}
0.4 * (1 000 / 16,25)~=~ 24,6
\end{equation}

En cas de nouvelle perte d'emploi, le ch\^omeur peut b\'en\'eficier d'un reliquat de droits \`a l'ASS s'il a \'epuis\'e ses droits \`a l'ARE et si la demande d'attribution du reliquat n'intervient pas plus de 4 ans apr\`es la date d'admission \`a l'ASS ou de son dernier renouvellement.

\texttt{R\'ef\'erences l\'egislatives}{:Art. R5425-1 \`a R5425-8 du Code du Travail; D\'ecret n°98-1070 du 27 novembre 1998 relatif aux modalit\'es de cumul de certains minima sociaux avec des revenus d'activit\'es}



\section{\label{section:sec2}Revenu de solidarit\'e active}

 

Introduit le 1\up{er} juin 2009 en remplacement du revenu minimum d'insertion, le revenu de solidarit\'e active (RSA) regroupe le revenu minimum, l'allocation de parent isol\'e et les dispositifs d'incitation \`a la reprise d'emploi pour fournir un revenu de substitution aux individus n'ayant pas ou peu de ressources.

 

\subsection{D\'etermination des crit\`eres d'\'eligibilit\'e}

 

Contrairement \`a la principale allocation de solidarit\'e, l'ASS, le b\'en\'efice du RSA n'est pas d\'etermin\'e par l'historique d'affiliation, puisqu'il s'adresse justement \`a toutes les personnes priv\'ees de ressources et ne pouvant pr\'etendre \`a un revenu de remplacement au titre de l'assurance ch\^omage. En revanche, le RSA impose une condition d'\^age et prend en compte dans son calcul l'ensemble des ressources du foyer l\`a o\`u l'ASS se concentre sur les ressources de l'individu et de son conjoint.\\
Les conditions d'\'eligibilit\'e sont les suivantes:
\begin{itemize}
\item Avoir 25 ans ou plus, ou entre 18 et 24 ans si le demandeur justifie de deux ans d'activi\'te \`a temps plein au cours des 3 ann\'ees pr\'ec\'edant la demande. Le crit\`ere d'\^age ne s'applique pas non plus si le demandeur sollicit\'e le RSA au titre de parent isol\'e;
\item Etre de nationalit\'e fran\c caise ou avoir un titre de s\'ejour r\'egulier;
\item R\'esider en France;
\item Ne pas \^etre \'etudiant;
\item Ne pas \^etre en cong\'e sabbatique, parental, sans solde, en disponibilit\'e. Les deux derniers crit\`eres ne s'appliquent pas aux parents isol\'es.
\item Avoir des ressources inf\'erieures \`a un certain plafond.
\end{itemize}


 

\subsection{D\'etermination des ressources prises en compte}

 

Les ressources prises en compte comprennent:
\begin{itemize}
\item La moyenne des revenus d'activit\'e salari\'ee ou non salari\'ee (stage de formation r\'emun\'er\'e compris) sur les trois derniers mois, ce qui inclut salaire, primes, heures suppl\'ementaires, indemnit\'es de licenciement, indemnit\'e compensatrice de cong\'es pay\'es, indemnit\'e de pr\'eavis, etc.
\item Les prestations familiales et aide au logement (forfait) sur le dernier mois, et d'autres ressources (pensions, allocations ch\^omage)
\item Les indemnit\'es journali\`eres de S\'ecurit\'e sociale (maladie, accident du travail, maladie professionnelle, maternit\'e, paternit\'e, adoption) sur les trois derniers mois;
\item Allocations ch\^omage, indemnit\'es de ch\^omage partiel
\item Pensions, retraite, rente
\item Pensions alimentaires
\item Prestations compensatoires
\item Ressources exceptionnelles (vente d'une maison, gain aux jeux, etc.)
\item Capitaux plac\'es (livret A, livret d'\'epargne, etc.)
\item Certaines prestations familiales (allocations familiales, compl\'ement familial, compl\'ement de libre choix d'activit\'e, etc.)
\item Loyer d'un immeuble lou\'e
\item Valeur locative d'un logement, local, terrain non lou\'e
\item Prime mensuelle forfaitaire pour reprise d'activit\'e
\item Allocation aux adultes handicapp\'es
\item La rente d'orphelin
\end{itemize}

La p\'eriode de r\'ef\'erence pour la comptabilisation de ces diff\'erents \'el\'ements est de trois mois (1 mois pour les allocations familiales et le forfait logement).\\

\texttt{R\'ef\'erences l\'egislatives}{: articles L262-2 \`a L262-12 du Code de l'action sociale et des familles}


 

\subsection{Calcul}

 

Le montant forfaitaire du RSA varie selon la composition du foyer (seul ou en couple, avec ou sans enfants). Il correspond au montant mensuel vers\'e aux invidivus d\'epourvus de toute ressource, que la soci\'et\'e garantit \`a chacun.\\

Si l'individu a des ressources qui ne proviennent pas d'une activit\'e (certaines prestations notamment) et qui sont inf\'erieures au montant socle du RSA, il re\c coit la diff\'erence entre le montant forfaitaire et ses propres resources. Le montant du RSA ainsi calcul\'e est encore r\'eduit d'un montant \'egal au "forfait logement" quand le demandeur est propri\'etaire de son logement; log\'e \`a titre gratuit; ou qu'il touche des aides personnelles, une allocation de logement \`a caract\`ere familial, une allocation de logement \`a caract\`ere social, une aide personnalis\'ee au logement. Le montant de ce "forfait logement" d\'epend de la composition du foyer.\\
On obtient donc la formule suivante: \\

$ RSA = \text{montant forfaitaire} - (\text{ressources du foyer} + \text{forfait logement})$ \\

Si l'individu per\c coit un revenu d'activit\'e, il peut toujours b\'en\'eficier d'un compl\'ement de revenu sous la forme du "RSA activit\'e": le montant forfaitaire qui lui est garanti correspond \`a la somme de 62\% de ses ressources et du RSA socle sp\'ecifique \`a sa situation familiale. Le montant de RSA qui lui sera vers\'e par la Caisse d'Allocation Familiale sera donc \'egal \`a la diff\'erence entre ce montant forfaitaire garanti et ses ressources (si cette somme est sup\'erieure \`a 6\euro~).\\ 
On obtient donc la formule suivante:\\

$RSA = 0.62 \times \text{ressources du foyer} + \text{montant du RSA socle} - \text{ressources du foyer}$ \\

Ainsi, pour un individu touchant le RSA socle et reprenant une activit\'e, sur chaque euro gagn\'e, il en conservera 62 centimes en termes de revenu disponible final. Ce mode de calcul est plus avantageux que celui qui pr\'evalait dans l'ancien syst\`eme du RMI et dans lequel chaque euro gagn\'e au titre d'une activit\'e professionnelle \'etait contrebalanc\'e par un euro de moins au titre du RMI. Cet ancien syst\`eme ne donnait pas d'incitations \`a travailler et avait tendance \`a enfermer les individus dans une "trappe \`a inactivit\'e". L'introduction du RSA r\'epondait en partie \`a cette probl\'ematique.

\begin{tab}
[15cm]                                                                           % Argument 1 [8 cm] : largeur du tableau (mettre 15 cm comme taille maximale)
{
Montant forfaitaire du RSA au 1\up{er} janvier 2016\label{tab:test1} % Argument 2 : titre du tableau
}
{
\begin{tabular}{lcccccc} \toprule                                                  % Argument 3 : contenu du tableau
Nombre d'enfants & \multicolumn{2}{c}{Personne seule} & \multicolumn{2}{c}{Parent isol\'e: majoration pour isolement}  & \multicolumn{2}{c}{Couple} \\
& Territoire fran\c cais & Mayotte & Territoire fran\c cais & Mayotte & Territoire fran\c cais & Mayotte \\
\midrule
0                  & 524,16\euro~ & 268,08\euro~  & 673,08\euro~ & 399,06\euro~ & 786,24\euro &  399,06\euro~    \\
 1                      & 786,24\euro~  & 399,12\euro~   & 897,44\euro~  & 530,1\euro   & 943,49\euro~ & 530,1\euro~   \\
2                  & 943,49\euro~  & 477,74\euro~  & 1 121\euro~  & 608,72\euro~  & 1 100,74\euro~ & 608,72\euro~    \\
 Chaque enfant suppl\'ementaire   & 209,66\euro~ & 26,21\euro~   & 224,36\euro~ & 26,21\euro~ & 209,66\euro~ & 26,21\euro~  \\
 \\
\bottomrule
\end{tabular}
}
{}
\end{tab}



\begin{tab}
[15cm]                                                                           % Argument 1 [8 cm] : largeur du tableau (mettre 15 cm comme taille maximale)
{
Montant du "forfait logement" au 1\up{er} janvier 2016\label{tab:test1} % Argument 2 : titre du tableau
}
{
\begin{tabular}{lll} \toprule                                                  % Argument 3 : contenu du tableau
Nombre de personnes dans le foyer & Forfait logement territoire fran\c cais & Forfait logement Mayotte \\
\midrule
1 & 62,90\euro& 30,83\euro\\
2 & 125,80\euro& 61,67\euro \\
3 & 155,68\euro & 76,31\euro \\
\bottomrule
\end{tabular}
}
{}
\end{tab}

 

\subsection{Prime d'activit\'e}

 


Le 1\up{er} janvier 2016 a \'et\'e introduit un nouveau dispositif -- la prime d'activit\'e -- fusionant la revenu de solidarit\'e active et la prime pour l'emploi pour les travailleurs modestes (le RSA socle continuant \`a s'appliquer). Elle fournit un compl\'ement de revenu, vers\'e le 5 de chaque mois par la caisse d'allocation familiale, l\`a o\`u la prime pour l'emploi avait l'inconv\'enient d'\^etre vers\'ee au moment de la d\'eclaration fiscale annuelle, et pouvait dont \^etre deconnect\'ee des besoins au jour le jour de l'individu.

Les conditions d'\'eligibilit\'e sont les suivantes:
\begin{itemize}
\item Etre de nationalit\'e fran\c caise ou en situation r\'eguli\`ere en France;
\item R\'esider en France;
\item Avoir au moins 18 ans;
\item Etre en activit\'e salari\'ee ou non salari\'ee;
\item Ne pas \^etre en conf\'e parental, sabbatique, sans solde, en disponibilit\'e ou travailleur d\'etach\'e;
\item Gagner moins de 1500\euro~ net (environ 1,3 Smic)
\item Pour les \'etudiants et apprentis, il est n\'ecessaire de totaliser 3 mois de travail et de gagner minimum 890 euros net chaque mois (78\% du smic).
\end{itemize}

Les ressources prises en compte pour le calcul de la prime d'activit\'e regroupent les revenus d'activit\'e nets, les revenus de remplacement (indemnit\'es ch\^omage, maladie, retraite, pension, etc.), les prestations et aides sociales et les autres revenus imposables (revenus du capital, du patrimoine, etc.). La p\'eriode de r\'ef\'erence correspond aux trois derniers mois pour les salari\'es, \'etudiants et apprentis (la prime \'etant calcul\'ee sur les derniers revenus professionnels connus pour les travailleurs ind\'ependants). Le montant de la prime est calcul\'e trimestriellement: elle est vers\'ee tous les mois mais reste fixe par p\'eriodes de trois mois, m\^eme si la situation professionnelle ou familiale du b\'en\'eficiaire change pendant cette p\'eriode.\\

\underline{Calcul de la prime d'activit\'e}: le calcul s'effectue en plusieurs \'etapes:
\begin{itemize}
\item La premi\`ere \'etape consiste \`a d\'eterminer le montant forfaitaire correspondant \`a la composition du foyer, et l'\'eventuelle bonification individuelle accord\'ee \`a chaque membre du foyer dont les revenus d'activit\'e sont \'egaux ou sup\'erieurs \`a 0,5 smic mensuel. Cette bonification augmente jusqu'\`a atteindre son maximum (67 \euro) au niveau du 0,8 smic mensuel;
\item Un premier calcul est effectu\'e, comme suit: $A = (\text{montant forfaitaire} + \text{bonification} + 62\% \text{revenus d'activit\'e}) - (\text{ressources du foyer} + \text{prestations familiales} + \text{forfait logement})$
\item Un deuxi\`eme calcul est effectu\'e comme suit: $B= \text{montant forfaitaire}  - (\text{revenus d'activit\'e} + \text{autres ressources du foyer} + \text{prestations familiales} + \text{forfait logement})$
\item Si B est positif, alors la prime vers\'ee correspond \`a A - B; si B est n\'egatif ou nul, le montant vers\'e correspond \`a A
\end{itemize}

La prime est vers\'ee tant que les b\'en\'eficiaires continuent \`a remplir les conditions d'\'eligibilit\'e. Le montant forfaitaire et le "forfait logement" sont les m\^emes que pour le calcul du RSA.\\

\texttt{R\'ef\'erences l\'egislatives:}{D\'ecret \no 2015-1709 du 21 d\'ecembre 2015 relatif \`a la prime d'activit\'e}

\section{\label{section:sec2}Allocation temporaire d'attente}

 
L'allocation temporaire d'attente (ATA) est destin\'ee \`a certaines cat\'egories particuli\`eres de ch\^omeurs, qui n'ont pas accumul\'e suffisamment de jours d'activiti\'e pour pouvoir b\'en\'eficier de l'allocation d'assurance ch\^omage. Elle remplace l'Allocation d'insertion depuis le 16 novembre 2006.

 

\subsection{D\'etermination des r\`egles d'\'eligibilit\'e}

 


Les cat\'egories de demandeur d'emploi concern\'ees par l'attribution de l'ATA sont les suivantes:
\begin{itemize}
\item Salari\'es expatri\'es non affili\'es \`a l'assurance ch\^omage de retour en France, s'ils justifient de 182 jours d'activit\'e au cours des 12 derniers mois pr\'ec\'edant la fin du contrat de travail dans un pays \'etranger, ou dans les territoires suivants: Mayotte, Nouvelle Cal\'edonie, Polyn\'esie fran\c caise, Wallis et Futuna, Terres australes et antarctiques fran\c caises. Depuis le 31 mars 2011, si le demandeur d'emploi a exerc\'e son activit\'e \`a Mayotte, il peut b\'en\'eficier de l'allocation d'assurance ch\^omage. Pour pouvoir b\'en\'eficier de l'ATA, le demandeur d'emploi doit fournir une copie des 12 derniers bulletins de paie ou du certificat de travail;
\item D\'etenus lib\'er\'es apr\`es deux mois ou plus de d\'etention. Pour pouvoir b\'en\'eficier de l'ATA, ils doivent pr\'esenter \`a P\^ole emploi le certificat \'etabli par l'administration p\'enitentiaire;
\item Demandeurs d'asile de 18 ans ou plus ayant fait une demande d'asile politique aupr\`es de l'office fran\c cais pour la protection des r\'efugi\'es et apatrides (OFPRA). Ils sont tenus de fournir l'autorisation provisoire de s\'ejour, ainsi que le r\'ec\'episs\'e de d\'ep\^ot de demande de statut de r\'efugi\'e \`a l'OFPRA, pour se voir attibuer l'ATA;
\item B\'en\'eficiaires de la protection temporaire sur pr\'esentation de leur autorisation provisoire de s\'ejour;
\item B\'en\'eficiaires de la protection subsidiaire sur pr\'esentation de l'autorisation provisoire de s\'ejour et de la d\'ecision de l'OFPRA ou de la Cour nationale du droit d'asile;
\item Victimes \'etrang\`eres de la traite des \^etres humains ou du prox\'enitisme sur pr\'esentation de leur autorisation provisoire de s\'ejour et de l'attestation de la protection de l'Etat fran\c cais;
\item Apatrides reconnus comme tels par l'OFPRA sur pr\'esentation de la d\'ecision d'attribution du statut d'apatride.
\end{itemize}

Les personnes justifiant leur appartenance \`a une des cat\'egories sus-cit\'ees dans les 12 mois pr\'ec\'edant leur inscription en tant que demandeur d'emploi doivent remplir certaines conditions pour pouvoir b\'en\'eficier de l'ATA:
\begin{itemize}
\item Etre inscrit comme demandeur d'emploi
\item Ne pas avoir d\'ej\`a per\c cu l'ATA pour le m\^eme motif
\item Ne pas disposer de revenus exc\'edant un plafond \'equivalent au Revenu de solidarit\'e active (RSA). Les ressources prises en compte sont toutes les ressources personnelles d\'eclar\'ees \`a l'administration fiscale, ainsi que celles du conjoint, concubin ou de la personne avec laquelle un PACS a \'et\'e conclu, sur les 12 mois civils pr\'ec\'edant la demande. Les prestations familiales et l'allocation logement ne sont pas int\'egr\'ees.
\end{itemize}

\texttt{R\'ef\'erences l\'egislatives:} {Art. L.5423-8,  R.5423-18, R.5423-20, R.5423-21, R.5423-24,  R.5423-26 du Code du Travail}

 

\subsection{Calcul}

 

Le montant journalier de l'ATA est forfaitaire, et r\'evis\'e chaque ann\'ee par d\'ecret: en 2015, il est \'egal \`a 11,45\euro. Le montant mensuel est \'equivalent au montant journalier multipli\'e par le nombre de jours du mois.\\

\texttt{R\'ef\'erences l\'egislatives:}{art. R.5423-12 du Code du Travail}

 

\subsection{Dur\'ee et mode de versement}

 

La dur\'ee de versement d\'epend \'egalement de la situation ayant donn\'e lieu au versement de l'ATA:
\begin{itemize}
\item Pour les apatrides, anciens d\'etenus et salari\'es expatri\'es, la dur\'ee maximale de versement est \'egale \`a 12 mois, sous r\'eserve d'un examen interm\'ediaire au 6\up{\`eme} mois;
\item Pour les demandeurs d'asile ayant fait une demande aupr\`es de l'OFPRA, l'indemnisation mensuelle continue tant que la proc\'edure d'instruction du dossier n'a pas abouti. Une fois la d\'ecision rendue, l'indemisation cesse \`a la fin du moins suivant celui de l'obtention ou du refus du statut de r\'efugi\'e. Si l'OFPRA d\'ecide d'accorder le statut d'apatride ou la protection subsidiaire au demandeur d'asile, il peut faire une nouvelle demande d'ATA pour 12 mois au titre de sa nouvelle situation. Si le demandeur d'asile obtient ou refuse un h\'ebergement en centre d'accueil des demandeurs d'asile (CADA), le versement de l'allocation est interrompu \`a la fin du moins suivant;
\item Pour les b\'en\'eficiaires de la protection subsidiaire et les victimes de la traite des \^etres humains ou du prox\'enitisme, l'ATA est vers\'ee tant que les conditions d'attribution sont respect\'ees. Le r\'eexamen de la situation se fait tous les 6 mois.
\item Pour les b\'en\'eficiaires de la protection temporaire, l'indemnisation mensuelle continue tant que les b\'en\'eficiaires sont pris en charge, sous r\'eserve d'un examen de leur situation tous les 6 mois. La dur\'ee de la prise en charge d\'epend d'une d\'ecision du Conseil de l'Union europ\'eenne et de son application sous la forme d'instructions minist\'erielles.
\end{itemize}

L'indemnisation peut cesser si la personne:
\begin{itemize}
\item Dispose de ressources exc\'edant le plafond;
\item Est malade;
\item Est interdit du b\'en\'efice des allocations par d\'ecision du pr\'efet ou par radiation par P\^ole emploi;
\item A atteint l'\^age de 60 ans et justifie du nombre de trimestres requis pour b\'en\'eficier d'une retraite \`a taux plein. Dans le cas contraire, l'indemnisation est prolong\'ee jusqu'\`a ce que cette condition soit remplie ou jusqu'\`a 65 ans;
\item Per\c coit l'allocation de pr\'esence parentale ou le compl\'ement de libre choix d'activit\'e \`a taux plein pour l'accueil du jeune enfant.
\end{itemize}


\texttt{R\'ef\'erences l\'egislatives:} {Art. R.5423-21 ; art. R.5423-9 ; art. R.5423-19 du Code du Travail}

 

\subsection{Cumul de l'ATA avec des revenus d'activit\'e professionnelle}

 


L'ATA peut \^etre per\c cue en cas de reprise d'une activit\'e r\'emun\'er\'ee pendant 12 mois \`a compter de la reprise d'activit\'e, ou jusqu'\`a l'\'epuisement des droits \`a l'allocation si cette dur\'ee est inf\'erieure \`a 12 mois. Les modalit\'es du cumul sont similaires \`a celles en vigueur dans le cas de l'ASS:
\begin{itemize}
\item Revenus mensuels bruts inf\'erieurs \`a 728,76\euro~: 
	\begin{itemize}
	\item Du 1\up{er} au 6\up{\`eme} mois, l'ATA est vers\'ee en int\'egralit\'e. 
	\item Du 7\up{\`eme} au 12\up{\`eme} mois, du montant total de l'ATA est d\'eduit le montant journalier multipli\'e par un nombre de jours \'equivalent \`a 40\% de la r\'emun\'eration brute divis\'ee par 11,45.
	\end{itemize}
\item Revenus mensuels bruts \'egaux ou sup\'erieurs \`a 728,76\euro~: 
	\begin{itemize}
	\item Du 1\up{er} au 6\up{\`eme} mois, du montant total de l'ATA est d\'eduit le montant journalier multipli\'e par un nombre de jours \'equivalent \`a 40\% de la partie de la r\'emun\'eration brute au-dessus de 728,76\euro~ divis\'ee par 11,45.
	\item Du 7\up{\`eme} au 12\up{\`eme} mois, du montant total de l'ATA est d\'eduit le montant journalier multipli\'e par un nombre de jours \'equivalent \`a 40\% de la r\'emun\'eration brute divis\'ee par 11,45.
	\end{itemize}
\end{itemize}

Si, au terme des 12 mois de cumul, le volume horaire total de l'activit\'e professionnelle est inf\'erieur \`a 750 heures, le cumul de l'ATA et des revenus professionnels peut se poursuivre jusqu'\`a ce que ce plafond de 750 heures soit atteint.\\

En cas de nouvelle perte d'emploi, le ch\^omeur peut b\'en\'eficier d'un reliquat de droits \`a l'ATA s'il a \'epuis\'e ses droits \`a l'ARE et si la demande d'attribution du reliquat n'intervient pas plus de 4 ans apr\`es la date d'admission \`a l'ATA.\\

\texttt{R\'ef\'erences l\'egislatives}{: Art. R.5425-1 et suite du Code du Travail; D\'ecret \no 2011-1421 du 02.11.11}

\section{\label{section:sec3}Allocation \'equivalent retraite}

 


L'allocation \'equivalent retraite (AER) a \'et\'e \'elabor\'ee comme un dispositif de pr\'e-retraite, destin\'e aux personnes ayant totalis\'e un nombre de trimestres suffisant pour b\'en\'eficier de la retraite \`a taux plein mais qui n'ont pas atteint l'\^age minimum de d\'epart \`a la retraite, pour leur assurer un revenu  jusqu'\`a ce qu'ils puissent toucher leur pension de retraite. Cette allocation a \'et\'e supprim\'ee le 1\up{er} janvier 2011, et remplac\'ee par l'allocation transitoire de solidarit\'e. Cependant, les personnes ayant commenc\'e \`a en b\'en\'eficier avant le 1\up{er} janvier 2011 continuent \`a la percevoir jusqu'\`a l'expiration de leurs droits.

 

\subsection{D\'etermination des r\`egles d'\'eligibilit\'e}

 

L'AER peut \^etre de deux types: de remplacement, pour les demandeurs d'emploi ayant \'epuis\'e leurs droits \`a l'allocation d'assurance et \`a la place de l'ASS ou du RSA; de compl\'ement, pour les demandeurs d'emploi dont l'ARE est inf\'erieure au montant de l'AER.\\

Pour en b\'en\'eficier, les personnes doivent \^etre inscrites comme demandeurs d'emploi et avoir totalis\'e le nombre de trimestres requis pour toucher une retraite \`a taux plein avant le 1\up{er} janvier 2011. S'ils en ont fait la demande avant cette date, ils continuent \`a en b\'en\'eficier jusqu'\`a expiration de leurs droits.\\
Ils doivent \'egalement disposer de revenus inf\'erieurs \`a un certain plafond, \'equivalent, pour l'AER remplacement, \`a 1 684,33\euro~ mensuels pour une personne seule, et 2 421,21\euro~ pour une personne vivant en couple (avec son \'epoux(se), partenaire pacs\'e(e), ou concubin(e)). Concernant l'AER compl\'ement, ces plafonds sont respectivement \'egaux \`a 1 052,70\euro~ et 2 421,21\euro~ (dont moins de 1052,70\euro~ \`a titre personnel). Les ressources prises en compte pour ce calcul sont les ressources soumises \`a l'imp\^ot sur le revenu et per\c cues sur les 12 mois pr\'ec\'edant la demande. Ce montant est ensuite divis\'e par 12 pour obtenir le montant mensuel. Ne sont cependant pas pris en consid\'eration les allocation d'assurance ch\^omage ou de solidarit\'e, les prestations familiales, l'allocation logement, les revenus d'activit\'e (ou stage) per\c cus au cours des 12 derniers mois pr\'ec\'edant la demande si leur versement est interrompu au moment de la demande d'AER et s'ils n'ont pas donn\'e lieu au versement d'un revenu de substitution. Dans le cas contraire, un abattement de 30\% est appliqu\'e sur la moyenne des ressources auquel ce revenu se substitue.\newline

Les b\'en\'eficiaires de l'AER peuvent, s'ils le souhaitent, se voir accorder une dispense de recherche d'emploi par P\^ole emploi\footnote{Cette dispense de recherche d'emploi concerne \'egalement, \`a leur demande, les personnes inscrites comme demandeurs d'emploi b\'en\'eficiaires de l'ARE et \^ag\'ees d'au moins 59 ans en 2010 et d'au moins 60 ans en 2011, ou bien les b\'en\'eficiaires de l'ASS \^ag\'ees d'au moins 58 ans en 2010 et d'au moins 60 ans en 2011. Certains demandeurs d'emploi ne peuvent b\'en\'eficier de cette dispense (car ils ne sont pas indemnis\'es par exemple), mais ils peuvent se soustraire aux obligations pr\'evues par l'article L. 5411-6 du Code du travail telles que l'actualisation du projet personnalis\'e d'acc\`es \`a l'emploi s'ils sont \^ag\'es d'au moins 58 ans en 2010 et d'au moins 60 ans en 2011. Les conditions de dispenses \'enonc\'ees ci-dessus sont r\'egies par la loi du 1\up{er} ao\^ut 2008 "relative aux droits et aux devoirs des demandeurs d'emploi", mais ont \'et\'e abrog\'ees \`a partir du 1\up{er} janver 2012. N\'eanmoins, les personnes b\'en\'eficiant d'une telle dispense avant cette date continuent \`a en b\'en\'eficier.}.\\

 
\texttt{R\'ef\'erences l\'egislatives}{: Loi \no 2010-1330 du 9 novembre 2010 portant r\'eforme des retraites (Article 106); D\'ecret \no 2010-458 du 6 mai 2010 instituant \`a titre exceptionnel une allocation \'equivalent retraite pour certains demandeurs d'emploi}



\subsection{Calcul}

 

Le montant journalier de l'AER est \'egal \`a 35,09\euro~ par jour, et doit \^etre multipli\'e par le nombre de jour du mois pour obtenir le montant mensuel. Cependant, l'AER n'est pas touch\'ee dans son int\'egralit\'e par tous les b\'en\'eficiaires, le calcul variant en fonction du type d'AER, du niveau de ressources et de la situation conjugale:

%\begin{table}[h!]\label{tab:4.2}
%\begin{center}
%\centering 
%\caption{Calcul mensuel de l'allocation \'equivalent retraite.}
%\vspace{0.2cm}
%\small
%\begin{tabular}{|>{\centering\arraybackslash}m{2.5cm}|| >{\centering\arraybackslash}m{2cm} >{\centering\arraybackslash}m{2.5cm} | >{\centering\arraybackslash}m{2cm} >{\centering\arraybackslash}m{2.5cm}|}
%\hline
%\multicolumn{4}{c}{AER de remplacement} \\
%\hline
%\multicolumn{2}{c}{Personne seule} & \multicolumn{2}{p}{Personne vivant en couple }\\
%\hline
%Revenus inf\'erieurs \`a 631,62\euro~ & Revenus entre 631,62 et 1 684,33\euro~ & Revenus inf\'erieurs \`a 1 368,51\euro~ & Revenus entre 1 368,51 et 2 421,21\euro~  \\
%\hline
%Montant journalier (35,09\euro) * nombre de jour du mois & Diff\'erence entre le plafond (1 684,33\euro) et le montant des revenus & Montant journalier (35,09\euro) * nombre de jour du mois & L'allocation d\'epend des revenus du conjoint: \begin{itemize}
%\item Pas de revenus: l'AER = diff\'erence entre le plafond (2 421,21\euro) et le montant des ressources 
%\item Revenu sup\'erieur \`a 1 368,51\euro: l'AER = diff\'erence entre 1 052,70\euro (30*montant journalier de l'AER) et le montant des ressources (revenu du conjoint non inclus) 
%\item Revenu inf\'erieur \`a 1 368,51\euro: l'AER = diff\'erence entre 2 421,21\euro et le montant des ressources (revenu du conjoint inclus)
%\end{itemize} \\
%\hline
%\multicolumn{4}{c}{AER de compl\'ement}\\
%\hline
%\multicolumn{2}{c}{Personne seule} & \multicolumn{2}{c}{Personne vivant en couple} \\
%\hline
%\multicolumn{2}{c}{Revenus inf\'erieurs \`a 1 052,70\euro~}  & \multicolumn{2}{c}{Revenus inf\'erieurs \`a 2 421,21\euro~ dont moins de 1 052,70\euro~ \`a titre personnel}  \\
%\hline
%\multicolumn{2}{c}{Allocation vers\'ee en compl\'ement des ressources \`a hauteur de 1052,70\euro~} & \multicolumn{2}{c}{Allocation vers\'ee en compl\'ement des ressources \`a hauteur de 1052,70\euro~ (sans prise en compte des ressources du conjoint)} \\
%\hline
%\end{tabular}
%\end{center}
%\end{table}

\begin{tab}
[15cm]                                                                           %  Argument 1 [8 cm] : largeur du tableau (mettre 15 cm comme taille maximale)
{
Calcul mensuel de l'allocation \'equivalent retraite\label{tab:AER} % Argument 2 : titre du tableau
}
{
\begin{tabular}{llll} \toprule                                                  % Argument 3 : contenu du tableau
\multicolumn{4}{c}{AER de remplacement}\\
\multicolumn{2}{c}{Personne seule} & \multicolumn{2}{c}{Personne vivant en couple }  \\
Revenus < 631,62\euro~ &  631,62 < Revenus < 1 684,33\euro~ & Revenus < 1 368,51\euro~ & 1 368,51 < Revenus < 2 421,21\euro~   \\
\midrule
Montant journalier (35,09\euro)$\times$ & Plafond (1 684,33\euro) -  & Montant journalier (35,09\euro) $\times$ & L'allocation d\'epend des revenus du conjoint: \\
nombre de jour du mois &  montant des revenus &  nombre de jour du mois &  - Pas de revenus: AER = plafond \\
	&	&	& (2 421,21\euro) - montant des ressources \\
	&	&	& -  Revenu > 1 368,51\euro: \\
	&	&	& AER = 1 052,70\euro~ (30*montant journalier AER) - \\
	&	&	&  montant des ressources (revenu du conjoint non inclus) \\
	&	&	& - Revenu < 1 368,51\euro: \\
	&	&	& AER = 2 421,21\euro~ - \\
	&	&	& montant des ressources (revenu du conjoint inclus) \\
\midrule
\midrule
\multicolumn{4}{c}{AER de compl\'ement} \\
\multicolumn{2}{c}{Personne seule} & \multicolumn{2}{c}{Personne vivant en couple}  \\
\multicolumn{2}{c}{Revenus < 1 052,70\euro~}  & \multicolumn{2}{c}{Revenus < 2 421,21\euro~ dont moins de 1 052,70\euro~ \`a titre personnel} \\
\midrule
\multicolumn{2}{l}{Allocation vers\'ee en compl\'ement}  & \multicolumn{2}{l}{Allocation vers\'ee en compl\'ement des ressources} \\
\multicolumn{2}{l}{des ressources \`a hauteur de 1052,70\euro~} &  \multicolumn{2}{l}{\`a hauteur de 1052,70\euro~ (sans prise en compte des ressources du conjoint)} \\
\\
\bottomrule
\end{tabular}
}
{}                                                                             % Argument 4 : notes du tableau (vide)
\end{tab}

\texttt{R\'ef\'erences l\'egislatives:}{D\'ecret \no 2014-1719 du 30 d\'ecembre 2014 revalorisant l'allocation temporaire d'attente, l'allocation de solidarit\'e sp\'ecifique, l'allocation \'equivalent retraite et l'allocation transitoire de solidarit\'e}
 

\subsection{Mode et dur\'ee de versement}

 

L'AER est vers\'ee chaque mois par P\^ole emploi, par p\'eriode de 12 mois renouvelables, sous r\'eserve que le b\'en\'eficiaire continue de remplir les crit\`eres d'attribution. Pour les b\'en\'eficiaires de l'AER de compl\'ement arrivant au terme de leurs droits \`a l'assurance ch\^omage, ils se voient attribuer l'AER calcul\'ee selon les modalit\'es de l'AER remplacement.



\subsection{Cumul de l'AER et des revenus d'activit\'e}

 

L'AER peut \^etre cumul\'ee avec des revenus issus d'une activit\'e r\'eduite ou occasionnelle, selon les conditions suivantes: l'allocation est r\'eduite d'un montant \'equivalent au montant journalier multipli\'e par un nombre de jours non indemnis\'es \'egal \`a 0.60 * (r\'emun\'eration brute/ 35.09\euro~). \\

\texttt{R\'ef\'erences l\'egislatives:} {Loi \no 2010-1330 du 9 novembre 2010 portant r\'eforme des retraites : Article 106; D\'ecret \no 2014-1719 du 30 d\'ecembre 2014; D\'ecret \no 2010-458 du 6 mai 2010 instituant \`a titre exceptionnel une allocation \'equivalent retraite pour certains demandeurs d'emploi}



\section{\label{section:sec4}Allocation transitoire de solidarit\'e}

 

L'allocation transitoire de solidarit\'e (ATS) est dirig\'ee aux personnes ayant \'epuis\'e leurs droits aux allocations d'assurance, et n'\'etant pas encore en \^age de percevoir leur retraite. Elle apporte donc un revenu de substitution transitoire entre une situation de ch\^omage et une situation de retraire. Elle a \'et\'e cr\'e\'ee pour remplacer l'allocation \'equivalent retraite \`a partir du 1\up{er} juillet 2011, comme un m\'ecanisme temporaire destin\'e \`a accompagner les ch\^omeurs remplissant certaines conditions de ressources jusqu'\`a la retraite, et elle a pris fin au 31 d\'ecembre 2015 (puisque les personnes n\'ees en 1953 ont toutes atteint l'\^age l\'egal de d\'epart \`a la retraite).

 

\subsection{D\'etermination des r\`egles d'\'eligibilit\'e}

 

Les conditions d'\'eligibilit\'e \`a l'ATS varient selon qu'elle joue le r\^ole de compl\`ement de revenu (ATS compl\'ement) ou qu'elle vienne en substitution des allocations d'assurance (ATS remplacement):
Les crit\`eres \`a respecter pour se voir attribuer l'ATS remplacement sont les suivants:
\begin{itemize}
\item Percevoir une allocation d'assurance ch\^omage ou bien remplir les conditions d'\'eligibilit\'e au 31/12/2010;
\item Avoir au moins 60 ans et justifier du nombre de trimestres requis pour toucher une retraite \`a taux plein \`a l'expiration des droits \`a l'allocation d'assurance;
\item Ne pas avoir atteint l'\^age l\'egal de d\'epart \`a la retraite;
\item Avoir des ressources inf\'erieures \`a un certain plafond \'equivalent \`a 1 684,32\euro~ pour une personne seule (c'est-\`a-dire 48 fois le montant journalier de l'ATS), et 2 421,21\euro~ pour une personne vivant en couple (69 fois le montant journalier de l'ATS).
\end{itemize}

Les ressources prises en compte pour d\'eterminer l'\'eligibilit\'e \`a l'ATS correspondent \`a toutes les ressources per\c cues dans les 12 mois pr\'ec\'edant la demande. Elles incluent les ressources mobili\`eres et immobili\`eres du demandeur lui m\^eme ou bien de son conjoint, concubin ou partenaire PACS, telles que d\'eclar\'ees \`a l'administration fiscale, avant abattement, mais elles excluent les prestations familiales, l'allocation logement, les allocations d'assurance ch\^omage et de solidarit\'e, les r\'emun\'erations de stage et les revenus d'activit\'e interrompus au moment de la demande et qui n'ont pas d\'ej\`a servi au calcul d'un revenu de substitution.

Les crit\`eres \`a respecter pour se voir attribuer l'ATS compl\'ement sont presque similaires:
\begin{itemize}
\item Percevoir l'allocation d'assurance ch\^omage au 10/11/2010 et avoir des droits expirant apr\`es 60 ans;
\item Avoir au moins 60 ans et justifier du nombre de trimestres requis pour toucher une retraite \`a taux plein;
\item Ne pas avoir atteint l'\^age l\'egal de d\'epart \`a la retraite;
\item Avoir des ressources inf\'erieures \`a un certain plafond \'equivalent \`a 1 684,32\euro~ pour une personne seule, et 2 421,21\euro~ pour une personne vivant en couple.
\end{itemize}

 

\subsection{Calcul et mode de versement}

 

Le montant journalier de l'ATS est forfaitaire et r\'evis\'e chaque ann\'ee par d\'ecret. En 2015, il est \'egal \`a 35,09\euro. Le montant mensuel est calcul\'e en multipliant le montant journalier par le nombre de jours du mois.\\
L'ATS est vers\'ee jusqu'\`a ce que le b\'en\'eficiaire atteigne l'\^age l\'egal de d\'epart \`a la retraite.\\

En cas de reprise d'activit\'e, le travailleur peut continuer \`a b\'en\'eficier de l'ATS s'il remplit les conditions d'\'eligibilit\'e. Un certain nombre de jours non indemnisables seront d\'eduits de son allocation mensuelle, calcul\'es comme suit: \\

Nombre de jours non indemnisables $= 0,60 * \frac{\text{r\'emun\'eration brute per\c cue}}{\text{montant journalier de l'ATS}}$ \\

\texttt{R\'ef\'erences l\'egislatives}{:D\'ecret \no 2013-187 du 4 mars 2013 instituant \`a titre exceptionnel une allocation transitoire de solidarit\'e pour certains demandeurs d'emploi; Instruction P\^ole emploi \no 2013-45 du 22 avril 2013 relative \`a l'allocation transitoire de solidarit\'e}

\section{\label{section:sec5}Prime transitoire de solidarit\'e}

 


La prime transitoire de solidarit\'e (PTS), entr\'ee en vigueur le 17 juillet 2015, prend la suite de l'allocation transitoire de solidarit\'e. Elle a \'et\'e cr\'e\'ee \`a destination des demandeurs d'emploi \^ag\'ees de 60 ans et plus jusqu'\`a ce qu'ils liquident leur retraite, \`a condition qu'ils remplissent certains crit\`eres.

 

\subsection{D\'etermination des r\`egles d'\'eligibilit\'e}

 

Ces crit\`eres sont les suivants:
\begin{itemize}
\item \^Etre n\'e en 1954 ou 1955;
\item Avoir 60 ans ou plus mais ne pas avoir atteint l'\^age l\'egal de d\'epart \`a la retraite;
\item Avoir totalis\'e un nombre de trimestres suffisant pour b\'en\'eficier d'une retraite \`a taux plein;
\item Percevoir l'ASS ou le RSA
\item Avoir per\c cu, entre le 1\up{er} janvier 2011 et le 31 d\'ecembre 2014, l'allocation de retour \`a l'emploi, l'allocation sp\'ecifique de reclassement ou l'allocation de s\'ecurisation professionnelle, et avoir \'epuis\'e ses droits.
\end{itemize}

 

\subsection{Calcul}

 

La PTS est une allocation forfaitaire qui s'\'el\`eve \`a 300\euro. Elle est vers\'ee mensuellement par P\^ole emploi

 

\subsection{Mode et dur\'ee de versement}

 

La demande doit \^etre adress\'ee \`a P\^ole emploi au plus tard le 31 d\'ecembre 2017. Elle est vers\'ee aux demandeurs d'emploi \'eligibles jusqu'\`a ce qu'ils liquident leur retraite ou, au plus tard, jusqu'\`a ce qu'ils atteignent l'\^age l\'egal de la retraite.\\


\texttt{R\'ef\'erences l\'egislatives}{:D\'ecret \no 2015-860 du 15 juillet 2015 instituant une prime transitoire de solidarit\'e pour certains demandeurs d'emploi}

\section{\label{section:sec6} Pr\'eretraite licenciement}

 

Ce dispositif de pr\'eretraite (convention AS-FNE entre l'employeur et l'Etat), financ\'e par le fonds national de l'emploi (FNE), n'a plus cours depuis le 10 octobre 2011. Cependant, les conventions conclues avant cette date continuent \`a \^etre en vigueur.\\

La pr\'eretraire licenciement assure aux salari\'es licenci\'es pour motif \'economique et \^ag\'es de 57 ans et plus une allocation (allocation sp\'eciale du FNE) et une protection sociale, jusqu'\`a la retraite, \`a condition que le salari\'e justifie de 10 ann\'ees d'affiliation dont une ann\'ee continue dans la derni\`ere entreprise.\\

Les allocations pr\'eretraite FNE sont cumulables, sous certaines conditions, avec la r\'emun\'eration issue d'une activit\'e salari\'ee. L'activit\'e ne doit pas exc\'eder 16 heures par mois, et la r\'emun\'eration correspondante doit \^etre inf\'erieure \`a 16/169\up{\`eme} de l'ancien salaire journalier brut. Ces conditions peuvent \^etre lev\'ees pour certaines activit\'es (mission d'expertise, de recherche, activit\'e agricole non salari\'ee, perception de droits d'auteur, etc.). En revanche, le cumul est impossible en cas de r\'eembauche chez le m\^eme employeur. \\
L'allocation mensuelle est diminu\'ee d'un nombre de jours non indemnisables (NJNI) \'egal \`a:\\

$NJNI = \frac{\text{salaires nets procur\'es par l'activit\'e}}{\text{montant de l'allocation jounrali\`ere nette}}$. \\


\texttt{R\'ef\'erences l\'egislatives:}{Art. R. 5123-12 et s. du Code du Travail; circulaire CDE 75-85 du 10/12/1985}


\section{\label{section:sec7}R\'emun\'eration de fin de formation (RFF)}

 

Cette allocation, qui remplace depuis le 1\up{er} janvier 2011 l'AFDEF, est distribu\'ee par P\^ole emploi aux ch\^omeurs ayant \'epuis\'e leurs droits \`a l'ARE sans avoir termin\'e leur formation, pour leur fournir un revenu jusqu'\`a la fin de celle-ci. Les b\'en\'eficiaires du CSP peuvent \'egalement y pr\'etendre.\\
Le montant journalier de la RFF est \'egal au dernier montant journalier de l'allocation percue par le demandeur d'emploi dans la limite de 652,02\euro~ par mois.\\

Les formations pouvant ouvrir l'acc\`es \`a la RFF sont celles qui permettent d'aqu\'erir une qualification reconnue et d'acc\'eder \`a un emploi pour lequel des difficult\'es de recrutement ont pr\'ealablement \'et\'e identifi\'ees. Si le ch\^omeur interrompt sa formation plus de 15 jours, le versement de l'allocation est suspendu. Cependant, le ch\^omeur peut continuer \`a toucher la RFF en cas de reprise d'une activit\'e compatible avec la poursuite de la formation.


%% checker autres alloc dans pr\'ecis 2015 
\ifx\isEmbedded\undefined
\newpage
\bibliography{../../Biblio/biblio-rapport}
\end{document}
\else \fi
