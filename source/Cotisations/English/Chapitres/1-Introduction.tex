%%%%%%%%%%%% Introduction %%%%%%%%%%%%%%%

The Social security scheme being mainly funded by contributions based on earnings, these contributions represent a substantial economic weight, while being at the core of public debate. The goal of this \emph{IPP legislative guide} is to provide an overview of social security contributions and other taxes based on earnings in France. Besides a detailed list of the different contributions and payroll taxes and precise information on their amount computation, the reader will find the present document elements of guidance on the Social security organizaton and its funding.

The French law regarding social security contribution being particularly complex, this guide does not seek comprehensiveness. It aims at making available information sufficiently detailed to allow economists and a non-expert audience interested in this topic to understand the way contributions and payroll taxes are computed in France. At the end of this guide, the reader should be able to better grasp the terms of the recurrent debates on the funding of the social protection scheme and on the labor cost in France. This document has been elaborated as a tool to be used in complement with the \emph{IPP tables - Social security contributions and payroll taxes} (\citet{ipp_baremes_prelevements}), compiling the values of all the legislative parameters necessary for the computation of the social security contributions and payroll taxes since 1930 (available on the IPP website: \texttt{http://www.ipp.eu/fr/outils/baremes-ipp/}). These tables and the information contained into this guide \emph{IPP legislative guide} have been used for the programming of the IPP static microsimulation model \TAXIPP, which allows to simulate the whole French social and fiscal legislation in general, and the social security contributions and payroll taxes in particular. This guide can also be considered a useful complement to the interactive payslip simulator developed by the IPP (free access on the IPP website:

 \texttt{http://www.ipp.eu/fr/outils/comprendre-son-bulletin-de-salaire/}).

Although legislative references have been inserted throughout the guide to allow the reader to have access to more detailed information by directly drawing on the raw sources, the objective of this guide is not to substitute to a law documentation, as it can be found in specialized social or tax law textbooks. Moreover, if we will endeavor to present the main evolutions having affected the social security contributions (change in the tax base and rates), and the political debates that came along with the implementation of some important measures (such as the policies of reductions in employers' social security contributions), this guide does not constitute a historical or socio-political analysis of social security contributions in France. Any reader interested in such approaches should refer to specialized publications on the Social security history.

The current draft essentially deals with contributions paid on the private and public-sectors earnings, and on self-employed workers' profit. It does not tackle the issue of special schemes. Besides, schemes specific to some occupations (salespersons, employees of hotels and restaurants, etc.) are not presented here, due to their complexity. Moreover, the reader should keep in mind that only contributions paid on \emph{earnings} are described in this guide: therefore, contributions and taxes paid on capital income and designed to finance the Social security scheme will not be looked over.

The guide starts with a general description of the Social security scheme in France: guidance elements on the Social security architecture, its revenue and spending are provided in chapter 1. We will then present the rules applying for the determination of the contribution and payroll tax bases and for the consideration of the different elements of earnings (chapter 2), before examining in further detail each of the existing contribution (chapter 3). Finally, chapter 4 will be entirely dedicated to the different measures of reduction in social security contributions which succeeded one another since the 1990's.

Sources that have been used to write this \emph{IPP legislative guide} are of several types. First of all, we used the laws and decrees published in the \emph{Official Journal} and the documents issued by competent administrations. The collections of the \emph{M\'emento Social Francis Lefebvre} (\citet{lefebvre2013social}), of the \emph{M\'emento Fiscal Francis Lefebvre} (\citet{lefebvre2013fiscal}) and the \emph{M\'emento Paie Francis Lefebvre} (\citet{lefebvre2013paie}) have also been valued sources of information. We also resorted to the handbook on social security law of J-J. Dupeyroux (\citet{Dupeyroux2011}) and to the book \emph{La S\'ecurit\'e sociale. Son histoire \`a travers les textes} (\citet{SecuriteSociale1988}). More specific references will be cited in each sub-section. \\
