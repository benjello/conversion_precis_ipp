\ifx\isEmbedded\undefined
\input{../../Style/ipp-separate}
\setcounter{chapter}{0}
\chapter{\label{Chapitre4}Les aides \`a la reprise d'activit\'e}
\else \fi


%%%%%%%%%%%%%%%%%%%%%%%%%%%%%%%%%%%%%%%%%%%%%%%%%%%%%%%%%%%%%%%%%%%%%%%%%%%%%%%%%%%%%%%%%%



\section{\label{section:sec1}Cumul de l'allocation d'assurance et des revenus d'activit\'e}




Le cumul partiel de l'ARE avec les revenus issus d'une activit\'e professionnelle a \'et\'e mis en place avec l'objectif d'inciter les ch\^omeurs \`a reprendre une activit\'e, m\^eme peu r\'emun\'eratrice, en faisant en sorte qu'il soit toujours gagnant en cas de retour sur le marche du travail. Ainsi, le total de son revenu en cas de reprise d'activit\'e est sup\'erieur \`a ce qu'il aurait touch\'e sans activit\'e professionnelle. Ce cumul peut \'egalement survenir dans le cas o\`u un salari\'e occupe plusieurs emplois, qu'il en perd un ou plusieurs et en conserve un ou plusieurs.



\subsection{D\'etermination des r\`egles d'\'eligibilit\'e}




Le demandeur d'emploi doit remplir les crit\`eres d'attribution de l'ARE, continuer de rechercher activement un emploi, et donc \^etre inscrit en tant que demandeur d'emploi. \\
Avant la signature de la Convention d'assurance ch\^omage 2014, entr\'ee en vigueur le 1\up{er} octobre 2014, le cumul n'\'etait possible que si le volume horaire de l'activit\'e reprise n'exc\'edait pas 110 heures par mois, et si le salaire issu de cette activit\'e ne d\'epassait pas 70\% de l'ancien salaire ayant servi au calcul de l'ARE. Apr\`es le 1\up{er} octobre 2014, la condition sur le volume horaire de l'activit\'e a \'et\'e lev\'ee.\\

Ainsi, au 30 juin 2014, on d\'enombrait 600 000 b\'en\'eficiaires du cumul ARE-r\'emun\'eration\cite{UNEDIC2015}, soit 54\% des allocataires qui exercent une activit\'e professionnelle.\\


\texttt{R\'ef\'erences l\'egislatives}{: Art. 30 \`a 34 du r\'eglement de l'assurance ch\^omage}



\subsection{Calcul}



On proc\`ede au calcul de l'allocation qui peut \^etre cumul\'ee avec le salaire comme suit: on d\'eduit du montant de l'ARE calcul\'e en l'absence d'activit\'e 70\% du salaire brut issu de la nouvelle activit\'e. Le r\'esultat est ensuite divis\'e par le montant journalier de l'ARE pour obtenir le nombre de jours indemnisables au cours du mois.\\
Le cumul de l'ARE et du salaire de l'activit\'e reprise doit \^etre inf\'erieur au montant mensuel du salaire journalier de r\'ef\'erence (SJR) qui s'obtient en multipliant le montant journalier par 30,42.\newline

Illustrons cette m\'ethode de calcul par un exemple: un ch\^omeur qui percevait un salaire mensuel brut de 3 500\euro~ avant la perte de son emploi re\c coit ensuite 1 995\euro~mensuels (66.5\euro~ par jour) au titre de l'ARE. S'il retrouve un travail lui rapportant 2 000\euro~ bruts par mois, l'ARE qu'il pourra alors cumuler avec son revenu est \'egal \`a: $1995 - 0.7 \times 2000 = 595$\euro. Le cumul peut \^etre accord\'e puisque l'addition de l'ARE partielle et du nouveau salaire ne d\'epasse pas l'ancien salaire (595 + 2 000 < 3 500).\\
Le nombre de jours indemnisables au cours du mois est \'egal \`a: $595 / 66,5 = 8,9$ soit 9 jours. L'ARE vers\'ee mensuellement en plus du revenu issu de la nouvelle activit\'e est \'egale \`a 598,5\euro~ ($66,5 \times 9$). Le total s'\'el\`eve donc \`a $598,5 + 2 000 = 2 598,5$\euro. \newline

\textbf{Certains cas particuliers existent}:
\begin{itemize}
\item R\'edaction d'articles de presse, mission d'expertise, activit\'e artistiques: si la r\'emun\'eration n'est pas vers\'ee mensuellement, le calcul du nombre de jours indemnisables se fait au moment de la perception des r\'emun\'erations selon le calcul suivant:\\
$(\text{ARE mensuelle} - 0,7 \times \text{r\'emun\'erations issues de l'activit\'e})\div \text{ARE journalier}$.
\item Activit\'e professionnelle non salari\'ee: le cumul est possible selon les m\^emes modalit\'es que pour une activit\'e salari\'ee, le montant \`a d\'eclarer correspondant aux r\'emun\'erations soumises \`a cotisations sociales. Le nombre de jours indemnisables est calcul\'e comme suit:\\
$(\text{ARE mensuelle} - 0,7 \times \text{r\'emun\'erations mensuelles d\'eclar\'ees aux assurances sociales})\div \text{ARE journalier}$\\
La r\'emun\'eration d\'eclar\'ee au titre des assurances sociales correspond au revenu professionnel soumis \`a l'imp\^ot sur le revenu, avant d\'eduction de certains abattements. Il s'agit, plus pr\'ecis\'ement, du revenu fix\'e par proc\`es-verbal lors de l'assembl\'ee g\'en\'erale pour les g\'erants de soci\'et\'e, \'eventuellement compl\'et\'e par la part des dividendes exc\'edant 10\% du capital social; des b\'en\'efices issus de l'activit\'e au titre des BIC, BNC ou BA pour les ind\'ependants; du chiffre d'affaire apr\`es d\'eduction de l'abattement pour frais professionnels pour les entrepreneurs individuels relevant du r\'egime micro-social. Dans le cas o\`u le revenu professionnel n'est pas encore connu, une base forfaitaire est utilis\'ee \`a titre provisoire pour le calcul, et fait l'objet d'une r\'egularisation annuelle lorsque les revenus r\'eels soumis \`a cotisations sociales. Pour plus d'informations sur la d\'efinition du revenu professionnel d\'eclar\'e au titre des assurance sociales, le lecteur peut se reporter au \textit{Guide l\'egislatif des cotisations sociales}\cite{IPP2014} de l'Institut des Politiques publiques. De m\^eme que pour une activit\'e salari\'ee, le cumul de l'ARE partielle et du revenu d'activit\'e ne peut d\'epasser le montant mensuel du salaire journalier de r\'ef\'erence ayant servi de base au calcul de l'ARE.
\item Salari\'es relevant d'une annexe: les r\'egles de cumul partiel allocations-r\'emun\'erations s'appliquent aux salari\'es relevant des annexes, except\'es pour ceux relevant des annexes VIII et X;
\item Les b\'en\'eficiaires de l'ARE formation peuvent b\'en\'eficier du cumul dans les m\^emes conditons que les b\'en\'eficiaires de l'ARE, alors que les b\'en\'eficaires du CSP ne sont pas autoris\'es \`a b\'en\'eficier de ce cumul (ils sont cependant couverts par des r\`egles sp\'ecifiques au CSP et peuvent b\'en\'eficier, sous certaines conditions, d'une allocation compl\'ementaire en cas de reprise d'activit\'e)
\item Cas des salari\'es d'employeurs mutliples qui perdent un ou plusieurs de leurs emplois, et en conservent un ou plusieurs: ils peuvent \^etre indemnis\'es au titre de ou des emploi(s) perdu(s) s'ils remplissent les conditions d'\'eligibilit\'e \`a l'ARE, et cumuler int\'egralement leurs allocations avec les revenus per\c cus au titre des activit\'es conserv\'ees (depuis le 1\up{er} octobre 2014, ce cumul int\'egral est \'egalement possible pour les demandeurs d'emploi relevant de l'annexe IV). Ils doivent alors \^etre inscrits comme demandeurs d'emploi, percevoir l'ARE calcul\'ee sur leur(s) emploi(s) perdu(s), et peuvent int\'egralement cumuler ARE et r\'emun\'eration issues de ou des activit\'e(s) conserv\'ee(s) pendant toute la dur\'ee du droit \`a l'ARE.
\end{itemize}

\texttt{R\'ef\'erences l\'egislatives}{: Art. 30 \`a 34; accord d'application \no 11 du r\'eglement annex\'e \`a la convention d'assurance ch\^omage du 14 mai 2014}



\subsection{Mode et dur\'ee de versement}



La dur\'ee du cumul correspond \`a la dur\'ee des droits \`a l'ARE initialement ouverts. Si l'activit\'e cesse entre temps, un reprise des droits est possible sous certaines conditions. Au moment de l'\'epuisement des droits, ceux-ci peuvent \^etre recharg\'es dans des conditions qui seront d\'evelopp\'ees dans un paragraphe ult\'erieur.\newline

L'activit\'e professionnelle doit \^etre d\'eclar\'ee mensuellement aux services de P\^ole emploi, et une copie du bulletin de salaire doit \^etre fournie. En cas d'impossibilit\'e de pr\'esenter les pi\`eces justifiant de la reprise d'activit\'e, un paiement anticip\'e peut avoir lieu, \`a hauteur de 80\% de l'allocation due pour le mois, la r\'egularisation d\'efinitive intervenant le mois suivant si les pi\`eces justificatives ont \'et\'e fournies. Dans le cas contraire, l'avance doit \^etre rendue \`a P\^ole emploi le ou les moins suivants.\\

\texttt{R\'ef\'erences l\'egislatives}{Art. 30 \`a 34 et accord d'application \no 11 du r\`eglement de l'Assurance Ch\^omage}

\section{\label{section:sec2}Cumul des allocations de solidarit\'e et des revenus d'activit\'e}



Le cumul des revenus issus d'une activit\'e avec les allocations de solidarit\'e est \'egalement possible, pour inciter les demandeurs d'emploi \`a retourner sur le march\'e du travail. Les modalit\'es de ce cumul varient selon le volume horaire de l'activit\'e reprise:
\begin{itemize}
\item \underline{Activit\'e sup\'erieure \`a 78 heures par mois}: les b\'en\'eficiaires de l'ASS reprenant une activit\'e salari\'ee de plus de 78 heures par mois ou une activit\'e non salari\'ee peuvent solliciter un cumul de leurs revenus avec l'ASS pendant 12 mois. Ce cumul sera int\'egral les 3 premiers mois, puis les 9 mois suivants, ils toucheront l'ASS diminu\'ee des revenus d'activit\'e, ainsi qu'une prime mensuelle de 150\euro.
\item \underline{Activit\'e inf\'erieure \`a 78 heures par mois}: les allocataires de l'ASS reprenant une activit\'e salari\'ee de moins de 78 heures par mois, et les b\'en\'eficiaires de l'ATA, quel que soit le volume horaire de leur activit\'e, peuvent \'egalement solliciter un cumul de leur allocation et leurs revenus pendant 12 mois. Pendant les 6 premiers mois, le cumul est int\'egral si les revenus issus de l'activit\'e ne d\'epassent pas la moiti\'e du SMIC. Dans le cas contraire, du cumul int\'egral est d\'eduit un montant correspondant \`a 40\% des sommes exc\'edentaires. Pendant les 6 mois suivants, du cumul int\'egral sont d\'eduits 40\% des revenus issus de l'activit\'e.
\end{itemize}

\section{\label{section:sec3}Cr\'eation et reprise d'entreprise}



L'aide \`a la reprise ou cr\'eation d'entreprise (ARCE) s'adresse, sous certaines conditions, aux demandeurs d'emploi qui cr\'eent ou reprennent une entreprise, d\`es le d\'ebut de leur activit\'e. Plus pr\'ecis\'ement, le public concern\'e comprend les personnes \`a qui l'on a accord\'e le versement de l'ARE (en cours d'indemnisation ou non encore indemnis\'es du fait de l'application d'un diff\'er\'e ou d'un d\'elai d'attente au moment du d\'emarrage de l'activit\'e), ainsi que les personnes licenci\'ees qui ont entam\'e les d\'emarches de cr\'eation ou de reprise d'une entreprise au cours de leur p\'eriode de pr\'eavis ou de leur cong\'e de reclassement ou de mobilit\'e.



\subsection{D\'etermination des r\`egles d'\'eligibilit\'e}



Le demandeur d'emploi voulant b\'en\'eficier de l'ARCE doit justifier de l'attribution de l'Aide aux ch\^omeurs cr\'eateurs ou repreneurs d'entreprises (ACCRE), et ne doit pas d\'ej\`a cumuler l'ARE avec les revenus provenant d'une activit\'e r\'emun\'er\'ee.



\subsection{Calcul}



L'ARCE est \'egale \`a la moiti\'e du montant des droits restants \`a l'ARE au moment du d\'ebut de l'activit\'e, ou au moment de la date d'attribution de l'ACCRE si celle-ci vient apr\`es. De ce montant est ensuite d\'eduite une participation de 3\% au financement des retraites compl\'ementaires.\\
Toutefois, l'allocation \'equivalente \`a 50\% des droits restants \`a l'ARE a \'et\'e provisoirement abaiss\'ee \`a 45\% pour la p\'eriode allant du 1\up{er} avril au 31 d\'ecembre 2013\footnote{Avenant \no 2 du 28/02/2013 \`a l'accord d'application \no 24 et avenant \no 4 du 28/02/2013 modifiant l'article 34 du r\`eglement g\'en\'eral annex\'e \`a la Convention du 06/05/2011 relative \`a l'indemnisation du ch\^omage.}.\\
Prenons l'exemple d'un demandeur d'emploi b\'en\'eficiant de l'ARE \`a hauteur de 50\euro~ par jour pour une dur\'ee maximale de 365 jours. En tenant compte du diff\'er\'e d'indemnisation et du d\'elai d'attente, on \'etablit que le versement a commenc\'e le 1\up{er} janvier 2014. Au moment de la cr\'eation de son entreprise, le 10 f\'evrier 2014, il lui reste donc 365 - 40 = 325 jours d'indemnisation. Il pourra alors b\'en\'eficier d'une aide \'egale \`a : $(50 \times 325)\div 2 = 8125$\euro. L'aide nette de la participation au financement des retraites compl\'ementaires est \'egale \`a 7 881,25\euro.



\subsection{Mode et dur\'ee de versement}



\textbf{Paiement en deux versements}: un premier versement \'equivalent \`a la moiti\'e du total de l'ARCE intervient \`a la date du d\'ebut de l'activit\'e ou de l'ouverture des droits \`a l'ARE si celle-ci vient apr\`es, \`a condition que le cr\'eateur ou repreneur d'entreprise ne soit plus inscrit en tant que demandeur d'emploi. Le deuxi\`eme versement des droits restants \`a l'ARCE intervient six mois apr\`es le d\'ebut de l'activit\'e, si celle-ci n'a pas cess\'e. En cas d'arr\^et de l'activit\'e, les droits restants \`a l'ARE avant le d\'ebut de l'activit\'e peuvent \^etre repris, apr\`es d\'eduction du montant de l'ARCE qui a \'et\'e vers\'e. Cependant, pour avoir droit \`a ce reliquat, la personne dispose d'un d\'elai de 3 ans, auquel s'ajoute la dur\'ee maximale de des droits \`a l'ARE, \`a compter de la date d'admission \`a l'ARE, pour s'inscrire en tant que demandeur d'emploi\footnote{Pour un demandeur d'emploi ayant ouvert ses droits \`a l'ARE un 1\up{er} janvier 2014 pour une dur\'ee maximale de 24 mois devra, si son activit\'e cesse et qu'il d\'esire b\'en\'eficier du reliquat de ses droits \`a l'ARE, s'inscrire comme demandeur d'emploi avant le 31 d\'ecembre 2018.}.\newline

L'ARCE ne peut \^etre vers\'ee alors que le cr\'eateur ou repreneur d'entreprise a d\'ej\`a opt\'e pour le cumul partiel entre ARE et revenus d'activit\'e. En revanche, si le cr\'eateur ou repreneur d'entreprise ne peut t\'emoigner de l'obtention de l'ACCRE au commencement de son activit\'e, il peut b\'en\'eficier du cumul partiel entre ARE et revenus d'activit\'e jusqu'\`a ce qu'il se voit attribuer l'ACCRE. L'ARCE peut alors lui \^etre vers\'ee, les droits \'etant calcul\'es comme le reliquat des droits restants le jour de l'obtention de l'ACCRE.\newline

\textbf{D\'emarches}: le projet de cr\'eation ou de reprise d'entreprise doit \^etre communiqu\'ee \`a P\^ole emploi (ou autre organisme en charge de l'accompagnement), aupr\`es duquel la demande d'ARCE est d\'epos\'ee. En revanche, la demande d'ACCRE se fait aupr\`es du centre de formalit\'es des entreprises (CFE) rattach\'e au lieu de la future entreprise.\\
L'attestation d'admission au b\'en\'efice de l'ACCRE est d\'elivr\'ee par le R\'egime social des ind\'ependants ou par l'Urssaf. Pour obtenir le b\'en\'efice de l'ARCE, cette attestation doit \^etre remise \`a P\^ole emploi au commencement de l'activit\'e. Toutefois, si cette attestation n'est pas d\'elivr\'ee dans un d\'elai d'un mois, les documents \`a fournir pour b\'en\'eficier de l'ARCE sont les suivants: le r\'ec\'episs\'e de d\'ep\^ot de dossier ACCRE d\'elivr\'e par le CFE, un extrait K-bis et une attestation sur l'honneur d'absence de notification de rejet par le RSI ou l'URSSAF (l'absence de notification valant comme acceptation).


\section{\label{section:sec4}Les droits rechargeables}



Les droits rechargeables sont un dispositif permettant aux demandeurs d'emploi de reprendre leurs droits entam\'es au terme d'une activit\'e reprise, et de les prolonger sur la base de leurs p\'eriodes d'activit\'e ayant eu lieu au cours de l'indemnisation.\\
Le droit \`a l'allocation d'assurance est d'abord ouvert si le demandeur d'emploi remplit toutes les conditions n\'ecessaires pour b\'en\'eficier de l'ARE, notamment celle portant sur la dur\'ee d'affiliation minimum de 122 jours au cours des 28 derniers mois. L'ARE est alors vers\'ee jusqu'\`a \'epuisement des droits. Cependant, en cas de reprise d'une activit\'e professionnelle au cours de la p\'eriode d'indemnisation, deux m\'ecanismes s'enclenchent:\newline

\textbf{Reprise des droits}: Si la reprise de l'activit\'e ne donne pas lieu au cumul partiel du salaire et de l'ARE, au moment o\`u l'activit\'e cesse, le demandeur d'emploi peut reprendre ses droits \`a l'ARE restants, s'il continue de remplir les conditions d'\'eligibilit\'e \`a cette allocation. La reprise des droits s'applique:
\begin{itemize}
\item S'il existe un reliquat de droits;
\item Si les droits \`a l'allocation d'assurance  ne sont pas d\'echus, c'est-\`a-dire qu'il ne faut pas qu'entre la date d'ouverture initiale des droits \`a l'ARE (sans prise en compte de l'\'eventuel diff\'er\'e ou d\'elai d'indemnisation) et la date de reprise du paiement il se soit \'ecoul\'e plus de 3 ans additionn\'es \`a la dur\'ee d'indemnisation accord\'ee initialement \`a l'allocataire. Par exemple, si un demandeur d'emploi se voit accorder une indemnisation pour une dur\'ee de 100 jours, il peut b\'en\'eficier d'une reprise de ses droits si la perte de son emploi intervient dans un d\'elai de 1195 jours apr\`es l'ouverture initiale de ses droits. Toutefois, le d\'elai de d\'ech\'eance peut \^etre allong\'e sous certaines conditions: il ne court pas durant la p\'eriode pendant laquelle la personne a repris une activit\'e sous contrat \`a dur\'ee d\'etermin\'ee; durant un contrat de service civique; en cas de versement du compl\'ement de libre choix d'activit\'e (remplac\'e par la prestation partag\'ee d'\'education de l'enfant pour les enfants n\'es ou adopt\'es \`a partir du 1\up{er} octobre 2014), ou de l'allocation journali\`ere de pr\'esence parentale. Le d\'elai de carence ne s'applique pas pour les personnes b\'en\'eficiant du maintien de leurs droits jusqu'\`a l'\^age de la retraite.
\item Si l'arr\^et de l'activit\'e n'est pas volontaire. Cette condition est lev\'ee si le ch\^omeur a travaill\'e moins de 91 jours ou 455 heures depuis la date d'ouverture des droits. En de\c c\`a de ce seuil, la perte de l'emploi, m\^eme volontaire, peut donner lieu \`a la reprise des droits, selon la nouvelle Convention d'assurance ch\^omage de 2014.
\item Si le ch\^omeur continue de remplir toutes les conditions d'\'eligibilit\'e \`a l'ARE.
\end{itemize}

\vspace{3mm}

\textbf{Rechargement des droits}: Si le ch\^omeur arrive \`a la fin de sa p\'eriode d'indemnisation, et qu'il a effectu\'e au cours de celle-ci des p\'eriodes d'activit\'e, il peut b\'en\'eficier du rechargement de ses droits \`a hauteur du nombre de jours pendant lesquels il a travaill\'e. Les conditions \`a remplir sont les suivantes:
\begin{itemize}
\item Avoir travaill\'e au moins 150 heures au cours de la p\'eriode d'indemnisation. Cette dur\'ee d'affiliation se mesure sur les p\'eriodes d'activit\'e exerc\'ees entre la date d'ouverture initiale des droits et la date d'\'epuisement de ces droits, dans la limite des 28 derniers mois (36 pour les plus de 50 ans) pr\'ec\'edant la fin du dernier contrat de travail. Ainsi, la p\'eriode consid\'er\'ee pour le d\'ecompte des jours travaill\'es correspond aux 28 derniers mois pr\'ec\'edant la fin du dernier contrat de travail et post\'erieurs \`a la date d'ouverture des droits initiale. Pour donner lieu au rechargement des droits, la perte du dernier emploi ne doit pas \^etre volontaire. Cette condition peut s'appliquer non pas \`a la derni\`ere activit\'e mais \`a une autre p\'eriode d'activit\'e s'il ne justifie pas de 91 jours ou de 455 heures depuis le d\'epart volontaire.
\item Continuer de remplir toutes les conditions d'\'eligibilit\'e \`a l'ARE.
\end{itemize}
Le rechargement des droits donne lieu \`a une dur\'ee d'indemnisation additionnelle correspondant au nombre de jours travaill\'es au cours de la p\'eriode de r\'ef\'erence prise en compte pour le rechargement des droits, ou bien du nombre d'heures travaill\'ees sur la m\^eme p\'eriode divis\'ee par cinq. La dur\'ee minimale d'indemnisation est de 30 jours.\\
Les diff\'er\'es d'indemnisation s'appliquant dans le cadre de l'ouverture des droits \`a l'ARE s'appliquent de la m\^eme mani\`ere lors du rechargement des droits, mais seulement si les sommes entrant de leur calcul n'ont pas d\'ej\`a \'et\'e prises en compte pour le calcul de diff\'er\'es lors d'une pr\'ec\'edente prise en charge. De m\^eme, le d\'elai d'attente applicable lors de toute prise en charge au titre de l'assurance ch\^omage prend \'egalement effet lors du rechargement des droits, \`a condition qu'il n'ait pas d\'ej\`a \'et\'e appliqu\'e dans les 12 mois pr\'ec\'edents.\\

Dans le cas o\`u le demandeur d'emploi ne totalise pas 150 heures de travail au moment de l'extinction des droits, il peut toutefois b\'en\'eficier d'une ouverture des droits ult\'erieure qui prend en compte les p\'eriodes d'activit\'e effectu\'ees lors de sa p\'eriode d'indemnisation, dans la limite des 28 derniers mois pr\'ec\'edant la fin du contrat de travail, s'il remplit les conditions d'\'eligibilit\'e \`a l'ARE. Par exemple, si une premi\`ere ouverture des droits est accord\'ee pour une dur\'ee de 200 jours, et que le demandeur d'emploi reprend une activit\'e pendant 15 jours soit 75 heures lors de son indemnisation, il ne peut b\'en\'eficier d'un rechargement de ses droits au terme des 200 jours (75 heures < 150 heures). Cependant, s'il reprend plus tard un travail pour une dur\'ee de 120 jours, \`a la fin de son contrat de travail, il totalise alors 120 + 15 = 135 jours d'affiliation dans les 28 derniers mois, et peut alors b\'en\'eficier de l'ARE, car sa dur\'ee d'affiliation est sup\'erieur \`a 122 jours.\\
A d\'efaut de pouvoir recharger ses droits au moment de leur extinction, le demandeur d'emploi pourra n\'eanmoins \^etre pris en charge au titre de l'Allocation de Solidarit\'e Sp\'ecifique.\\

Pour mieux comprendre comment s'appliquent la r\'egles du rechargement des droits, nous pouvons illustrer \c ca par deux exemples: un premier cas o\`u le rechargement des droits s'applique:

\begin{fig}
[15cm]                                                                          % Argument 1 (15 cm) : largeur de la figure
{                                                                               % Argument 2 : titre de la figure
Application des droits rechargeables\label{fig:DR}
}
{                                                                               % Argument 3 : lien vers le fichier pdf de la figure (� stocker imp�rativement dans le dossier \Sections\3-Section1\Figures\)
\graphique{Droits_rechargeables.pdf}
}
{}                                                                              % Argument 4 : notes de la figure (vide)
\end{fig}

Un deuxi\`eme cas o\`u le rechargement ne peut se faire:
\begin{fig}
[15cm]                                                                          % Argument 1 (15 cm) : largeur de la figure
{                                                                               % Argument 2 : titre de la figure
Cas o\`u le rechargement des droits ne peut s'effectuer\label{fig:DNR}
}
{                                                                               % Argument 3 : lien vers le fichier pdf de la figure (� stocker imp�rativement dans le dossier \Sections\3-Section1\Figures\)
\graphique{Droits_non_rechargeables.pdf}
}
{}                                                                              % Argument 4 : notes de la figure (vide)
\end{fig}

\texttt{R\'ef\'erences l\'egislatives:}{Convention du 14/05/2014, art. 13 paragraphe 4 entr\'e en vigueur le 01/10/2014 ; Circ. Un\'edic \no 2014-19 du 02/07/2014}

\section{\label{section:sec5}Reprise d'activit\'es moins r\'emun\'er\'ees}



Participant \`a l'ensemble des mesures visant \`a inciter la reprise d'un emploi, l'aide diff\'erentielle de reclassement (ADR) peut \^etre vers\'ee dans le cas o\`u le demandeur d'emploi reprend un activit\'e dont le salaire est au moins 15\% inf\'erieure \`a celui de sont emploi pr\'ec\'edent, pour le m\^eme nombre d'heures travaill\'ees.\\

Nous allons d\'etailler les modalit\'es d'application de l'ADR dans les paragraphes qui suivent. Cependant, l'ADR a \'et\'e supprim\'ee \`a partir du 1\up{er} avril 2015 (Avenant n° 1 du 25 mars 2015), mais elle continue de s'appliquer pour les reprises d'emploi ant\'erieures \`a cette date.



\subsection{D\'etermination des r\`egles d'\'eligibilit\'e}



Les conditions pour b\'en\'eficier de l'ADR sont les suivantes:
\begin{enumerate}
\item \^etre \^ag\'e de moins de 50 ans, et \^etre indemnis\'e au titre de l'ARE depuis plus de 12 mois ou bien \^etre \^ag\'e de 50 ans et plus, et \^etre indemnis\'e au titre de l'ARE, quelle que soit la dur\'ee de l'indemnisation;
\item Ne pas remplir les conditions d'un cumul entre ARE et revenus d'activit\'e (si le cumul de l'ARE partiel et des revenus d'activit\'e d\'epasse 70\% de l'ancien salaire par exemple);
\item L'activit\'e reprise doit \^etre salari\'ee, mais ne peut \^etre chez l'ancien employeur. Dans le cas d'un CDD, il doit \^etre d'une dur\'ee de 30 jours minimum;
\item Le salaire brut mensuel doit \^etre inf\'erieur de 15\% ou plus \`a l'ancien salaire mensuel brut utilis\'e dans le calcul de l'ARE, pour le m\^eme nombre d'heures de travail.
\end{enumerate}
Au 31 d\'ecembre 2014, on comptait 7 074 b\'en\'eficiaires de l'ADR dans la France enti\`ere, dont 70\% avaient 50 ans ou plus.


\subsection{Calcul}



L'ADR est calcul\'ee selon la formule suivante: \\
\begin{equation}
\begin{split}
ADR=\text{salaire mensuel ant\'erieur ayant servi au calcul de l'ARE} \\
-  \text{nouveau salaire brut}
\end{split}
\end{equation}

Cependant, le montant total de l'ADR per\c cu ne peut d\'epasser la moiti\'e des droits restants \`a l'ARE.



\subsection{Mode et dur\'ee de versement}



L'ADR est vers\'ee mensuellement pendant une dur\'ee ne pouvant exc\'eder la dur\'ee des droits restants \`a l'ARE \`a la vieille de la reprise d'activit\'e. Le versement peut \^etre interrompu en cas de suspension temporaire du contrat de travail de plus de 15 jours (cong\'e maternit\'e, fermeture temporaire de l'entreprise, etc.).\\

Prenons l'exemple d'un demandeur d'emploi reprenant une activit\'e r\'emun\'er\'ee \`a hauteur de 2 000\euro~ bruts mensuels pour 35 heures. Son salaire mensuel brut pr\'ec\'edant ayant servi au calcul de l'ARE est \'egal \`a 3 000\euro~ pour la m\^eme dur\'ee de travail.\\
L'ARE journali\`ere est \'egale \`a 57\euro~, et, le nouveau salaire \'etant inf\'erieur \`a 85\% de l'ancien salaire, le travailleur peut pr\'etendre au versement de l'ADR. Son montant journalier est calcul\'e comme suit:\\
\begin{align}
3 000 - 2 000 &= 1000\\
1000 \div 30 &= 33,33
\end{align}
L'ADR journali\`ere est \'egale \`a 33,33 \euro~.  Si l'on fait l'hypoth\`ese que la dur\'ee des droits \`a l'ARE restant est de 100 jours, l'ADR sera vers\'ee dans la limite de:\\

$(57 \times 100)\div 2 = 2 850\text{\euro} $ \\

car 3 333\euro~ est sup\'erieur \`a 2 850\euro.\\

Si jamais le nouvel emploi est perdu, le versement de l'ARE peut reprendre. Toutefois, du reliquat des droits est d\'eduit la p\'eriode de versement de l'ADR. Ainsi, si le m\^eme travailleur perd l'emploi retrouv\'e au bout de 50 jours, il aura alors touch\'e:\\
 $33,33 \times 50 = 1666,67 \text{\euro}$. \\

 Ainsi, le nombre de jours \`a d\'eduire du reliquat des 100 jours est \'egal \`a: $1666,67 \div 57 = 29.2$



\subsection{Cas particuliers}



Les personnes \^ag\'ees de 58 ans ou plus au moment o\`u leur contrat de travail prend fin retrouvent automatiquement leurs anciens droits, m\^eme dans le cas o\`u la perte de l'emploi est volontaire. \\
La m\^eme r\`egle s'applique pour les demandeurs d'emploi de plus de 61 ans r\'eunissant les conditions n\'ecessaires au maintien des droits \`a l'allocation jusqu'\`a l'\^age de la retraite.\\

La reprise d'une activit\'e professionnelle non salari\'ee n'ouvre pas de nouveaux droits \`a l'ARE, mais peut seulement conduire au versement du solde des allocations.\\

\texttt{R\'ef\'erences l\'egislatives}{: Art. 35 et accord d'application \no 23 du r\`eglement de l'assurance ch\^omage}

\section{\label{section:sec6}Aides mat\'erielles}


En parall\`ele du versement des allocations ch\^omage d'assurance et des incitations \`a la reprise d'une activit\'e par le cumul partiel des allocations et des revenus professionnelles, certaines aides sont mises en place pour des cat\'egories sp\'ecifiques de demandeurs d'emploi.
\paragraph{}
\textbf{L'aide pour cong\'es non pay\'es}: si un ch\^omeur reprend une activit\'e salari\'ee et qu'il ne peut b\'en\'eficier de cong\'es pay\'es au moment ou l'entreprise qui l'emploie ferme annuellement pour cong\'es, le salari\'e peut faire la demande d'une aide pour cong\'es non pay\'es aupr\`es de P\^ole emploi. Cette aide, vers\'ee en une fois, est calcul\'ee selon le nombre de jours de fermeture, et ne peut d\'epasser le montant des allocations qui auraient \'et\'e per\c cues si le demandeur d'emploi n'avait pas repris une activit\'e salari\'ee.\newline

\texttt{R\'ef\'erences l\'egislatives:} {art. 36 - R\`eglement g\'en\'eral annex\'e \`a la convention du 6 mai 2011}

\paragraph{}
\textbf{Aide \`a l'allocataire arrivant au terme de ses droits au titre de l'assurance ch\^omage}: Certains demandeurs d'emploi, \`a l'extinction de leurs droits \`a l'assurance ch\^omage, ne peuvent b\'en\'eficier de l'ASS, car ils ne remplissent pas tous les crit\`eres requis. Si la raison pour laquelle ils ne peuvent b\'en\'eficier de l'ASS n'est pas li\'ee aux ressources (mais \`a une dur\'ee d'affiliation insuffisante par exemple), une aide pour ch\^omeur en fin de droits peut \^etre distribu\'ee. Son montant, forfaitaire, est \'egal \`a 27 fois la partie fixe de l'ARE.\newline

\texttt{R\'ef\'erences l\'egislatives:} {art. 37 - R\`eglement g\'en\'eral annex\'e \`a la convention du 6 mai 2011}

\paragraph{}
\textbf{Allocation d\'ec\`es}: En cas de d\'ec\`es d'un demandeur d'emploi indemnis\'e (ou en p\'eriode de diff\'er\'e ou de d\'elai d'attente), une allocation peut \^etre vers\'ee au conjoint du demandeur d'emploi d\'ec\'ed\'e. Elle est \'egale \`a 120 fois le montant journalier de l'allocation que touchait, ou aurait touch\'e, le demandeur d'emploi, major\'ee de 45 fois le montant de cette allocation journali\`ere pour chaque enfant \`a charge au sens de la s\'ecurit\'e sociale.\newline

\texttt{R\'ef\'erences l\'egislatives:} {art. 35 - R\`eglement g\'en\'eral annex\'e \`a la convention du 6 mai 2011}

\ifx\isEmbedded\undefined
\newpage
%\bibliography{../../Biblio/biblio-rapport}
\end{document}
\else \fi
