Le pr�sent chapitre pr�sente les transferts sociaux dont peuvent b�n�ficier les m�nages. Ces transferts se calculent � l'�chelle du \textbf{foyer social} (et non pas � l'�chelle du foyer fiscal comme c'est le cas pour l'IRPP). Cette entit� se compose d'une personne ou d'un couple de r�f�rence ainsi que des personnes � charge. � la diff�rence du foyer fiscal, les membres du couple de r�f�rence ne doivent pas obligatoirement �tre mari�s ou pacs�s pour faire partir du m�me foyer social. Ils peuvent �tre concubins. Quant � la d�finition de la notion de personnes � charge, elle varie selon les transferts.

Dans ce qui suit, nous d�crivons les r�gles de calcul des transferts sociaux. Les transferts peuvent �tre divis�s en trois cat�gories que nous pr�sentons successivement : les prestations familiales (Partie~\ref{sec_prest}), les allocations logement (Partie~\ref{sec_logt}) et les minima sociaux (Partie~\ref{sec_minima}).

\section{Architecture institutionnelle}


\section{Les d�penses sociales en France}