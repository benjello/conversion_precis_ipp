\ifx\isEmbedded\undefined
\input{../../Style/ipp-separate}
\setcounter{chapter}{0}
\chapter{\label{Chapitre1}Introduction}
\else \fi


%%%%%%%%%%%%%%%%%%%%%%%%%%%%%%%%%%%%%%%%%%%%%%%%%%%%%%%%%%%%%%%%%%%%%%%%%%%%%%%%%%%%%%%%%%


L'assurance chômage a été créée à l'échelle nationale par la Convention collective nationale et interprofessionnelle du 31 Décembre 1958, instituant le principe d'une indemnisation en période de chômage harmonisée sur tout le territoire, et distribuée de manière centralisée, pour les salariés de l'industrie et du commerce. Selon les mots du Général De Gaulle, l'indemnisation était censée fournir un supplément portant le revenu aux alentours du salaire minimum après la perte de l'emploi. Ce régime d'assurance chômage -- obligatoire pour les entreprises appartenant à une branche d'activité représentée au CNPF (Conseil National du Patronat Français devenu ensuite le MEDEF) -- vient compléter un régime public d'assistance déjà existant. En effet, le régime d'assurance -- comme une assurance classique -- vise à transférer une partie du revenu de l'état positif -- en emploi -- à l'état négatif -- au chômage -- en cas de réalisation du risque. Les allocations sont donc financées par des cotisations payées sur le salaire brut en période d'emploi, témoignant de l'affiliation des travailleurs au système d'assurance chômage. Les allocation distribuées en période de chômage sont proportionnelles au salaire, et donc aux cotisations contribuant au financement de l'assurance. A l'inverse, le régime public d'assistance est financé par l'État directement, et répond à un objectif de solidarité visant à fournir un revenu décent en cas de perte du salaire, pour ceux qui ne sont plus ou pas éligibles aux allocations d'assurance. Des initiatives autonomes existaient auparavant, organisées au niveau de chaque métier, sous forme de caisses syndicales et patronales. L'Office du Travail, créé en 1891, contrôle ces caisses et classifie les types de chômage pour déterminer quels chômeurs ont droit à une indemnisation. L'État apporte un soutien financier à ces caisses, et va même jusqu'à créer des fonds locaux de chômage qui couvrent une plus grande partie de la population, au moment de la crise économique des années 1930. Cependant, ce n'est qu'à partir de 1958 qu'est mis en place un véritable système d'indemnisation obligatoire et national.\\

Le 13 Juillet 1967, le régime d'assurance devient obligatoire pour toutes les entreprises relevant de l'industrie et du commerce. La première ANPE (Agence Nationale Pour l'Emploi) est créée: les chômeurs doivent obligatoirement y être affiliés pour bénéficier des allocations d'assurance chômage. La différence de principe qui existait entre les allocations d'assurance et d'assistance a été atténuée par la Convention d'assurance chômage de 1979 qui a fusionné les deux régimes en un système unique. Cette distinction est à nouveau consacrée par l'ordonnance du 21 Février 1984, qui dessine, dans ses grands traits, l'architecture actuelle du système d'indemnisation. \\

% 1974: allocation spéciale d'attente: 100% de l'ancien salaire pdt 1 an. Supprimée en 1979, coûte tp cher
% 1976: cap des 1 million de chômeurs franci
%13 juillet 1067: 1ere ANPE
% 1982: 2 millions de chômeurs, baisse des prestations et hausse des cotis sous peine de faillite
% 1984: allocation spécifique de solidarité pr chômeurs en fin de droits
% 93: cap des 3 millions de chômeurs, creation de l'AUD

L'assurance chômage fait partie de la catégorie des assurances sociales en ce sens qu'elle est organisée à l'échelle de l'État. Son caractère public s'explique par le fait que la mise en place des assurances soulève deux problèmes: dans un monde où l'assureur ne peut observer le niveau d'effort que chaque travailleur exerce pour conserver ou pour chercher un travail, il est impossible pour l'assureur de distinguer les niveaux de risque parmi les chômeurs. Face à ce manque d'information, le prix fixé par l'assureur ne pourra être qu'un prix moyen correspondant à ce que l'assureur perçoit comme étant le risque moyen, conduisant à l'exclusion d'une partie des travailleur dont le niveau de risque rend le prix moyen non attractif pour eux. Face à ce problème, appelé \textit{sélection adverse} dans la littérature sur l'élaboration des assurances, il apparaît nécessaire d'organiser l'assurance chômage publiquement, pour que tous les travailleurs soient couverts contre le risque chômage. C'est principalement ce problème qui explique que l'assurance chômage n'ait jamais été organisée au niveau privé à grande échelle. Tous les systèmes existant aujourd'hui à travers le monde sont publics.

Si l'assurance chômage est organisée à l'échelle de l'État, elle est néanmoins gérée par les partenaires sociaux représentants les syndicats des travailleurs et des employeurs. Ils sont en effet chargés de négocier les règles d'indemnisation et de contribution relatives à l'assurance chômage au sein d'une Convention renouvelée régulièrement. Ils font également partie du Conseil d'administration de l'Unédic (Union nationale interprofessionnelle pour l'emploi dans l'industrie et le commerce) en charge d'appliquer les règles et de veiller au bon fonctionnement de l'assurance chômage en gérant le recouvrement des cotisations et l'indemnisation des demandeurs d'emploi. Ainsi, la gestion de l'assurance chômage repose sur plusieurs acteurs: l'Unédic, qui a pour mission de transcrire dans la loi les décisions des partenaires sociaux, de s'assurer que l'ACOSS collecte les cotisations sociales servant au financement de l'assurance chômage, et que Pôle Emploi distribue les prestations aux demandeurs d'emploi. L'Unédic est aussi le premier contributeur au budget de Pôle Emploi, à hauteur de 64~\%, et appuie les partenaires sociaux dans la négociation des conventions en fournissant études, analyses et statistiques relatives au marché du travail et au système d'indemnisation chômage. De son côté, Pôle Emploi est une agence nationale qui s'occupe de mettre en relation les entreprises et les demandeurs d'emploi, et d'accompagner les chômeurs dans leur projet de réinsertion sur le marché du travail: en plus de leur fournir les allocations auxquelles ils sont éligibles, les conseillers de Pôle Emploi définissent avec les chômeurs un projet professionnel, les orientent et les conseillent dans leur recherche d'information, d'emploi ou de formation. Ils sont également chargés de veiller à ce que les demandeurs d'emploi cherchent activement un travail, car cela constitue une condition nécessaire au maintien des droits à l'indemnisation. C'est donc cette conjonction d'acteurs qui doit assurer le bon fonctionnement du système d'indemnisation chômage et contribuer à la stabilisation du marché du travail.

Depuis sa création, l'assurance chômage, et le régime d'assistance qui la complémente, ont fait l'objet de nombreux débats et de nombreuses modifications; certaines affectant simplement le montant des allocations ou les taux de cotisations, d'autres changeant le système en profondeur. Ainsi, les débats ont notamment porté sur le profil temporel des allocations, qui, selon certains économistes, devrait être dégressif pour mieux inciter à la reprise d'un emploi. Selon ce principe, les allocations seraient perçues de manière stable pendant un certain nombre de mois, puis multipliées par un coefficient inférieur à un au fil des mois. Ce système a prévalu de 1993 à 2001 sous forme d'une \textit{Allocation unique dégressive}. Le coefficient de dégressivité ainsi que la rapidité de la diminution dépendaient de l'âge du demandeur d'emploi et de la durée de sa période d'emploi passée. Depuis 2001, nous sommes revenus à un profil d'indemnisation plat, avec le même montant distribué tous les mois jusqu'à atteindre la durée maximale d'indemnisation. Cependant, un possible retour à la dégressivité fait toujours partie des débats, notamment dans le cadre des négociations sur l'assurance chômage actuelles\footnote{\url{http://abonnes.lemonde.fr/economie/article/2016/02/01/le-retour-de-la-degressivite-des-allocations-chomage_4857413_3234.html}}.\\

Ce guide se propose de fournir un descriptif de la législation régissant le système d'indemnisation chômage dans son versant d'assurance et de solidarité, pour permettre à chacun d'évaluer ses droits en cas de perte d'emploi. Il est important de noter que nous allons nous concentrer ici sur le cas des salariés -- du secteur privé et du secteur public. Nous évoquerons le cas des travailleurs indépendants, mais ceux ci dépendent d'un régime de protection sociale différent, qui n'inclut pas une assurance chômage obligatoire. Ils ont donc la possibilité de s'affilier de manière volontaire à des assurances chômages qui peuvent prendre la forme d'un contrat d'assurance perte d'emploi ou garantie chômage auprès d'une assurance privée, d'un contrat collectif d'assurance réservé aux membres adhérant à une association, ou d'un contrat individuel. Les modalités de ces assurances varient selon le type de contrat. Nous n'en détaillerons donc pas la présentation dans ce guide. De même, les dispositions spécifique détaillées dans les différentes annexes au règlement général de l'assurance chômage (intermittents du spectacle, intérimaires, apprentis, etc.) ne seront pas présentées dans ce document. \\

\texttt{Références législatives:} Loi \no 94-126 du 11 Février 1994 (JO du 13/02/1994).

\ifx\isEmbedded\undefined
\newpage
%\bibliography{../../Biblio/biblio-precis}
\end{document}
\else \fi





















