\ifx\isEmbedded\undefined
\input{../../Style/ipp-separate}
\setcounter{chapter}{0}
\chapter{\label{Chapitre1}Histoire de la rémunération des fonctionnaires}
\else \fi


%%%%%%%%%%%%%%%%%%%%%%%%%%%%%%%%%%%%%%%%%%%%%%%%%%%%%%%%%%%%%%%%%%%%%%%%%%%%%%%%%%%%%%%%%%

\indent Cette partie vise à présenter succinctement l'histoire des traitements dans la fonction publique, en présentant brièvement les origines anciennes de la rémunération des fonctionnaires et le système de traitement tel qu'il est défini depuis 1946 et approfondi depuis le 13 Juillet 1983. Nous invitons le lecteur souhaitant approfondir ses connaissances historiques de la fonction publique à se plonger dans  l'\oe{}uvre fondatrice de Marcel Pinet, "Histoire de la Fonction publique en France" et à consulter la première partie du livre d'Arnaud Freyer : "La fonction publique : Chronique d'une révolution silencieuse". Pour une analyse fine des traitements sous la Révolution, nous référons à l'essai de J-P Jourdan "Pour une histoire des traitements des fonctionnaires de l'Administration au XIXe siècle". On présentera dans cette partie les éléments de contexte historique nécessaires à la compréhension des grands enjeux de l'histoire des traitements. En particulier, les caractères plus ou moins égalitaires des régimes et cultures politiques, la définition de la fonction publique ainsi que l'état des finances publiques déterminent les systèmes de traitement et leurs applications au cours du temps.\\

Retracer les grandes lignes de l'histoire des traitements nous permet de comprendre l'état actuel de la fonction publique, l'ancrage profond du statut des fonctionnaires dans notre histoire et les contraintes de l'action publique actuelle dans le domaine. Nous réalisons par là même la constance des problèmes liés aux rémunérations des fonctionnaires. Les rémunérations des fonctionnaires coûtent-elles trop cher à l'état et aux contribuables ? Les fonctionnaires méritent-ils bien une rémunération régulière et qui augmente au cours de leurs carrière ? Peut-on rendre les fonctionnaires plus performants en les rémunérant davantage ou différemment ? Comment les contribuables et les fonctionnaires peuvent-ils s'approprier une législation des rémunérations si complexe, peut-on la réformer ? Les complexité des systèmes cachent-elles de profondes inégalités de salaires dans la fonction publique ? Comment les inégalités sont-elles justifiées et pouvons-nous accepter collectivement ces justifications ? Comment réformer les traitements de certaines catégories d'agent sans déséquilibrer les finances publiques, commettre une injustice ou remettre en cause le statut unifié des fonctionnaires ? Une logique par contrat doit-elle dominer ? Qui doit décider de la composition des salaires et des montants attribués aux individus ? Comment les rémunérations de la fonction publique doivent-elles évoluer par rapport à celles du secteur privé, complémentaire et concurrent ? La liste des problématiques est longue. Une chose est sûre, la majorité de ces problèmes ont été rencontrés sous une forme ou sous une autre depuis que l'état, et avec lui la nécessité des agents publics, s'est imposé en France.

\section{L'émergence de la nécessité des agents publics et de leur contrôle} 

Les ancêtres de nos fonctionnaires actuels sont peut-être les légistes de Philippe le Bel, qui sont chargé de l'organisation et de la conservation des documents royaux. Les légistes ont des perspectives de carrière dans la chancellerie. A la même époque, les baillis, en charge de la levée d'impôt, se multiplient et sont contrôlée par la nouvelle Chambre des Comptes, qui vérifie leurs comptes. Pour assurer leur contrôle par le roi, leur loyauté et prévenir leur corruption, les baillis sont régulièrement mutés.
Au début du XIV\up{ème}, les officiers royaux sont mis en difficulté par les états généraux. Ce n'est qu'avec la victoire de Charles V en 1359 que sont réaffirmés les principes de nomination, contrôle et révocation des officiers par le roi. L'impopularité des agents publics conduit le roi à diminuer les effectifs et à instaurer l'élection comme nouveau mode de nomination des officiers, qui les rend à la fois plus légitimes et plus autonomes.\\ 

Sous Charles V, les descendants des légistes de Philippe le Bel contribuent à l'enracinement de l'état et au développement des fonctions administratives dans des domaines nouveaux. A la même époque, une première hiérarchie et un premier statut des agents publics émergent. Par exemple, les baillis ont autorité sur les fermiers. Les officiers bénéficient d'un premier statut particulier : ils sont exempts de responsabilités, "sauvegardés" par le roi, lorsqu'ils sont dans l'exercice de leurs fonctions. L'élection comme mode de nomination des agents de généralise et les premières écoles d'administration sont créées. On cherche à rationaliser et uniformiser les paies : la rémunération des agents du Parlement augmente avec leur "antiquité", les clercs reçoivent leurs salaires des officiers, qui sont payés en gage par le roi. Toutefois, la plus grande partie de la paie des agents est constituée d'avantages en nature, d'exemptions fiscales et d'accès facilité au crédit. Ce sont ces bénéfices, davantage que les salaires, qui rendent la fonction publique attractive. Les perspectives de carrière existent : on nourrit au des espoirs d'anoblissement au sein de l'administration. Parallèlement, et ce n'est pas surprenant, le coût des dépenses de personnels de l'état et la corruption des agents publics deviennent des thèmes récurrents de doléances des contribuables. Au XV\up{ème} siècle, les suppressions de poste d'officiers deviennent naturellement les variables d'ajustement des finances publiques. De plus, les agents publics sont extrêmement impopulaires (boucs émissaires après la peste) et les luttes politiques au sein du Parlement font reculer l'élection comme mode de nomination.\\

Après une courte période de double administration après la prise de Paris par les Bourguignons (1418-1436), Paris est à nouveau capitale administrative après la victoire de Charles VII. Le XV\up{ème} siècle marque la reconstruction de la fonction publique et les premières vagues de déconcentration, avec notamment la création des Parlement provinciaux.


\section{\label{section:sec1}Ancien Régime : aux origines du statut d'agent public}

\ La fonction publique n'est pas clairement définie ni unifiée sous l'Ancien Régime. On entendra toutefois par fonction publique l'ensemble hétérogène des commissaires, des officiers civils et militaires, des intendants et des ministres. Le nombre d'agents du roi s'accroît à cette période, notamment pour étendre les pouvoirs fiscaux et militaires de l'Etat monarchique. Dans cette société d'ordres, les membres du Conseil royal sont nommés par le roi ou le sont par héritage, et leur rémunération est faite de privilèges et de rentes. Ils ne paient pas d'impôt. A côté des très rares ministères, proches du pouvoir royal, se développent les offices et les commissions. Les rémunérations des officiers, "salariés", et des commissaires, "indemnisés", préfigurent les deux éléments de la paie de nos fonctionnaires : le traitement et les indemnités.\\

Nous présentons ici brièvement les modes de rémunération des officiers et des commissaires sous l'Ancien Régime.

\subsection{Les offices : un marché de la fonction publique} 
 Le mode de rémunération des officiers nous intéresse particulièrement en ce qu'il est lié à un véritable marché de la fonction publique sous l'Ancien Régime.\\

Les \og offices \fg apparaissent au milieu du XV\up{ème} siècle. L'office désigne la charge publique attribuée à un particulier, qui paie le roi pour l'obtenir et à qui est en échange donnée une rémunération et un certain nombre de privilèges. En 1467, Louis XI impose l'inamovibilité des offices : seule est permise, peu avant la mort de l'officier, la transmission de son office. Cette transmission se monnaie. Ce marché de la fonction publique est appelé \og vénalité des offices \fg et se développe jusqu'à la révolution. En effet, les rois et les particuliers trouvent leur intérêt dans ces échanges. Le roi vend des offices pour renflouer les finances publiques. La vente d'office est particulièrement utile en ces temps de couteuses guerres d'annexion. Les particuliers acquièrent avec l'office un statut social lié à la fonction publique, un salaire stable, un investissement permanent peu risqué ou des privilèges tels que les crédits d'impôts ou des gratifications sociales liées à la proximité avec les élites.\\

En 1604, l'édit de Paulette consacre le droit des officiers à conserver, transmettre ou révoquer leurs offices contre paiement d'un impôt annuel. L'office constitue alors également un bien patrimonial qui est transmis, sans risque, de génération en génération. A la même époque, la vénalité des offices est officiellement interdite. Cependant, François I\up{er} lui-même créé en 1522 un  \og{} bureau des parties casuelles \fg{}...en charge de la vente d'offices. Les officiers peuvent alors recevoir des charges aussi variées que des charges militaires ou de la vente du beurre. La valeur de l'office est alors fixée par les ministres, qui l'évaluent selon ses bénéfices attendus, le statut social qui lui est attaché et l'intérêt espéré des acheteurs. La vente d'offices entre particuliers se développe. Les ministres réévaluent les offices selon ce qu'elles rapportent effectivement, selon l'inflation etc. et n'hésite pas à faire augmenter leurs prix alors même qu'elles sont déjà attribuées. La force est rarement nécessaire pour ce faire, les officiers étant largement attachés à leurs offices (pour le statut social conféré, la sécurité familiale etc.). La menace de la perte de privilège ou de la mise en vente d'une office similaire était parfois brandie par l'état, mais rarement réalisée (en particulier, l'Etat n'a aucun intérêt à saturer le marché des offices qui aurait conduit à leur dépréciation généralisée et à la mise à mal des finances publiques). Les offices les plus prestigieuses sont le monopole quasi-exclusif de la noblesse, et une première distinction entre haute fonction publique et basse fonction publique se développe.\\

La généralisation des offices conduit à la dilution de l'action publique, et dans une certaine mesure à son discrédit. Toutefois, l'administration étend ses services publics à de nouveaux domaines d'intérêt général tels que l'éducation et l'action sociale. En province, les guerres forcent les communes à se mobiliser en matière sociale et hospitalière. En raison de la multiplication des offices, le roi perd progressivement le contrôle sur ses officiers et créée, en réaction, les commissions. La vénalité des offices, reposant sur les privilèges, l'arbitraire et l'inégalité entre les membres de corps et le tiers-état, a sans doute une large responsabilité dans l'établissement d'une culture révolutionnaire égalitariste et dans les grandes réformes révolutionnaires du statut et de la rémunération des fonctionnaires qui en découlent.

\subsection{La paie des commissaires} A côté des offices sont également créés les commissaires, qui ne sont pas propriétaires de leurs charges et qui sont révocables par le roi. L'ordonnance de 1493 distingue les officiers et les commissaires : les officiers sont inamovibles et Les commissaires ont une charge temporaire et révocable par le roi à tout instant. Les commissaires comprennent par exemple le Chancelier de France et le connétable. Avec la création des commissions sont également définies les règles embryonnaires du recrutement des agents : des examens et diplômes sont requis pour accéder à certains postes. Dans les faits, les nominations arbitraires sont la norme. Au XVIIème siècle, le nombre de commis croît. En 1663 sont codifiés leurs droits et obligation. Les commis ont des perspectives de carrière en tant que secrétaires d'état, fonction qui se développe alors. Les commissaires sont \og{} appointés et gagé \fg{} et n'ont pas de droit sur leur poste. A l'inverse, les officiers reçoivent un sorte de traitement garantit et des \og{} épices \fg{}. Les commissaires peuvent cumuler leurs missions mais leur rémunération par mission n'évolue presque pas avec l'expérience.\\

Au XVIII\up{ème} siècle apparaissent également les commis, qui ressemblent davantage à nos fonctionnaires modernes. Ils sont nommés par les chefs de services. Leur rémunération est composée d'un traitement de base et de primes, ils font partie d'une hiérarchie et ont droit à la retraite.


\subsection{Commis et ingénieurs : ancêtres des fonctionnaires ?}
Au XVII\up{ème} siècle, l'état se stabilise et les premiers corps des fonctionnaires émergent. 

Au XVIII\up{ème} siècle, les attributs des fonctionnaires et des intendants se confondent et leurs mode de recrutement et de rémunération fusionnent peu à peu. Par exemple, les carrières des conseillers d'Etat sont organisées selon une hiérarchie stricte, ils sont régulièrement notés. 

\section{\label{section:sec2}Révolution : égalitarisme et inégalités de traitement }

\ En 1872 est pour la première fois définie la fonction publique, comme l'ensemble des \og personnes que leur fonction rattachent d'une manière plus ou moins directe au gouvernement, qui concourent à la gestion de la chose publique \fg C'est sous la révolution qu'est consacré le terme d'\og agent public \fg, et que le nombre d'agents augmente véritablement. Il y avait en 1789 670 agents de l'administration centrale, 6 500 sont dénombrés en 1794 [Church, 1981]. Deux catégories d'agents sont distinguées : les élus (députés, juges, membres de conseils généraux) et les personnels nommés par la Convention et le Directoire. Des ensembles cohérents de la fonction publique sont fixés : le clergé, le conseil d'Etat, la cour des comptes, les préfets, la magistrature, l'administration financière, l'enseignement et la presse (des censeurs nommés par le pouvoir travaillent en effet dans les journaux). Au même moment, une \og  nouvelle définition de la rémunération des fonctionnaires émerge autour de leur nationalisation \fg{} \cite{Jourdan1991}. Le décret du 27 Novembre 1789 met fins aux gratifications de l'Ancien Régime. En 1789, on interdit les dons d'étrennes, les gratifications et cadeaux dès 1789. En 1791, l'emploi de fonctionnaire ne peut être occupé en situation de faillite. En 1793, les marchés publics sont réglementés et les fonctionnaires obligés à déclarer leur patrimoine. En 1794, le cumul des fonctions est interdit par décret. Il s'agit d'en finir avec la vénalité des offices et les déviances des agents publics de l'Ancien Régime, et de garantir l'égalité, d'affirmer que la fonction publique est un devoir à remplir de façon désintéressée avec fidélité à la loi, et plus généralement de \og régénér[er] les moeurs \fg. En particulier, les textes révolutionnaires garantissent l'égalité face à l'accès à la fonction publique, assurée par le recrutement sur concours. Les seuls critères de la compétence professionnelle et de la formation, assurée par les \og écoles centrales de la République \fg doivent être considérés pour la sélection et la rémunération des fonctionnaires.  Avec la Révolution naît également une volonté de rationalisation de la fonction publique. La notion de carrière émerge : les conditions de promotion et de traitement sur échelon sont définies. Peu de places est donnée à la rémunération au mérite, seule l'assiduité est alors récompensée. Toutefois, les fonctionnaires conservent certains avantages, comme l'indemnité de déplacement, l'exmption de la patente et des avantages en nature. En 1795, ils acquièrent le droit à la pension, financée par des prélèvements de 3 \% sur les traitements.\\
\indent Malgré les efforts pour plus d'égalité et de transparence, l'organisation de la fonction publique et le système de rémunération sont extrêmement complexes. De larges écarts de traitements existent entre l'Empire, Paris et les provinces, et entre statuts des agents (En 1810, les traitements s'étalent de 300 F à 10 000 F). Les traitements élevés sont réservés aux agents du haut de la hiérarchie, exerçant des fonctions de représentation de l'état ou des fonctions régaliennes, alors que les rémunérations des petits fonctionnaires égalent celles des ouvriers. Les indemnités se développent, avec en 1790 notamment, à l'occasion de la crise financière, la mise en place de pensions pour fonctionnaires dont l'emploi est supprimé. Toutefois, les retards de paiement des traitements et indemnités sont fréquents, et la politique inflationniste de l'époque entraîne la diminution du pouvoir d'achat des fonctionnaires. La première revalorisation des traitements pour les ajuster à l'inflation a lieu en 1794, mais les traitements ne s'améliorent réellement qu'avec la baisse des prix liée au retour à la monnaie métallique.

\section{Napoléon : entre rationalisation et favoritisme}


\ En 1795, sous le Directoire, naît une première esquisse de la grille de rémunération des fonctionnaires, prenant pour modèle la rémunération des ministères parisiens [Freyder, 2013]. Des éléments du régime indemnitaire apparaissent également : une part de rémunération au mérite peut être décidée par les chefs de service, des indemnités de mobilité et des avantages en nature peuvent être attribués. L'exercice de responsabilité est également pris en compte dans le calcul de la rémunération. Les traitements sont bimensuels, versés sous la forme d'assignats. Sous Napoléon, de nouvelles administrations sont créées, tel que le Conseil d'Etat, et les corps d'emploi des agents publics et les conditions d'avancement basées sur un système de notation sont plus précisément définis. Toutefois, l'accès aux corps et les promotions dépendent largement des faveurs accordées par le chef d'Etat ou les ministres. Le recrutement sur concours est alors limité au corps des ingénieurs. Les inégalités de traitement sont également très larges : les salaires des fonctionnaires varient de 1 à 15 à la fin de l'Empire [repère fp]). Cependant, la condition des fonctionnaires du bas de la hiérarchie s'améliore. Leur traitement égale en moyenne le salaire des commerçants et artisans, bien que les rémunérations varient largement avec la valeur de l'assignat et selon les ministères. Les fonctionnaires ont droit à la retraite. Ils ont également des perspectives d'avancement de carrière.

La première moitié du XIX\up{ème} est une période de réflexion sur le nombre des fonctionnaires, la taille de l'Etat et le bien fondé de son intervention dans un ensemble toujours plus large de domaines de la vie. C'est également l'âge d'or de la science administrative : la chambre des député discute des règles d'avancement et de traitement des fonctionnaires. Toutefois, il n'y a pas de statut de la fonction publique, ni grille permettant l'uniformisation des traitements : de larges écarts perdurent entre haut fonctionnaires et petits employés de la fonction publique. A Paris, la hiérarchie des haut fonctionnaires se clarifie. Si leur nomination par le roi est toujours la norme, les diplômes, notamment de l'école Polytechnique et de l'ENS, sont pris en compte pour l'accès à la fonction publique. Les critères fondamentaux restent la nationalité et la moralité. Des examens d'aptitudes sont mis en place, notamment pour le recrutement au sein du conseil d'Etat. L'idée de la sélection sur concours est considérée, mais entre en trop forte contradiction avec le pouvoir discrétionnaire du roi. La loi Gouvion-St-Cyr, en 1817, approfondit les définitions de l'avancement et de la carrière. Si Louis XVIII définit précisément les conditions de sélection et de d'avancement des militaires, la traduction de ces normes dans la fonction publique civile est difficile. En effet, le projet de loi Dufaure est par exemple jugé contraire à la Charte. Toutefois, en pratique, l'ancienneté détermine déjà pour partie le montant des rémunérations. Par exemple, les fonds de fin d'exercice sont attribués selon l'expérience des agents.\\

Le rémunérations sont très hétérogènes et leur composition peu codifiées. Les salaires sont effectivement composés de gages, d'appointements, d'avantages en nature divers...et chaque corps d'emploi trouve son complément de revenu et ses avantages ! Ainsi, une forme de rémunération annexe à la tâche émerge aux finances, les greffiers peuvent choisir leurs successeurs moyennant paiement, les cantonniers bénéficient de congés aménagés pour participer aux vendanges, et bien des fonctionnaires cumulent fonction publique et autres activités lucratives. Des décorations permettent de valoriser et de motiver les agents de la FP militaires, et la tradition s'étend aux fonctionnaires civil.

La stabilité de l'emploi public n'est pas encore assurée. Les emplois de la FP dépendent de la disponibilité des crédits. Cependant, en 1843, une ordonnance oblige pour la première fois l'administration à clarifier les postes et emplois de la FP. On peut y voir un premier pas vers la sécurisation des emplois publics.

Au milieu du XIX\up{ème} siècle, le suffrage universel élargit le corps électoral français de 250 000 électeurs à 9 millions. Le nombre de fonctionnaires croît de moins de 500 000 sous la seconde république à plus de 700 000. Les inégalités de traitement au sein de la fonction publique deviennent un sujet de revendication d'un syndicalisme de la fonction publique illégal mais naissant. Les petits fonctionnaires se plaignent de l'insuffisance de leurs traitement par rapport au coût de la vie, à Paris notamment, et du caractère arbitraire des nominations. Les revendications des syndicalistes sont toutefois largement ignorées par les pouvoirs en place.\\

Si les inégalités de traitement s'accroissent, les conditions de promotion internes se dessinent. On peut ainsi passer d'expéditionnaires à commis, commis principal etc.). Bien que la feuille de présence ait été instaurée en 1850, l'image du fonctionnaire absent ou occupé à des tâches futile est semble t-il largement dans les esprits des citoyens français.\\

Les évolutions en terme de salaires viennent souvent, à l'époque, des collectivités locales. C'est par exemple le préfet Haussmann qui fixe un salaire minimum par catégorie, qui définit les conditions d'un avancement automatique, ou soumis à choix du prince président. \\

\section{Troisième république XXX}

La priorité de la III\up{ème} République pour la fonction publique est son ancrage et sa légitimité auprès des français. Le champ d'action de la fonction publique s'étend encore. Le nombre de ministères augmente. Si l'on comptait 250 000 fonctionnaires en 1870, 1 million est dénombré en 1945. L'administration est structurée au sein de chaque ministère : des corpus de règles font office de statuts ministériels. Le droit administratif théorise les spécificités de la fonction publique. La jurisprudence du Conseil d'Etat joue un rôle critique dans l'organisation de l'administration et dans la clarification du statut des fonctionnaires sur la période. Notamment, il fixe les notions de grades et de corps à cette période. Les femmes rejoignent peu à peu la fonction publique. Toutefois, ces évolutions sont insuffisantes pour permettre à la FP de s'adapter au changement social et technologique. La faible productivité et les inefficiences de la FP font débat. Les années 20 voient émerger de nombreux plans de gestion et de suppressions de postes jugés inutiles. Le gouvernement Poincaré supprime notamment des centaines de tribunaux de premières instances et de sous-préfectures. Un comité supérieur des économies est créé en 1932. Dans les années trente, les obligations liées à la rémunération des fonctionnaires sont peu à peu codifiées. Par exemple, la loi du 29 Octobre 1936 restreint les possibilités de cumul d'emplois et de traitements. Ces efforts sont presque annihilés par l'instabilité ministériel, les résistances administrative et la déclaration de guerre.\\Au début de la III\up{ème} République, l'attribution des places est largement discrétionnaire. C'est seulement le développement extrême des services publics, la masse de candidats diplômés et sans doute une culture égalitariste qui permettent l'imposition du recrutement sur concours. En 1911, la loi fixe une série de règles d'avancement. Toutefois, les rémunérations sont très hétérogènes, même si un système d'échelle est mis en place après la première guerre mondiale. Le mouvement associatifs de la fonction publique, qui voient dans ces disparités de rémunérations un favoritisme \og parfois scandaleux \fg (Steeg), se développent, notamment autour de la Fédération générale des associations professionnelles des employés de l'Etat. Les syndicats de la FP ne sont alors pas reconnus, et ce jusqu'au Cartel des Gauches d'Herriot en 1924.\\

Des mesures d'exceptions sont prises durant la seconde guerre mondiale. En particulier, sous Vichy, des lois discriminatoires interdisent l'accès aux Juifs, aux personnes naturalisées et aux femmes mariées à la fonction publique. Le recrutement se fait précaire et révocable, l'idéologie autoritaire et collaborationniste domine. Un statut des fonctionnaire est créé en 1941. Un de ses mérites est de codifier les statuts particuliers des ministères et la jurisprudence du Conseil d'Etat. Toutefois, l'obéissance, la fidélité et l'autorité hiérarchique dans la fonction publique sont poussés à l'extrême dans ce statut qui ne sera appliqué que partiellement, jusqu'à son abrogation à la Libération.
 


\section{1946 : un premier statut des fonctionnaires et de leur rémunération}

Au sortir de la guerre, l'état providence, la croissance de la population et l'élargissement des domaines d'intervention de l'Etat entraînent une rapide augmentation du nombre d'agents de la fonction publique (+ 150 \% de 1948 à 1990) et ce malgré l'épuration de 5 000 fonctionnaires accusés de collaboration. A la libération, les agents publics sont consacrés comme serviteurs de l'Etat, unis par un objectif commun de reconstruction. Le 9 Octobre 1945, le gouvernement provisoire réforme la fonction publique en profondeur, à commencer par la formation des fonctionnaire (création de l'ENA), d'un corps d'administrateurs interministériels, une direction de la fonction publique, et un conseil permanent de l'administration civile en charge du dialogue social.

La loi du 19 octobre 1946 définit le premier statut général des fonctionnaires de l'Etat (le statut des fonctionnaires communaux est créé par la loi du 29 Avril 1952, et celui des agents de la fonction hospitalière par le décret du 20 Mai 1955). Le statut est créé sous l'impulsion de Maurice Thorez. Les syndicats s'y opposent, y voyant une tentative de l'administration de les annihiler. Le décret \no 48-1108 du 10 juillet 1948 définit le classement hiérarchique des grades et emplois de la fonction publique ainsi que les rémunérations associées pour les personnels civils et militaires de l'Etat relevant du régime des retraites. Un indice de classement hiérarchique est créé, allant de 100 à 800. A certains corps d'emploi est toutefois associé, par décret, un indice supérieur à 800. La valeur du point d'indice, le nombre d'échelons de chaque grades sont fixés par arrêtés ministériels. Quatre catégories sont instituées, correspondants à quatre intervalles d'indices de rémunération. Au traitement de base s'ajoutent une indemnité résidentielle, un supplément pour charge familiale et des primes de rendements pour les individus ou les groupes qui dépassent les objectifs qui leur sont assignés ou qui effectuent des tâches difficiles. Les conditions d'accès à la fonction publique sont définies. Un salaire minimum de l'agent est fixé à 120 \% du minimum vital, fixé par décret tous les deux ans. Une part de rémunération au mérite est instituée, déterminée selon une grille de notation. Par exemple, le décret du 6 Août 1945 stipule que ces primes "variables et personnelles sont attribuées par décision du ministre des finances, compte tenu de la valeur et de l'action de chacun des agents". Les primes au mérite sont alors plafonnées à 18\% du traitement le plus élevé du grade et des montants maximum par catégorie sont également déterminés. \\
Le recrutement par concours interne et externe s'impose. Une note et une appréciation annuelles sont mises en place afin de complémenter l'avancement à l'ancienneté. Ce premier statut exclue les collectivités locales, car la puissante CGT s'oppose à leur inclusion, craignant l'étatisation de la fonction publique territoriale. Les trois versants de la fonction publique ne sont pas définis, même si un statut de 1952 étend certaines garanties aux personnels des communes.\\
La mise en place du statut s'étale jusqu'à 1950, en raison notamment de son coût élevé. Les premières réformes du statuts apparaissent dès le milieu des années cinquante, avec en particulier le reclassement des fonctionnaires hors échelles en 1955-1957. La hiérarchie des traitements tend alors à s'écraser, dû notamment aux nécessaires revalorisations des bas salaires pour compenser l'inflation. On découvre les rigidité de la grille indiciaire et notamment les effets de demande en cascades, \og reconventionnelles \fg. En effet, le reclassement d'une partie des corps d'emploi dans l'échelle indiciaire aboutit immanquablement à des demandes de reclassement d'autres corps. Les DOM commencent à bénéficier d'avantages spécifiques tels que les majorations et les primes d'éloignement. Les réformes sont majoritairement contraintes par l'instabilité ministérielle et par les résistances de l'administration centrale à la déconcentration.

\begin{tab}
	[5cm]                                                                           % Argument 1 [8 cm] : largeur du tableau (mettre 15 cm comme taille maximale)
	{
		Catégories et indices de rémunération associés dans la FPE en 1946
	}
	{
		\begin{tabular}{lcc}
			\toprule     
Catégorie A & 225 - 800 \\
Catégorie B & 185 - 360 \\
Catégorie C & 130 - 250 \\
Catégorie D & 100 - 185 \\

			\bottomrule
		\end{tabular}
	}
	
\end{tab}

\newpage

La loi \no 72-662 du 13 juillet 1972 créé le premier statut général des militaires, qui a été modifiée par la loi \no 2005-270 du 24 mars 2005. Le statut général "s'applique aux militaires de carrière, aux militaires servant en vertu d'un contrat, aux militaires réservistes qui exercent une activité au titre d'un engagement à servir dans la réserve opérationnelle ou au titre de la disponibilité et aux fonctionnaires en détachement qui exercent, en qualité de militaires, certaines fonctions spécifiques nécessaires aux forces armées."

\section{Années 1970 : négocier la modernisation}

A la suite du mouvement de Mai 68, et sans doute inspirées des accords de Grenelle dans le secteur privé, les négociations salariales dans la fonction publique d'Etat naissent dans les années 1970.
En particulier, le Protocole Oudinot, signé en 1968 est un premier accord sur les rémunérations et le temps de travail. Les traitements sont relevés, la catégorie C est revalorisée, on bénéficie de plus de congés et d'une semaine de 5 jours dans la fonction publique. Toutefois, ces mesures excluent largement certains ensembles d'agents, ceux des entreprises publiques notamment, qui entrent en grève. 1968 peut être considéré comme l'acte de naissance des négociations salariales dans la fonction publique d'Etat, qui ont lieu annuellement à partir de 1970. Les autres versants de la fonction publique rejoignent le dialogue lorsque les grilles de rémunération sont harmonisées dans les années 1980, ce qui rend les rémunération comparables. En 1970, une circulaire fonde les modalités du syndicalisme de la FP. L'équilibre entre participation démocratique des agents de la fonction publique au dialogue social et forte hiérarchie dans la FP penche alors vers l'inclusion des fonctionnaires dans la négociation. Peu à peu, de nouveaux domaines de revendications s'ouvrent, comme préconisé dans le rapport Guilhamon de 1988. La formation devient notamment un enjeux de négociation. Toutefois, les revendications salariales restent centrales avec notamment la remise en cause des traitements et de l'indépendance des hauts fonctionnaires vis-à-vis des pouvoirs politiques.

%http://www.vie-publique.fr/documents-vp/dsfonctionpublique_ena2004.pdf

\section{Loi de 1983 : les fondations du statut général}

Entre 1981 et 1983, on compte 200 000 nouveau fonctionnaires en France. La réforme du droit de la FP devient prioritaire sous Mittérand : les traitements sont revalorisés de 10.5 \% en 1982 et on promet leur progression en fonction de l'inflation. L'ENA est réformée, de grandes vagues de nationalisation font des fonctionnaires 1/3 de la population active, on titularise et oeuvre à la décentralisation en 1982. 

C'est la loi dite Le Pors du 13 Juillet 1983 qui porte droits et obligations des fonctionnaires et qui  fonde le statut général des fonctionnaires actuels. Les principes d'égalité (concours, égalité des traitements au sein des corps et dans les trois versants), d'indépendance (principe de carrière, différenciation du grade et de l'emploi) et de citoyenneté du fonctionnaire qui bénéficie des mêmes droits et obligations que le reste de la population.  En particulier, les trois versants de la fonction publique sont définit. La loi du 26 janvier 1984 porte statut général de la fonction publique territoriale. La loi du 9 janvier 1986 définit les dispositions relatives à la fonction publique hospitalière.

Depuis les années 1980, la croissance de l'emploi public est essentiellement due au développement de la FPT. En effet, les vagues de décentralisation permettent le transfert de missions du centre vers les collectivités locales, et le statut de 1984 rend les postes plus attractifs en définissant la progression salariale des agents de la FPT. Le statut permet également de fixer les règles de recrutement, et de relever les exigences en termes de qualification des agents de la FPT. Les salaires de la FPT, encore aujourd'hui, n'égalent pas ceux de la FPE. En effet, les femmes et les agents de catégorie C sont particulièrement sur représentés dans la FPT et le recours au temps partiels y est plus important.

Les années 80 marquent également l'apparition du critère de productivité et de performance dans la fonction publique, que des rémunérations adaptées peuvent aider à satisfaire. Cette tendance culmine avec la Loi Organique relative aux Lois de Finances en 2001, qui fournit un cadre de gestion des fonctionnaires, affirme la nécessité de l'évaluation des politiques publiques. Le ralentissement de la croissance à la fin des trente Glorieuses, la réévaluation périodique des grilles, notamment pour suivre l'augmentation des qualifications des fonctionnaires, contraignent l'évolution des salaires de la fonction publique. De plus, les négociations salariales, prises en étaux entre contraintes budgétaires et rigidités due à la largesse du champ d'application, s'essoufflent dans la période récente, sauf dans la FPH, qui signe de nombreux accords depuis les années 2000.




\ifx\isEmbedded\undefined
\newpage
\bibliography{../../Biblio/biblio-rapport}
\end{document}
\else \fi
