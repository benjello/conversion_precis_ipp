
The standard French payslip is often difficult to read, because of the multiplicity of elements which appear on it. Aside from in-kind compensation and employers' participation (public transportation, luncheon vouchers, contingent fund, etc.), are also mentioned the different contributions and payroll taxes based on earnings. For private-sector employees, we can generally find about twenty lines corresponding to as many contributions. The present chapter looks over the contributions and payroll taxes based on private-sector earnings in 2014, as well as contributions based on civil servants' pay. If some contributions are jointly paid by  the private and public sectors, other contributions are, however, specific to the private-sector. And, \emph{a contrario}, there are some contributions meant to fund social insurance schemes specific to the public-sector.

Contributions specific to special schemes are not presented here. Contribution rates particular to the special schemes can be found in the Excel file \emph{Bar\`emes IPP - Pr\'el\`evements sociaux} (\citet{ipp_baremes_prelevements}). Informations available in this chapter can be used as a useful complement to the use of the online payslip simulator developed by the IPP\footnote{Available on the IPP website: \texttt{http://www.ipp.eu/fr/outils/comprendre-son-bulletin-de-salaire/}.}.

We will start by giving a detailed presentation of the Social security contributions funding the health insurance, the old-age insurance, the family allowance branch and the benefits resulting from a work accident or a professional disease. Then, a review of unemployment insurance contributions as well as contributions funding the complementary pension schemes, often assimilated to social security contributions, will be provided. After having presented the flat-rate income taxes, CSG and CRDS, funding the different branches of he Social security scheme, we will give some insights on the recently implemented contributions and taxes aiming at taxing extra elements of earnings exempted from social security contributions. We will then describe contributions specific to the public sector, before listing in details the payroll taxes (PT), directed to the general State budget or to specific socio-economic action funds (apprenticeship, building, transports).


\section{Social contributions}

\subsection{Social security contributions}


%%%% MMID

\subsubsection{Sickness-maternity-disability-death insurance contributions (SMDD)}

\subsubsection*{SMDD contributions of the general scheme}

The first branch of the Social security aims at covering sickness, maternity, disability and death risks. Since 1980, SMDD contributions (employees' and employers') are uncapped, and have increased to a significant extent until 1997. In 1998, the employees' contribution, however, has been brought from 5.50~\% down to 0.75~\% as the funding of the Social security scheme had been transfered from social security contributions to CSG. The employer's part is at a 12.80~\% rate since 1992.

Individuals affiliated to the Alsace-Moselle local scheme for the health insurance benefit from additional benefits in return of an extra SMDD contribution paid as a supplement of the contribution the French general scheme.

\subsubsection*{SMDD for tenured civil servants}

Contributions to acquire a health insurance paid by employers and employees in the public sector differ from the ones transfered to the National health insurance fund (NHIF) by companies and employees in the private sector. Contributions are directly paid to the mutual health insurances for civil servants (\textit{Mutuelle autonome g\'en\'erale de l'\'education}, MAGE, \textit{Mutuelle g\'en\'erale de l'\'education nationale}, MGEN), which have the same function as the local health insurance funds. The contribution, uncapped, is paid at a uniform rate (whatever the mutual health insurance chosen). Since 1984, this rate prevailing in the State civil service (SCS) differs both from the one used for the Hospital civil service (HCS) and from the one used in the Territory civil service (TCS). Between 1967 - date when the contribution has been created -- and 1984, the rates were the same in the three civil services.\newline

\texttt{Legislative reference:} Decree 67-850 of 09/30/1967 (OJ of 10/03/1967); Decree 83-1196 of 12/30/1983 (OJ of 12/31/1983), art. 5. \newline

\underline{Rate:} Initially, the statutory sickness contribution burden was borne both be employers and employees in the public sector. The contribution could be subdivided into a contribution on earnings below the Social security ceiling (repealed in 1984), and a contribution computed on the whole pay. Since January 1\up{st}, 1998, only the contribution on the whole pay paid by the employer has been maintained. Its rate in 2014 is equal to 11.50~\% for civil servants in the HCS and in the local and regional authorities, and to 9.70~\% for the SCS employees.\newline

\underline{Tax base:} For the three civil services, the contribution tax base is made of the sum of the base pay (\textit{traitement indiciaire brut}, TIB) and the rank-related bonus (\textit{nouvelle bonification indiciaire}, NBI).\newline

\texttt{Legislative references:}  Decree 97-1249 of 12/29/1997 (OJ of 12/30/1997), art. 8.\newline


\subsubsection*{SMDD contribution for non-permanent civil servants}

Non-permanent civil servants contribute to the general scheme for the sickness insurance. The contribution tax base, is made of the totality of earnings, that is to say the sum of the base pay (TBI), of a bonus related to the number of children of the agent, of the residence compensation, of bonuses, as well as in kind compensation. The contribution rate is equal to 0.75~\% in 2014, and is entirely paid by employees.

\subsubsection*{SMDD contribution for non-farming self-employed workers}

The self-employed health insurance is managed by the Self-employed social scheme (\textit{R\'egime Social des Ind\'ependants}, RSI) since July 1\up{st} 2009, both for liberal professions and for craftsmen, traders, and manufacturers. Nonetheless, the emission, the sending of notices of appeal, the receipt, the amicable recovery, as well as the inspection functions are delegated to the Urssaf.

All workers belonging to these categories -- more precisely defined in the first section -- have to register at the self-employed scheme. The affiliation is also mandatory for pensioners and disabled persons formerly self-employed, for surviving spouses benefiting from a survivor's or widower's pension and being more than 55 years-old, and, in some cases, for spouses associated to the craft or trade activity, for unique associate in non farming \textit{EURL} (legal type of company), and for authorized representative for the social protection of adult persons. This list does not include self-employed workers who are already covered by a legal health insurance scheme as assignees, or medical practitioners or auxiliaries authorized by the Social security and benefiting from a particular health insurance scheme.\newline

\textbf{Contributions of liberal professions}\newline

Since April 1\up{st}, 1970, liberal professions contribute for the funding of the sickness-maternity-disability-death insurance. Although the contribution was capped initially, the ceiling has been lifted on January 1\up{up}, 2013, to be paid through a unique 6.5~\% rate on the totality of earnings, as for craftsmen and traders. Since January 1\up{st}, 1998, insured pensioners do not contribute anymore to the SMDD branch.
On the whole, the SMDD rate has increased until 1995, then has declined in 1998, to remain roughly stable after.

The RSI also requires the payment of a minimal contribution if earnings are lower than 40~\% of the Social security ceiling: its amount lies between  659\euro ~ (for earnings equal or lower than zero) and 976\euro ~. \newline

\texttt{Legislative references:} Art. 2 of decree \no 2012-1551 of December 28\up{th}, 2012 (OJ of December 30\up{th}, 2012).\newline

\textbf{Contributions of craftsmen, traders and manufacturers}\newline

The evolution of the SMDD contribution due by craftsmen, traders and manufacturers was similar to the one due by liberal professions, apart from the fact that for craftsmen, traders and manufacturers, there is an additional SMDD contribution equivalent to 0.70~\% of earnings under five SSC for the funding of sick leave daily allowances.

A minimal contribution has to be paid for the funding of sickness insurance in case earnings are lower than 10~\% SSC, and is computed on this tax base applying the rate in effect. In the case of the sick leave daily allowances, the earnings threshold used as tax base for the computation of the minimal contribution is equal to 40~\% of the SSC. 

Special schemes which include reductions in contributions have been implemented for some special areas such as urban duty free zones (\textit{zones franches urbaines}), urban revitalization zones (\textit{zones de redynamisation urbaine}), overseas departments, and for micro-enterprises.\newline

\texttt{Legislative references:} Art. 2 of decree \no 2012-1551 of December 28\up{th}, 2012 (OJ of December 30\up{th}, 2012); Art. 20 of decree \no 2004-565 ; Art. L 756-4 and L 756-5 of the SSC.


\subsubsection{Solidarity and autonomy contributions}

\subsubsection*{The Solidarity-Autonomy contribution (\textit{La contribution solidarit\'e-autonomie}, CSA)}

In 2004, a new contribution, sometimes aggregated with the sickness insurance contribution on the payslip, has been introduced to finance the National solidarity fund for the autonomy of elderly and disabled persons (\textit{Caisse nationale de solidarit\'e pour l'autonomie des personnes \^ag\'ees et des personnes handicap\'ees}, CNSA). The Solidarity-Autonomy contribution is paid by employers at a 0.30~\% rate in 2014, both on private-sector and public-sector earnings.


\subsubsection*{The Additional Solidarity-Autonomy contribution (\textit{Contribution Additionnelle Solidarit\'e-Autonomie}, CASA)}


The CASA has been introduced in April 1\up{st}, 2013 to make retired people contribution to the funding of dependency policies. Retirement, early-retirement and disability pensions are subjected to a 0.3~\% contribution.  \\
There are several exemption conditions from the payment of this contribution:
\begin{itemize}
\item Individuals exempted from income tax on year N for income of year N-1 are exempted from the CASE payment on year N+1;
\item Individuals exempted from the CSG or benefiting from a reduced rate \footnote{See the CSG section.} are not liable for the payment of the CASA;
\item When old-age or disability benefits are converted into capital with a unique payment (which happens when amounts are low), the CASA is already included in the 15.5~\% levy on capital income;
\item ASPA recipients (equivalent to a minimum income for elderly people) are exempted. \\
\end{itemize}

\texttt{Legislative references:} Art. 17 2013 Social security finance act (SSFA) (Law \no 2012-1404 of December 3\up{rd}, 2012).



%%%% AT-MP %%%%%%%%%%%%%
%%%%%%%%%%%%%%%%%%%%%%%
\subsubsection{Work accidents-professional diseases contributions (WA--PD)}

\subsubsection*{WA--PD contributions of the general scheme}

The work accidents--professional diseases contribution is due by employers on the totality of earnings (before 1991, the tax base was capped at 1 SSC). As this contribution has the double objective of funding the inability benefits following a work accident or a professional disease and of providing incentives for companies to reduce the occupational hazard, its rate is determined according to complex rules. It depends both on a hazard scale specific to the type of activity carried out by the employee and on accidents that have happened within the company, and their seriousness. If the companies of 150 employees and more benefit from an individual pricing system, a collective pricing system is applied in companies of less than 20 employees, whereas, for companies between 20 and 149 employees, a mixed pricing system is used. Thus, the rate in effect each year in each company is notified by the National health insurance fund for wage-earners (NHIF-WE), or by the Regional health insurance fund (RHIF) for companies set up in the Ile-de-France region.


\subsubsection*{WA-PD contributions in the public sector}

While contractual civil servants are covered by the general scheme for the WA--PD risk, tenured civil servants get their insurance from the State pension service (\textit{Service des retraites de l'Etat}, SRE) for the State civil service, and from the National fund for local and regional authorities' agents, for the Hospital and Territory civil services, two funds which also manage the old-age contributions and the tenured agents' pensions.

\subsubsection*{WA--PD contributions for non-farming self-employed}

Non-farming self-employed workers are not covered for the WA-PD risk as part of the mandatory social security scheme. However, they can register to a voluntary insurance scheme, and then pay optional contributions that are deductible from the earnings to compute the taxable income. 


%%%%%%%%%%% FAMILLE %%%%%%

\subsubsection{The family allowances contributions}

Since its creation in 1946, the statutory liability of the contribution designed to fund the family branch of the Social security scheme has always been on employers. Initially capped, the family contribution tax base corresponds to the totality of earnings since 1990. Unlike other social security contributions, the family contribution rate has followed a declining trend, going from 7~\% in 1990 to 5.40~\% in 1991, as the revenue from the CSG (established in 1991) has been entirely assigned to the funding of the family branch the first years it existed (from 1991 to 1993).

Besides, the first reductions in employers' contributions for low earnings introduced in 1993 consisted in a partial or total exemption from the family contribution. The contribution rate decreased again on January 1\up{st}, 2014, and the government announced a progressive removal of the contribution as part of a reduction in employers' contributions policy~(see chapter on exemptions).

Private-sector as well as public-sector employees, contractual or tenured civil servants, contribute to the family branch of the Social security scheme. The National family allowances fund manages contributions and benefits for private and public sectors.

\subsubsection{The family allowances contributions for non-farming self-employed workers}

Craftsmen, traders and manufacturers, as well as liberal professions, pay a 5.25~\% contribution on the whole earnings to fund the family branch of the Social security scheme. This contribution may be supplemented with the family contribution due by self-employed workers as employers, described above.

Before 1974, this contribution was due on a fixed-amount basis, determined according to the self-employed worker's earnings. The rate in effect today is the same as in the general scheme, the harmonization has been however gradual.

\texttt{Legislative references:} Decree \no 2013-1290 of December 27\up{th}, 2013 (OJ of 12/31/2013).




%%%%%%%%% RETRAITE DE BASE %%%%%%%%%%%%%%

\subsubsection{Old-age insurance contributions (base pension)}

\subsubsection*{Contributions of the general scheme}

Composed of an employees' and an employers' part, the contribution funding the old-age insurance (base pension) has undergone some changes regarding its tax base: the ceiling has been partly removed in 1991 and 2004. Private-sector employees, but also contractual workers in the public-sector, contribute to the general scheme for their basic old-age insurance benefits.

When the CSG has been instituted in 1991, to neutralize the impact of a rise in contributions for employees, the employees' old-age contribution has been diminished from 7.6~\% to 6.55~\%. As the old-age contribution is capped whereas the CSG is not, a fixed-amount reduction in old-age contribution equivalent to 42 FRF per month has been introduced at the same time, so that earnings above one SSC also benefit from the counterbalancing decrease in the old-age contribution. To compensate for the diminution in public revenue induced by the fixed-amount reduction in old-age contribution, an uncapped employers' contribution of 1.60~\% has been implemented in 1991. It should be noticed however that the introduction of the CSG has been neutral for employers' contributions as the institution of the uncapped old-age contribution has only counterbalanced the decrease in employers' contributions for the family branch, from 7~\% to 5.40~\% (see above).

To cope with the deterioration of the financial situation of the old-age branch observed since the 2000's, the old-age contribution rates have been raised; moreover, a decree following the 2013 pension reform planned an gradual increase in the rate between 2014 and 2016.

\texttt{Legislative references:} Law 91-73 of 01/18/1991 (OJ of 01/19/1991); Decree 2013-1290 of 12/27/2013 (OJ of 12/31/2013).


\subsubsection*{The contributions of tenured civil servants in the State civil service}

\textbf{Deduction on Pension (DP)}\newline

Instituted in 1925, the deduction on pension (DP) is a contribution paid only by tenured civil servants in the State public service to fund the basic pension benefits.

The tax base of this contribution, formally borne by the civil servant, is made of the sum of the base pay (\textit{Traitement indiciaire brut}, TIB) and the rank-related bonus (\textit{Nouvelle bonification indiciaire}, NIB). Unlike what prevails for contractual workers of the public sector regarding the old-age contribution tax base of the general scheme, the deduction on pension tax base does not include bonus and benefits.\newline

\underline{Tax base:}
\begin{equation*}
Tax~base~=~TBI + NBI
\end{equation*}

Initially set at 6~\%, the deduction on pension rate gradually increased, until reaching 9.14~\% in 2014. A decree published in the end of 2013 plans a gradual increase until 2020, date when it should be equal to 10.86~\%.\newline

\texttt{Legislative references:} Law of 04/14/1924 (OJ of 04/15/1924); Decree 2013-1290 of 12/27/2013 amending the old-age insurance contribution rate of different social security schemes, and the family allowances contribution rate (OJ of 12/31/2013), art. 8.\newline

\textbf{State's contribution on civil and military pay}\newline

The State must pay a contribution designed to fund the basic old-age benefits for civilian and military agents, as well as the temporary disability benefits --TDB -- (\textit{allocations temporaires d'invalidit\'e}). The contribution rate is set each year by decree: it is computed such that the State manages to strike a balance between its spending in terms of pensions paid to retired agents and its revenue through contributions. As the retirement conditions are more favorable to soldiers, the contribution on their pay is at a higher rate than the contribution rate on civilian's pay (126.07~\% versus 74.28~\% in 2014). As the number of retired persons in the State civil service increased since the 2000's, the contribution followed the same increasing path the last ten years (the rate was equal to 56.80~\% in 2004 for civilians' pensions). Finally, both for civilians' and soldiers', the State must contribution at a 0.32~\% rate to finance the temporary disability benefits (TDB).\newline

\underline{Tax base:}
\begin{equation*}
Tax~base~=~TBI + NBI
\end{equation*}


These contributions have been created in 2006: before this date, an implicit contribution rate was computed for the employer State to finance the old-age benefits on the General Budget. To make the civil servants' social protection scheme's accounts more transparent, and to encourage a sound management, the Budget framework Law \footnote{The Budget framework law, BFL, is a legislative text determining the legal framework of the finance act voted each year.} (\textit{Loi organique relative aux lois de finance}) made mandatory the setting and announcement of a explicit contribution rate in the Finance Act.

Since 2009, the State pensions service (SPS), connected to the Public finance general directorate (\textit{Direction g\'en\'erale des finances publiques}, PFGD), is in charge of the financial and administrative management of the State civil service pensions.\newline

\texttt{Legislative references:} Decree 2009-1052 of 08/26/2009 (OJ of 08/29/2009) creating the State pensions service.

\subsubsection*{Contributions to the National pension fund for local and regional authorities' agents}

Established in 1947 and connected to the Deposits and Consignments fund (\textit{Caisse des D\'ep\^ots et des Consignations}, a public establishment in charge of receiving and preserving funds transfered, in some cases by private individuals, or by public establishments), the National pension fund for local and regional authorities' agents (\textit{Caisse nationale des retraites des agents des collectivit\'es locales}) manages base contributions and pensions for civil servants of the Territory civil service and of the State civil service.  This mandatory points-based scheme is funded both by contributions nominally paid by employers and by contributions nominally paid by the civil servant.

Between 1947 and 1983, the NPFLRA contribution rate for civil servants was the same as the deduction on pension one paid on the State civil service agents' pay.\newline

\underline{Tax base:}
\begin{equation*}
Tax~base~=~TBI + NBI
\end{equation*}

Moreover, civil servants affiliated to the NPFLRA must pay a contribution to fund the insurance for the work accidents--professional diseases risk for State and Hospital civil services, allowing the casualties to benefit from the temporary disability benefits of local and regional authorities' agents. Initially optional when it has been created in 1961, it became mandatory in 1969. Its rate was equal to 0.40~\% of gross pay and back pay, excluding any benefit, bonus, compensation, or non-permanent rank-related bonus (NBI).

\subsubsection*{Contributions of the non-farming self-employed workers}

The old-age insurance scheme for non-farming self-employed workers has been created in 1948, along with other branches of the self-employed social protection scheme. Texts determining the rules of the old-age insurance set forth a precise definition of workers covered by the non-farming self-employed scheme, which is then used as a reference to the delimit the scope of other social protection branches.

Four categories of self-employed workers can be distinguished: liberal professionals outside lawyers, lawyers, traders and manufacturers, and craftsmen. The basic old-age insurance is mandatory for all of these categories whereas the complementary pension schemes are either optional or mandatory, and some supplementary insurances can be taken out.

We can notice a gradual convergence of the scheme towards the wage-earners scheme, more pronounced for craftsmen, traders and manufacturers.\newline

\texttt{Legislative references:} Law \no 48-101 of 01/17/1948; Art. L 622-3, L 622-4, L 622-5, L 622-7 of the SSC.\newline

\textbf{Contributions of liberal professions}\newline

The basic pension scheme jointly covers all the liberal professions outside lawyers, whereas complementary pension schemes and disability-death insurance are specific to each occupational section.\newline

When the system has been created, the old-age fund financing rested upon a fixed-amount contribution specific to each liberal profession. In 1993, a proportional contribution paid by all the liberal occupational sections has been added. The structure in charge of uniting all the different sections and of administering the basic pension scheme is the National old-age insurance fund for liberal professions (NOIFLP, \textit{Caisse Nationale d'Assurance Vieillesse des Professions Lib\'erales}). The group gathering the NOIFLP and the ten pension funds for liberal professions forms the Autonomous old-age insurance organization for liberal professions (AOIOLP, \textit{Organisation autonome d'assurance vieillesse des professions lib\'erales}). As an example, the annual fixed-amount contribution was equal, in 2003, to 2000\euro ~ for law and public legal officers, and legal companies, 1580\euro ~ for doctors, 2000\euro ~ for dental surgeons, and 1708\euro ~ for pharmacists. The fixed-amount contribution is repealed in 2004, and from this date, the funding only rests upon a capped proportional contribution: so far, it was equal to 10.1~\% of earnings below 0.85 SSC, and 1.87~\% of earnings between 0.85 and 5 SSC. In 2015, the scheme has been reformed, by amending the threshold defining the two brackets of earnings and introducing two different rates from the first euro earned: thus, liberal professional must pay 8.23~\% of their earnings below one SSC, and 1.87~\% of their earnings below 5 SSC. Then, the rate remains unchanged for earnings between 0 and 0.85 SSC (as 8.23 + 1.87 = 10.10), but liberal professional earning between 0.85 and 2 SSC have their contributions and entitlement significantly increased.

In addition to the base contribution, doctors authorized by the Social security organization are liable for the payment of an extra contribution to fund the early job termination scheme.

Contributions are due, except in case of adjournment or staggering, from the first day of the calendar quarter following the start of the activity, until the last day of the calendar quarter when the employment termination occurs. These contributions open entitlements to a certain number of points.\newline

Lawyers constitute a peculiar sub-category among the liberal professions regarding old-age insurance: they pay their contributions to the National fund of French Bars (NFFB, \textit{Caisse nationale des barreaux fran\c{c}ais}). The contribution to the funding of the base scheme is made of plea rights, a fixed-amount contribution set each year, and a proportional contribution based on earnings.\newline

\texttt{Legislative references:} Art. 19 of decree \no 2014-1637 of December 26, 2014 (OJ of 12/28/2014); Art. D 642-3 and L 642-1 of the SSC.\newline

\textbf{Contributions of craftsmen, traders and manufacturers}

Until 2013, the effective rate was equal to the sum of the old-age contribution rates in the general scheme, whatever they are capped or uncapped, paid by employers or employees. The contribution rate to fund the base pension scheme of craftsmen, traders and manufacturers corresponds to 17.05~\% of earnings under one SSC in 2015. A 0.20~\% additional contribution on the whole earnings has been introduced in 2014, and has been brought to 0.35~\% in 2015

\texttt{Legislative references:} Art. 5 of decree \no 2012-847 of 07/02/2012 (OJ of 07/03/2012); Decree \no 2013-1290 of 12/27/2013 (OJ of 12/31/2013); Law \no 2013-1203 of 12/23/2013 (2014 Social security finance act, OJ of 12/24/2013) ; Art. D 633-3 of the SSC


%%%%%%%%%%%%%%%%%%%%%%%%%%%%%%%%%%%%%%%%%%%%%%%%%%%%%%%%%%%%%%%%%%%%%%%%%%%%%%%%%%%%%%%%%%
%%%
\subsection{Unemployment insurance contributions}


\subsubsection*{Unemployment contributions of the general scheme}

The unemployment contribution tax bas is the same as the social security contributions tax base, aside from the fact that wage-earners above 65 years-old are exempted from unemployment insurance contribution.

Bounded to four social security ceilings, the unemployment contribution is paid at a uniform rate since 2002. In 2014, the employer's rate was equal to 2.40~\% and the employee's rate to 4.00~\% of gross earnings.

Historically, the contribution rate has increased from the creation of the insurance scheme in 1958, to 2003, and then stabilized. The quasi-doubling of the contribution rate in 1984 has been introduced as a response to the continuous rise in unemployment, in a context of crisis for the Social security general scheme. In 1993, after several years of low economic growth, the rate has been substantially increased again to compensate for the very low weak inflation and then the very low increase in the unemployment contribution tax base.


\subsubsection*{Unemployment contribution of non-farming self-employed workers}

Self-employed workers affiliated to the non-farming non wage-earning scheme for the sickness and old-age risks are not covered for the unemployment risk: thus, they are not liable for the payment of a contribution nor they can receive any benefits as part of the unemployment insurance.

However, they can subscribe to an unemployment insurance contract on a voluntary basis, which can take three forms: an insurance contract against job separation or unemployment guarantee in a private scheme, a collective insurance contract reserved for members of an association, or an individual contract. According to the type of contract chosen, the contribution differs.

Let us take the example of the "group insurance contracts", ruled by the Madelin law: these contracts are purchased from insurance companies and require the regular payment of a contribution whose amount is agreed upon in the contract. These contributions are deductible from taxable income up the highest limit among these two: 1.875~\% of taxable earnings below 8 SSC; 2.5~\% of the SSC. Companies offering this type of contract must conform to different rules: we can mention the Managers' social guarantee (\textit{Garantie sociale des chefs d'entreprise}), the Association for the protection of independent managers (\textit{Association pour la protection des patrons ind\'ependants}), April Insurances, and Cameic.\newline

\texttt{Legislative references:} Law \no 94-126 of February 11\up{th}, 1994 (OJ of 02/13/1994).

\subsubsection*{AGS contribution}

In addition to the unemployment insurance, employers must contribute to the funding of an employers' scheme designed to finance a fund for the guarantee of wages, which secures the payment of wages in case of tax adjustment or compulsory liquidation. Initially called \textit{Fonds national de garantie des salaires} (FNGS), it has been renamed \textit{Association pour la gestion du r\'egime de garantie des cr\'eances des salari\'es} (AGS). The employer's participation translates into a contribution paid on earnings under four SSC, at a 0.30~\% rate in 2014.


\subsubsection*{Contribution to solidarity funds}

In 1982, faced with a high and persistent unemployment rate, a public establishment has been created to fund benefits for unemployed workers. The Solidarity fund (\textit{Fonds de solidarit\'e}) funding rests upon a solidarity levy (also called exceptional solidarity contribution) at a 1~\% rate.

The solidarity contribution tax base corresponds to the sum of the base pay (TBI), the grade-related bonus and other additional revenues minus the employee's social security contributions ($cot_e$). The contribution tax base is delimited by an upper threshold (equal to four SSC) and a liability threshold. Between 1982 and 1988, the value of this threshold was defined in net monthly Francs; since 1989, it is the increased benchmark index (a number related to the civil servant's grade) which defines the liability of the agent. Since January 1\up{st}, 2013, are only liable the agents whose increased benchmark index is greater or equal to 309.\newline

\underline{Tax base:}
\begin{equation*}
Tax~base~=~Min[4 SSC, (TIB + NBI - cot_e)]~if~Index~number~> 309
\end{equation*}

%% TIB ou TDI?
This contribution both affects the tenured and non-permanent agents of the public sector. Unlike the unemployment insurance private-sector employees contribute to, it is not a social security contribution as it does not directly open any social entitlement.


%%%%%%%%%%%% RET COMPLEMENTAIRE %%%%%%%%%%%%%%%%%%%%%%%


\subsection{The complementary pension scheme}

\subsubsection{The private-sector wage earners' complementary pension scheme.}

The complementary pension contributions are mandatory for employees, but differ according to whether or not the employee has an executive status: a non-executive worker must contribute to a scheme affiliated to Arrco (\textit{Association pour le r\'egime de retraite compl\'ementaire des salari\'es}, Association for the employees' complementary pension scheme), whereas an executive worker contributes to a scheme affiliated to Agirc (\textit{Association g\'en\'erale des institutions de retraite des cadres}, General association for executive workers' pension scheme), and, since 1972, also to an institution affiliated to the Arrco for the part of his earnings below one SSC. In addition to contributing to these two complementary pension schemes, employees must pay contributions to two funds administered by the complementary pension schemes, the ASF and the APEC.

The contribution tax base is the same as the one for social security contribution since 1996 in most cases. However, whereas social security contributions can be paid on a fixed-amount tax base for some particular occupations (artists, models, etc.), contributions to Agirc and Arrco are always based on the employee's effective earnings. Moreover, employees can keep on paying complementary pension contributions on a fictional wage in case of expatriation or inactivity (mobility leave, parental leave, reduction in earnings or working time, etc.), which is impossible for social security contributions.


\subsubsection*{Contributions defining complementary pension entitlements}

Both for executives and non-executives, the principle of complementary pension contributions, with an employer's and an employee's part, is the same: contributions paid to Agirc and Arrco allow to determine the amount of entitlements to a complementary pension acquired, as a function of a parameter called the point value, and whose value is defined such that contributions paid by affiliated people and pensions paid to retired recipients are balanced\footnote{For more information on the use of contractual and effective rates in the determination of the complementary pension amount, please refer to \citet{ipp_retraites}}.

In the 1940's and the 1950's, right after their implementation, the complementary pension schemes were entirely optional and the contribution rates varied from one institution to the other. In the 1970's, the subscription to a complementary pension scheme has become mandatory (1972 for Arrco, 1974 for Agirc). Contribution rates were still varying, as companies were free to set their own rate between an upper and lower bound. Finally, it is in 1983 for the Agirc, and 1993 for the Arrco that a unique rate per bracket has been (gradually) implemented. From this date on, any new enterprise must compute its contributions to Agirc and Arcco according to a unique mandatory rate; however, older companies having defined their own rate before 1983 or 1993 are allowed to keep it.

In 1971, a parameter called adjustment rate has been introduced in order to increase contributions without mechanically increasing entitlements acquired by insured persons retiring (nor decreasing the point's value): the effective contribution rate is indeed computed as the product of the adjustment rate (reaching 125~\% as soon as 1996) and the contractual contribution rate, whereas it is the latter that is used in computation of the pension entitlements.

\begin{equation}\nonumber
\tau_e=\tau_c \times \tau_a
\end{equation}

where $\tau_e$ corresponds to the effective rate, $\tau_c$ the contractual rate and $\tau_a$ the adjustment rate.

The Agirc and the Arrco have experienced several reforms in the 1990's to cope with their financial difficulties, due to the slowing down of the wage inflation and to the demographical evolutions (the rise in life expectancy having made the number of years retired increase, then triggering a rise in pensions paid to insured persons). The 1993 and 1994 agreements gave rise to an important increase in contribution rates and removed the possibility for employers to contribute over the mandatory rate (up to a 8~\% limit on the first bracket or earnings, and 16~\% on the second one). On the other hand, complementary pension schemes tried to contain their spending by reducing the value of the points granted for free according to the number of children, and, as a general rule, by reducing the contributions profitability (increase in the adjustment rate and decrease in the point's value).

The 1996 agreement introduced a financial solidarity between Agirc and Arrco, as Agirc was facing more difficulties than Arrco, because the SSC was increasing more rapidly than executives' wages\footnote{For more details on the implementation and operating of the complementary pension schemes, please refer to \citet{ipp_retraites}, chapter 3.}.\newline


\texttt{Legislative references:} Law \no 72-1223 of 12/29/1972 (OJ of 12/30/1972).

\subsubsection*{Contributions of non-executive workers}

For non-executive workers, complementary pension contributions either for the employer's or the employee's parts, are levied at different rates on two different brackets of earnings since January 1\up{st}, 2000. The portion of earnings below one SSC, called bracket 1, is subjected to an effective rate of 3.05~\% for the employee's part and of 4.58~\% for the employer's part (in 2014) (corresponding to an overall contractual rate of 6.20~\%). The portion of earnings lying between 1 and 3 SSC, called bracket 2, is subjected to effective rates of 8~\% for employees' part, and of 12~\% for employers' part. 

The division between employers' and employees' Arrco contributions correspond to a 60~\% - 40~\% division, except if the company has been created before January 1\up{st}, 1999 and wishes to maintain the effective distribution until 1998, or if the company has finalized a collective contract before April 25\up{th}, 1996, or if the company has reached an agreement on a distribution more favorable to employees. As a consequence, the formal division of Arrco contributions between employers and employees varies from one company to the other.


\subsubsection*{Contributions of executive workers}

For executive workers, complementary pension contributions either for the employer's or the employee's parts, are levied at different rates on two different brackets of earnings since January 1\up{st}, 1998. Bracket A is equivalent to bracket 1 for non-executives, whereas bracket B (taxable as soon as the scheme has been introduced in 1948) refers to part of earnings between 1 and 4 SSC. Finally, the last bracket, created in 1998, corresponds to the portion of earnings between 4 and 8 SSC. Earnings above 8 SSC are not subjected to contributions and then do not provide additional complementary pension entitlements.

Contributions paid on bracket A are actually transfered not to Agirc but to Arrco. However, contributions paid on brackets B and C are collected by Agirc. Overall rates -- employees' and employers' -- are equal on bracket B and bracket C, but the distribution can vary between bracket B and C if an agreement within the company states so (even though the last increase in rate had to be distributed the following way: 0.10 additional points for employees, and 0.20 additional points for employers). Thus, as for Arrco, the division of the contribution tax burden between employers and employees differs from one company to the other. Companies falling under the most important complementary pension fund attached to Agirc have today a contribution formally borne for 1/3 by employees, versus only 1/4 at the time the scheme has been implemented.

To cope with increased funding needs caused by the retirement of baby-boomers, Agirc and Arrco agreed upon a gradual rise in employers' and employees' contribution rates on all brackets in 2014 and 2015 (\texttt{Agirc-Arrco agreement of 03/13/2013}).


\subsubsection*{\textit{La Garantie Minimale de Points} (GMP)}

Instituted in 1989, the \textit{Garantie minimale de points} (GMP) aims at guaranteeing to each person affiliated to Agirc a minimum number of pension points for a given year. A minimum fixed-amount contribution must be paid in return of this guarantee.

The GMP contribution, designed to secure this guarantee, is a contribution paid only in the case that the Agirc contribution on bracket B is lower than the fixed-amount contribution set each year by Agirc (according to the point's value and it purchasing value, such that the fixed-amount contribution entitles to get 144 Agric points). This means that the GMP contribution is only levied on executives with annual earning below a critical wage (equal to 41 445 euros in 2014). The GMP contribution is computed as the difference between the fixed-amount contribution and the Agirc contribution paid on bracket B.

\begin{equation}\nonumber
GMP~CONT = F-A~CONT - Agirc_B~CONT \hspace{0.5cm} if~w < CW
\end{equation}

with $GMP~CONT$ being the GMP contribution to determine, $F-A~CONT$ the fixed-amount contribution set each year by Agirc, $Agirc_B~CONT$ the Agirc contribution paid on bracket B, $w$ the gross wage and $CW$ the critical wage.

The distribution of the GMP contribution between employers' and employees' parts is the same as for the Agirc contribution on bracket B (that is to say $2/3$ borne by employers and $1/3$ borne by employees) \newline

\texttt{Legislative references:} National executives collective agreement of 03/14/1947, art. 6, \S 2, F..


\subsubsection*{Other contributions managed by complementary pension schemes}

\textbf{ASF and AGFF contributions}\newline

Introduced in 1984, the Association for the management on the funding structure (\textit{Association pour la gestion de la structure financi\`ere}, AGF) allowed to compensate for the decrease in the retirement age of the complementary pension schemes Agirc and Arrco. As the decline in the minimum retirement age had made possible the stabilization of the unemployment insurance accounts by letting unemployed elder people retire earlier, the Unedic was required to transfer part of its receipts to the ASF and to collect the ASF contribution meant to provide to the fund additional revenue. This contribution, levied on brackets A and B at different rates, was made of an employers' and an employees' part.

Since April 1\up{st}, 2001, the ASF has been replaced by the Association for the management of the financing fund of Agirc and Arrco (\textit{Association pour la gestion du Fonds de financement de l'Agirc et de l'Arrco}, AGFF), administered by employers' and employees' representatives. Since it has been created, the complementary pension schemes are in charge of collecting the AGFF contribution, which is levied both on executives' and non-executives' earnings (on brackets 1 and 2 for non-executives, and on brackets A and B for executives). The AGFF agreement allows employees retiring as soon as at 62 years-old to get their complementary pension without waiting for the full-pension retirement age of the base scheme (65 years-old), provided that employees have reached the required contribution duration (varying according to the age). \newline

\texttt{Legislative references:} Arrco-Agirc agreement of 02/10/2001 (ch. III) and of 11/12/2003; circular 2001-053 of 04/04/2001.\newline


\textbf{The APEC contribution}\newline

The \textit{Association pour l'emploi des cadres} (APEC -- an association for the employment of executives, engineers and technicians working on the placement of unemployed people) has been introduced by the National executives collective aggrement of November 18\up{th}, 1966. The collective agreement stipulates that a contribution must fund the APEC benefits. Nonetheless, it seems that this contribution has been implemented \textit{de facto} only in 1971 (\texttt{Amendment \no1 of January 3\up{rd}, 1969 and Amendment \no2 of June 9\up{th}, 1969)}, as a proportional contribution on bracket B of earnings (between 1 and 4 SSC).

The rate, initially set at 0.04~\%, has been brought to 0.06~\% on January 1\up{st}, 1975. Moreover, a 1975 amendment to the 1966 collective agreement extended the contribution tax base to the portion of earnings below one SSC (bracket A). In practice, however, the Agirc not having available any information on earnings below one SSC, this contribution has been levied on a fixed-amount basis equal to 0.06~\% of the annual SSC, from its implementation in 1976 to 2010. \\
In 2011, the contribution became entirely proportional, and is now levied on the whole earnings below four SSC (\texttt{Agirc circular 2010-5-DF of 07/29/2010}).

The contribution, being either proportional or a fixed-amount, is divided into an employers' and an employees' part (60/40).\newline

\texttt{Legislative references:} National executives collective agreement of 11/18/1966, art. 3; Amendment \no5 of 11/17/1975 to the collective agreement of 11/18/1966 approved by the order of 02/16/1976 (OJ of 02/26/1976).\newline

\textbf{Exceptional temporary contribution}\newline

This contribution is due on earnings of employees affiliated to Agirc, at a unique rate for earnings below eight SSC (earnings above this threshold are exempted). Before it has been created on January 1\up{st}, 1997, the complementary pension funds have established systems of fixed-amount contributions and guarantees made mandatory through collective or company-wide agreements, and supposed to allow some affiliated persons who did not contribute enough to still receive an old-age benefit. The exceptional temporary contribution replaced this multiplicity of schemes. Thus, the exceptional temporary contribution is a solidarity contribution: it contributes to the financial balance of the Agirc accounts but does not open any entitlement for those who pay it.

Today equal to 0.35~\%, the contribution is divided between employers (0.22~\% rate) and employees (0.13~\% rate). Although its name suggests a provisional vocation, the exceptional temporary contribution is still effective since it has been created in 1997, and has been regularly increased since it has been instituted.\newline

\texttt{Legislative references:} National executives collective agreement of 03/14/1947, art. 2 of appendix III.



\subsubsection{Complementary pension schemes of public-sector employees}

\subsubsection*{The APPS contributions}

The Additional pension scheme of the public-sector (APPS) (\textit{Retraite additionnelle de la Fonction publique}, is a point-based scheme, mandatory since 2005, allowing tenured civil servants of the three civil services to acquire an extra old-age benefit. The APPS contribution is made of an employer's and an employee's part, both at a 5~\% rate.

The APPS contribution tax base is made of all the earnings subjected to CSG and not subjected to the basic old-age contribution, that is to say bonuses and compensations. This tax base is however capped to 20~\% of the base pay (TBI) earned the given year. In other words, knowing that $Tot~earn$ refers to total earnings:\newline

\underline{Tax base:}
\begin{equation*}
Tax~base~=~Max[(Tot~earn - TBI), 0.2\times TBI]
\end{equation*}

\texttt{Legislative references:} Decree 2004-569 of 06/18/2004 (OJ of 06/19/2004).


\subsubsection*{The IRCANTEC contributions}

Non-permanent public agents must also contribute to a mandatory complementary pension scheme. Their complementary pension scheme is however different both from Agirc-Arrco schemes of private sector employees and from the APPS scheme of tenured civil servants. Non-permanent public agents from the three civil services contribute to their own specific scheme, called the Complementary pension scheme of non-permanent agents in the State and public authorities (\textit{Institution de retraite compl\'ementaire des agents non titulaires de l'Etat et des collectivit\'es publiques}, IRCANTEC). Created in 1971, the IRCANTEC is a point-based scheme which replaced the other two former complementary pension schemes (the IPACTE and IGRANTE, respectively instituted in 1951 and 1959)

Contributions paid to the IRCANTEC fall into an employers' and an employee's part. As for the private-sector Agirc and Arrco schemes, the effective contribution rate $\tau_e$ is computed as the product of the contractual contribution rate $\tau_c$ and an adjustment rate $\tau_a$. Initially set at 80~\%, the latter has reached 100~\% in 1988, and is now equal to 125~\%. Thus, only 4/5\up{th} of contributions paid provide entitlements to complementary pension.

\begin{equation*}
\tau_e = \tau_a \times \tau_c
\end{equation*}

The contribution rate varies with the level of earnings: earnings under one SSC correspond to bracket A, whereas the portion of earnings lying between one and eight SSC is defined as bracket B. Earnings above 8 SSC are exempted from contributions and then do not provide additional entitlement to complementary old-age benefits. Until December 1\up{st}, 1991, the bracket B upper threshold was equivalent to 4.75 SSC.

The IRCANTEC contribution tax base consists of the contractual gross wage, compensations attached to the employment, the overtime hours payment and the value of in-kind compensation. Are exempted the \textit{suppl\'ement familial de traitement} (benefit granted according to the number of children) and some exceptional compensations and compensations for expenses incurred. \newline

\texttt{Legislative references:} Decree 70-1277 of December 23\up{rd}, 1970 creating a complementary pension scheme in favor of non permanent agents of the State and public authorities; Decree 91-1375 of 12/30/1991 (OJ 12/31/1991).

\subsubsection{Complementary pension schemes of non-farming self-employed workers}

\subsubsection*{Contributions of liberal professions}

The management of complementary pension schemes is organized at the level of occupational section, the contributions and benefits varying from one section to the other (some schemes including or not a disability-death insurance, for example). Nonetheless, they have common features, such as the fact that they are mandatory, of a mixed type between funded and pay-as-you-go pension scheme, and, in general, with a retirement age equal to 65 years.

The contribution is set, according to the occupational section, as a function of the level of earnings and/or the level of benefits demanded. For example, general insurance brokers have their own complementary pension scheme, and their own disability-death insurance, such as the authorized architects, engineers, technicians, surveyors, experts, or the physiotherapists, nurses, speech therapists and orthoptists.\newline

Here again, lawyers must be considered a special case: they contribute to the National fund of French Bars as part of their complementary pension scheme and contingency fund. The contribution is applied on two brackets of professional earnings.\newline

\texttt{Legislative references:} Art. L 642-1 of the SSC; Decree \no 67-1169 of 12/22/1967; Decree \no 2003-1273 of 12/26/2003; Decree \no 79-262 of 03/21/1979; Decree \no 79-263 of 03/21/1979; Decree \no 84-143 of 02/22/1984; Decree \no 68-884 o 10/10/1968 \newline

\subsubsection*{Contributions of craftsmen, traders and manufacturers} 

The craftsmen, traders and manufacturers must also contribute to a mandatory complementary pension scheme, also administered by the Social self-employed regime (\textit{R\'egime social des ind\'ependants}, RSI), as for the basic pension scheme. This uniting of the two pension funds if the result of the merging, on July 1\up{st} 2006, of the former two pension funds: the AVA (craftsmen' pension scheme) and the Organic (Traders' and manufacturers' pension scheme). This complementary pension scheme is point-based. Until 2004, contributions were optional for traders and manufacturers. The complementary pension contribution has always been capped; since January 1\up{st} 2013, the contribution rate is the same for craftsmen, traders and manufacturers.

It is today equal to 7~\% for earnings below the RSI threshold, and 8~\% between one and four RSI threshold. The RSI threshold is equal to 37 513\euro ~ and is specific to the self-employed scheme: in 2013 it was similar to the SSC, but it has been revalued after.

The contributions' computation is done following the same method than for the other Social security branches, that is to say on a provisional basis on earnings of year N-2, and being then revalued. However, before 2009, the complementary pension contributions were paid on a permanent way on earnings of the penultimate year.

Craftsmen, traders and manufacturers can also subscribe to complementary pension contracts and contingency funds in addition to the mandatory scheme, on a voluntary basis.\newline

\texttt{Legislative references:} Art. L 635-1 of the SSC; Decree \no 2012-139 of 01/30/2012 (OJ of 31/12/2012); Decree \no 2012-443 of 04/03/2012 (OJ of 04/04/2012); RSI Circular \no 2013-004 of 01/17/2013




%%%%%%%%%%%%%%%%%%%%%%%%%%%%%%%%%%%%%%%%%%%%%%%%%%%%%%%%%%%%%%%%%%%%%%%%%%%%%%%%%%%%%%%%%
%%%%%%%%
\section{Flat-rate income tax (CSG and CRDS)}

The Generalized social contribution (\textit{Contribution sociale g\'en\'eralis\'ee}, CSG) and the Contribution to the repayment of the social debt (\textit{Contribution au remboursement de la dette sociale}, CRDS) have been created in the 1990's to diversify and increase the sources of funding of the Social security scheme. Explicitly designed to fund different Social security branches, these two flat-rate taxes are also levied on capital and substitution income, at different rate according to the type of income. Although the CSG was initially meant to fund exclusively the family branch of the Social security scheme, there are actually today five CSG funding different Social security funds.

CSG and CRDS can be either considered proportional income tax (on income from work here) or social contributions. Unlike the income tax (IT), CSG and CRDS tax base is gross income (from work) and not taxable income (from work); in addition, they are levied at the same rate whatever the level of earnings, whereas the IT rate is progressive. The CSG is paid by any individual domiciled in France for the payment of the income tax \emph{and} affiliated to a French sickness insurance scheme (whereas this last condition is not required to be subjected to the IT\footnote{As a reminder, the conditions defining the fiscal residence are the following: \emph{(i)} having one's household, or \emph{(ii)} having one's main place of residence in France, or \emph{(iii)} carrying out a non-accessory professional activity in France, or \emph{(iv)} having the center of one's economic interests in France.}, which has then a territorial scope larger than the CSG). Moreover, two elements mentioned as impeding the CSG and income tax merging in the debate on the fiscal reform in 2013 and 2014 are worth being recalled: CSG and CRDS are computed on individual income and deducted at source while the IT is at the household level and levied after notice.


If this \emph{IPP legislative Guide} focuses on taxes and contributions on earnings, we will should however remind to the reader about the fact that there are four CSG: the CSG on earnings and substitution income, created in 1990\footnote{But enforced on February 1\up{st}, 1991.}, the one of interest here, but also the CSG on property income and the CSG on capital income (also created in 1990), as well as the CSG on gains from games (instituted in 1997). Although the CSG and CRDS paid on capital income are considered taxes and therefore collected by the Treasury department, the CSG and CRDS on benefits income and earnings are collected by the Urssaf, as for the social security contributions. However, unlike the unemployment and old-age insurances contributions, the amount of CSG and CRDS paid has no impact on the social rights the individual is entitled to receive. Moreover, the CSG and CRDS tax base differs from the one of the social security contributions: a 1.75~\% deduction for business expenses is applied to earnings up to four SSC, and some extra elements of earnings are subjected to the CSG and CRDS while they are exempted from social security contributions, and in some cases, from payroll taxes (see the table summing up the social regime of all the different elements of earnings at the end of the chapter \ref{Regimes fiscal social}).



\subsection{The \textit{Contribution sociale g\'en\'eralis\'ee} (CSG)}

Introduced in 1991 to cope with the funding needs of the Social security scheme, the CSG rate has increased to a great extent, from 1.10\% when it has been created to 7.5~\% in 1998, whereas the abatement rate for business expenses has been reduced from 5~\% to 3~\% in 2004, before being brought down to 1.75~\% in 2012. Also levied on substitution and capital income, the CSG can be partly deducted from income tax, at a rate varying with the type of income.

The CSG tax base has also undergone some changes since it has been created. Initially supposed to be identical to the social security contributions tax base, it is today larger because of the numerous social security exemptions applying for some elements of earnings (as an example, bonus and compensations for tenured civil servants, contribution to an employee's saving plan in the private-sector).

The CSG yields today more revenue than the IT (92 billions euros of receipt from the CSG versus 69 for the IT in 2013). Its receipts, initially intended to fund the family branch of the general scheme, are now used as follows: 
\begin{itemize}
\item For the family branch of the Social security scheme (at a 1.08~ \% rate on earnings and unemployment benefits, and at a 1.10~\% rate on other incomes);
\item For the sickness branch of the Social security scheme (at a rate varying between 3.95~\% for some substitution incomes and 7.25~\% for gains from games); this distribution has been decided in 1998, to counterbalance a strong decrease in employees' sickness contribution (from 5.50~\% to 0.75~\%, this employee's part funding only the daily sickness leave benefits);
\item For the Solidarity old-age fund (\textit{Fonds de solidarit\'e vieillesse}) in order to fund the Solidarity to elderly people allowance (\textit{Allocation de solidarit\'e aux personnes \^ag\'ees}, formerly called the old-age minimum income). This allocation has been instituted by the 1993 reform on pensions (called Balladur reform); it takes the form of a 1.03~\% rate for earnings and a 1.05~\% rate for other incomes;
\item For the National solidarity fund in favor of autonomy (\textit{Caisse nationale de solidarit\'e pour l'autonomie}), created in 2005 to contribute to the functioning of establishments for elderly or disabled people;
\item For the \textit{Caisse nationale d'amortissement de la dette sociale} (CADES, i.e. a fund for the paying off of the social debt) since 2008 (at a 0.48~\% rate). This fund, created in 1996, was initially only funded by the CRDS (cf.\emph{infra}). \\
\end{itemize}

%\begin{tab}[16cm]{Taux de CSG en 2014 \label{CSG}}
%{\begin{tabular}{p{3cm}|p{3cm}|p{3cm}|p{3cm}|p{3cm}|p{3cm}|p{3cm}}
%\textbf{Revenus d'activit�: taux global} & \textbf{Revenus d'activit�: taux d�ductible} & \textbf{Allocations ch�mage: taux global} & \textbf{Allocations ch�mage: taux d�ductible} & \textbf{Abattement: sous 4 PSS} & \textbf{Abattement: Au-dessus de 4 PSS} & \textbf{Pensions de retraite et d'incapacit�: taux global} \\
%\hline
%7,50\% & 5,10\% & 6,20\% & 3,80\% & 1,75\% & 0,00\% & 6,60\%  \\
%\hline
%\textbf{Pensions de retraite et d'incapacit�: taux d�ductible} & \textbf{Pensions de retraite et d'incapacit�: taux r�duit} & \textbf{Pensions de pr�retraite: taux global} & \textbf{Pensions de pr�retraite: taux d�ductible} & \textbf{Pensions de pr�retraite: taux r�duit} & \textbf{Indemnit�s journali�res: taux global} & \textbf{Indemnit�s journali�res: taux r�duit} \\
%\hline
% 4,20\% & 3,80\% & 7,50\% & 4,20\% & 3,80\% & 6,20\% & 3,80\% \\
%\end{tabular}}
%{\emph{}}
%\end{tab}

The CSG has been implemented along with a set of adjustments of the old-age contributions of the general scheme, and of the family allowances contributions, as shown in the table \ref{tab:4:2}.


\begin{table}[h!]\label{tab:4:2}
\begin{center}
\centering
\caption{Increase and decrease in contributions consecutive to the CSG implementation.}
\vspace{0.2cm}
\small
\begin{tabular}{|>{\centering\arraybackslash}m{2.5cm}|| >{\centering\arraybackslash}m{2cm} >{\centering\arraybackslash}m{2.5cm} | >{\centering\arraybackslash}m{2cm} >{\centering\arraybackslash}m{2.5cm}|}
\hline
Contribution affected & \multicolumn{2}{c}{Employer's contributions} & \multicolumn{2}{|c|}{Employee's contributions} \\
\hline
\hline
CSG & - & - & $+$ 1.1 ppt & Total earnings \\
\hline
Family contribution & $-$ 1.6 ppt & Total earnings & - & - \\
\hline
\multicolumn{1}{|c||}{\multirow{2}{*}{Old-age contribution}} & $+$ 1.6 ppt & Total earnings & $-$ 1.05 ppt & below the ceiling \\
\multicolumn{1}{|c||}{\multirow{2}{*}{}} & & & - 42 FRF & All levels of earnings \\
\hline
\end{tabular}
\end{center}
\end{table}


\texttt{Legislative references:} {Art. L136-8 of the SSC}.

\subsection{The \textit{Contribution au remboursement de la dette sociale} (CRDS)}

In 1996, the \textit{Caisse d'amortissement de la dette sociale} (CADES) has been instituted to work on the repayment of the Social security debt. General scheme receipts having been on the whole lower than benefits granted to the affiliated people, the Social security scheme accumulated deficits which have been partially transfered to the CADES. This fund, under the administrative supervision of the Ministry of Finance and Economy, must stay active until the social debt disappears.

The CADES issues equities on the international capital markets to bring some liquid assets necessary to the funding of the Social security scheme, and to turn the short-run debt into mid-run and long-run debt. The resources needed by the CADES to be operational are provided by the CRDS. This \emph{flat-rate} income tax has the same characteristics than the CSG: levied at a uniform rate of 0.5~\% since it has been created in 1996, it is paid on earnings, substitution and capital incomes\footnote{The CRDS tax base is however larger than the CSG one with regards to benefits, as the family and housing allowances as well as the minimum income are subjected to CRDS but not to CSG.}.\newline

\texttt{Legislative references:} {Order 96-50 of 01/24/1996 creating the CADES; Order 2004-810 of 08/13/2004.}



%%%%%%%%%%%%%%%%%%%%%%%%%%%%%%%%%%%%%%%%%%%%%%%%%%%%%%%%%%%%%%%%%%%%%%%%%%%%%%%%%%%%%%%%%
%%%%%%%%%%%%%%%%%%%%
\section{Contributions specific to some extra elements of earnings}

Some elements of earnings benefit from a more favorable social regime, as they are excluded from the social security contributions tax base, only the CSG and the CRDS (and in some cases payroll taxes) being deducted on them. If the social advantage granted through this dispensation generally aims at giving incentives for employers and employees to develop some peculiar forms of salary (especially employees' saving plan and shareholding), the substitution of earnings subjected to the general social regime with partially taxes earnings induced a decrease in revenue. This decrease led the public authorities to set up specific contributions for these elements of earnings in the 2000's.


\subsection{The taxation of employers' contribution to a contingency fund}

To give incentives for employers to participate in the funding of a contingency fund covering their employees, the employers' contributions to third-party organizations funding complementary provident benefits have been exempted from CSG. Nonetheless, starting from 1996, a specific tax, initially at a 6~\% rate, has been introduced to tax these employers' contributions. Since 1998, the rate is equal to 8~\% whereas a tax exemption is granted to companies. In 2010, employers' contributions to contingency funds for \emph{former} employees are included in the provident tax base. This tax has been removed in January 1\up{st} 2012, at the moment when employers' contributions to contingency funds have been included in the \textit{forfait social}, at a special rate (see \emph{infra}).\newline

\texttt{Legislative references:} Order 96-51 of 01/24/1996; Acoss circular of 02/20/2012.

\subsection{The \textit{forfait social}}

The \textit{forfait social} (i.e. an employer's contribution) has been introduced on January 1\up{st}, 2009 at a 2~\% rate; this rate has been raised at a 2 percentage points pace each year until 2012, and then has been increased from 8~\% up to 20~\% in August 2012, in keeping with a recommendation from the \textit{Cour de Comptes} (audit office). Since January 1\up{st}, 2014, the revenue from this contribution has been allocated to the National old-age insurance fund and to the Old-age solidarity fund.

The \textit{forfait social} tax base includes sums of money paid as part of incentive bonus and profit-sharing scheme, the employers' contribution to an employees' saving plan, attendance fees paid to managers and members of a supervisory board of some types of companies (\textit{SA} and \textit{SELAFA}) since 2010, as well as the voluntary payment by the employer of the employee's contribution to a complementary pension scheme.


Besides, since January 1\up{st} 2012, the employers' contributions to a contingency fund are not subjected anymore to the special tax created in 1996 (see \emph{supra}), but are subjected to the \textit{forfait social} (at the reduced rate of 8~\%).\newline

\texttt{Legislative references:} art. L 137-15 to L 137-17 of the SSC, created by the art. 13 of the Law \no2008-1330 of December 17\up{th}, 2008.


\subsection{Contribution on bonus shares}

Employees' salary can take the form of the allocation of free equities and of stock-options which are subjected to the CSG and the CRDS, but exempted from social security contributions. A special contribution is paid on bonus shares and stock-options allocated since October 16\up{th}, 2007.

The tax is decomposed between an employer's and an employee's part; the revenue yielded by the first part, formerly allocated to the funding of the sickness insurance, has been allocated to the funding of the family branch since January 1\up{st}, 2014. The employee's contribution is considered a tax and not a social security contribution: its yield is allocated to the State budget.

The overall rate, equal to 12.5~\% in 2007 (10~\% for the employer's part, 2.5~\% for the employee's part), has been raised in 2001 and 2012, to reach now 40~\% (including 30 percentage points for the employer's part).\newline

\texttt{Legislative references:} art. L 137-13 and art. L 137-14 of the SSC, created by the art. 13 of the Law \no2007-1786 of December 19\up{th}, 2008; for the employee's contribution, seer Art. 80 bis and 80 quaterdecies of the General Tax code.


\subsection{Contribution on the \og top-hat retirement pension schemes\fg }

Sums of money paid by employer to fund supplementary defined benefits pension plans (also called "top-hat pension plans") are not subjected to social security contributions. This special regime by derogation -- as well as the sums of money at stake for some "big bosses" -- have been highly criticized in the 2000's. A specific tax has finally been set up on this sums of money on January 1\up{st}, 2004, at a rate varying with the type of tax base chosen by  the employer (see \emph{infra}). This contribution rate has been doubled in 2010, and again in 2013.

When the employer takes the scheme on, he can opt for (in a definitive way) a tax base made up of annuities (32~\% rate) or of bonuses paid to fund the scheme (24~\% rate if the scheme is managed externally, or 48~\% rate if the scheme is managed internally).

Moreover, to deter employers from paying very high annuities, a 30 points increase in the rate has been implemented on annuities exceeding 8 times the SSC.

With regard to taxes, this contribution is deductible from the income tax base up to a limit equal to the fraction paid on the first 1 000 euros of monthly return. The "top-hat pensions", which triggered heated debate when the crisis burst, are subjected to a large extent to a form of social taxation.\newline

\texttt{Legislative references:} art. L 137-11 of the SSC, created by the art. 115 of the Law \no2003-775 of 08/22/2003; on the tax regime, art. 83, 2-0 quater of the General Tax Code (GTC).




%%%%%%%%%%%%%%%%%%%%%%%%%%%%%%%%%%%%%%%%%%%%%%%%%%%%%%%%%%%%%%%%%%%%%%%%%%%%%%%%%%%%%%%%%
%%%%%%%%%%%%%%%%%%%%
\section{Payroll taxes}

Payroll taxes bring together different taxes explicitly designed to fund some public spending or allocated to the general budget. The tax burden is always borne by employers, and the amount due generally varies with the workforce of the company.

Public-sector employees are only subjected to the transportation payment (i.e. a contribution to finance the local public transport infrastructures) and the contribution to the National fund for housing aid 

\subsection{The wage tax}

The wage tax has been instituted in 1968, to replace the flat-payment on individual wages in effect between 1949 and 1967. It is levied by the Public Finances General Directorate (\textit{Direction g\'en\'erale des finances publiques}, DGFIP). Only some employers are liable for the payment of this tax. The liability conditions are the following:
\begin{itemize}
\item Being domiciled or established in France;
\item Not being subjected to the VAT (or not being subjected on more than 90~\% of the turnover made the year preceding the year of the earnings payment).
\end{itemize}

In practice, employers liable for the payment of the wage tax fall under the following categories: 
\begin{itemize}
\item Healthcare facilities (clinic, hospital, retirement home, etc.);
\item Some liberal professions (mainly the medical and paramedical occupations);
\item Educational institutions;
\item Retirement and contingency fund and Social security organizations;
\item Cooperating agencies, mutual benefit companies and farming occupations;
\item Associations ruled by the Law of July 1\up{st}, 1901 socially, culturally or sport oriented (recreation associations or federations) and those with less than 30 employees;
\item Banking and financial establishments;
\item Insurance companies;
\item Corporate services (activities of consultancy, architecture, engineering, survey, security, cleaning). \\
\end{itemize}

Public establishments not subjected to the VAT are also subjected to the wage tax.

\subsubsection*{Tax base}

\underline{Private sector:}\newline

For the private sector, since the 2014 Finance act, the wage tax base is identical to the CSG tax base. Therefore, for earnings paid starting from January 1\up{st}, 2014, the base pay, but also, for instance, the profit-sharing bonus and the employers' contributions to contingency funds are subjected to the wage tax. \newline

\begin{table}[h!]
\begin{center}
\centering
\caption{Evolution of the wage tax base.}
\vspace{0.2cm}
\small
\begin{tabular}{|c|c|}
\hline
Earnings' year & Tax's year \\
\hline
Until 2001 & Specific tax base \\
Between 2002 and 2013  & Social security contributions tax base \\
Since 2014         & CSG-CRDS tax base \\
\hline
\end{tabular}
\end{center}
\end{table}

\texttt{Legislative references:} {2002 Finance Act; 2014 Finance Act}.\newline

\underline{Public sector:}\newline

The tax base covers all types of earnings received by public agents (contractual or tenured): base pay, bonus, compensation, benefits for children.\newline

\texttt{Legislative references:} art. L136-2 (I.) of the SSC.

\subsubsection*{Computation method}

The wage tax is computed on each individual earnings according to a progressive tax scale divided into 4 brackets (three until 2013). The brackets' thresholds are annually revaluated at the same pace than IT brackets.

\subsubsection*{Exemptions and abatement}

Companies whose wage tax amount due is low, as they have a small workforce, are entitled to a reduction and even a total exemption from wage tax. Besides, the associations ruled by the 1901 law benefit from wage tax abatement when the overall amount on the whole year is low (under a 20 000\euro ~ ceiling for earnings paid in 2014). Finally, there are some dispensations: amounts paid to employees hired under special contracts (\textit{Contrat d'accompagnement dans l'emploi}, special work contracts designed to help the integration into the labor market) are not included to the wage tax base.\newline

\texttt{Legislative references:} {art. 231 of the General Tax code (GTC); art. 1679 of the GTC} (reduction and exemption).

\subsection{Employer's participation in in-house training and fixed-term contract fee}

Except for the State, local and regional authorities and public establishments, all employers established in France must participate in the funding of occupational training since 1972. The contribution rate is computed on total earnings. It varies with the size of the company: it is equal to 1.05~\% for companies between 10 and 19 employees, and to 1.60~\% for companies of 20 employees and more. Furthermore, since 1992, companies of less than 10 employees are also subjected to this contribution, at a reduced rate of 0.55~ \%. Employment contractors are subjected to a higher rates, because of the greater turnover and the lower skill level of the workforce. Rates are respectively equal to 1.35~\% and 2.00~\% for companies between 10 and 19 employees and companies of 20 employees and more. For all companies, the tax base is the same than the social security contributions tax base.

The sum of money paid as part of this contribution by companies is designed to fund the occupational training within the company. It falls into two parts: one part is meant to finance contracts and periods of professionalization, and individual right to training (at a 0.50~\% rate for companies of 10 employees or more), the other is used for training projects within the company.

Moreover, companies hiring workers under fixed-term contracts must pay a fee equivalent to 1~\% of earnings paid to these employees, to finance training and occupational qualification projects. \newline

\texttt{Legislative references:} {Art. 235 ter KA and ter D of the General Tax Code (GTC)}.

\subsection{Contribution to occupational training for self-employed workers}

All non-farming self-employed workers, being either liberal professional, craftsman, trader, manufacturer or in freelance -- as well as the co-worker or associate spouse, are required to contribute to the financing of their own occupational training. This contribution is also due by farming workers and artist-authors, but we will not go into detail as our study does not focus on them.

This contribution is collected by the RSI for traders and manufacturers, by the tax office which transfers it to the Guild of arts and crafts for craftsmen, and by the Urssaf for liberal professions. In case the traders or liberal professionals are helped in their activity by a co-worker of associate spouse, or by a family auxiliary non-wage earner, the contribution is increased. Regarding self-employed workers having chosen the micro-social regime (see the preceding chapter), the contribution is paid in addition to the overall social security contribution rate. In 2015, the occupational training contribution was equal to:
\begin{itemize}
\item 0.25\% of the SSC for the traders, manufacturers and liberal professionals, and 0.34\% for the co-worker or associate spouse, or non-wage earner auxiliary worker;
\item 0.29\% of the SSC for craftsmen (0.17\% of the SSC in Alsace-Moselle);
\item 0.3\% of the turnover for craftsmen with a micro-enterprise, 0.1\% of the turnover for the trader with a micro-enterprise, and 0.2\% of the turnover for the liberal professionals and service providers with a micro-enterprise
\end{itemize}

\texttt{Legislative references:} Art. L 6331-48 to L 6331-54 of the Labor Code (LC); Art. L 6331-65 to L 6331-68 of the Labor Code (LC); Art. 38 of the Law \no 2012-958 of 08/16/2012


\subsection{Apprenticeship taxes}

Introduced as soon as 1925 to fund the apprenticeship, the proper apprenticeship tax is based on total earnings and is due by all the trade, industrial, artistic or craft companies. It is today equal to 0.50~\%. An additional contribution of 0.10~\% has been instituted in 1977, but repealed on January 1\up{st}, 1993, and replaced by a contribution of the same rate designed to fund the vocational block release education. Finally, an additional contribution called the Contribution to the development of apprenticeship (\textit{Contribution au d\'eveloppement de l'apprentissage}, CDA) has been set up on January 1\up{st}, 2004, as part of a policy in favor of the integration of young people into the labor market. Initially levied at a 0.06~\% rate, the CDA has been raised to a 0.18~\% rate as soon as 2011. An additional 0.26~\% tax specific to the local Alsace-Moselle scheme has been introduced in the 1970's.


Besides, companies of 250 employees and more are required to pay an Additional apprenticeship contribution (\textit{Contribution suppl\'ementaire \`a l'apprentissage}, AAC). Instituted on January 1\up{st}, 2010, its rate differs according to both the workforce size and the proportion of employees under a work-study contract.

It is worth noting that companies not subjected to the IT or to the corporate tax (CT), as well as companies hiring apprentices, are exempted from the apprenticeship tax and the AAC. The local Alsace-Moselle scheme has its own tax scale for the AAC.

The apprenticeship tax base is the same as for social security contributions.\newline

\texttt{Legislative references:} {Finance act of 07/13/1925}.

\subsection{Employer's participation in the house building effort}

Created in 1953, the employer's participation in house building effort (\textit{participation des employeurs \`a l'effort de construction}, EPHBE) is a contribution meant to make companies participate in the building of new housing. Since January 1\up{st} 2005, it is only paid by companies of 20 employees or more, at a 0.45~\% rate.

This contribution, initially called the \og housing 1~\%  \fg{} because of its 1~\% rate, has been renamed \og Housing Action \fg{} in 2010.

\subsection{Contributions to the National fund for housing aid}

All employers are liable for the payment of a contribution to the National fund for housing aid (\textit{Fonds national d'aide au logement}, FNAL), designed to finance housing benefits. Instituted in 1972, this contribution, called \og FNAL contribution \fg, is paid only on the part of earnings below one SSC, at a 0.10~\% rate. Furthermore, a second contribution called \og Additional FNAL contribution \fg{} has been created in 1986. Due only by employers of 20 employees and more (except employers affiliated to the farming scheme), its rate has regularly increased since it has been created. This additional contribution is levied at a different rate for the part of earnings below the SSC (currently 0.40~\%) and above the SSC (0.50~\%).

The FNAL contribution and additional FNAL contribution tax base is the same as the social security contributions tax base. 

\subsection{The transportation payment}

Instituted in 1971 to finance the public transportation infrastructures, the transportation payment (TP, \textit{versement transport}) has to be paid by all the companies established in the region Ile-de France, and, since 1974, in some French cities outside the Parisian urban area. Are subjected the cities of more than 30 000 inhabitants (before 1982, the threshold was at 100 000 inhabitants). Are exempted companies employing less than 10 workers in the affected areas. The liability is determined at the company level: if a company owns several establishments in different areas, it is subjected to the tax only in establishments of more than 9 employees. The workforce count must not take into consideration students in internship and itinerant workers.

The transport payment tax base is the same as for social security contributions (earnings to take into consideration being only those in the affected establishments). The tax rate depends on the geographical area: maximal rates are set for cities outside Ile-de-France, and they vary with the city size. Therefore, in 2014, the maximal rate effective for cities of more than 100 000 inhabitants outside Ile-de-France is 1.75~\% (with a possible increase in touristic urban areas, as in Marseille for example). In Paris, the rate in zone 1 goes up to 2.70~\%. It is up to the urban areas to set the transport payment rate, within the limits stipulated in the law.  For each federation of municipalities, the rate can be revised twice a year. It results from this setup a very high number of transportation payment rates (thousands), available for consultation on the Urssaf website\footnote{Website adress: \href{http://www.urssaf.fr/profil/employeurs/baremes/baremes/versement_transport_01.html}.}.

Unlike other payroll taxes, the transportation payment is levied by the Urssaf which transfer the collected sum of money to the local and regional authorities concerned.


\subsection{The exceptional tax on high earnings}

The exceptional tax on high earnings (\textit{La taxe exceptionnelle sur les hauts salaires}, ETHE) corresponds to the second draft of the \og 75~\% tax\fg{} put forward by the President of the Republic Fran\c{c}ois Hollande during his presidential campaign. Actually, it consists in a tax equal to 50~\% of the part of individual gross earnings exceeding one million euros. The tax base includes, in particular, the base pay, in-kind or cash compensations, as well as retirement pensions and additional old-age benefits, bonus or compensation granted for the retirement, the allocation of subscription of purchasing possibilities on stock-options, the allocation of bonus shares.

Effective on earnings paid starting from January 1\up{st}, 2013, this tax is destined to be repealed on January 1\up{st}, 2015. Furthermore, the total amount paid by an employer as part of the ETHE cannot exceed 5~\% of the company's turnover\footnote{This measure has been introduced as a response to the strike threat from professional football clubs which pay earnings falling under the scope of the ETHE to attract and keep international players.}.\newline

\texttt{Legislative references:} {Law 2013-1278 of 12/29/2013 (2014 Initial Finance Act, OJ of 12/30/2013), art. 15 (tax base defined in II.)}.


%%%%%%%%%%%%%%%%%%%%%%%%%%%%%%%%%%%%%%%%%%%%%%%%%%%%%%%%%%%%%%%%%%%%%%%%%%%%%%%%%
%%%%%%%%%%%%%%%%%%%%%
\section{Summing-up tables}

\begin{center}
\begin{fig}[16cm]{Social security contributions and flat-rate income tax: effective tax base and rates on January 1\up{st}, 2014. \label{tableall}}
{\graphique[0.8]{tableau_SSC_TSMO.png}} {\footnotesize \textsc{Notes:}~(1) Fixed-amount or real value estimation ; (2) The IT exemption is total only if the rights' blocked period is respected.}
\end{fig}
\end{center}

\begin{center}
\begin{fig}[18cm]{Payroll taxes:es: effective tax base and rates on January 1\up{st}, 2014. \label{tableall}}
{\graphique[0.8]{tableau_SSC_TSMO_2.png}} {\footnotesize \textsc{Note:} An additional apprenticeship contribution is due by companies of 250 employees or more or which do not comply with the quota of young people employment (in apprenticeship or under a youth training contract). The contribution tax base is total earnings, and the rate varies between 0.05 and 2.00~\%.}
\end{fig}
\end{center}

\begin{center}
\begin{fig}[18cm]{Social security contributions and flat-rate income taxes with respect to status and earnings bracket (outside payroll taxes), effective on January 1\up{st}, 2014. \label{tableall}}
{\graphique[0.9]{Tableau_prelevements_salaires.png}}{\emph{}}
\end{fig}
\end{center}


\begin{table}[h!]
\centering
\caption {Tax base, rate and terms and conditions for the payment of contributions for craftsmen, traders and manufacturers.}
\small
\begin{center}
\begin{tabular}{|>{\centering\arraybackslash}m{2cm}| >{\centering\arraybackslash}m{2cm} | >{\centering\arraybackslash}m{2cm} | >{\centering\arraybackslash}m{1cm}|>{\centering\arraybackslash}m{1cm}|>{\centering\arraybackslash}m{1cm}|>{\centering\arraybackslash}m{2.5cm}|>{\centering\arraybackslash}m{2.5cm}|}
\cline{1-8}
Contribution & Organization & Tax base & Rate & Minimal contribution (ceiling) & Minimal contribution amount & Dispensation from minimal contribution & Exemptions \\
\hline
\hline
\multicolumn{8}{|c|}{Social security scheme} \\
\hline
SMDD                  & RSI   & Total professional income & 6.5\% & 10\% SSC & Depends on the level of income & Retired people; minimum income recipients; workers carrying out a wage-earning activity as their main activity & Unemployed entrepreneurs benefiting from the Accre on their earnings below 120\% SMIC during 12 months \\
\hline
Daily sickness leave benefits& RSI & Below 190 200\euro & 0.70\% & 40\% PSS & 107\euro & Retired people: minimum income recipients;workers carrying out a wage-earnings activity as their main activity & Unemployed entrepreneurs benefiting from the Accre on their earnings below 120\% SMIC during 12 months\\
\hline
\multirow{2}{2cm}{Base retirement pension} & RSI &	 Professional income under 1 SSC	& 17.40\%	& \multirow{2}{1cm}{7.7\% PSS}& \multirow{2}{1cm}{510\euro} & In case of inability to work & Unemployed entrepreneurs benefiting from the Accre \\	
  &  & Professional income above 1 SSC	& 0.35\% & & & & on their earnings below 120\% SMIC during 12 months \\				
\hline
\multirow{3}{2cm}{Family contributions} & RSI & Professional income under 110\% SSC	& 2.15\%	&	& & & Unemployed entrepreneurs benefiting from the Accre \\
 & & 	Professional income between 110\% and 140\% SSC	& between 2.15\% and 5.25\%	& & & &  on their earnings below 120\% SMIC \\			
 & & 	Professional income above 140\% SSC	& 5,25\% & & & & during 12 months \\			
\hline
\end{tabular}
\end{center}
\end{table}

\begin{table}[h!]
\centering
\caption {Tax base, rate and terms and conditions for the payment of contributions for craftsmen, traders and manufacturers.}
\small
\begin{center}
\begin{tabular}{|>{\centering\arraybackslash}m{2.5cm}| >{\centering\arraybackslash}m{1cm} | >{\centering\arraybackslash}m{2cm} | >{\centering\arraybackslash}m{1cm}|>{\centering\arraybackslash}m{1cm}|>{\centering\arraybackslash}m{1cm}|>{\centering\arraybackslash}m{1.5cm}|>{\centering\arraybackslash}m{2cm}|}
\cline{2-8}
\multicolumn{1}{c|}{} & \multicolumn{1}{|c|}{Organization} & Tax base & Rate & Minimal contribution (ceiling) & Minimal contribution amount & Dispensation from minimal contribution & Exemptions \\
\hline
\hline
\multicolumn{8}{|c|}{Complementary schemes} \\
\hline
\multirow{2}{2.5cm}{Complementary pension} & RSI &	Below 37 513\euro & 7\% & 5.25\% SSC & 140\euro & & \\
 & & Between 37 513\euro and 152 160\euro & 8\% & & & & \\
\hline
Disability-death	& RSI & Below 38 040\euro & 1.30\% & 20\% SSC & 99\euro & & Unemployed entrepreneurs benefiting from the Accre on their earnings below 120\% SMIC during 12 months \\
\hline
\multicolumn{8}{|c|}{Payroll and flat-rate taxes} \\
\hline
CSG CRDS	& Urssaf	& Total professional income + Mandatory  social security contributions &	8\% (7.5\% for CSG + 0.5\% for CRDS) & & & & \\
\hline
Occupational training contribution & Urssaf	& Professional income below 1 SSC & 0.25\% & & & & \\
\hline
\end{tabular}
\end{center}
\end{table}

\begin{table}[h!]
\centering
\caption {Tax base, rate and terms and conditions for the payment of contributions for liberal professionals.}
\small
\begin{center}
\begin{tabular}{|>{\centering\arraybackslash}m{2cm}| >{\centering\arraybackslash}m{2cm} | >{\centering\arraybackslash}m{2cm} | >{\centering\arraybackslash}m{1cm}|>{\centering\arraybackslash}m{1cm}|>{\centering\arraybackslash}m{1cm}|>{\centering\arraybackslash}m{2.5cm}|>{\centering\arraybackslash}m{2.5cm}|}
\cline{2-8}
\multicolumn{1}{c|}{} & Organization & Tax base & Rate & Minimal contribution (ceiling) & Minimal contribution amount & Dispensation from minimal contribution & Exemptions \\
\hline
\hline
\multicolumn{8}{|c|}{Social security scheme} \\
\hline
SMDD                  & RSI   & Total professional income & 6.5\% & 10\% SSC & Depends on the level of income & Retired people; minimum income recipients; & Unemployed entrepreneurs benefiting from the Accre \\
& & & & & &  workers carrying out a wage-earning activity as their main activity & on their earnings below 120\% SMIC during 12 months \\
\hline
\multirow{2}{2cm}{Base retirement pension} & CNAVPL &	Professional income under 1 SSC	& 8.23\%	& \multirow{2}{1cm}{7.7\% PSS}& \multirow{2}{1cm}{296\euro} & retirement or disability pension recipients;  workers & Unemployed entrepreneurs benefiting from the Accre \\	
  &  & Professional income under 5 SSC	& 1.27\% & & & carrying out a wage-earning activity as their main activity & on their earnings below 120\% SMIC during 12 months \\				
\hline
\multirow{3}{2cm}{Family contributions} & Urssaf & Professional income under 110\% SSC	& 2.15\%	&	& & & Unemployed entrepreneurs benefiting from the Accre \\
 & & 	Professional income between 110\% and 140\% SSC	& between 2.15\% and 5.25\%	& & & &  on their earnings below 120\% SMIC \\			
 & & 	Professional income above 140\% SSC	& 5,25\% & & & & during 12 months \\			
\hline
\end{tabular}
\end{center}
\end{table}

\begin{table}[h!]
\centering
\caption {Tax base, rate and terms and conditions for the payment of contributions for liberal professionals.}
\small
\begin{center}
\begin{tabular}{|>{\centering\arraybackslash}m{2.5cm}| >{\centering\arraybackslash}m{2cm} | >{\centering\arraybackslash}m{2cm} | >{\centering\arraybackslash}m{2cm}|>{\centering\arraybackslash}m{1cm}|>{\centering\arraybackslash}m{1cm}|>{\centering\arraybackslash}m{1.5cm}|>{\centering\arraybackslash}m{2cm}|}
\cline{2-8}
\multicolumn{1}{c|}{} & \multicolumn{1}{|c|}{Organization} & Tax base & Rate & Minimal contribution (ceiling) & Minimal contribution amount & Dispensation from minimal contribution & Exemptions \\
\hline
\hline
\multicolumn{8}{|c|}{Complementary schemes} \\
\hline
Complementary pension & Depends on the professional section &	Depends on the professional section & Depends on the professional section &  & & & \\
\hline
Disability-death	& Depends on the professional section &	Depends on the professional section & Depends on the professional section &  & & & Unemployed entrepreneurs benefiting from the Accre on their earnings below 120\% SMIC during 12 months \\
\hline
\multicolumn{8}{|c|}{Payroll and flat-rate taxes} \\
\hline
CSG CRDS	& Urssaf	& Total professional income + Mandatory  social security contributions &	8\% (7.5\% for CSG + 0.5\% for CRDS) & & & & \\
\hline
Occupational training contribution & Urssaf	& 1 SSC & 0.25\% & & & & \\
\hline
\end{tabular}
\end{center}
\end{table}


%\begin{center}
%\begin{fig}[16cm]{Tax base, rate and terms and conditions for the payment of contributions for craftsmen, traders and manufacturers.}
%{\graphique[1.1]{Cotis_global_ac.pdf}}
%{\footnotesize \textsc{Note:}~Rates mentioned are those effective on January 1\up{st}, 2015.}
%\end{fig}
%\end{center}

%\begin{center}
%\begin{fig}[15cm]{Tax base, rate and terms and conditions for the payment of contributions for liberal professions.}
%{\graphique[1.2]{Cotis_global_pl.pdf}}
%{\footnotesize \textsc{Note:}~Rates mentioned are those effective on January 1\up{st}, 2015.}
%\end{fig}
%\end{center}

\begin{table}[h!]
\centering
\caption {Public and private sectors: joint and specific social security contributions and taxes.}
\small
\begin{center}
\begin{tabular}{|>{\centering\arraybackslash}m{3.5cm}|| >{\centering\arraybackslash}m{4cm} | >{\centering\arraybackslash}m{3.5cm} | >{\centering\arraybackslash}m{3.5cm}|}
\cline{2-4}
\multicolumn{1}{c|}{} & \multicolumn{1}{|c|}{Joint contributions} & Contributions specific to the private-sector & Contributions specific to the public sector \\
\hline
\hline
Family contributions                  & contributions to the National family allowances fund    & - & - \\
\hline
Sickness contributions and equivalent   & Autonomy solidarity contribution         & contribution to the National health insurance fund  & contributions to public sector mutual fund  \\
\hline
Basic pension contributions         & contribution to the National old-age insurance fund & - & Deduction on pensions and contribution \\
                                                    &                                                                                   &      &      to the additional pension scheme of the public sector \\
\hline
Complementary pension contributions & - & Agirc and Arrco contribution & IRCANTEC and CNARCL contributions (complementary pension schemes \\
&            &                 & for contractual public agents and local and regional authorities civil servants) \\
\hline
Flat-rate income taxes              & CSG and CRDS   & - & contribution to the Solidarity fund \\
\hline
Payroll taxes  & Wage tax, FNAL contribution, transportation payment & apprenticeship and training contributions & - \\
\hline
\end{tabular}
\end{center}
\end{table}

\begin{table}[h!]
\centering
\caption {Public sector: the different social insurance schemes.}
\small
\begin{center}
\begin{tabular}{|>{\centering\arraybackslash}m{4cm}|| >{\centering\arraybackslash}m{2.5cm} | >{\centering\arraybackslash}m{2.5cm} | >{\centering\arraybackslash}m{2.5cm} | >{\centering\arraybackslash}m{2.5cm}|}
\cline{2-5}
\multicolumn{1}{c|}{} & Tenured civil servants of the State civil service & Tenured civil servants of the Hospital civil service & Tenured civil servants of the Territory civil service & Contractual agents \\
\hline
\hline
Family allowances     & \multicolumn{4}{c|}{General scheme} \\
\hline
Sickness, maternity, disability and death insurance & \multicolumn{4}{c|}{General scheme} \\
\hline
WA-PD insurance  & State pension service & \multicolumn{2}{c|}{National fund for local} & {General scheme} \\
& & \multicolumn{2}{c|}{and regional authorities agents} & {} \\
\hline
Old-age insurance & State pension service & \multicolumn{2}{c}{National fund for local|} & National old-age \\
                               &                          & \multicolumn{2}{c}{and regional authorities agents|} & insurance fund \\
\hline
Complementary old-age insurance & \multicolumn{3}{c|}{Additional pension scheme of the public sector} & Complementary pension scheme of non-permanent agents in the State and public authorities \\
\hline
\end{tabular}
\end{center}
\end{table}

Subject to exceptions, the measures described hereafter apply both to contractual public sector agents and to tenured civil servants.
