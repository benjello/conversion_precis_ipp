

To determine the value of a contribution or payroll tax, it is first necessary to delimit the tax base, in order then to apply the rate corresponding to the specific tax. As a general rule, contributions are based on gross earnings before any contribution deduction (\texttt{art. L 242-1, 1\up{st} al of the SSC}). If the contributions and payroll tax bases are constituted of some elements of earnings, their exact composition differs from one contribution to the other.

In the case of capped contributions or payroll taxes with rates varying according to the wage level, we need to define the threshold necessary to determine which rate must be applied. This chapter is an attempt to draw up a detailed presentation of the rules regulating the threshold and tax base (sometimes called taxable income) that must be used for the computation of contributions and payroll taxes. We look over the statute of each element of earnings in the private and public sectors, as well as for self-employed workers, regarding contributions and payroll taxes. Finally, the final chapter will provide a global review of contributions and payroll tax bases.


%%%%%%%%%%%%%%% SECTION 1: r�gles g�n�rales de d�termination de l'assiette}

\section{General rules}

\subsection{Determination of the upper threshold}

\subsubsection{Setting of the periodic threshold}

\textbf {Case \no 1: Regular frequency of the salary}
\begin{itemize}
\item \underline{General case :} Pay period corresponding to a calendar month. The ceiling used is the monthly ceiling.
\item \underline{Special cases :} For the salespersons (\textit{Vendeurs repr\'esentants placiers}, VRP), it is the bi-annual ceiling which is used. In some temporary employment firms, the reference sometimes used is the weekly ceiling, when the salary is paid each week.\\
\end{itemize}

\textbf{Case \no 2: Irregular frequency of the salary}
\begin{itemize}
\item  \underline{General case:} When the salary is paid on an occasional basis, or at an irregular frequency, the ceiling is set by decomposing the period which corresponds to the salary paid into months, calendar two weeks period, calendar weeks, working days and hours.  In this case, it is the \textit{sum} of a given number of monthly, weekly, daily and hourly ceilings which applies.
\item \underline{Special case 1:} Earnings are explicitly expressed in \textbf{days}. The employer can choose either to use the daily ceiling or to add up as many thirtieths of the monthly ceiling as the number of days (working days or not) included in the pay period, up to a limit of 30 thirtieths. 
\item \underline{Special case 2:} Earnings are explicitly expressed in \textbf{hours}. In that case, the following rule applies:
    \begin{eqnarray*} \nonumber
    PC=MC \times \frac{NBHP}{151.67}
    \end{eqnarray*}
    where $PC$ refers to the period ceiling, $MC$ to the monthly ceiling, $NBHP$ to the number of hours paid over the working period, and 151.67 corresponds to the monthly number of hours paid for a full-time employee. \\
\end{itemize}

These rules do not apply to specific categories of occupations (artists, medical professions, etc.).\newline

\texttt{Legislatives references:} {art.  L241-3 al. 1 and art. R242-2 al. 1 of the SSC.}

\subsubsection{Annual regularization}

To account for the variation in wage across months, contributions paid by the employer on earnings paid to a unique employee over the year are computed using an \textit{annualized} ceiling. To determine to which wage bracket the employee's earnings correspond to, with regards with the sum of all earnings received throughout the year (from the same employer), we should add up all these earnings. The amount obtained is then compared to the ceiling computed as the sum of the ceilings effective for all the working periods of the employee. Then, it allows us to identify the amount of contributions that should have been paid over the year.

The tax base of the annual regularization is equal to the difference between the ceiling to be applied at the end of the calendar year (called \og applicable ceiling \fg) and the sum of the ceilings under which the employee has effectively contributed. If this difference is positive, the employer is required to pay additional contributions; on the contrary, if the regularization base is negative, the Urssaf must give back part of the contributions paid over the year.

\underline{Nota Bene:} The contributions' annual regularization only potentially affects the computation of capped contributions, as contributions and payroll taxes paid on the whole wage do not refer, by definition, to the Social security ceiling. It does not affect either employees whose earnings have been stable throughout the year. \newline

\textbf{\textit{Example of an annual regularization}:}

Let us take the example of an employee working in the same firm the all year 2014, and earning a monthly gross salary equal to 2 700 euros. However, in July and December, he earned holidays bonuses and performance rewards: as he is paid each month, we should refer to the monthly ceiling for the computation of social security contributions, as shown in the following table \ref{tab:exemple_1}.

\begin{table}\label{tab:exemple_1}
\centering
\caption{Example of an annual regularization of social security contributions: periodic ceiling applicable.}
\small
\begin{center}
\begin{tabular}{|c|c|c|}
\hline
Employment    & Gross earnings paid & Periodic ceiling \\
periods    & each month   & used each month  \\
\hline
January     & 2700  & 3129 \\
February     & 2700  & 3129 \\
March        & 2700  & 3129 \\
April       & 2700  & 3129 \\
May         & 2700  & 3129 \\
June        & 2700  & 3129 \\
July     & 3400  & 3129 \\
August        & 2700  & 3129 \\
September   & 2700  & 3129 \\
October     & 2700  & 3129 \\
November    & 2700  & 3129 \\
December    & 3700  & 3129 \\
\hline
Total       & 34100 & 37548  \\
\hline
\end{tabular}
\end{center}
\end{table}

His earnings having exceeded the monthly SSC twice (in July and December), he contributed on bracket 2 (that is to say the section of his earnings above the SSC). Yet, regarding his earnings throughout the calendar year (34 100 euros), he is below the annual SSC (37 548 euros): then, he should not contribute on bracket 2, but only on bracket 1.

If we assume that the contribution rate (employer and employee) on the part of earnings below one SSC is equal to 60~\%, and that the contribution rate on the part of earnings above one SSC is equal to 35~\%, the contributions paid each month are as described in the following table\ref{ta:exemple_2}.

\begin{table}\label{tab:exemple_2}
\centering
\caption{Example of an annual regularization of social security contributions: detailed presentation of the monthly contributions paid.}
\small
\begin{center}
\begin{tabular}{|c|c|c|c|}
\hline
Employment & Contributions paid & Contributions paid & Total of \\
period & on bracket 1 & on bracket 2 & contributions paid \\
\hline
January     & 1620  & 0     & 1620  \\
February     & 1620  & 0     & 1620  \\
March        & 1620  & 0     & 1620  \\
April       & 1620  & 0     & 1620  \\
May         & 1620  & 0     & 1620  \\
June        & 1620  & 0     & 1620  \\
July     & 1877.4 & 94.85 & 1972.25 \\
August        & 1620  & 0     & 1620  \\
September   & 1620  & 0     & 1620  \\
October     & 1620  & 0     & 1620  \\
November    & 1620  & 0     & 1620  \\
December    & 1877.4  & 199.85 & 20249.5 \\
\hline
Total       & 19954.8 & 294.7 & 20249.5 \\
\hline
\end{tabular}
\end{center}
\end{table}

The 294.7 euros of contributions paid on bracket 2 must be repaid. Nonetheless, 20460 euros ($0.6 \times 34100$) of contributions should have been paid on earnings below one SSC, whereas only 19954.8 euros have been paid. The annual regularization is then equal to + 210.5 euros (that is to say $(20460-19954,8)-294,7$), that employer and employee must pay in addition to what has been already contributed throughout the year.

Employers can also decide to set up a progressive regularization, month by month.\newline

\texttt{Legislative references:} {art. R243-10 and R243-11 of SSC}

\subsubsection{Reduction and counteraction of the ceiling}

\subsubsection*{Part-time employees}

The ceiling is prorated for employees working part-time, according to the number of hours worked relative to the full-time working time.
In some cases, such as the transition from a full-time to a part-time job, social security contributions (in particular those paid to complementary  pension schemes) can be computed on a tax base higher than gross earnings. \newline

\texttt{Legislative references:} {art. R242-7 of the SSC}

\subsubsection*{Hiring and leaving during the year}

For employees working only part of the year, the social security  ceiling is also prorated according to the length of the employed period. For instance, for an employee present within a given firm three months, two weeks and two days, we refer to a ceiling equal to 3 times the monthly SSC $+$ 2 times the weekly SSC $+$ 2 times the daily SSC.

\subsection{Minimum wage and minimum tax base}

The social security contributions tax base cannot be lower than the hourly Smic multiplied by the number of hours worked:

\begin{equation}\nonumber
Minimum~tax~base=Hourly~SMIC \times NBH
\end{equation}

where $NBH$ corresponds to the number of hours worked.

In particular, for employees paid around the hourly Smic level, the employer needs to check that the contribution tax base is not lower than the minimum tax base. This rule is effective for employees whose employer chose to apply the 10~\% flat-rate deduction for professional expenses .

However, the minimum tax base does not apply in the following cases:
\begin{itemize}
\item for employees whose contributions have been computed based on a fixed-amount; 
\item for students as part of a contracted internship in enterprise; 
\item for salespersons working for several companies (called in French \textit{VRP multicartes}).  
\end{itemize}



\subsection{Number of hours paid}

\subsubsection{General case}

The number of hours paid in a month or a year is used to compute the number of overtime hours, in order to check whether the employer abides by the social legislation regarding working time. Since January 1\up{st}, 2000 (January 1\up{st}, 2002 for firms with 20 employees or less), the statutory working time being equal to 35 hours a week, any hour starting from the 36\up{th} must be paid at an increased hourly wage rate. This increase depends on the number of overtime hours worked.

Moreover, the number of hours paid is a parameter taken into consideration in the computation of some contributions abatement on low earnings (Fillon reduction, among others).

\subsubsection{Fixed-wage employees}

There is a category of employees who are paid not on the basis of the number of hours effectively worked, but on the basis of a fixed number of hours initially agreed upon in a contract between employer and employee. These employees, so called \og fixed-wage employees \fg, are, in general, managers, as their autonomy and responsibilities lead them to often work overtime, and make the count of effective working time complicated.
 
The work contract of these employees stipulates a fixed number of hours or days of work to make within a month or a year. To determine the number of hours officially paid, conversion formulas are defined in the social law. \newline

\underline{Employees working a fixed contractual number of \emph{days} over the year:}

\begin{equation} \nonumber
NBHP = 151.67 \times \Big[\frac{NBDC}{218}\Big] \times \Big[\frac{52}{12}\Big]
\end{equation}

where NBHP corresponds to the number of hours paid in a month, and NBDC is the fixed number of days set in the contract (for the whole year). \newline

\underline{Employees working a fixed contractual number of \emph{hours} over the year:}

\begin{equation} \nonumber
NBHP = 151.67 \times \Big[\frac{NBHC}{45.7}\Big] \times \Big[\frac{52}{12}\Big]
\end{equation}

where NBHP corresponds to the number of hours paid in a month, and NBHC is the fixed number of hours set in the contract (for the whole year). \newline

In both cases, if the employment period of the employee only covers part of the calendar month because of a recruitment or departure, the number of hours paid computed using the above mentioned formulas must be prorated.

\begin{equation} \nonumber
NBHP_{pro} = NBHP \times \Big[\frac{NBDEP}{30}\Big]
\end{equation}

where $NBHP_{pro}$ corresponds to the number of hours paid after it has been prorated, NBHP is the number of hours paid before it has been prorated and NBDEP is the number of calendar days corresponding to the employment period.


\subsection{Rules applying to the public sector}

Social law and rules regarding wages are not the same in the public and private sectors. Nonetheless, the definition of the social security contributions tax base in the three civil services follows the similar rules as those presented above: earnings (gross or net, as the case may be) received under a certain period is compared to the Social security ceiling of the corresponding period. Elements of earnings specific to the public sector and the corresponding social regime will be described in further detailed later in this chapter.

Apart from a base pay inscribed on the work contract, an employer can pay his employee in different ways: in-kind benefits, bonuses, contribution to an employee's savings plan, etc. Elements additional to the base pay are subjected to different social and tax regimes, which will be presented in  this chapter.




%%%%%%%%%%%%%%%%%%%%%%%%%%%%%%%%%%%%%%%%%%%%%%%%%%%%%%%%%%%%%%%%%%%%%%%%%%%%%%%%%%%
%%%% Section 2: r�gimes social et fiscal des diff�rents �l�ments de r�mun�ration %%%

\section{Elements of earnings in the private sector}

The liability of an element of earnings to a contribution or payroll tax determines its \textbf{social regime}, but does not tell anything about its \textbf{tax regime}, that is to say its statute regarding income tax (IT), as established by the principle of autonomy between social and tax legislations. Although we will focus on the statute of the so called \og extra \fg~elements of earnings regarding social security contributions and payroll taxes, we will also provide some information on the taxation system of each category of earnings. To fully understand all the dimensions of the debates and measures with regard to social security contributions, it is indeed crucial to be aware that the effective tax rate of earnings depends both on the contribution and income tax rates. From the point of view of public finance and tax payers, the way social and tax systems intertwine has great consequences that are too often unknown. 

Apart from a base pay inscribed on the work contract, an employer can pay his employee in different ways: in-kind benefits, bonuses, contribution to an employee's savings plan, etc. Elements additional to the base pay are subjected to different social and tax regimes, which will be presented in the next sections. Moreover, as earnings in the public sector are subjected to specific rules, we will present in details the elements of earnings peculiar to the public sector and we will set out their status with regard to contributions and payroll taxes. 


%%%%%%%%%%%%%%%%%%%%%%%%%%%%%%%%%%%%%%%%%%%%%%%%%%%%%%%%%%%%%%%%%%%%%%%%%%%%%%%%%%%%%%%%%%%
%%%%
\subsection{Earnings strictly speaking}

The base pay is the element constituting the main form of salary for private-sector employees, and whose social and tax regimes are the most simple. Base pay is part of all the contributions and payroll tax base (social security contributions, unemployment contribution, social contributions and payroll taxes). It is also fully subjected to income tax (after having deducted contributions and deductible CSG).

In general, social and tax regimes do not differ whether earnings are paid under contractual hours or under overtime hours. Yet, the \og TEPA law \fg, intoduced by the Fillon government in 2007 and remained into force until it has been abrogated in 2012, made the taxation of earnings paid under overtime hours more complex (see chapter \ref{5-Exonerations}).

%%%%%%%%%%%%%%%%%%%%%%%%%%%%%%%%%%%%%%%%%%%%%%%%%%%%%%%%%%%%%%%%%%%%%%%%%%%%%%%%%%%
%%%%
\subsection{Allowances, bonuses, and cash compensation}

As a general rule, allowances, bonuses and cash benefits are subjected to social security contributions. Allowances for business expenses and profit-related bonus belong to specific social and tax systems (see below).

%%%%%%%%%%%%%%%%%%%%%%%%%%%%%%%%%%%%%%%%%%%%%%%%%%%%%%%%%%%%%%%%%%%%%%%%%%%%%%%%%%%
%%%%

\subsection{In-kind compensations}

\og Are considered in-kind compensations the goods or services provided by the employer on a free basis (or for a price lower than the real value) to employees for their private use. \fg (\citet{lefebvre2013paie}). As the case may be, the employer can provide an accommodation, some food, a mobile telephony equipment, a company car, or deliver reduced rate credits, etc.
%% Mettre des guillemets alors que traduction?
 
\subsubsection{Estimation of the value of in-kind compensation}
\begin{itemize}
\item \textbf{Real value estimation:} it is equivalent, for the recipient employee, to the amount he is able to save thanks to this compensation;
\item  \textbf{Flat-rate estimations:} the value of accommodation, food, car, personal computer and cellphone is assessed by default on a flat-rate basis (and according to what has been actually spent, if the employer decides so). These fixed rates are revalued each year on January 1\up{st}, based on the progression of the consumption price index (excluding tobacco).
\end{itemize}

\textbf{Example}
The in-kind compensation -- a car for instance -- must be estimated either on the basis of expenses incurred, or on a flat-rate basis.
\begin{itemize}
\item In the case of real value estimation, the expenses incurred for the employee include the amortization over a 5-years period at a 20~\% rate per year, the insurance, the maintenance cost and the cost of the fuel used for a private use but financed by the employer.  
\item In the case of a flat-rate estimation, for a car bought and used permanently by the employee who covers himself the fuel costs, the in-kind compensation is estimated to 9~\% of the purchasing cost of the car, taxes included, if the car has less than 5 years, and to 6~\% if it has more than 5 years. 
\end{itemize}

If the employee partly finances the in-kind compensation, his participation has to be deducted from the flat-rate or real value estimation: it is the difference between the value of the in-kind compensation and the employee's participation that must be included to the contributions tax base \footnote{In the case of food, if the employee's participation is equal or higher than 50~\% of the flat-rate estimation, the employer's participation is not integrated to the contributions tax base.}.

\underline{Nota Bene:} The supply of in-kind compensation should be distinguished from the provision of the necessary equipment to carry out the professional activity. Providing a company car to a commercial manager, for example, leads the employer to bear the cost of it as part of business expenses. If the company car is is privately used, this use would be considered an in-kind compensation. 

\subsubsection{Social and tax regimes}

In-kind compensations are subjected to all contributions and payroll taxes, as well as to the income tax (as earnings).\newline

\texttt{Legislative references:} {art. 82, al. 2 of the GCT ; art. L 242-1 of the SSC}

\subsubsection{Special cases}

\subsubsection*{Employees paid at the Smic level}

If the employee's salary is partly made up (in an usual way) of the provision of food or an accommodation, the employer deducts from the Smic amount a fixed-amount to obtain the \textit{cash} minimum wage that should be paid. However, whether the flat-rates needed to estimate the value of in-kind compensations are used for the computation of the contributions tax base, or are used for the computation of the cash minimum wage, they differ:

\begin{tabular}{c l}
\textit{Tax base} & = \textit{Hourly gross Smic $\times$ number of hours worked} \\
 & - \textit{fixed-rate compensations} \\
 & + \textit{in-kind compensations} \\
\end{tabular} \\

The value of fixed-rate compensations used for the minimum wage computation being higher than the value of in-kind compensations, the provision of in-kind compensation to employees paid at the Smic level gives rise to a contributions tax base \textit{greater} than cash gross earnings of an employee paid at the Smic level without in-kind compensations.


\subsubsection*{Employees whose contributions tax base or amount is determined on a fixed-amount basis}

For employees contributing on fixed-amount bases, in-kind compensations are not included in the contributions tax base.
This is the case, for example, for students doing an internship regulated by a contract between the educational institution, the company and the intern.


\subsubsection*{Categories of employees belonging to a special scheme}

Some categories of employees or recipients must comply with different rules in terms of in-kind compensations accounting and taxation. This is especially the case for:
 
\begin{itemize}
\item Employees in hotels, restaurants or caf\'es;
\item Canteens' staff;
\item Apprentices;
\item Corporate executive;
\item Pensioners formerly employed within the firm.
\end{itemize}

For more detailed informations on these specific professional categories, please refer to a reference handbook in social law or related to wages (for example \citet{lefebvre2013social} and \citet{lefebvre2013paie}).



%%%%%%%%%%%%%%%%%%%%%%%%%%%%%%%%%%%%%%%%%%%%%%%%%%%%%%%%%%%%%%%%%%%%%%%%%%%%%%%%%%%
%%%%

\subsection{Canteen and luncheon vouchers}

Employer's participation to meals taken at the canteen or to the purchase of luncheon vouchers by their employees is not mandatory, but is common and imposed by numerous collective bargaining agreements. The only thing the employer is bounded to do is to make available a catering place when at least 25 employees are willing to take their meal in the workplace. When they are less than 25, the employer is still required to provide a place allowing his employees to eat in good conditions.

\subsubsection{Luncheon vouchers}

Employees whose working time includes a meal time are entitled to receive luncheon vouchers of a value freely set by the employer. Sick workers, employees in training or in paid leave cannot benefit from them when they are absent from work.

\subsubsection*{Employers' contribution}

Since January 1\up{st}, 2014, the employer's participation to the purchase of luncheon vouchers is entirely exempted from social security contributions if it lies between 50 and 60~\% of the voucher's value, and is inferior or equal to 5.33 euros per voucher. 

However, when the employer's participation is lower than 50~\% \emph{or} higher than 60~\% of the voucher's value, it is entirely included in the contributions tax base. In the case that the employer's participation exceeds 5.33 euros per voucher, only the part above this amount is integrated to the contributions tax base.

The additional compensation granted through the employer's contribution to the purchase of luncheon vouchers is income tax exempted, up to the limit of 5.33 euros per voucher.

\subsubsection*{Employee's contribution}

The employee's participation to the purchase of luncheon vouchers must be included in the contributions tax base. Moreover, it is subjected to income tax\footnote{On payslips, this employee's participation is \emph{added} to net earnings to pay to get the taxable earnings, as luncheon vouchers are directly provided by the employer, the employee's participation is deducted from the amount received in cash.}.

\subsubsection{Professional canteen}

If employees can enjoy free meals at the company's canteen, the provision of the meal is considered an in-kind compensation and then must be estimated as such. If the employee's participation is at least equal to 50\% of the fixed-amount value of a meal, the tax administration tolerates that the employer's participation (lower or equal to 50~\%)  is not considered an in-kind compensation. However, if the employer's participation exceeds 50~\% of the fixed-amount value of the meal, we must add to the contribution tax base the value of the food in-kind compensation, equivalent to the fixed-value of the meal minus the employee's participation. \newline

\textbf{Example}

In 2014, the fixed-value of a meal is equal to 4.60 euros. Let the employee's participation to each meal being equal to 1 euro (then less than 50~\% of the fixed-value of the meal). The amount to be incorporated to the contributions tax base is equal to 3.60 euros per meal:
\begin{equation}\nonumber
IKC = (4,6 - 1)\times NBM
\end{equation}

where $IKC$ corresponds to the value of the in-kind compensation, and $NBM$ is the number of meals the employee actually had.


%%%%%%%%%%%%%%%%%%%%%%%%%%%%%%%%%%%%%%%%%%%%%%%%%%%%%%%%%%%%%%%%%%%%%%%%%%%%%%%%%%%
%%%%

\subsection{Business expenses}

The labor code draws a distinction between the concept of \textit{salary} and the one of \textit{business expenses}. The employer can be asked to pay some compensation for what their employees had to spend as part of their professional activity, and which is deduced from the salary. In particular, it includes:
\begin{itemize}
\item Expenses incurred by the employee when they make business travels using their own private mean of transport;
\item Meal expenses;
\item Double residence costs;
\item House-moving costs;
\item Professional premises expenses;
\item Training spending;
\item Clothing expenses;
\item Contributions paid to an employees' trade union;
\item Interests paid for the loan taken to buy the company's shares.
\end{itemize}


\subsection{Accounting and social and tax regimes}

The compensation for business expenses can take one of the three following forms:
\begin{itemize}
\item The repayment of actual spending (upon presentation of a written evidence);
\item The payment of a fixed-amount compensation;
\item The application of flat-rate specific deductions for some occupations which must bear high business expenses.
\end{itemize}

\subsubsection*{Actual cost scheme}

If the employer has in his possession the evidence that his employee is forced to incur expenses (in compliance with the object declared for these expenses), and the written proof of these spending, the amounts the employer repays as part of business expenses are \emph{excluded} from the social security contributions and income tax base.

Spending incurred for the purchase of new information and communication technologies equipment, or of material allowing the employee to telecommute, are necessarily counted up under the actual cost scheme.

\subsubsection*{Fixed-amount compensation scheme}

\textbf{General case}

Although the employee is not required to provide written evidence, the employer must be able to prove that the compensations are used in compliance with their object if he wants the fixed-amount compensations paid to his employee to be excluded from the contributions tax base. Fixed-amount compensation can be paid to the employee for travels expenses (use of the company or private car when the employee cannot use public transportation and when his place of residence is not abnormally far from his workplace) as well as for occupational mobility. 

In case of transportation cost, kilometric scale defined by the tax administration are used by the employer to decide of the fixed-amount compensations.

If no evidence can be given, the compensation paid must be considered an in-kind compensation and incorporated to the social security contributions and income tax base. This is especially the case when the employee uses his car for a personal convenience motive. \newline
%% Manque un mot?
%Si la preuve ne peut �tre produite que les indemnit�s forfaitaires, l'indemnit� vers�e doit �tre consid�r�e comme un avantage en nature et r�int�gr�e  � la base des cotisations sociales et au revenu imposable. C'est le cas notamment lorsque le salari� utilise sont v�hicule pour des raisons de commodit� personnelle. \newline

\textbf{Accommodation and food}

In the case of accommodation and food expenses, it is said to be more difficult than for other business expenses purposes to provide a proof that fixed-amount compensations have been used in compliance with their purpose. Exemption limits have been set for repayments taking the form of fixed-amount compensations, in the case of meals for employees having a particular work schedule, or of accommodation expenses of employees during a business trip. The exemption limits differ whether the business trip takes place in mainland France, in the Overseas departments and territories, or abroad\footnote{Exemptions limits amounts can be looked up in paragraphs 22630 to 226670 of the \emph{M\'emento pratique Social 2013}, ed. Francis Lef\`ebvre.}.
 
In the event that fixed-amount compensations paid by the employer exceed the exemption limits defined, two cases can emerge: 
\begin{enumerate}
\item If the employer provides written evidence that the fixed-amount compensation has been used in compliance with its purpose, the whole compensation is excluded from the contributions tax base;
\item If the employer cannot provide such evidence, the difference between the fixed-amount compensation paid and the limits set must be incorporated to the contributions and income tax base, as an in-kind compensation.
\end{enumerate}



%%%%%%%%%%%%%%%%%%%%%%%%%%%%%%%%%%%%%%%%%%%%%%%%%%%%%%%%%%%%%%%%%%%%%%%%%%%%%%%%%%%
%%%%

\subsection{Voluntary employer's contributions}

\subsubsection{The employer's support of the employees social security contributions' cost}

\subsubsection*{Social regime}

Until the 2003 pension reform, the employer taking charge of the cost of the employee's contributions for the complementary pension scheme and contingency funds was \emph{not} considered part of the salary: although it was a voluntary contribution from the employer, it was not included within the contributions tax base. In 2014, it is still the case for the employer taking charge of the cost of the employee's contribution to a \textbf{contingency fund} and to a \textbf{supplementary pension plan}. However, with regards to the \textbf{complementary  pension scheme}, if the employer chooses to bear the cost of the employee's contribution, it is considered part of the salary and subjected to social security contributions. Yet, if it is made in order to maintain part-time employees' social security contributions, or for contributions paid during  a family leave, the employer's support is subjected to the \textit{forfait social} (i.e. a contribution paid by the employer on direct or indirect compensations not subjected to contributions but subjected to the CSG).

The following table sums up the social regime of the amounts paid by the employer and corresponding to some social security contributions\footnote{It should be noticed that when the employees' contributions paid by the employer are subjected to social security contributions, they are also subjected to payroll taxes.}.

\begin{table}[h!]
\begin{center}
\caption {ocial regime of the employer's taking over of some employees' contributions} \label{tab:title}
\small
\begin{tabular}{|c|c|c|c|}
\hline
\multicolumn{2}{|c|}{Complementary pension scheme} & Supplementary &   Complementary \\
\textit{General case} & \textit{Special cases} & pension plan  & Contingency fund \\ \hline
Social security contributions & Forfait social & & Forfait social \\
 and CSG-CRDS & (full rate) & CSG-CRDS & (reduced rate) \\
 &  and CSG-CRDS & & and CSG-CRDS \\
\hline
\end{tabular}
\end{center}
{\footnotesize \textsc{Note:} The \textit{forfait social} is a contribution created in 2009. Until 2011, the contingency funds contributions were subjected to a specific 8\% contribution. It has been incorporated to the \textit{forfait social}.}
\end{table}

Other special cases: refer to the \S6910 of \citet{lefebvre2013paie}.

\texttt{Legislative references:} {Law 2003-775 of 08/21/2003 reforming pensions schemes (OJ of 08/22/2003).}

\subsubsection*{Tax regime}

The contributions funding the supplementary pensions schemes and complementary contingency funds can be deducted from the income tax base (as the social security contributions) up to some annual limits: beyond these, contributions are incorporated to the income tax base as an element of earnings.

\subsubsection{Employers' contributions}

Contributions and taxes paid by employers are not considered part of earnings, and as such, are not subjected to contributions and payroll taxes. Three exceptions should be mentioned:
\begin{enumerate}
\item Employers' contributions for the funding of \textbf{mandatory complementary pension schemes} are exempted from contributions, payroll taxes and CSG-CRDS without any limit;
\item Employers' contributions for the funding of \textbf{supplementary pension plans} and for \textbf{contingency funds} are subjected to CSG-CRDS, and excluded from the social security contributions up to some limits and under some conditions. They are then subjected to the \textit{forfait social};
\item Employers' contributions to \textbf{defined benefits pension plans} are not part either of the contributions (and payroll taxes) tax base nor of the CSG-CRDS tax base, nor of the \textit{forfait social} tax base. But, since 2004, they are subjected to a special contribution (levy on "top hat pension plans").
\end{enumerate}

\texttt{Legislative references:} {Art. L 242-1 of the SSC}

\subsubsection*{Employers' contributions for supplementary pension schemes and complementary contingency funds}

Employers' voluntary contributions intended to finance supplementary pension schemes and complementary contingency funds can be exempted from social security contributions under some conditions. These conditions depend on the type of benefits provided by the pension and contingency plan, on the organization in charge of managing it, on the terms and conditions of its implementation, and on the collective and mandatory feature of the scheme. For additional details, see \citet{lefebvre2013paie}, \S 6920 to 6931.

Exemption limits are defined according to the SSC, to the earnings subjected to contributions and to a potential employer's contribution to an employee's savings plan. They are determined each year by decree. For additional details, see \citet{lefebvre2013paie},  \S 6934 to 6936.

Moreover, these contributions are subjected to the CSG and the CRDS \textit{without} applying the flat-rate deduction for business expenses since January 1\up{st}, 2012 \footnote{The tax abatement effective for other elements of earnings being equal to 1.75~\% since 2012.}.

\subsubsection{The employer's participation to a mutual health insurance}

The employer can partly support the purchase of a mutual health insurance for his employees. Before, this employer's participation was not taxable: although it is a form of in-kind compensation, it was not included in taxable earnings. The 2014 Finance act has changed this rule: the employer's participation to the purchase of a mutual health insurance on earnings paid starting from January 1\up{st}, 2013 is incorporated to the taxable income as part of earnings. \newline

\textbf{Example:}

We assume for what follows monthly that gross earnings equal to 3000 euros, a total employer's contribution rate of 60~\% and a total employee's contribution rate (CSG and CRDS excluded) of 15~\%\footnote{We also assume that employers'  and employees' contribution rates remained unchanged between 2012 and 2013.}. The deductible CSG rate is equal to 5.10~\% in 2012 and 2013. The CSG must be computed by applying first a tax abatement equivalent to 1.75~\% on gross earnings.
 
\begin{table}
\begin{center}
\centering
\small
\begin{tabular}{|l|c|c|}
\cline{2-3}
\multicolumn{1}{c|}{} & Until 2013 & Since the 2014 Finance Act \\
\hline
Gross earnings & 3000 \euro & 3000 \euro \\
Employer's participation to the mutual health insurance & 90 \euro & 90 \euro \\
Employee's participation to the mutual health insurance & 40 \euro & 40 \euro \\
Employer's contributions & 1800 \euro & 1800 \euro \\
Employee's contributions & 450 \euro & 450 \euro \\
Deductible CSG & 150 \euro & 150 \euro \\
Labor cost & 4890 \euro & 4890 \euro \\
Taxable earnings &  2400 \euro & 2490 \euro \\
\hline
\end{tabular}
\end{center}
\end{table}

Until the income tax paid in 2013 (on 2012 earnings), the taxable earnings was equal to: \textit{gross earnings - employees' contributions (CSG and CRDS not included) - deductible CSG}.

However, the taxable earnings for the 2014 income tax (paid on 2013 earnings) is equal to: \textit{gross earnings - employees' contributions (CSG and CRDS not included) - deductible CSG + employer's participation to the purchase of a mutual health insurance}. We can recover the 90 euros of employer's participation in the gap between taxable earnings under the new legislation and what the taxable earnings would have been if we were to apply to former tax regime.

Then, since 2014, although the employer's participation to the purchase of a mutual health fund is still fully exempted from social security contributions, it is however classified as an in-kind compensation when it comes to employee's income taxation. It should be noticed that the employee's participation to the purchase of a mutual health insurance, if deducted from gross earnings for the computation of net earnings to be paid in cash, is included in the net table earnings, and then is not deducted from gross earnings in the computation. \newline
 
\texttt{Legislative references:} {Law 2013-1278 (2014 Initial Finance Act) of 12/29/203 (OJ of 12/30/2013), art. 4}.


%%%%%%%%%%%%%%%%%%%%%%%%%%%%%%%%%%%%%%%%%%%%%%%%%%%%%%%%%%%%%%%%%%%%%%%%%%%%%%%%%%%
%%%%

\subsection{Employees' savings, incentive bonus and profit-sharing}

Some measures on earnings, coupled with attractive social and tax regimes, have been implemented to favor employees' shareholding, to give incentives for employees to save, and to include employees to the firm's financial performance. A given employee within the same company can benefit from several of these schemes; they  have a collective feature, to the extent that they are generally defined for the whole workforce of a given employer. Social and taxation privileges are granted only if these elements of earnings do not replace the base pay.

The incentive bonus and employee's shareholding and saving plans are not mandatory; on the contrary, the set up of a profit-sharing scheme is mandatory for companies of 50 employees and more, voluntary for the others.

\subsubsection{Profit-sharing}

Firms must amass a special profit-sharing fund (defined according to the net profit, to the shareholder's equity, to the added value and to the earnings paid). Each recipient employee owns a specific entitlement to a part of the fund. Each time the company transfers money to the fund, any employee can decide either to have it available only after a blocked period, or to enjoy it immediately. In the first case, the blocked period is equal to 5 or 8 years. 


\subsubsection*{Social regime}

The amounts transfered to the special fund are not considered an element of earnings (including) for the Smic computation and the increase in wage rate for overtime hours. Besides, these amounts are not subjected to social security contributions nor to equivalent contributions (unemployment contributions, Fnal, contribution to local public transport infrastructures). However, they are subjected to CSG-CRDS and to the \textit{forfait social}. If these amounts are withdrawn before the blocked period comes to an end, the amounts earned as part of profit-sharing are subjected to contributions and payroll taxes, as an element of earnings.

\subsubsection*{Tax regime}

The sum of money into the special fund is excluded from the emloyee's income tax base, except if the blocked period is not respected.


\subsubsection{Incentive bonus}

The incentive bonus consists in paying extra earnings to employees, calculated according to the company's performance. This scheme is not mandatory, except if the company does not provide a saving plan to its employees, in which case it becomes mandatory to negotiate an incentive bonus scheme each year within the company.

The individual bonus amount is defined according to the job tenure or to the earnings, but the employer can also choose an even distribution. Each bonus is bounded by an upper limit. Moreover, the total of individual bonuses cannot exceed 20~\% of total wage bill in a year. It the bonus exceeds this ceiling, the above mentioned social and taxation privileges do not apply anymore to the fraction of the bonus above the threshold. 


\subsubsection*{Social regime}

Incentive bonus is not considered part of earnings: in particular, it is not taken into consideration for the Smic computation and the increase in wage rate for overtime hours. Besides, these amounts are not subjected to social security contributions nor to equivalent contributions (unemployment contributions, Fnal, contribution to local public transport infrastructures). However, they are subjected to CSG-CRDS, to the \textit{forfait social}, and, since January 1\up{st}, 2013, to payroll taxes.


\subsubsection*{Tax regime}

Incentive bonus is subjected to income tax, except if it is assigned to an employee's saving plan.


\subsubsection{Employees' saving plan}

Employees' saving plans allow employees to increase their income by constituting a securities portfolio. We can distinguish three types of employees' saving plans:
\begin{enumerate}
\item The \textbf{Company saving plan} (\textit{Plan d'\'epargne entreprise (PEE)});
\item The \textbf{Inter-companies saving plan} (\textit{Plan d'\'epargne inter-entreprises (PEI)});
\item The \textbf{Saving plan for collective pension} (\textit{Plan d'\'epargne de retraite collectif (Perco)}). The sum of money invested in the Perco has the peculiarity to be available only when the employee retires (except in case of early release). Indeed, these plans aim at amassing savings for the employees' pensions, available as annuity or capital, depending on the collective agreement that has been reached within the company.
\end{enumerate}

The employee is required to regularly transfer money in the saving plan, completed with transfers made by the employer. The employee can also choose to transfer the sum of money he earned as part of the incentive bonus or the profit sharing scheme. The employer's contribution to the saving plan is capped (to 3 004 euros per year per employee in 2014 for PEE and PEI, and 6 008 for a Perco, equivalent respectively to 8~\% and 16~\% of the 2014 SSC), and cannot exceed three times what has been transferred by the employee during the year.

The amounts of money transferred in the plan are not available for a certain period of time (varying according to the type of plan), but can be released in advance under some specific conditions (death or disability of the employee or his spouse, indebtness, etc.). If the money release occurs outside these conditions and before the end of the blocked period, the amounts of money are considered classic elements of earnings.

% http://vosdroits.service-public.fr/particuliers/F10260.xhtml#N101A7

\subsubsection{Social and tax regimes of the employer's contribution}

\subsubsection*{Employer's contribution of the company saving plan (\textit{plan d'\'epargne entreprise} (PEE))}

As for the profit-sharing scheme, the employer's contribution is not considered part of earnings by the labor law. It is not subjected to social security and equivalent contributions (unemployment contribution, contribution Fnal (National fund for housing aid), contribution to finance the local public transport infrastructures...), and to payroll taxes (apprenticeship tax, etc.). However, it is subjected to CSG-CRDS and to the \textit{forfait social}, as well as the wage tax since January 1\up{st} 2013.

Recipient employees do not have to include the employer's contribution to their taxable income, provided that the sums of money transferred by the employer are kept at least 5 years in the saving plan.

\subsubsection*{Employer's contribution to an inter-companies saving plan (\textit{plan d'\'epargne inter-entreprise} (PEI))}

As its name suggests, a PEI is a joint saving plan shared by several companies. Apart from some specificities, the PEI can work either as a Perco, either as a PEE, depending on what the signatory companies have decided.

\subsubsection*{Employer's contribution to a common saving plan for pension (\textit{plan de retraite collectif} (Perco))}

The employer is not liable for the payment of social security contributions, in general, for its contribution to a Perco. But, its contribution is taken into account to compute the (capped) deduction of employer's contributions for contingency funds and supplementary pension schemes from the contributions tax base.  It is not subjected to CSG-CRDS and the \textit{forfait social}.

Besides, if the employer's contribution exceeds a certain amount (2 300 euros in 2014) per employee, the sum above the threshold is subjected to a special contribution of 8.2~\% owed by the employer, and different from the \textit{forfait social}.

\subsubsection{Social and tax regimes of the employees' payments}

These payments are not subjected to social security contributions nor deductible from the income tax, either if they are mandatory or voluntary, except if they come from the profit-sharing scheme. The incentive bonus transferred into a Perco is exempted from income tax but only up to half of the SSC.

\subsubsection{Social and tax regimes of the withdrawals from the Perco}

The sums of money withdrawn from the Perco (usually when the employee retires) can be paid in two ways:
\begin{itemize}
\item as capital: interests are not subjected to income tax but are subjected to taxes and contributions on capital.
\item as life annuities: the sums of money earned are subjected to income tax as well as taxes and contributions on capital. To delimit the taxable base, a tax abatement is applied to the annuity, whose rate increases with the recipient age when he starts benefiting from the annuity.
\end{itemize}


%%%%%%%%%%%%%%%%%%%%%%%%%%%%%%%%%%%%%%%%%%%%%%%%%%%%%%%%%%%%%%%%%%%%%%%%%%%%%%%%%%%
%%%%

\subsection{Shareholding of employees}

We can distinguish two main employees' shareholding schemes:  purchase of stock options or free script issue. They can be implemented only in joint stock companies, and are not mandatory.

\subsubsection{Purchase of stock options}

This type of scheme consists in allowing the employees (or directors) to buy or subscribe to stock options of the company or of another company belonging to the same group, under attractive conditions.

When the employee is allowed to buy a company's share, he potentially enjoys two advantages:
\begin{enumerate}
\item The purchase gain, defined as the difference between the value of the option when it is purchased and its buying price; 
\item The capital gain, defined as the difference between the value of the option the day it is sold and the value of the option the day it is purchased.
\end{enumerate}

Since 2012, the beneficiary employees are liable for the payment of a 10~\% contribution on the purchase gain\footnote{The company itself is liable for a specific contribution. The specific contributions of the beneficiary employee and of the company are owed only for share attributed to employees affiliated to a French Social security scheme for sickness insurance.}. As a general rule, the purchase gains are exempted from social security contributions and payroll taxes (except from the wage tax). Yet, they are subjected to the CSG-CRDS (with the same rates as those which apply for earnings, and not for capital income) and to the wage tax if they have been allocated since September 28\up{th} 2012 or if they do not respect some conditions (especially the nonavailability condition). 

The tax regime is similar to the classic tax regime on property sales (income tax and contributions and payroll taxes).

\subsubsection{Free script issue}

Companies can decide to allocate their shares for free to some of their employees. Shareholders gathered in a general assembly choose the identity of the recipients as well as the shares' purchase and preservation periods. 

When the employee is granted a bonus share, he potentially enjoys two advantages:
\begin{enumerate}
\item The direct advantage related to the free allocation of the share, equal to the value of the share the day it is allocated;
\item The capital gain from the sale.
\end{enumerate}

\begin{table}[h!]
\begin{center}
\caption {Social and tax regimes of the bonus share issue.}
\small
\begin{tabular}{|l|c|c|}
\cline{2-3}
\multicolumn{1}{c|}{} & Shares allocated  & Shares allocated \\
\multicolumn{1}{c|}{} & before the 09/29/2012 & since the 09/29/2012 \\
\hline
Gain at the allocation & Exempted from  & Exempted from   \\
& Social security contributions & Social security contributions \\
 & Exempted from CSG-CRDS           & Subjected to CSG-CRDS \\
& 30\% taxation            & Subjected to the  \\
& and contributions and payroll taxes       & income tax scale as earnings \\
& on property\textsuperscript{*}                & \\
\hline
Gain on the sale & Conditions of the    & Conditions of the \\
 & capital gains taxation     & capital gains taxation    \\
\hline
\end{tabular}
\end{center}
{\footnotesize \textsc{Note:} \textsuperscript{*} The taxation is made according to the income tax scale on earnings when the preservation period has not been respected.}
\end{table}


Moreover, bonus shares allocated since 10/16/2007 are subjected to specific employers' and employees' contributions, respectively at 30\% and 10\% in 2014 (applied on the gain at the allocation).\newline

\texttt{Legislative references:} {art. L 137-13 and art. L 137-14 of the SSC, created by the Law 2007-1786 of 12/19/2007 (2008 Social Security Finance Act), art. 13} \footnote{For taxation conditions on capital gains, whether they are made on shares purchased at their market value or on bonus shares, see \texttt{Art. 150-0 A of the GTC}.}.


%%%%%%%%%%%%%%%%%%%%%%%%%%%%%%%%%%%%%%%%%%%%%%%%%%%%%%%%%%%%%%%%%%%%%%%%%%%%%%%%%%%
%%%%

\subsection{Daily Social security benefits}

Base daily social security benefits (\textit{Les indemnit\'es journali\`eres de S\'ecurit\'e sociale} (DSSB), i.e. benefits for sickness leave) are not subjected to contributions, but to CSG and CRDS, at specific rates but without applying the tax abatement for business expenses. Besides, they are subjected to income tax. Some daily social security benefits are partially or fully exempted; as an example,  the DSSB paid as a compensation for a work accident or a professional disease are exempted at 50~\%. 

The employer can pay additional benefits to his employees, as a complement to DSSB. Apart from any agreement reached within the company or the industry, their amount depends on the worker's company seniority. These additional benefits are subjected to CSG and CRDS as earnings (thus, at a higher rate than for DSSB).


%%%%%%%%%%%%%%%%%%%%%%%%%%%%%%%%%%%%%%%%%%%%%%%%%%%%%%%%%%%%%%%%%%%%%%%%%%%%%%%%%%%
%%%%

\subsection{Termination benefits}

There are different types of benefits that could be paid in case of work contract severance. The main ones are the following:
\begin{itemize}
\item Layoff pay;
\item Benefits for irregular and unfair dismissal;
\item Benefits for the early termination of a fixed-term contract;
\item Benefits paid as part of a job-protection plan.
\end{itemize}

These benefits are limited by an upper bound generally set at 10 SSC (it was equal to 30 SSC until 2012). The social and tax regime of these termination benefits is generally pretty complex because of the numerous types of benefits; the following table draws a simplified overview of the social and tax regimes these benefits belong to.

\begin{table}[h!]
\begin{center}
\caption{Social and tax regimes of some termination benefits.}
\vspace{0.2cm}
\small
\begin{tabular}{|>{\centering\arraybackslash}m{3cm}| >{\centering\arraybackslash}m{3.1cm}| >{\centering\arraybackslash}m{3.1cm}|>{\centering\arraybackslash}m{3.1cm}|}
\cline{2-4}
\multicolumn{1}{c|}{} & Social security contributions & CSG-CRDS & Income tax \\
\hline
Layoff pay & Exemption limited to 2 SSC, except if the benefits are higher then 10 SSC &  Exemption whose amount is limited by a collective agreement (maximum is 2 SSC) & Full or partial exemption  \\
\hline
Voluntary separation package & Subjected & Subjected after a tax abatement for business expenses & Taxable (except approved contractual termination) \\
\hline
\end{tabular}
\end{center}
{\footnotesize \textsc{Note:} The social scheme regarding social security contributions is effective for all the contributions and taxes whose tax base is harmonized with the social security contributions tax base.}
\end{table}


%%%%%%%%%%%%%%%%%%%%%%%%%%%%%%%%%%%%%%%%%%%%%%%%%%%%%%%%%%%%%%%%%%%%%%%%%%%%%%%%%%%
% Section 3: r�mun�rations dans le secteur public
%%%%%%%%%%%%%%%%%%%%%%%%%%%%%%%%%%%%%%%%%%%%%%%%%%%%%%%%%%%%%%%%%%%%%%%%%%%%%%%%%%%

\section{Elements of earnings in the public sector}

Earnings of tenured civil servants is determined according to statutory or regulatory provisions decided by the legislator. Then, whereas it is not the case in the private sector, earnings paid to civil servants do not depend on individual or collective bargaining. Moreover, the civil servants' earnings are composed of several elements that are different from the extra elements of earnings in the private sector. Earnings comprise a base pay and some benefits and statutory bonuses, as well as additional earnings defined according to the number of children of the worker\footnote{\texttt{law 83-634 of 07/13/1983 related to the rights and commitments of the civil servants (OJ of 07/14/1983), art. 20.}}.

With regards to non-permanent civil servants, their earnings are, by nature, more similar to the salary earned in the private sector, as they are set in a contract. 

\subsection{Tenured civil servants}

\subsubsection{Main earnings}

The main earnings are composed of two elements: the \textit{Traitement indiciaire brut} (TIB) and the \textit{Nouvelle bonification indiciaire} (NBI).

The TIB corresponds to the base pay. It is defined as the product of a salary scale grade (called \textit{indice de traitement}) and the value of the index point. The index point has a unique definition applying for all the civil servants whereas the value of the salary scale grade depends on the position and the rank of the civil servant into the hierarchy, as well as on the professional corps he belongs to.

The NBI, created in 1990, consists in allocating a certain number of grade points on the salary scale, which, multiplied by the value of the index point, give extra earnings added to the base pay (TIB). The NBI is granted to some civil servants who have a specific position into the hierarchy, or who belong to particular corps. The number of grade points granted depend on the category (A, B or C) of the civil servant. Although it is considered a bonus, the NBI is assimilated to the base pay regarding social security contributions. \newline

\underline{Social regime:}

The TBI and the NBI are subjected to all the contributions and payroll taxes (social security contributions, equivalent contributions, contribution to the national fund for housing aid, transportation payment).


\subsubsection{Family supplement to base pay}

The family supplement to base pay (\textit{Supplement familial de traitement} (SFT)) corresponds to extra earnings granted to civil servants having at least one dependent child.\newline

\underline{Social regime:}

The SFT is not subjected to all contributions and payroll taxes, but only to the flat-rate income taxes (CSG and CRDS) and to the contribution for the additional pension scheme for the public sector (\textit{R\'egime de la retraite additionnelle de la fonction publique} (RAFP)). It is also included in the tax base of the exceptional solidarity contribution and of the wage tax.


\subsubsection{Benefits and bonus}

In addition to their base pay, tenured civil servants earn, on a statutory basis, some benefits and bonus related to their situation: mobility allowance, residence allowance, hourly allowance for additional work, profession bonus, performance reward, etc.\newline

\underline{Social regime:}

Bonus and allowances are excluded from most of the contributions tax bases. However, they are incorporated to the tax base of the contribution for the mandatory complementary pension scheme (RAFP) as well as the tax base for the exceptional solidarity contribution. They are also subjected to the wage tax and to flat-rate income taxes.


\subsubsection{In-kind compensation}

Along with the private sector employees, civil servants can benefit from some goods or services necessary for carrying out their activity (accommodation or company car, portable computer equipment, etc.). The provision of these goods and services is considered an in-kind compensation, whose value is estimated according to the rules applying in the private sector (see \emph{supra}).\newline

\underline{Social regime:}

In-kind compensation is only subjected to the exceptional solidarity contribution, to the contribution for the funding of the mandatory complementary pension scheme (RAFP), and to the CSG and CRDS. They are excluded from the social security contributions tax base (contributions for old-age insurance, sickness-maternity-disability-death insurance, family benefits).


\subsection{Non-permanent civil servants}

Non-permanent civil servants are liable for the payment of contributions for old-age insurance and sickness insurance to the general social security scheme, of contributions for their own complementary pension scheme (IRCANTEC), of the exceptional solidarity contribution, and of flat-rate income taxes (CSG-CRDS) on all their earnings -- the base pay, including bonuses and in-kind compensation. 


%%%%%%%%%%%%%%%%%%%%%%%%%%%%%%%%%%%%%%%%%%%%%%%%%%%%%%%%%%%%%%%%%%%%%%%%%%%%%%%%%%%%%%%%%%
%Section 4: travaileurs ind\'ependants
%%%%%%%%%%%%%%%%%%%%%%%%%%%%%%%%%%%%%%%%%%%%%%%%%%%%%%%%%%%%%%%%%%%%%%%%%%%%%%%%%


\section{Elements of earnings of non-farming self-employed workers}

Self-employed workers benefit from a specific social and tax regime, different both from the one of wage-earners, and from the one of the self-employed workers in the farming sector. The social security contribution tax base delimitation is then simplified, as self-employed workers have not as many elements of earnings as private and public sectors employees. We can draw a distinction between two categories of self-employed workers: on the one hand, the liberal professions -- bringing together doctors, lawyers, etc. -- and on the other hand, the craftsmen, traders and manufacturer. Liberal professions are required to pay their contributions on their non-commercial profit (NCP, \textit{b\'en\'efices non commerciaux}) whereas craftsmen, traders and manufacturers have their contributions levied on their industrial and commercial profit (ICP, \textit{b\'en\'efices industriels et commerciaux}). The tax base definition -- described in further detail hereafter -- applies for most of the social security contributions, the CSG and the CRDS.

\subsection{Non-commercial profit of liberal professions}

A liberal profession can be carried out either under an individual enterprise, with the possibility to take on a special status -- the \textit{auto-entrepreneur} status -- or under a company form. Each status is regulated by different rules in terms of social and tax regime.

As the tax base used for the social security contributions is defined in reference to taxable earnings, we should first define the tax regime applying to non-commercial profit of liberal professionals, and then examine the social regime.


\subsubsection{Tax regime}

The tax regime the liberal professional is subjected to depends on whether his activity takes the form of a sole proprietorship or a company, and on the profit amount.

In the case of a sole proprietorship, profit is subjected to income tax as non-commercial profit. Two cases must be distinguished:
\begin{itemize}
\item If annual gross receipts are lower than 32 900\euro\: the liberal professional benefits from the micro-enterprise regime. The income tax base corresponds to fixed-amount profit, that is to say gross receipts from which we deduct a flat-rate abatement for business expenses equal to 34~\% of gross receipts. The minimum abatement is equal to 305\euro\.\\
If the liberal professional opted for the payment in full discharge of the income tax (\textit{le versement lib\'eratoire de l'imp\^ot sur le revenu}) it means that he will pay the income tax at a flat-rate, at the same time that he pays his social security contributions. A 2.2~\% rate is therefore applied the total gross receipts earned within the period.
\item If annual gross receipts exceed 32 900\euro\, or if the liberal professional decided so, he can benefit from a controlled tax return (\textit{d\'eclaration contr\^ol\'ee}). In that case, the tax base is made up of receipts actually earned over the year minus professional expenses actually paid over the same year, except if the professional opts to take into account the established debts and the expenses incurred. The receipts are taken into consideration whatever the method of payment and the date the service will be made. In case of a deficit, it is charged to the income tax base of the household.
\end{itemize}

Liberal professional can also choose to pay the corporate tax if their activity falls under the status of the \textit{Entreprise individuelle \`a r\'esponsabilit\'e limit\'ee} (sole proprietorship with limited liability).\newline

%Selon le montant des recettes encaiss\'ees au cours de l'ann\'ee civile, le professionnel lib\'eral est plac\'e sous un r\'egime d'imposition diff\'erent: dans tous les cas, l'assiette consid\'er\'ee correspond aux b\'en\'efices de l'entreprise, ce qui \'equivaut au montant des recettes bruttes auquel on d\'eduit un abattement forfaitaire pour frais professionnels \'egal \`a 34\% des recettes bruttes. L'abattement minimum est de 305\euro\ 
In the case of an activity carried out within a company, here again, two cases should be distinguished:
\begin{itemize}
\item People belonging to a non regulated occupation can choose all the "classic" types of companies, even commercial forms, their activity being still liberal;
\item In the case of regulated liberal occupations, the professionals can choose either commercial or private companies types (SCM, SCP or SEL). According to the type of company chosen, the tax regime can vary. For instance, in the case of a Professional civil company (PCC, \textit{Soci\'et\'e civile professionnelle}) allowing several members of the same liberal profession to create a partnership with a shared activity, each professional is individually taxed on his own part of the profit, as non-commercial profit (NCP). If the professional opts for a SEL (\textit{Soci\'et\'e d'exercice lib\'eral}, i.e. a joint-stock company for liberal professions), he is subjected to the commercial accounting rules. Therefore, the income tax base takes into consideration established debts -- even those not paid off -- and the incurred expenses -- even those not paid.
\end{itemize}

To get the income tax base, we must deduct from gross professional earnings -- as defined above -- the personal mandatory social security contributions paid within the considered year for earnings of the preceding year. 


\subsubsection{Social regime}

The social security contributions tax base for liberal professions consists of total taxable earnings from the liberal activity. If the liberal activity is supplemented with a secondary craft or trade activity, the earnings yielded are added to the tax base for the sickness contribution. \\

More precisely gross earnings from the self-employed activity are made up of \textbf{gross profit}, in the case the liberal professional works in a sole proprietorship, or of the professional's \textbf{part of gross profit}  if he works within a company subjected to income tax, or of the professional's \textbf{salary} in case he works within a company subjected to corporate tax.

To get the income tax base, as defined in the preceding paragraph, we must deduct from gross professional earnings the personal mandatory social security contributions paid within the considered year for earnings of the preceding year. 

To get the social security  contributions tax base, starting from the income tax base, we must:
\begin{itemize}
\item \textbf{Not} take into consideration the professional long-term capital gains or losses, the losses carried forward (defined in art. D 612-7 of the SSC), the reserves for investment or for compliance upgrade expenses in terms of food security for sole proprietorships subjected the real value taxation regime, and for EURL (limited liability enterprise with a unique associate). We must not take into account either the multiplier coefficient mentioned in 7 of article 158 in the GTC (General Tax Code), granted to self-employed workers who do not benefit from the income tax abatement applying for members of authorized management centers and associations;
\item \textbf{Add} the part of \textbf{dividends} earned by the professional and exceeding 10~\% of the enterprise's capital stock or 10~\% of the property assets allocated to the enterprise in case of \textit{Entreprise Individuelle \`a Responsabilit\'e Limit\'ee} (sole proprietorship with limited liability);
\item Reincorporate the \textbf{10~\% abatement} for business expenses used for the computation of the income tax base;
\item \textbf{Not} apply the different tax reductions for companies located in priority areas for the urban planning, in urban free zones, in areas of research and development of a competitive cluster, for young innovative companies, etc.;
\item Partly deduct optional personal contributions paid within the year. For the part of the contribution paid to mandatory complementary pension schemes exceeding the mandatory minimal contribution, and for payments to a "group insurance contract"\footnote{Voluntary collective insurance contract offering additional sickness or old-age benefits}, is deductible:
	\begin{itemize}
	\item  For the old-age insurance, the amount up to a limit equal to the highest of the following two amounts: 10~\% of the fraction of taxable profit under 8 annual SSC plus 15~\% of the fraction of taxable profit between 1 and 8 annual SSC; or 10~\% of the annual SSC. If the company contributes to a collective saving plan for retirement pension, the amounts paid are deducted from these limits;
	\item For contingency funds, the thresholds corresponds to 7~\% of the annual SSC plus 3.75~\% of taxable profit, if the total does not exceed 3~\% of eight annual SSC; 
	\item For layoff risk, the limit is equal to the highest of the following two amounts: 1.875\% of the fraction of taxable profit under 8 annual SSC, or 2.5~\% of the annual SSC.
	\end{itemize}
\item \textbf{Add} income from the renting of part or all the business, if it is a craft, trade or industrial establishment equipped with all the necessary furniture for its operation and if this income is earned by an individual carrying out an activity as part of the rented enterprise;
\item \textbf{Exclude} all other types of income such as property or investment income, wage, etc. However, farming self-employed income can possibly be incorporated.
\item \textbf{Not reincorporate} mandatory social security contributions deducted for the computation of the income tax base.
\end{itemize}

This definition of earnings is used as tax base for sickness-maternity-disability-death-daily benefits contributions, for basic pension contributions of liberal professionals, for basic and complementary pensions contributions for self-employed lawyers, for family allowances contributions, and, subjected to some particularities, for the CSG and CRDS. Disability or retirement pensions are not subjected to contributions except for self-employed workers whose fiscal residence is abroad and who contribute at a 2.8~\% rate on their basic pension (directly levied by the old-age insurance fund). \newline


\texttt{Legislative references:} Art. L 131-6 of the SSC; \textit{Memento pratique social 2012}, ed. Francis Lef\`ebvre; \textit{Memento pratique fiscal 2012}, ed. Francis Lef\`ebvre

\subsubsection{CSG and CRDS tax base}

Tha CSG and CRDS tax base is identical for all non-farming self-employed workers. It is roughly similar to the social security contributions tax base, with some differences:

We must reincorporate to the social security contributions tax base the personal social security contributions paid by the self-employed worker and his spouse when they are deductible from from the social security  contributions and income tax base.

Sickness-maternity insurance cash benefits are included in the income tax base, and then, in the social security contributions tax base. For the CSG payment, a reduced rate is applied to these sums of money. Regarding the beneficiaries of the micro-tax regime (see above), these cash benefits are excluded from the social security contributions tax base, but included to the CSG tax base (applying a reduced rate) and to the CRDS tax base.
For the beneficiaries of the micro-social regime, the CSG and the CRDS are included in the overall rate comprising the social security contributions.

With regard to exemptions, they affect the same persons as in the social security contributions case, except for the unemployed entrepreneurs.


\subsubsection{Exemptions and minimal contribution}

A minimal contribution is due in case earnings are below a certain threshold: for the sickness insurance, the threshold is equal to 40~\% of the SSC, and the minimal contribution varies with the level of earnings; for the old-age insurance, the minimal contribution is the one computed with the standard rate on a tax base equal to 7.7~\% of the SSC.
Nonetheless, the sickness insurance minimal contribution is not effective if the worker is retired, benefits from the minimum income or carries out a wage-earning activity as his main activity. Similarly, old-age insurance contributions are paid on real earnings no matter their level for retired people, active people receiving a disability pension, or for people carrying out a wage-earning activity as their main activity. Moreover, the exemption from family contribution and CSG-CRDS which used to apply for people with earnings lower than 13~\% of the SSC and for people aged over 65 who had at least four dependent children, has been removed in 2015. Since January 1\up{st}, 2015, the minimal contribution for the family branch and the CSG and CRDS is not applied anymore.

Exemptions are granted to unemployed entrepreneurs, as well as for entrepreneurs still carrying out a wage-earning activity, for maternity-sickness insurance, basic pension, family allowances and disability-death insurance contributions on the portion of earnings from the self-employed activity below 120~\% of the minimum wage (equivalent to 20 989\euro\ in 2015) during 12 months. Large exemptions have also been implemented for complementary pension contributions. Regarding the old-age insurance contribution, an exemption can be granted to liberal professionals unable to work for more than six months. 

% ref\'erence exo alloc fami CF CSS art L 242-11 al2 et R 242-15

\subsubsection{Terms and conditions of the payment}

Contributions paid by the liberal professionals to fund the health, old-age, family branches of the Social security scheme, as well as the CSG and CRDS, are computed on a provisional and retroactive basis: it means that contributions paid the year $t$ are computed based on profit earned in the calendar year $t-2$. They are then adjusted the next year based on actual profit earned in year $t-1$.

Therefore, the first two years, as earnings from year $t-1$ and $t-2$ are not known yet, the maternity-sickness contribution, the old-age contribution, the family allowances contribution and the CSG and CRDS for liberal professionals outside lawyers, are computed on a tax based equal to 19~\% of the annual SSC the first year, and 27~\% of the annual SSC the second year. 

The contributions amounts are then adjusted once the professional earnings are known.


\subsubsection{Special situations}

Some particular situations go with specific rules with respect to social security contributions, especially on the tax base considered. Although we do no seek to address all these situations, we can however mention the most common ones.


\subsubsection*{Micro-enterprise}

Since January 1\up{st}, 2009, self-employed workers can opt for the micro-enterprise scheme, if their turnover does not exceed a certain threshold, equal to 32 900\euro ~ a year in 2015 (34 900\euro~ if the preceding year's turnover was not over 32 900\euro~). This scheme is coupled with a specific taxation, already mentioned in the paragraph on the tax regime. Regarding social security contributions, they can also choose to join the micro-social regime. This scheme replaces the "social shield" and the possibility  to opt for the quarterly payment of contributions based on actual turnover for the first three years of activity.
At the tax level, this scheme implies the payment of the income tax applying the standard progressive income tax scale to the total income of the tax household and on the turnover after the deduction of a 34~\% abatement for business expenses (with a minimum abatement of 305\euro~). This profit computed in this way, using a flat-rate abatement, is also used as the social security contributions tax base, except if the liberal professional chooses the micro-social regime. Liberal professional can decide to pay the income tax under the income tax payment in full discharge (\textit{versement lib\'eratoire de l'imp\^ot sur le revenu}) if they have chosen the micro-social regime and if their earnings are under a certain threshold: in this case, they are required to add to the quarterly or monthly payment of their social security contributions 2.2~\% of their turnover as part of the income tax.\\

Regarding the social security contributions, micro-entrepreneurs can benefit from the micro-social regime, which consists in applying monthly or quarterly a unique rate to the non commercial profit actually yielded the preceding month or quarter. For liberal professions, this rate is equal to 22,9~\% (outside the option for the income tax payment in full discharge).\newline

 \texttt{Legislative references:} Art L 133-6-8, R 133-30-1, R 133-30-10, and D 131-6-1/2 of the SSC.


\subsubsection*{Unemployed entrepreneurs and the micro-enterprise scheme}

Unemployed entrepreneurs benefiting from the Accre\footnote{The Accre is a scheme designed for unemployed people wishing to create or take over an enterprise consisting in contributions exemptions, the conservation of some benefits and guidance in the different procedures} and having subscribed to the micro-enterprise scheme are granted reduced social security contributions rates.


\subsection{Industrial and commercial profit of craftsmen, traders and manufacturers}

Craftsmen, traders and manufacturers, as well as their spouses, belong to a social regime which depends on the legal status of the enterprise, as well as their participation in the activity. Craftsmen, traders and manufacturers considered non wage-earners, that is to say managing an enterprise belonging to specific legal categories -- \textit{Entreprise individuelle} and \textit{entreprise individuelle \`a responsabilit\'e limit\'ee} (sole proprietorship and sole proprietorship with limited liability), \textit{soci\'et\'es en nom collectif, entreprise unipersonnelle \`a responsablit\'e limit\'ee} and \textit{soci\'et\'e \`a responsabilit\'e limit\'ee} (collective company, limited liability enterprise with a unique associate and limited liability company) -- fall under the Social self-employed scheme (\textit{r\'egime social des ind\'ependants}, RSI).

The social security contributions tax base defined in the section on liberal professions is also effective for craftsmen, traders and manufacturers, who pay their contributions on their industrial and commercial profit (ICP). As the liberal professionals, they can be taxed on their actual profit, or on the profit computed with a flat-rate abatement as part of the micro-enterprise scheme if their profit does not exceed a certain threshold (for more detail, see the paragraph on micro-enterprise). In the case of a standard taxation on actual profit, the taxable profit is equal to the difference between what is yielded and the expenses of the enterprise. This includes the selling of assets. Then, the taxable profit corresponds to the difference between value of net assets at the end and at the beginning of the period on which the income tax is paid, from which we deduct input supplements and to which we add direct debt of the corresponding period. To get the social security contributions tax base starting from the taxable profit, we must add and deduct the same elements as in the case of liberal professions.

This tax base is effective for the computation of the sickness-maternity-disability-death-daily benefits, basic pensions, complementary pension, and family allowances contributions for the craftsmen, traders and manufacturers. The CSG and CRDS tax base is the same as for liberal professions.

However, some particularities are sometimes applied: for instance, in the case of base old-age insurance, craftsmen, traders and manufacturers pay their contributions on the same tax base as liberal professionals, but only up the the Social security ceiling.

%% Assiette CF CSS art L 131-6


\subsubsection{Exemptions and minimal contributions}

The same rules apply for liberal professions on the one hand, and for craftsmen, traders and manufacturers on the other hand, even if some precisions are needed:
In addition to the minimal contributions due by liberal professionals, craftsmen, traders and manufacturers are liable for the payment of other minimal contributions, such as for the daily social security benefits (for sickness leave), the complementary pension and the disability-death insurance contributions. They are respectively equal to the contribution owed on a tax base equivalent  to 40~\% of the SSC, 5.25~\% of the PSS, and 20~\% of the PSS. However, as for the liberal professions, the minimal contribution for the family branch and the CSG and CRDS is not applied anymore since January 1\up{st}, 2015.

The same exemptions granted to liberal professions apply to craftsmen, traders and manufacturers, as for the daily social security benefits. Similarly, the dispensation from family allowances contributions and CSG and CRDS has been removed. Yet, the old-age contribution exemption is effective only in case of incapacity of carrying out the activity, due in particular to a disease, a maternity, a damage, etc. Unemployed entrepreneurs are also excluded from the payment of the contributions on their earnings from the self-employed activity below 120~\% of the minimum wage (20 989\euro~ in 2015) for 12 months, except for the CSG, the CRDS, the contribution to professional training and the complementary pension contributions.


\subsubsection{Terms and conditions of the payment}

The terms and conditions are similar to the ones for the liberal professions, but we must point out some differences: in the case of the complementary pension schemes, two mandatory schemes coexist, for the craftsmen on the one hand, and for traders and manufacturers on the other hand, whereas each liberal professional section is organized independently.\\

Regarding craftsmen, traders and manufacturers, the complementary pension contributions were paid on a permanent basis on earnings of the penultimate year before 2009. After 2009, the payment conditions have been harmonized with those of other contributions. Thus, complementary pension contributions are paid on a provisional basis on earnings of the penultimate year before being adjusted the next year.\\

At the beginning of the activity, contributions to finance  daily social security benefits for sickness leave are paid on a tax base equivalent to 15 216\euro~, that is to say 40~\% of the SSC.\\

Contributions financing the disability-death insurance of craftsmen, traders and manufacturers are paid on a permanent basis on earnings of the penultimate year. At the beginning of the activity, the contribution is paid on a fixed-amount tax base equal to 7608\euro~ in 2015, equivalent to 20~\% of the SSC the first year, and 27~\% of the SSC the second year .


\subsubsection{Special situations}

\subsubsection*{Micro-enterprise}

Similarly to liberal professionals, craftsmen, traders and manufacturers entrepreneurs can choose to join the micro-enterprise scheme. The turnover upper threshold differs according to the type of activity: it is equal to 82 200\euro ~ a year (or 90 300\euro ~ if the preceding year's turnover was lower than 82 200\euro~) for entrepreneurs whose main activity is the selling of goods, objects, furniture and commodities to take away or to consume on site, or the provision of accommodation. For other entrepreneurs whose profit enters the category of industrial and commercial profit (ICP), the threshold is equal to 32 900\euro ~ (34 900\euro ~ if the preceding year's profit was lower than 32 900\euro~).

As for liberal professionals, craftsmen, traders and manufacturers falling under the micro-enterprise scheme can be taxed under the standard tax regime, under the specific micro-tax regime, or under the payment in full discharge. Here again, only thresholds change: regarding the micro-tax regime, the abatement for business expenses on the turnover is equal to 71~\% for activities of purchase and reselling and the provision of accommodation, and to 50~\% for other activities in the ICP category, with the same minimal abatement of 305\euro~. Regarding the payment in full discharge, it represents 1~\% of the turnover for entrepreneurs whose main activity is the selling of goods, objects, furniture and commodities to take away or to consume on site, or the provision of accommodation, and 1.7~\%  for other entrepreneurs whose profit enters the category of ICP. 

As part of social security contributions, craftsmen, traders and manufacturers can benefit from the micro-social regime, which consists in applying monthly or quarterly a unique rate to the ICP actually yielded the preceding month or quarter. For activities of purchase and reselling of commodities to consume on site, and of provision of accommodation (outside accommodation renting), this rate is equal to 13.3~\% whereas it is equal to 22,9~\% for other service providers (outside the option for the income tax payment in full discharge).\newline

 \texttt{Legislative references:} Art L 133-6-8, R 133-30-1, R 133-30-10, and D 131-6-1/2 of the SSC.


\subsubsection*{Unemployed entrepreneurs and the micro-enterprise scheme}

Unemployed entrepreneurs benefiting from the Accre\footnote{The Accre is a scheme designed for unemployed people wishing to create or take over an enterprise consisting in contributions exemptions, the conservation of some benefits and guidance in the different procedures} and having subscribed to the micro-enterprise scheme are granted reduced social security contributions rates.




%%%%%%%%%%%%%%%%%%%%%%%%%%%%%%%%%%%%%%%%%%%%%%%%%%%%%%%%%%%%%%%%%%%%%%%%%%%%%%%%%%%%%%%%%%%
% Section 5: vue d'ensemble
%%%%%%%%%%%%%%%%%%%%%%%%%%%%%%%%%%%%%%%%%%%%%%%%%%%%%%%%%%%%%%%%%%%%%%%%%%%%%%%%%%%%%%%%%%

\section{Summary elements}


\subsection{Different tax bases for different contributions and payroll taxes}

\subsubsection{Towards a tax base convergence?}

Although all the previously presented contributions and payroll taxes have all in common to be based on elements of earnings, the delimitation of their tax base differs. This variability in tax base can be underlined at three levels:

Regarding the employee, some elements of his earnings can be subjected to a contribution or tax, other not. At the level of the firm, some employees can be subjected to a contribution or payroll tax, whereas others are not (fixed-term contract employees \textit{vs} permanent contract employees, employees domiciled in France \textit{vs} employees living abroad, etc.). Finally, some companies can be liable for the payment of some contributions and payroll taxes (as the wage tax and the transportation payment) whereas others are not.

We can distinguish five contributions and payroll tax bases:
\begin{enumerate}
\item The tax base for social security contributions and complementary pension schemes contributions;
\item The tax base for unemployment contributions;
\item The tax base for flat-rate income tax (CSG-CRDS);
\item The tax base for payroll taxes, outside the wage tax;
\item The tax base for the wage tax.
\end{enumerate}

Defining different tax bases can be theoretically justified, as contributions and payroll taxes are intended to fund special social entitlements, to increase the state budget, or to finance defined economic and social projects (apprenticeship development, construction, etc.).

Nonetheless, in practice, to limit incentives for the employers and employees to replace heavily taxed elements of earnings by partly taxed ones, the different tax bases have been gradually harmonized:
\begin{itemize}
\item February 1\up{st}, 1991: when the CSG has been created, a ministerial circular (\texttt{Circ. DSS 16-1-1991}) stated that \og for earnings [the CSG tax base] is the same as the one used for the contributions owed to the general scheme \fg;
\item January 1\up{st}, 1996: conformity of the tax base for the contributions funding training, building, apprenticeship with the tax base for the social security contributions;
\item January 1\up{st}, 2002: conformity of the wage tax base with the social security contributions tax base;
\item January 1\up{st}, 2014: conformity of the wage tax base with the CSG tax base.
\end{itemize}

However, despite the legal principle of tax base unity, there are exemptions and rules specific to some contributions and payroll taxes\footnote{As an example, companies liable for the payment of the VAT are exempted from the wage tax.}.


\subsubsection{Contributions on earnings exempted from CSG}

Though the CSG and CRDS tax base has been gradually harmonized with the social security contributions tax base, there are still some direct or indirect elements of earnings that are not subjected to CSG. To increase the Social security receipts, some contributions based on the parts of earnings subjected to social security contributions but exempted from CSG have been introduced in the 2000's. The reduction of the \og contributions loopholes \fg{} triggered by these exemptions, which started with the tax base convergence, has continued with the creation of new contributions, whose rates have been progressively raised. These contributions, whose tax base is explicitly defined as the gap between two tax bases, will be presented in further details in the next chapter.



\subsection{Social and tax regimes: review tables}

\begin{center}
\begin{fig}[18cm]{Social and tax regimes of the main elements of earnings (private sector). \label{tableall}}
{\graphique[0.9]{Tableau_bases_1.png}} {\emph{}}
\end{fig}
\end{center}

\begin{center}
\begin{fig}[18cm]{Social and tax regimes of the daily social security benefits, of termination benefits and of extra-legal family allowances (private sector). \label{tableall}}
{\graphique[0.9]{Tableau_bases_2.png}} {\emph{}}
\end{fig}
\end{center}

\begin{table}[h!]\label{tab:4.2}
\begin{center}
\centering
\caption{Contributions and payroll tax bases in the public sector.}
\vspace{0.2cm}
\small
\begin{tabular}{|>{\centering\arraybackslash}m{5cm}|| >{\centering\arraybackslash}m{5cm} | >{\centering\arraybackslash}m{5cm} |}
\cline{2-3}
\multicolumn{1}{c|}{} & Tenured & Non-permanent \\
\hline
\hline
Sickness-maternity-disability-death (SMDD) contributions             & TIB + NBI & Total gross earnings \\
Work accident-Professional diseases (WA-PD) contributions            & TIB + NBI & Total gross earnings \\
Old-age insurance contributions & TIB + NBI & Total gross earnings \\
Family allowances contributions  & TIB + NBI & Total gross earnings \\
\hline
Complementary pension scheme contribution & Max[Total gross earnings - (TIB + NBI), 20~\% of the TIB] & Total gross earnings \\
\hline
Exceptional solidarity contribution & Total gross earnings - employees' pension scheme contributions & Total gross contributions - employees' pension scheme contributions \\
\hline
CSG and CRDS                 & Total gross earnings & Total gross earnings \\
\hline
Contributions to the National fund for housing aid          & TBI + NBI & Total gross earnings \\
Contribution for funding local public transport infrastructures         & TBI + NBI & Total gross earnings \\
Wage tax       & Total gross earnings & Total gross earnings \\
\hline
\end{tabular}
\end{center}
\footnotesize \textsc{Notes:} TIB: \textit{traitement indiciaire brut}, equivalent to base pay (see above) ; NBI: \textit{Nouvelle bonification indiciaire}, equivalent to a position-related bonus (see above).
\end{table}

