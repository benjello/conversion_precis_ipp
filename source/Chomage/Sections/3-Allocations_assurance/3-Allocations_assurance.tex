\ifx\isEmbedded\undefined
\input{../../Style/ipp-separate}
\setcounter{chapter}{0}
\chapter{\label{Chapitre2}Le r\'egime d'assurance}
\else \fi


%%%%%%%%%%%%%%%%%%%%%%%%%%%%%%%%%%%%%%%%%%%%%%%%%%%%%%%%%%%%%%%%%%%%%%%%%%%%%%%%%%%%%%%%%%

Le r\'egime d'assurance ch\^omage se caract\'erise par une allocation principale d\'estin\'ee aux demandeurs d'emploi ayant contribu\'e un minimum au syst\`eme lors de périodes d'emploi, et autoris\'es à ce titre \'a \^etre indemnis\`es au moment de la perte d'emploi. Le montant et les modalit\'es de versement sont donc \'etroitement connect\'ees à la carri\`ere pass\'ee du demandeur d'emploi, et conditionn\'es, dans la plupart des cas, \`a la recherche active d'un emploi de la part du ch\^omeur. \\
Le r\'egime d'assurance couvre 54.4\% des personnes inscrites comme demandeurs d'emploi, soit environ 3,1 million de personnes au 2\up{\`eme} trimestre 2015. \cite{REF16}


\section{L'Allocation de retour \`a l'emploi pour les ch\^omeurs de moins de 50 ans}

L'Allocation de Retour \`a l'Emploi (ARE) cr\'e\'ee par la Convention du 1\up{er} Janvier 2001 constitue la principale allocation d'assurance ch\^omage. Elle s'inscrit dans un projet plus global de retour \`a l'emploi, qui consiste \`a d\'efinir, \`a travers une collaboration entre le ch\^omeur et le conseiller de l'agence P\^ole Emploi, quelles sont les offres d'emploi raisonnables -- en prenant en compte les crit\`eres de distance, de salaire, de domaine de comp\'etences, etc. -- tout en offrant aux demandeurs d'emploi conseils, information, formation, ou aides \`a la mobilit\'e.

\subsection{D\'etermination des r\`egles d'\'eligibilit\'e}

L'ouverture des droits \`a l'ARE est conditionn\'ee aux crit\`eres suivants:
\begin{itemize}
\item Ne pas avoir volontairement d\'emissionn\'e de son dernier emploi\footnote{Cependant, certains d\' eparts volontaires sont explicitement jug\'es comme l\'egitimes dans les textes (Accord d'application \no 14 au r\`eglement de l'assurance ch\^omage, chapitres 1 et 2).} (ou avant-dernier si le dernier a dur\'e moins de 3 mois). Les ruptures conventionnelles de contrats \`a dur\'ee ind\'etermin\'ee et les licenciements pour motif \'economique ne sont pas consid\'er\'es comme des d\'emissions;
\item Avoir \'et\'e affili\'e au r\'egime d'assurance ch\^omage au moins 122 jours dans les 28 derniers mois. Cette dur\'ee d'affiliaton n'est pas n\'ecessairement continue, et les p\'eriodes de formation qui n'ont pas donn\'e lieu au paiment d'une indemnisation sont comptabilis\'ees dans la p\'eriode d'affiliation;
\item \^Etre inscrit \`a P\^ole Emploi dans les 12 mois suivant la rupture du contrat de travail (ou suivant une formation);
\item Chercher activement du travail;
\item \^Etre physiquement en mesure de travailler;
\item Ne pas avoir d\'epass\'e l'\^age minimum de d\'epart \`a la retraite. Cependant, dans certains cas, si la personne est plus \^ag\'ee mais n'a pas contribu\'e assez pour b\'en\'eficier d'une retraite \`a taux plein, elle peut continuer \`a b\'en\'eficier de l'ARE en attendant de toucher sa retraite (voir section \textit{L'Allocation de retour \`a l'emploi pour les ch\^omeurs de 50 ans et plus}).
\end{itemize}

Si ces crit\`ere paraissent simples, ils sont cependant sujet \`a d\'ebat, notamment quand il s'agit d'\'evaluer si le demandeur d'emploi cherche activement du travail. En effet, il est particuli\`erement difficile de surveiller le comportement de recherche d'emploi, m\^eme si tout un syst\`eme de contr\^ole et de sanction a \'et\'e mis en place par P\^ole Emploi. \\
Une des fa\c cons de s'assurer que le demandeur d'emploi recherche activement un travail est de d\'efinir, au d\'ebut de l'indemnisation, les crit\`eres d\'elimitant une \textit{offre raisonnable d'emploi}. En effet, le projet personnalis\'e d'acc\`es \`a l'emploi (PPAE) pr\'ecise les caract\'eristiques du travail souhait\'e (type de contrat, nombre d'heures, niveau de salaire, zone g\'eographique, etc.) et sert \`a d\'efinir l'offre raisonnable d'emploi selon des conditions plus ou moins restrictives en fonction du temps d'inscription \`a P\^ole emploi: si l'inscription a \'et\'e faite depuis plus de trois mois, une offre est consid\'er\'ee raisonnable si l'emploi est compatible avec les qualitifications du ch\^omeur et est r\'emun\'er\'e au minimum \`a 95\% du salaire de r\'ef\'erence. Si l'inscription a \'et\'e faite depuis plus de six mois, le crit\`ere du salaire est abaiss\'e \`a 85\%,  et la zone g\'eographique correspond \`a un p\'erim\`etre de 30km autour du domicile, ou d'une heure en transport en commun. Au-del\`a d'un an, les crit\`eres g\'eographiques restent inchang\'es, mais la r\'emun\'eration minimale acceptable devient \'egale au revenu de remplacement per\c cu. \\
Cependant, le salaire ne peut \^etre inf\'erieur ni au salaire normalement pratiqu\'e dans la r\'egion pour la m\^eme profession, ni au salaire correspondant aux dispositions l\'egales et conventionnelles en la mati\`ere. De m\^eme, l'offre raisonnable d'emploi doit correspondre aux crit\`eres d\'efinis dans le PPAE en termes de temps de travail (partiel ou complet) et de type de contrat (\`a dur\'ee d\'etermin\'ee ou ind\'etermin\'ee).\\

Ainsi, si le demandeur d'emploi refuse \`a 2 reprises une \textit{offre raisonnable d'emploi} sans motif l\'egitime, il s'expose \`a une radiation par P\^ole emploi, au m\^eme titre que s'il refuse de d\'efinir ou actualiser son PPAE sans motif l\'egitime, ou s'il refuse de suivre les actions pr\'evues dans le cadre du PPAE sans motif l\'egitime.\\


%Malgr\'e des crit\`eres d'\'eligibilit\'e plut\^ot larges, un grand nombre de demandeurs d'emploi continue \`a ne pas percevoir d'indemnisation. L'explication principale n'est pas qu'ils ne remplissent pas les crit\`eres n\'ecessaires mais que, bien qu'ils aient droit \`a une indemnisation, ils n'en font pas la demande.
%% QLQS stat sur le taux de take up, les facteurs explicatifs, etc


\subsection{Assiette}

L'ARE est calcul\'ee -- au moins en partie -- comme une proportion de l'ancien salaire touch\'e en p\'eriode d'emploi. Cette base de calcul est appel\'ee \textit{salaire journalier de r\'ef\'erence}, et correspond \`a la r\'emun\'eration brute telle que d\'efinie pour le calcul des cotisations sociales\cite{IPP2014} touch\'ee pendant les 12 derniers mois pr\'ec\'edant la rupture du contrat de travail, et divis\'ee par le nombre de jours travaill\'es. Ainsi, si le demandeur d'emploi t\'emoigne, sur les 12 derniers mois, d'une p\'eriode d'emploi correspondant au salaire A, d'une p\'eriode d'inactivit\'e ne donnant pas lieu \`a une indemnisation, et d'une seconde p\'eriode d'emploi correspondant au salaire B, le salaire journalier de r\'ef\'erence sera alors \'egal \`a une moyenne journali\`ere des salaires A et B pond\'er\'ees par les dur\'ees des p\'eriodes d'emploi respectives. \\
%% Mettre un exemple?

De m\^eme que pour le paiement des cotisations, la base de calcul des allocations correspond au salaire journalier de r\'ef\'erence dans la limite de 4 plafonds de la S\'ecurit\'e sociale (12 872 euros mensuels en 2015). Cette limite sup\'erieure tr\`es \'elev\'ee -- une des plus hautes du monde -- a fait l'objet de nombreux d\'ebats, car elle ne correspond en rien aux niveaux observ\'es dans les autres pays europ\'eens, et parce qu'elle peut conduire \`a des indemnisations tr\`es \'elev\'ees pour les demandeurs d'emploi ayant un salaire proche de cette limite. En effet, en 2015, l'indemnisation brute maximale qu'un demandeur d'emploi peut percevoir s'\'el\`eve \`a 7 227,6\euro~ mensuels, soit un peu moins de 3 fois le salaire moyen en France. Cependant, il convient de garder en t\^ete que les ch\^omeurs concern\'es par de tels montants d'allocations repr\'esentent une part tr\`es marginale de la population de demandeurs d'emploi. D'autre part, abaisser le plafond de l'assiette consid\'er\'ee pour le calcul des allocations (de 4 \`a 1 PSS par exemple) n\'ecessiterait, si l'on veut conserver la logique assurantielle du syst\`eme d'indemnisation ch\^omage, d'abaisser de la m\^eme mani\`ere le plafond de l'assiette consid\'er\'ee pour le calcul de la cotisation d'assurance ch\^omage (aujourd'hui aussi \'egal \`a 4 PSS). Or, les individus dont la r\'emun\'eration se situe entre 1 et 4 PSS sont des nets contributeurs au syst\`eme d'indemnisation ch\^omage (signifiant qu'ils contribuent davantage qu'ils ne re\c coivent). L'effet net total d'une telle r\'eforme pourrait donc s'av\'erer n\'egatif pour le financement de l'assurance ch\^omage.


%% CHECKER POURCENTAGE DE CHOMEURS CONCERNES DANS PARAMETRES UTILES

\subsection{Calcul}

Le m\'ethode de calcul de l'ARE consiste \`a choisir entre deux formules, en optant pour celle qui engendre l'allocation la plus \'elev\'ee. En effet, le demandeur d'emploi peut percevoir soit 57\% de son salaire journalier de r\'ef\'erence (ce pourcentage \'etant pass\'e de 57.4 \`a 57\% dans la derni\`ere Convention de 2014), soit 40.4\% de son salaire journalier de r\'ef\'erence, plus un montant fixe, correspondant \`a 11,76\euro~ par jour en 2016. Il existe \'egalement une allocation minimum, touch\'ee par ceux dont le pr\'ec\'edant calcul engendrerait une allocation inf\'erieure \`a ce montant minimal. Enfin, une proportion maximale a \'egalement \'et\'e fix\'ee, selon laquelle le demandeur d'emploi ne peut percevoir une allocation \'equivalente \`a plus de 75\% du salaire journalier de r\'ef\'erence. Ainsi, selon le niveau de salaire, le lien entre ancien salaire et allocation ch\^omage diff\`ere. Pour l'ann\'ee 2016, la distribution se fait ainsi:

%\begin{tab}[15cm]{Calcul de l'ARE pour 2015\label{ARE2015}}
%{\begin{tabular}{|c|c|}
%\hline
%\textbf{Salaire journalier de r\'ef\'erence} & \textbf{Allocation}  \\
%\hline
% SJR $<$ 38.23\euro~ & 75\% SJR \\
%\hline
% 38.23\euro~ $<$ SJR $<$ 41.86\euro~ & 28,67\euro \\
%\hline
%41.86\euro~ $<$ SJR $<$ 70.84\euro~ & 40.4\% SJR + 11.76\euro \\
%\hline
%SJR $>$ 70.84\euro & 57\% SJR \\
%\hline
%\end{tabular}}
%{\emph{}}
%\end{tab}


\begin{tab}
[12cm]                                                                           % Argument 1 [8 cm] : largeur du tableau (mettre 15 cm comme taille maximale)
{
Calcul de l'ARE\label{tab:ARE} % Argument 2 : titre du tableau
}
{
\begin{tabular}{ll} \toprule                                                  % Argument 3 : contenu du tableau
Salaire journalier de r\'ef\'erence   & Allocation  \\
\midrule
 SJR $\leq$ 38,23\euro~ & 75\% SJR \\
38,24\euro~ $<$ SJR $\leq$ 41,86\euro~ & 28,67\euro \\
41,87\euro~ $<$ SJR $\leq$ 70,84\euro~ & 40,4\% SJR + 11,76\euro \\
SJR $>$ 70,84\euro & 57\% SJR \\   
\\
\bottomrule
\end{tabular}
}
{
\textsc{Notes:} Les param\`etres sont ceux en vigueur depuis le 01/07/2015
}
\end{tab}

\begin{tab}
[12cm]                                                                           % Argument 1 [8 cm] : largeur du tableau (mettre 15 cm comme taille maximale)
{
Calcul de l'ARE en Mayotte\label{tab:AREM} % Argument 2 : titre du tableau
}
{
\begin{tabular}{lll} \toprule                                                  % Argument 3 : contenu du tableau
  & Jusqu'au 30/04/2016 & A partir du 01/05/2016  \\
\midrule
Allocation minimale & 14,33\euro & 14,33\euro  \\
Calcul du montant de l'ARE-Mayotte & 75\% du SJR les 3 premiers mois & 70\% du SJR les 3 premiers mois \\
	& 50\% du SJR les 4 mois suivants & 50\% du SJR les mois suivants \\
	& 35\% du SJR les mois suivants &  \\
\\
\bottomrule
\end{tabular}
}
{
\textsc{Notes:} Un accord national interprofessionnel relatif \`a l'indemnisation ch\^omage en Mayotte est en cours d'agr\'ement minist\'eriel, et devrait entrer en vigueur le 1\up{er} mai 2016.
}
\end{tab}

Si la dur\'ee de travail \'etait inf\'erieure \`a la dur\'ee l\'egale ou conventionnelle, l'allocation minimale et la partie fixe de la formule de calcul doivent \^etre r\'eduits au prorata du nombre d'heures travaill\'ees par rapport \`a la dur\'ee l\'egale du travail.


\subsection{Mode et dur\'ee de versement}

Le versement de l'ARE d\'ebute apr\`es l'application, le cas \'echant, d'un diff\'er\'e d'indemnisation, et d'un d\'elai d'attente: 
\begin{itemize}
\item Un diff\'er\'e d'indemnisation s'applique si le ch\^omeur touche des indemnit\'es compensatrices de cong\'es pay\'es. Il est \'egal au montant des indemnit\'es compensatrices de cong\'es pay\'es (ICCP) divis\'ees par le salaire journalier de r\'ef\'erence.
\item Un diff\'er\'e d'indemnisation peut \'egalement s'appliquer si le ch\^omeur touche des indemnit\'es de licenciement au del\`a du montant fix\'e par la loi. Il est \'egal au montant des indemnit\'es supra-l\'egales divis\'ees par 90 (arr\^et\'e \`a l'entier sup\'erieur), avec un plafonnement du diff\'er\'e \`a 75 jours pour les licenciements \'economiques ou rupture de contrat pour motif \'economique, et \`a 180 jours dans les autres cas.
\item Le d\'elai d'attente de 7 jours s'applique dans toutes les prises en charge au titre de l'assurance ch\^omage si un autre d\'elai d'attente n'a pas \'et\'e appliqu\'e dans les 12 mois pr\'ec\'edents.
\end{itemize}


Le calcul du diff\'er\'e d'indemnisation prend comme point de d\'epart le lendemain de la rupture du contrat de travail. Le d\'elai d'attente commence au lendemain de la rupture du contrat de travail ou au lendemain des deux diff\'er\'es le cas \'ech\'eant, si le demandeur d'emploi est d\'ej\`a inscrit sur les registres de P\^ole emploi. S'il n'est pas d\'ej\`a inscrit, le d\'elai d'attente prend pour point de d\'epart le jour de l'inscription en tant que demandeur d'emploi.\newline

Le versement de l'ARE est conditionnel au respect des engagements du ch\^omeur, notamment concernant sa recherche active d'un emploi, et \`a l'actualisation de sa situation tous les mois. En effet, au plus tard 15 jours apr\`es son inscription comme demandeur d'emploi, le ch\^omeur est tenu d'\'etablir avec son conseiller P\^ole emploi un projet personnalis\'e d'acc\`es \`a l'emploi (PPAE) fixant certaines de ses obligations, et servant \`a la d\'efinition des offres raisonnables d'emploi. Le PPAE d\'etaille notamment quel type d'emploi est recherch\'e (type de contrat, temps de travail, domaine de comp\'etence), la zone g\'eographique privil\'egi\'ee, le niveau de salaire souhait\'e, et les actions de formation et reclassement que P\^ole emploi s'engage \`a mettre en place. Ce projet, actualis\'e tous les trois mois, sert de feuille de route pendant toute la dur\'ee d'indemnisation du demandeur d'emploi. \\


Le calcul de la dur\'ee maximale d'indemnisation a \'et\'e largement simplifi\'e suite \`a la r\'eforme de 2009. En effet, jusqu'\`a cette date, diff\'erentes cat\'egories d'affiliation existaient, qui donnaient lieu \`a diff\'erentes dur\'ee d'indemnisation, suivant le principe que plus la dur\'ee d'affiliation \'etait longue, plus elle ouvrait droit \`a une dur\'ee d'indemnisation longue, mais ceci de mani\`ere non lin\'eaire. Ainsi, ce syst\`eme \'etait susceptible de g\'en\'erer des effets de seuils au passage d'une cat\'egorie \`a une autre, et n'offrait pas un lien entre dur\'ee d'affiliation et dur\'ee d'indemnisation uniforme au sein des ch\^omeurs.\\
La Convention d'assurance ch\^omage de 2009 a supprim\'e toutes ces cat\'egories pour instituer une r\`egle unique: chaque jour travaill\'e, dans les 28 derniers derniers mois (respectivement 36 derniers mois pour les ch\^omeurs de plus de 50 ans), donne droit \`a un jour indemnis\'e, dans la limite de 24 (respectivement 36) mois. La dur\'ee d'affiliation minimale n\'ecessaire \`a l'ouverture des droits correspond \`a 4 mois. \\
Un plafond relatif \`a la dur\'ee d'indemnisation maximale a \'egalement \'et\'e fix\'e \`a hauteur de 75\% du salaire de r\'ef\'erence rapport\'e aux p\'eriodes d'affiliation retenues: en effet, une journ\'ee de travail normale étant comptabilis\'ee comme \'equivalente \`a 5 heures, dans le cas de personnes ayant travaill\'e peu de temps avec des journ\'ees de travail intenses, le calcul de leurs droits donnait lieu \`a des dur\'ees d'indemnisation parfois tr\`es \'elev\'ee. Par exemple, pour une p\'eriode d'affiliation de 80 jours \`a hauteur de 10 heures de travail par jour, la dur\'ee d'indemnisation correspondante serait de $80 \times \frac{10}{5} = 160$ jours, soit une dur\'ee d'indemnisation \'egale au double de la dur\'ee d'affiliation, et le capital total serait \'egal \`a $68.4 \times 160 = 10 944$\euro. Pour un salaire horaire de 12\euro, le salaire de r\'ef\'erence sur la p\'eriode est de 9 600\euro, le plafond \`a ne pas d\'epasser correspond donc \`a 7 200\euro. Sans l'application du plafond de 75\%, cela conduit \`a obtenir un capital et une dur\'ee d'indemnisation tr\`es \'elev\'es par rapport \`a la p\'eriode d'affiliation. L'application du plafond donne droit \`a 105 jours d'indemnisation (7 200\euro/68.4\euro), corrigeant ainsi la majoration excessive.\\
Dans tous les cas, la dur\'ee d'indemnisation ne pourra \^etre inf\'erieure à 4 mois ou 30 jours lors d'un rechargement.\\


Concernant les ch\^omeurs partiels, la dur\'ee maximale d'indemnisation est fix\'ee \`a 182 jours. Cependant, si la suspension de l'activit\'e est due \`a un sinistre ou \`a une catastrophe naturelle, la dur\'ee de versement peut \^etre prolong\'ee jusqu'\`a la reprise de l'activit\'e.

\subsection{R\'egimes social et fiscal}

L'allocation de retour \`a l'emploi est soumise au paiement d'une cotisation \'equivalente \`a 3\% du SJR pour le financement des retraites compl\`ementaires, pr\'elev\'ee sur le montant brut de l'allocation, sauf dans le cas o\`u cela r\'eduirait l'allocation journali\`ere en-de\c c\`a de l'allocation minimale.\\
De m\^eme, la CSG et le CRDS sont pay\'ees \`a des taux r\'eduits (6.2\% au lieu de 7.5\% pour la CSG sur 98,25\% des allocations brutes), sauf si le montant brut de l'allocation est inf\'erieure au montant journalier du Smic (49\euro\ au 1\up{er} janvier 2016). Lorsque le pr\'el\`evement de la CSG et de la CRDS conduit \`a r\'eduire l'allocation nette en dessous du montant journalier du Smic, sont alors op\'er\'es une exon\'eration ou un ecr\^etement.\\

Ainsi, durant la p\'eriode d'indemnisation, des points de retraite compl\'ementaire sont attribu\'es, et le ch\^omeur peut accumuler des trimestres de retraite valid\'es par la caisse d'assurance vieillesse au rythme d'un trimestre pour 50 jours de ch\^omage indemnis\'e, dans la limite de 4 par an.\\


\texttt{R\'ef\'erences l\'egislatives:}{Circulaire Un\'edic \no 2015-14 du 1er juillet 2015; Convention du 14 mai 2014 relative \`a l'indemnisation du ch\^omage; R\`eglement g\'en\'eral annex\'e \`a la Convention du 14 mai 2014; art. 79-81 ter, art. 156-163 quinquies C bis du Code g\'en\'eral des imp\^ots}


\section{ARE pour les ch\^omeurs de 50 ans et plus}

L'ARE peut \^etre vers\'ee pour les demandeurs d'emploi de 50 ans et plus au m\^eme titre que pour les demandeurs d'emploi de moins de 50 ans s'ils remplissent les conditions d'\'eligibilit\'e.

\subsection{D\'etermination des r\`egles d'\'eligibilit\'e}

\begin{itemize}
\item Etre inscrit comme demandeur d'emploi;
\item Ne pas avoir atteint l'\^age l\'egal de d\'epart \`a la retraite (entre 60 et 62 ans selon l'ann\'ee de naissance). Cependant, si le demandeur d'emploi atteint cet \^age sans avoir totalis\'e un nombre de trimestres suffisant pour b\'en\'eficier d'une retraite \`a taux plein, il pourra continuer \`a toucher l'ARE, au maximum jusqu'\`a avoir atteint l'\^age du droit \`a une retraite \`a taux plein (de 65 \`a 67 ans suivant l'ann\'ee de naissance). Par ailleurs, les demandeurs d'emploi b\'en\'eficiant d'une retraite anticip\'ee (travailleurs handicapp\'es, titulaires d'une carri\`ere longue par exemple) ne peuvent b\'en\'eficier de l'ARE;
\item Avoir travaill\'e au moins 122 jours (610 heures) au cours des 36 derniers mois. Les p\'eriodes de travail retenues sont toutes celles qui n'ont pas donn\'e lieu \`a une indemnisation, qu'elles aient \'et\'e effectu\'ees de mani\`ere continue ou discontinue, chez un employeur unique ou chez diff\'erents employeurs. Chaque jour de suspension du contrat de travail \'equivaut \`a une journ\'ee d'affiliation (ou 5 heures) en moins. Les p\'eriodes de formation professionnelles peuvent \^etre comptabilis\'ees comme p\'eriodes d'affiliation sous certaines conditions;
\item Etre involontairement priv\'e d'emploi. Si le demandeur d'emploi a d\'emission\'e de son dernier emploi (ou de l'avant-dernier si le dernier a dur\'e moins de 91 jours), il ne pourra pas \^etre indemnis\'e, sauf dans certains cas de d\'emission jug\'es l\'egitimes (d\'epart volontaire pour suivre son conjoint suite \`a une mutation par exemple). Par ailleurs, m\^eme en cas de d\'epart, une indemnisation peut finalement \^etre accord\'ee s'il est constat\'e sur une certaine p\'eriode que le ch\^omeur recherche effectivement un emploi;
\item Etre physiquement apte \`a l'exercice d'un emploi;
\item Rechercher effectivement et activement un emploi;
\item R\'esider sur le territoire couvert par l'assurance ch\^omage, c'est-\`a-dire la France m\'etropolitaine, les DOM hors Mayotte, Saint-Pierre-et-Miquelon, Saint Barth\'el\'emy, Saint-Martin, et la principaut\'e de Monaco.
\end{itemize}


\subsection{Calcul, mode et dur\'ee de versement}

L'ARE est vers\'ee de la m\^eme mani\`ere aux allocataires de moins de 50 ans qu'aux allocataires de 50 ans et plus: le mode de calcul et les modalit\'es de versement restent inchang\'ees. La dur\'ee d'indemnisation potentielle est calcul\'ee selon la m\^eme m\'ethode, \`a la diff\'erence que la p\'eriode de r\'ef\'erence ainsi que la dur\'ee d'indemnisation maximale sont \'egales \`a 36 mois: un demandeur d'emploi de 50 ans ou plus sera donc indemnis\'e pendant un nombre de jours \'egal \`a sa dur\'ee d'affiliation dans les 36 derniers mois (contre 28 pour les moins de 50 ans), avec un maximum de 36 mois (24 mois pour les moins de 50 ans).\\
On obtient donc les dur\'ees d'indemnisation suivantes:


\begin{tab}
[15cm]                                                                           % Argument 1 [8 cm] : largeur du tableau (mettre 15 cm comme taille maximale)
{
Dur\'ees d'indemnisation maximales \label{tab:AREdu} % Argument 2 : titre du tableau
}
{
\begin{tabular}{llllll} \toprule                                                  % Argument 3 : contenu du tableau
\multicolumn{3}{c}{Moins de 50 ans} &  \multicolumn{3}{c}{Plus de 50 ans}  \\

Dur\'ee d'affiliation & P\'eriode de r\'ef\'erence & Dur\'ee d'indemnisation & Dur\'ee d'affiliation & P\'eriode de r\'ef\'erence & Dur\'ee d'indemnisation \\
 & & maximale & & & maximale \\
\midrule
 Moins de 4 mois & 28 derniers mois & Pas d'indemnisation & Moins de 4 mois & 36 derniers mois & Pas d'indemnisation \\
Entre 4 et 24 mois & 28 derniers mois & Dur\'ee d'indemnisation = & Entre 4 et 36 mois & 36 derniers mois & Dur\'ee d'indemnisation = \\
	&	& dur\'ee d'emploi &	&	& dur\'ee d'emploi \\
Plus de 24 mois & 28 derniers mois & 24 mois & Plus de 36 mois & 36 derniers mois & 36 mois \\
\\
\bottomrule
\end{tabular}
}
{
}
\end{tab}


\begin{tab}
[15cm]                                                                           % Argument 1 [8 cm] : largeur du tableau (mettre 15 cm comme taille maximale)
{
Dur\'ees d'indemnisation maximales en Mayotte\label{tab:AREMdu} % Argument 2 : titre du tableau
}
{
\begin{tabular}{llll} \toprule                                                  % Argument 3 : contenu du tableau
 &  & Jusqu'au 30/04/2016 & A partir du 01/05/2016  \\
\midrule
Condition d'affiliation minimale & & 9 mois (271 jours ou 2246 heures) & 6 mois (182 jours ou 1 020 heures)  \\
P\'eriode de r\'ef\'erence &  & 24 derniers mois & 24 derniers mois \\
\multirow{3}{*}{Dur\'ee d'indemnisation} & Moins de 50 ans & 7 mois (212 jours) & \multirow{3}{*}{1 jour affili\'e = 1 jour indemnis\'e} \\
	& Entre 50 et 57 ans & 20 mois (609 jours) &  \\
	& 57 ans et plus & 30 mois (912 jours) &  \\ 
\\
\bottomrule
\end{tabular}
}
{
\textsc{Notes: Un accord national interprofessionnel relatif \`a l'indemnisation ch\^omage en Mayotte est en cours d'agr\'ement minist\'eriel, et devrait entrer en vigueur le 1\up{er} mai 2016.}
}
\end{tab}

De m\^eme, le r\'egime social et fiscal de l'ARE est le m\^eme quel que soit l'\^age du demandeur d'emploi.

\subsection{Maintien des droits jusqu'\`a la retraite \`a taux plein}

Commen mentionn\'e dans le paragraphe sur les crit\`eres d'\'eligibilit\'e, les droits \`a l'indemnisation ch\^omage peuvent \^etre prolong\'es au-del\`a de la dur\'ee maximale d'indemnisation et jusqu'\`a la liquidation de la retraite, si certaines conditions sont remplies:
\begin{itemize}
\item Etre en cours d'indemnisation \`a l'\^age l\'egal de d\'epart \`a la retraite, correspondant \`a 61 ans et 2 mois pour les personnes n\'ees en 1953, 61 ans et 7 mois pour les personnes n\'ees en 1954, et 62 ans pour les autres;
\item Etre en cours d'indemnisation depuis un an au moins;
\item Ne pas avoir totalis\'e assez de trimestres valid\'es par l'assurance vieillesse pour pouvoir b\'en\'eficier d'une retraite \`a taux plein;
\item Justifier de 12 ans d'affiliation \`a l'assurance ch\^omage, dont une ann\'ee continue ou 2 ann\'ees discontinues dans les 5 derni\`eres ann\'ees;
\item Avoir totalis\'e au moins 100 trimestres valid\'es par l'assurance vieillesse.
\end{itemize}


\section{ARE formation}

Un cas particulier de l'assurance ch\^omage concerne les demandeurs d'emploi en formation, qui peuvent toucher une allocation sp\'ecifique, l'Allocation de retour \`a l'emploi formation (Aref).

\subsection{D\'etermination des r\`egles d'\'eligibilit\'e}

Pour b\'en\'eficier de l'Aref, les conditions d'\'eligibilit\'e sont les m\^emes que dans le cas de l'ARE. Elle est vers\'ee \`a tous les demandeurs d'emploi susceptibles de toucher l'ARE et qui sont en p\'eriode de formation dans le cadre de leur projet personnalis\'e d'acc\`es \`a l'emploi (PPAE). Ainsi, cela concerne \'egalement les personnes licenci\'ees en cours de cong\'e individuel de formation \`a condition que celles-ci s'inscrivent comme demandeur d'emploi et que leur formation soit jug\'ee comme adapt\'ee au projet personnalis\'e d'acc\`es \`a l'emploi par P\^ole emploi. \\

%Accord d’appli. n° 20 - R\`eglement g\'en\'eral annex\'e \`a la convention du 6 mai 2011

Il existe \'egalement certains cas dans lesquels les personnes suivent une formation qui n'entre pas dans le PPAE: si elles restent inscrites en tant que demandeurs d'emploi, elle peuvent b\'en\'eficier de l'Aref. On peut citer l'exemple des stagiaires dont le stage n'exc\`ede pas 40 heures ou dont le stage ne l'emp\^eche pas d'\^etre disponible pour un nouveau travail.

%R\'ef\'erence r\'eglementaire ⇒ Art. R5411-10 du Code du Travail


\subsection{Assiette}

La d\'etermination de l'assiette se fait selon les m\^emes r\`egles que dans le cas de l'ARE.

\subsection{Calcul}

Le montant brut de l'Aref est \'egal au montant de l'ARE vers\'ee au demandeur d'emploi quand il n'est pas en formation. L'allocation minimale est la m\^eme que pour l'ARE, sauf pour les demandeurs d'emploi qui travaillaient \`a temps partiel auparavant: dans ce cas, elle est fix\'ee \`a 20,54\euro~ depuis le 1\up{er} Juillet 2015.


\subsection{Mode et dur\'ee de versement}

L'Aref est vers\'ee au demandeur d'emploi dans la limite de la dur\'ee maximale d'indemnisation calcul\'ee selon les modalit\'es d\'ecrites dans le cas de l'ARE.\\

Si jamais le b\'en\'eficiaire de l'Aref \'epuise ses droits avant la fin de sa formation, il peut continuer \`a toucher une allocation durant le reste de sa formation \`a condition, entre autres, que la formation suivie permette d'acc\'eder \`a un m\'etier figurant sur une liste fix\'ee par arr\^et par le Pr\'efet de la r\'egion.

\subsection{R\'egimes social et fiscal}

Le r\'egime social de l'Aref est plus avantageux que dans le cas de l'ARE, car n'est pr\'elev\'ee qu'une participation de 3\% au titre des retraites compl\'ementaires. Le r\'egime fiscal est le m\^eme que pour l'ARE.

\subsection{Cas particuliers}

Certains cas particuliers n\'ecessitent des pr\'ecisions suppl\'ementaires. En cas:
\begin{itemize}
\item D'interruption de stage: si elle est inf\'erieure \`a 15 jours, le demandeur d'emploi continue de percevoir l'Aref. Si elle est sup\'erieure \`a 15 jours, le demandeur d'emploi est transf\'er\'e dans la cat\'egorie des demandeurs d'emploi \`a la recherche d'un emploi, et per\c coit donc l'ARE \`a ce titre;
\item De modification de stage, P\^ole emploi se charge d'actualiser le PPAE;
\item D'abandon de stage, la situation doit \^etre examin\'ee par P\^ole emploi. Cela peut se traduire par une radiation, si l'abandon est assimil\'e \`a un refus de formation, ou bien une r\'eduction ou une suppression temporaire ou d\'efinitive des allocations, mais sans que cela n'affecte la dur\'ee maximale d'indemnisation du demandeur d'emploi;
\item De reprise d'activit\'e, l'allocation peut \^etre partiellement cumul\'ee avec le salaire si l'activit\'e est compatible avec le suivi de la formation.
\end{itemize}


\section{Allocations en cas de licenciement \'economique}

Les demandeurs d'emploi ayant subi un licenciement \'economique peuvent b\'en\'eficier d'un suivi sp\'ecifique dans le cadre du \textit{Contrat de s\'ecurisation professionnelle} (CSP), qui leur offre une indemnisation et un accompagnement individualis\'e pour aider \`a leur r\'einsertion, et une protection sociale. Ce contrat est propos\'e par les employeurs des entreprises de moins de mille salari\'es et des entreprises en redressement ou liquidation judiciaire -- quel que soit leur effectif -- aux employ\'es vis\'es par une proc\'edure de licenciement \'economique \`a partir du 1\up{er} septembre 2011 (le CSP est en effet issu de la fusion de deux pr\'ec\'edents dispositifs, la convention de reclassement personnalis\'ee et le contrat de transition personnelle, act\'ee lors de la convention du 19 juillet 2011). Concernant les entreprises de plus de 1 000 salari\'es, elles sont charg\'ees de mettre en place une proc\'edure de cong\'e de reclassement.\\

Le CSP est propos\'e par les employeurs concern\'es, soit au moment de l'entretien pr\'ealable de licenciement, soit apr\`es la derni\`ere r\'eunion des instances repr\'esentatives du personnel, soit au plus tard au dernier jour du cong\'e l\'egal de maternit\'e le cas \'ech\'eant, ou bien par le conseiller P\^ole emploi. L'employeur est alors redevable d'une contribution \'egale \`a deux mois de salaire brut si le salari\'e refuse le CSP, ou trois mois de salaire superbrut (incluant cotisations salariales et patronales) si le salari\'e l'accepte.\\

Le salari\'e dispose alors d'un d\'elai de r\'eflexion de 21 jours \`a compter du lendemain de la proposition (pour les salari\'es dont le licenciement est soumis \`a autorisation, la limite du d\'elai de r\'eflexion correspond au lendemain de la date \`a laquelle l'employeur a pris connaissance de la d\'ecision de l'autorit\'e administrative) pour d\'ecider s'il accepte ou  non ce contrat. Durant cette p\'eriode, le salari\'e peut b\'en\'eficier d'un entretien informatif avec un  conseiller P\^ole emploi, pour l'\'eclairer sur son choix. Si le salari\'e accepte, le contrat de travail prend fin et le CSP commence \`a l'issue du d\'elai de r\'eflexion. En cas de refus ou d'absence de r\'eponse, le licenciement pour motif \'economique poursuit son cours conform\'ement au droit commun.\\

\texttt{R\'ef\'erences l\'egislatives:}{Arr\^et\'e du 6 octobre 2011 relatif \`a l'agr\'ement de la convention du 19 juillet 2011 relative au contrat de s\'ecurisation professionnelle; Art. L.1233-66 du Code du Travail} 


\subsection{D\'etermination des r\`egles d'\'eligibilit\'e}

Le salari\'e faisant l'objet d'une proc\'edure de licenciement \'economique peut b\'en\'eficier d'un contrat de s\'ecurisation professionnelle \`a condition qu'il remplisse quatre crit\`eres:
\begin{itemize}
\item Avoir un an d'anciennet\'e dans l'entreprise. Ainsi, il peut percevoir une allocation de s\'ecurisation professionnelle (ASP) \'equivalente \`a 80\% du salaire journalier de r\'ef\'erence. A d\'efaut de remplir cette condition d'anciennet\'e, il doit avoir \'et\'e affili\'e au r\'egime d'assurance ch\^omage au moins 4 mois dans les 28 derniers mois (36 derniers mois s'il a plus de 50 ans au moment de la fin du contrat de travail), et peut alors b\'en\'eficier d'une allocation \'egale \`a ce qu'il aurait per\c cu au titre de l'ARE s'il avait refus\'e le CSP.
\item Ne pas avoir atteint l'\^age auquel il peut toucher une retraite \`a taux plein
\item R\'esider sur un territoire entrant dans le champ d'application de l'assurance ch\^omage (France, DOM, Saint-Pierre et Miquelon, Saint-Barth\'el\'emy et Saint-Martin)
\item Etre physiquement apte \`a l'exercice d'une activit\'e. Cependant, si le salari\'e est en cong\'e maternit\'e, arr\^et maladie ou per\c coit une pension d'invalidit\'e, il peut quand m\^eme b\'en\'eficier du CSP au terme de l'\'ev\`enement ayant engendr\'e ce cong\'e ou cette pension.
\end{itemize}

Au 31 d\'ecembre 2014, les b\'en\'eficiaires du CSP \'etaient 93 000 au total, dont 89 500 indemnis\'es au titre de l'Allocation de s\'ecurisation professionnelle, et 3 500 recevant une allocation \'equivalente \`a l'ARE (ASP-ARE).\\

\texttt{R\'ef\'erences l\'egislatives:}{Arr\^et\'e du 6 octobre 2011 relatif \`a l'agr\'ement de la convention du 19 juillet 2011 relative au contrat de s\'ecurisation professionnelle; Convention d'assurance ch\^omage 2014}

\subsection{Assiette}

Le salaire journalier de r\'ef\'erence est calcul\'e conform\'ement aux conditions \'enonc\'ees dans le paragraphe sur l'allocation de retour \`a l'emploi.

\subsection{Calcul}

L'allocation de s\'ecurisation professionnelle correspond \`a 80\% du salaire journalier de r\'ef\'erence brut, ce qui \'equivaut \`a peu pr\`es au salaire net total, avec une limite inf\'erieure correspondant \`a l'ARE-formation minimale (20,54\euro~ par jour) ou a l'ARE, et une limite sup\'erieure correspondant \`a l'ARE brute maximale (241,22\euro~ par jour). Si  le salari\'e ne peut t\'emoigner d'un an d'anciennet\'e dans l'entreprise mais totalise au moins 4 mois d'affiliation au r\'egime d'assurance ch\^omage dans les 28 derniers mois, l'ASP est alors \'egale au montant de l'ARE que le salari\'e aurait touch\'e s'il avait refus\'e le CSP.\\
L'indemnit\'e de pr\'eavis est vers\'ee par l'employeur \`a P\^ole emploi afin de financer le dispositif dans la limite de trois mois (charges salariales et patronales incluses) concernant les employ\'es ayant au moins un an d'anciennet\'e dans l'entreprise; la fraction de l'indemnit\'e au-del\`a de trois mois de salaire revient \`a l'employ\'e. Dans le cas o\`u l'employ\'e ne totalise pas un an d'anciennet\'e dans l'entreprise, il per\c coit la totalit\'e de l'indemnit\'e de pr\'eavis. Quelle que soit l'anciennet\'e du salari\'e, il per\c coit la totalit\'e des indemnit\'es relatives \`a la rupture du contrat de travail.\\

Si le salari\'e b\'en\'eficiant d'un CSP retrouve un emploi moins r\'emun\'er\'e que son emploi pr\'ec\'edant d'au moins 15\% (\`a heures de travail \'equivalentes) avant d'arriver au terme de son CSP, il peut recevoir une indemnit\'e diff\'erentiellle de reclassement (IDR). Le montant et la dur\'ee de versement sont soumis \`a certaines limites. En effet, l'IDR est \'egale \`a la diff\'erence entre l'ancien salaire brut et le nouveau salaire brut (=SJR $\times 30 - $ Salaire brut du nouvel emploi). Elle est vers\'ee pour une dur\'ee maximale de 12 mois, et tant que le contrat de travail est effectif.\\

Dans le cas o\`u un salari\'e occupait plusieurs emplois, en a perdu un ou plusieurs pour cause de licenciement pour motif \'economique, et qu'il est indemnis\'e au titre de l'ASP, il peut int\'egralement cumuler l'ASP calcul\'ee sur la base de ou des emploi(s) perdu(s) pour motif \'economique, et les r\'emun\'erations issues de sa ou ses activit\'e(s) conserv\'ee(s). Ce cumul est conditionn\'e au respect de ses devoirs figurant dans son plan de s\'ecurisation professionnelle, et sa dur\'ee n'est pas limit\'ee dans le temps. \\

L'ASP est soumise \`a la participation de 3\% aux retraites compl\'ementaires, mais \'echappe \`a la CSG et la CRDS.\\

\texttt{R\'ef\'erences l\'egislatives:}{Arr\^et\'e du 6 octobre 2011 relatif \`a l'agr\'ement de la convention du 19 juillet 2011 relative au contrat de s\'ecurisation professionnelle}

\subsection{Mode et dur\'ee de versement}

L'allocation de s\'ecurisation professionnelle (ASP) est vers\'ee d\`es que le CSP entre en effet, sans application de diff\'er\'e ou de d\'elai d'attente. Le versement a lieu pour une dur\'ee maximale de 12 mois, \`a condition que l'allocataire actualise mensuellement sa situation. Cependant, le versement peut \^etre interrompu pour certain motifs (cong\'e maternit\'e, arr\^et maladie) et reprendre \`a la disparition du dit motif. L'interruption peut \^etre d\'efinitive si le b\'en\'eficiaire cesse de remplir les obligations stipul\'ees dans le CSP.\\

Le b\'en\'eficiaire du CSP a le statut de stagiaire de la formation professionnelle, et b\'en\'eficie d'un suivi r\'egulier, renforc\'e et personnalis\'e de la part de P\^ole emploi pour une dur\'ee maximale de 12 moins, qui peut \^etre port\'ee \`a 18 moins si le b\'en\'eficiaire retrouve un emploi entre le 6\up{\`eme} et le 12\up{\`eme} mois.\\
Le b\'en\'efice du CSP est compatible avec la reprise d'une activit\'e professionnelle: en effet, l'allocataire peut r\'ealiser de courtes p\'eriodes d'emploi en entreprise, sous la forme d'un contrat de travail \`a dur\'ee d\'etermin\'ee ou temporaire, renouvelable une fois. Chaque contrat doit avoir une dur\'ee minimum de deux semaines et la dur\'ee cumul\'ee de la totalit\'e des contrats ne peut d\'epasser 6 mois. Chaque p\'eriode d'emploi conduit \`a l'interruption du versement de l'ASP, qui reprend \`a la fin du contrat de travail pour la dur\'ee du CSP restant \`a courir.\\

Si, au terme de son CSP, le b\'en\'eficiaire n'a pas retrouv\'e un emploi, il peut s'inscrire en tant que demandeur d'emploi, et \`a ce titre, percevoir l'allocation d'aide au retour \`a l'emploi, s'il remplit les conditions, et sans aucun diff\'er\'e ou d\'elai d'attente. La dur\'ee de versement est calcul\'ee en d\'eduisant la dur\'ee de versement de l'ASP.\\

\texttt{R\'ef\'erences l\'egislatives:}{Arr\^et\'e du 6 octobre 2011 relatif \`a l'agr\'ement de la convention du 19 juillet 2011 relative au contrat de s\'ecurisation professionnelle}

%%%%%%%%%% FAIRE UN ENCART %%%
\fbox{ %fbox est utilis\'e pour voir les bords de la minipage
\begin{minipage}[c]{15cm}
La convention relative au contrat de s\'ecurisation professionnelle sign\'ee le 26 Janvier 2015 et entr\'ee en vigueur le 1\up{er} f\'evrier 2015 jusqu'au 31 d\'ecembre 2016 a apport\'e quelques modifications au dispositif. \\
\textbf{Pour une meilleure int\'egration au march\'e du travail}: en vue de favoriser le retour \`a l'emploi, la nouvelle convention:
\begin{itemize}
\item a introduit la possibilit\'e de prolonger la dur\'ee du CSP en cas de reprise d'une activit\'e r\'emun\'er\'ee \`a partir du 7\up{\`eme} mois, jusqu'\`a trois mois suppl\'ementaires (permettant donc de totaliser 15 mois maximum). A partir du 1\up{er} mars 2015, les p\'eriodes d'activit\'e r\'emun\'er\'ees au cours du CSP peuvent \^etre de 3 jours minimum;
\item a facilit\'e l'acc\`es \`a la formation, notamment concernant les formations \'eligibles au compte personnel de formation (CPF) si elles correspondent au projet professionnel de l'allocataire;
\item a cr\'ee la prime au reclassement \'equivalente \`a 50\% des droits restants \`a l'ASP de l'allocataire s'il retrouve un travail d'au moins 6 mois avant la fin du 10\up{\`eme} mois du CSP;
\item a assoupli les crit\`eres d'attribution de l'indemnit\'e diff\'erentielle de reclassement: elle peut \^etre vers\'ee en cas de reprise d'un emploi moins r\'emun\'er\'e que le pr\'ec\'edent avant le terme du CSP, sans consid\'eration d'un seuil relatif \`a la baisse de la r\'emun\'eration. Le montant doit \^etre inf\'erieur \`a 50\% des droits restants \`a l'ASP, et la dur\'ee de versement ne peut d\'epasser 12 mois.
\end{itemize}
\textbf{Evolution du montant de l'indemnisation}: Si le salari\'e justifie d'au moins deux ans d'anciennet\'e au sein de l'entreprise, il per\c coit une ASP \'egale \`a 75\% (et non plus 80\%) de l'ancien salaire brut, qui ne peut \^etre inf\'erieur au montant maximal de l'ARE \`a laquelle il aurait droit, pendant 12 mois maximum. Si l'anciennet\'e du salari\'e au sein de l'entreprise n'exc\`ede pas 1 an, le salari\'e per\c coit une allocation \'equivalente \`a l'ARE dans les conditions r\'egissant l'ARE. Si la salari\'e justifie d'une anciennet\'e comprise entre 1 et 2 ans, il peut b\'en\'eficier de l'ASP selon des conditions qui seront d\'efinies dans une prochaine convention entre l'Etat et l'Un\'edic.
\end{minipage}
}

\vspace{3mm}

\texttt{R\'ef\'erences l\'egislatives}{: Arr\^et\'e du 16 avril 2015 relatif \`a l'agr\'ement de la convention du 26 janvier 2015 relative au contrat de s\'ecurisation professionnelle}  

%\section{\label{section:sec4}Autres prestations d'assurance ch\^omage}

 

%En compl\'ement ou remplacement de l'ARE, certaines aides mat\'erielles peuvent \^etre accord\'ees:
%\begin{itemize}
%\item Aide pour cong\'es non pay\'es: si le ch\^omeur reprend un activit\'e mais qu'au moment de la fermeture annuelle de son \'etablissement, il n'a pas accumul\'e de cong\'es pay\'es pour la p\'eriode en cours, il peut solliciter cette aide aupr\`es de P\^ole emploi. Elle est \'equivalente au moment des allocations qui auraients \'et\'e per\c cues sur la p\'eriode si le ch\^omeur n'avait pas repris un emploi, et est pay\'ee en une fois;
%\item Aide pour les ch\^omeurs en fin de droits \`a l'assurance ch\^omage: si le ch\^omeur n'est pas \'eligibile \`a l'Allocation de solidarit\'e sp\'ecifique pour un motif autre que celui des ressources, il peut percevoir une aide \'equivalente \`a 27 fois la partie fixe de l'ARE;
%\item Allocation d\'ec\`es: elle peut \^etre distribu\'ee au conjoint d'un allocataire qui \'etait en cours d'indemnisation (ou au cours d'une p\'eriode de diff\'er\'e ou de d\'elai d'attente) et qui est d\'ec\'ed\'e. Son montant est de 120 fois le montant brut journal de l'allocation pr\'ec\'edemment per\c cue par le conjoint, auquel s'ajoute 45 fois le montant brut journalier par enfant \`a charge.
%\end{itemize}


\ifx\isEmbedded\undefined
\newpage
\bibliography{../../Biblio/biblio-rapport}
\end{document}
\else \fi

