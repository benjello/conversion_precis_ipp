\ifx\isEmbedded\undefined
\input{../../Style/ipp-separate}
\setcounter{chapter}{0}
\chapter{\label{Chapitre1}Le financement du syst\`eme d'indemnisation ch\^omage}
\else \fi


%%%%%%%%%%%%%%%%%%%%%%%%%%%%%%%%%%%%%%%%%%%%%%%%%%%%%%%%%%%%%%%%%%%%%%%%%%%%%%%%%%%%%%%%%%





%%%%%%%%%%%%%%%%%%%%%%%%%%%%%%%% SECTION: FINANCEMENT DU REGIME D'ASSURANCE %%%%%%%%%%%%%%%%%%%%%%%%%


\section{\label{section:sec1}Financement du r\'egime d'assurance}



Le r\'egime d'assurance ch\^omage, comme toute assurance classique, consiste \`a pr\'elever une partie du revenu \`a l'\'etat positif, correspondant \`a l'emploi, pour financer l'indemnisation en cas de r\'ealisation du risque ch\^omage, et ainsi lisser la consommation en cas de choc de revenu. Ainsi, les allocations d'assurance ch\^omage sont financ\'ees gr\^ace aux cotisations pay\'ees par chaque travailleur: ces cotisations comprennent une part employeur et une part employ\'e pay\'ees sur le salaire brut. 

 

\subsection{D\'etermination de l'assiette}

 

Les cotisations d'assurance ch\^omage sont assises sur la r\'emun\'eration brute avant d\'eduction de toute cotisation et contribution sociale (\texttt{art. L 242-1, al 1\up{er} du CSS}). La r\'emun\'eration brute est compos\'ee du salaire \textit{stricto sensu} ainsi que d'autres \'el\'ements tels que les avantages en nature ou en argent, les indemnit\'es, les primes, etc, qui ob\'eissent \`a des r\'egimes sociaux et fiscaux complexes. Toutes les informations concernant le traitement social et fiscal de chaque \'el\'ement de r\'emun\'eration peuvent \^etre trouv\'ees dans le Guide l\'egislatif sur les cotisations sociales de l'Institut des Politiques publiques\cite{IPP2014}; nous en fournissons ici une description r\'esum\'ee.\\

L'assiette des cotisations d'assurance ch\^omage pay\'ees par les employ\'es du secteur priv\'e comprend:
\begin{itemize}
\item Le salaire de base, qu'il soit touch\'e au titre des heures contractuelles ou suppl\'ementaires, m\^eme si la \og Loi TEPA \fg, introduite par le gouvernement Fillon en 2007 et rest\'ee en vigueur jusqu'\`a son abrogation en 2012, a complexifi\'e l'imposition des r\'emun\'erations vers\'ees au titre des heures suppl\'ementaires;
\item Les avantages en nature sont pris en compte -- qu'ils soient \'evalu\'es sur leur valeur r\'eelle ou de mani\`ere forfaitaire -- en d\'eduisant la participation de l'employ\'e au financement de l'avantage en nature:  c'est la diff\'erence entre la valeur de l'avantage et la participation de l'employ\'e qui doit \^etre int\'egr\'ee \`a l'assiette des cotisations\footnote{Dans le cas de la nourriture, lorsque la participation salariale est au moins \'egale \`a 50\% du forfait, la participation de l'employeur n'est pas r\'eint\'egr\'ee \`a l'assiette des cotisations sociales.};
\item Les contributions facultatives de l'employeur pay\'ees au titre de la prise en charge de la part salariale des cotisations \`a un r\'egime de retraite compl\'ementaire;
\item Les sommes retir\'ees de la r\'eserve sp\'eciale de participation de l'entreprise ou d'un plan d'\'epargne salariale si elles sont touch\'ees avant la fin de la p\'eriode d'indisponibilit\'e. Dans le cas contraire, ces sommes ne sont pas soumises \`a la cotisation ch\^omage;
\item Les indemnit\'es de rupture du contrat de travail sont, selon le type d'indemnit\'es, soit partiellement soit totalement assujetties.
\end{itemize}

Inversement, sont exclus de cette assiette:
\begin{itemize}
\item  Les frais professionnels comptabilis\'es sous le r\'egime des d\'epenses r\'eelles ou sous forme d'une allocation forfaitaire sont exclus de l'assiette des cotisations sociales et du salaire imposable sur pr\'esentation de justificatifs. Les frais professionnels indemnis\'es sous forme d'une d\'eduction forfaitaire sp\'ecifique dans le cas de certaines professions qui doivent supporter des frais professionnels \'elev\'es sont donc soustraits de l'assiette de la r\'emun\'eration soumise aux cotisations sociales;
\item La prise en charge patronale des cotisations pay\'ees par les employ\'es \`a un r\'egime de pr\'evoyance compl\'ementaire ou \`a un r\'egime de retraite suppl\'ementaire;
\item Les sommes retir\'ees de la r\'eserve sp\'eciale de participation de l'entreprise ou d'un plan d'\'epargne salariale si elles sont touch\'ees apr\`es la p\'eriode d'indisponibilit\'e;
\item Les primes d'int\'eressement et l'abondement de l'employeur \`a un plan d'\'epargne salariale;
\item Les plus-values d'acquisition d'actions par les salari\'es, ainsi que l'avantage \`a l'attribution et la plus-value de cession dans le cas des actions gratuites;
\item Les indemnit\'es journali\`eres de S\'ecurit\'e sociale.
\end{itemize}

Une fois circonscrite la d\'efinition de la r\'emun\'eration soumise \`a la cotisation d'assurance ch\^omage, il convient de rappeler que l'assiette sur laquelle est pay\'ee cette cotisation correspond \`a la r\'emun\'eration ci-dessus d\'efinie jusqu'\`a quatre plafonds de la S\'ecurit\'e sociale\footnote{Le plafond de la S\'ecurit\'e sociale est un montant fix\'e par arr\^et\'e publi\'e au Journal officiel, et qui sert de base au calcul de la plupart des pr\'el\`evements sociaux, car les diff\'erentes tranches d'imposition sont d\'efinies par rapport \`a ce montant. En 2015, le plafond mensuel correspond \`a 3 218\euro~.} (PSS). Avant le 1\up{er} janvier 2002, un taux diff\'erent s'appliquait \`a la r\'emun\'eration situ\'ee entre 0 et 1 plafond de la S\'ecurit\'e sociale (tranche A), et celle situ\'ee entre 1 et 4 plafonds de la S\'ecurit\'e sociale (tranche B). Ces taux s'appliquant aux tranches A et B ont \'et\'e harmonis\'es depuis 2002, si bien qu'un taux unique est pay\'e sur la totalit\'e de la r\'emun\'eration en-dessous de 4 PSS.\newline

En dehors de la cotisation d'assurance ch\^omage principale pay\'ee par chaque travailleur, d'autres contributions plus sp\'ecifiques ont \'et\'e cr\'e\'ees: la cotisation AGS est due par les employeurs pour le financement d'un fonds de garantie des salaires destin\'e \`a assurer le paiement des salaires aux employ\'es en cas de liquidation judiciaire de l'entreprise. L'assiette correspond \'egalement \`a la r\'emun\'eration brute sous 4 PSS.

%% VERIFIER SI COTISATION CHOMAGE PAYEE DE LA MEME MANIERE PAR SALARIES DU PUBLIC, ET SI OUI, DETAILLER COMPOSITION DE L'ASSIETTE.
%% CHECKER LA COTISATION ASF
\texttt{R\'ef\'erences l\'egislatives}{: Article L242-1 code de la S\'ecurit\'e sociale}

 

\subsection{Mode de calcul}

 

Les cotisations d'assurance ch\^omage -- de m\^eme que pour les autres cotisations sociales -- sont pr\'elev\'ees chaque mois \`a la source, c'est-\`a-dire d\'eduites directement du salaire brut avant qu'il ne soit pay\'e au travailleur.\newline

Il existe deux types de cotisations: 
\begin{itemize}
\item La cotisation d'assurance ch\^omage principale comporte une partie pay\'ee par l'employeur, qui correspond, dans le cas g\'en\'eral, \`a 4\% de la r\'emun\'eration brute en-dessous de 4 PSS, et une partie pay\'ee par l'employ\'e, qui correspond \`a 2.4\% de la r\'emun\'eration brute en-dessous de 4 PSS. Le taux global a augment\'e avec le temps, mais la r\'epartition a toujours \'et\'e plus favorable aux employ\'es.
\item La cotisation du r\'egime de garantie des salaires (AGS), exclusivement patronale, permet de payer les r\'emun\'erations, pr\'eavis et indemnit\'es des salari\'es en cas de redressement ou liquidation judiciaire. Depuis le 1\up{er} janvier 2016, son taux est de 0,25\% (contre 0,30\% auparavant).
\end{itemize}
%%%%%% Parler de la cotis AGS et de la modulation des contributions!!!
Depuis le 1\up{er} juillet 2014, les cotisations assurance ch\^omage et AGS sont pay\'ees sur les r\'emun\'erations de tous les salari\'es, y compris les salari\'es de plus de 65 ans.\\

La modulation des contributions, applicable \`a partir du 1\up{er} juillet 2013, a introduit une variation des contributions patronales selon le type et la dur\'ee du contrat de travail et l'\^age du salari\'e pour tous les employeurs du secteur priv\'e et les employeurs du secteur public ayant choisi d'adh\'erer \`a l'assurance ch\^omage (sont donc exclus les employeurs en auto-assurance et les particuliers employeurs).\\
La majoration pour les CDD s'applique comme suit:

\begin{tab}
[15cm]                                                                           %  Argument 1 [8 cm] : largeur du tableau (mettre 15 cm comme taille maximale)
{
Majoration de la contribution \`a l'assurance ch\^omage pour les CDD\label{tab:modcontrib} % Argument 2 : titre du tableau
}
{
\begin{tabular}{lcc} \toprule                                                  % Argument 3 : contenu du tableau
Motif du CDD             & Dur\'ee du contrat de travail     & Majoration de la contribution \\
\midrule
\multirow{2}{*}{Acroissement temporaire d'activit\'e}                  & Inf\'erieure ou \'egale \`a 1 mois  & 3\%        \\
                        & Sup\'erieure \`a 1 mois et inf\'erieure \`a 3 mois  & 1,5\%     \\
Contrat d'usage & Inf\'erieure ou \'egale \`a 3 mois    & 0,5\%        \\
\bottomrule
\end{tabular}
}
{\textsc{Notes:} Ces majorations ne s'appliquent pas dans le cas o\`u le salari\'e est embauch\'e en CDI \`a la suite de son CDD, dans le cas de CDD sup\'erieurs \`a 3 mois, et dans le cas de CDD conclus pour d'autres motifs que ceux \'enonc\'es ci-dessus (travail temporaire, CDD de remplacement, saisonniers, etc.)}                                                                              % Argument 4 : notes du tableau (vide)
\end{tab}


%% ON A DIT QU'ON PARLAIT PAS DES INTERMITTENTS
%Concernant les employeurs d'intermittents du spectacle, les contributions pay\'ees diff\`erent: dans le cas g\'en\'eral, la part patronale est de 8\%, et la part salariale de 4.80\%. Les m\^emes majorations \'enonc\'ees ci-dessus s'appliquent.\\

Enfin, des taux sp\'ecifiques de contribution \`a l'assurance ch\^omage s'appliquent au d\'epartement de Mayotte: ils sont fix\'es, dans le cas g\'en\'eral, \`a 1,75\% pour la part employeur, et 1,05\% pour la part salariale.

Les employeurs peuvent aussi b\'en\'eficier de certaines exon\'erations: \\
\begin{tab}
[15cm]                                                                           %  Argument 1 [8 cm] : largeur du tableau (mettre 15 cm comme taille maximale)
{
Exonerations de contribution \`a l'assurance ch\^omage et de contribution AGS\label{tab:exocontrib} % Argument 2 : titre du tableau
}
{
\begin{tabular}{lll} \toprule                                                  % Argument 3 : contenu du tableau
Salari\'es concern\'es             & Contribution \`a l'assurance ch\^omage  & Cotisation AGS \\
\midrule
\multirow{6}{*}{Apprentis}& Employeurs inscrits au r\'epertoire & Employeurs inscrits au r\'epertoire des m\'etiers  \\
	& des m\'etiers et employeurs de moins de 11 salari\'es sont &  et employeurs de moins de 11 salari\'es sont \\
	& exon\'er\'es de la part patronale et salariale & exon\'er\'es de cotisation AGS \\
           & Employeurs de plus de 10 salari\'es non inscrits &  Employeurs de plus de 10 salari\'es \\
	& au r\'epertoire des m\'etiers sont exon\'er\'es & non inscrits au r\'epertoire des m\'etiers sont \\
	& de la seule part salariale &  exon\'er\'es de cotisation AGS \\
\multirow{5}{*}{Marins du commerce} & Exon\'eration de la part  & Pas d'exon\'eration        \\
	& patronale des contributions d'assurance ch\^omage &  \\
	&  dues pour les marins employ\'es \`a bord de navires & \\
	& de transport de passagers battant pavillon fran\c cais et exploit\'es & \\
	& \`a titre principal en situation de concurrence internationale & \\
Salari\'es des syndicats & Pas d'exon\'eration & Exon\'eration \\
de copropri\'et\'e & & \\
\multirow{5}{*}{Tous les salari\'es} & Employeurs embauchant un jeune de & \\ 
	&  moins de 26 ans en CDI et dont le contrat se poursuit & \\
	& au-del\`a de la p\'eriode d'essai sera exon\'er\'e -- pour la part patronale -- & \\
	& de contribution \`a l'assurance ch\^omage pendant 3 mois  & \\
	& (4 mois dans les entreprises de moins de 50 salari\'es) & \\
\bottomrule
\end{tabular}
}
{}                                                                              % Argument 4 : notes du tableau (vide)
\end{tab}


\texttt{R\'ef\'erences l\'egislatives}{: article L. 3253-18 du Code du travail; article 11 de la LOI \no 2013-504 du 14 juin 2013 relative \`a la s\'ecurisation de l'emploi; Circulaire Un\'edic \no 2013-04 du 21 janvier 2013 : Indemnisation du ch\^omage \`a Mayotte; Convention du 14 mai 2014 relative \`a l'indemnisation du ch\^omage (art.4); R\`eglement g\'en\'eral annex\'e \`a la convention du 14 mai 2014 (art. 50 \`a 61); Annexe VII au r\`eglement g\'en\'eral annex\'e et aux annexes au r\`eglement g\'en\'eral de la convention du 14 mai 2014; Annexe VIII au r\`eglement g\'en\'eral annex\'e \`a la convention du 14 mai 2014 (art. 59 \`a 60); Annexe X au r\`eglement g\'en\'eral annex\'e \`a la convention du 14 mai 2014 : (art. 59 \`a 60); Code du travail articles L. 6243-2 et D. 6243-5; Loi n° 2006-1666 du 21 d\'ecembre 2006 de finances pour 2007, article 137; Loi n° 2009-526 du 12 mai 2009, article 7}

\section{\label{section:sec2}Financement du r\'egime de solidarit\'e}

 

Les allocations du r\'egime de solidarit\'e sont financ\'ees selon des modalit\'es diff\'erentes: r\'epondant \`a un objectif de solidarit\'e nationale, elles ne sont pas contributives. Leur financement ne repose donc pas sur des cotisations pay\'ees par chaque travailleur et sp\'ecialement d\'edi\'ees au financement d'une branche de la S\'ecurit\'e sociale ou bien de l'assurance ch\^omage, mais il est directement issu du budget de l'\'etat \`a travers le Fonds de Solidarit\'e.\\

Le Fonds de Solidarit\'e est donc en charge de la gestion financi\`ere de l'Allocation de solidarit\'e sp\'ecifique, de l'Allocation \'equivalent retraite, de la Prime forfaitaire de reprise d'activiti\'e, de l'Allocation du fonds de professionnalisation et de solidarit\'e, et de l'Aide aux ch\^omeurs repreneurs ou cr\'eateurs d'entreprise. Toutefois, le versement des allocations est assur\'e par P\^ole emploi, au m\^eme titre que les allocations d'assurance ch\^omage.\newline

Le Fonds de Solidarit\'e tire ses ressources financi\`eres du paiement d'une contribution de solidarit\'e pr\'elev\'ee \`a la source par les employeurs sur les r\'emun\'eration des fonctionnaires et agents publics relevant de l'\'Etat, des collectivit\'es locales, des \'etablissements hospitaliers et autres organismes \'enum\'er\'es dans les articles L. 5424-1, L 5424-2 et R 5424-1 du Code du Travail.
Elle correspond \`a 1\% de leur r\'emun\'eration, et est compl\'et\'ee par une subvention de l'Etat, pour permettre au Fond de Solidarit\'e de faire face \`a toutes ses d\'epenses.\\

A titre indicatif, on pourra noter qu'en 2013, le montant total des allocations financ\'ees par le Fonds de Solidarit\'e a repr\'esent\'e 2,6 milliards d'euros, alors que l'Allocation de retour \`a l'emploi -- principale allocation d'assurance ch\^omage -- a repr\'esent\'e \`a elle seule 27,8536 milliards d'euros. Le total des allocations d'assurance distribu\'ees en 2013 est \'egal \`a 30,8247 milliards d'euros, alors que les contributions servant \`a financer le r\'egime d'assurance s'\'elevaient \`a 33,4531 milliards d'euros en 2013.

 

\subsection{D\'etermination du champ d'application}

 

Le versement des allocations de solidarit\'e \'etant en partie assur\'e gr\^ace au paiement de la contribution de solidarit\'e, il convient de d\'efinir plus pr\'ecis\'ement les employeurs et employ\'es vis\'es par ce paiement.\\
Les employeurs concern\'es sont les employeurs des salari\'es du secteur public et parapublics, tels que d\'efinis aux articles L.5424-1 et L 5424-2 du Code du travail:
\begin{itemize}
\item L'Etat;
\item Les \'etablissements publics administratifs de l'\'etat;  
\item Les \'etablissements publics administratifs autres que ceux de l'\'etat;
\item Les collectivit\'es territoriales;
\item Les groupements d'int\'er\^et public;
\item Les entreprises inscrites au r\'epertoire national des entreprises contr\^l\'ees majoritairement par l'\'etat;
\item Les \'etablissements publics \`a caract\`ere industriel et commercial, les \'etablissements publics \`a caract\`ere industriel et commercial des collectivit\'es territoriales, les soci\'et\'es d'\'economie mixte dans lesquelles ces collectivit\'es ont une participation majoritaire;
\item Les chambres de m\'etiers, les chambres de commerce et d'industrie territoriales, les services \`a caract\`ere industriel et commercial g\'er\'es par les chambres de commerce et d'industrie, les chambres d'agriculture, les \'etablissements et services d'utilit\'e agricole de ces chambres;
\item L'entreprise France-T\'el\'ecom et ses filiales;
\item Les \'etablissements d'enseignement mentionn\'es \`a l'article L 916-1 du Code de l'\'education;
\item Les \'etablissements publics d'enseignement sup\'erieur et les \'etablissements publics \`a caract\`ere scientifique et technologique;
\item Les entreprises de la branche professionnelle des industries \'electriques et gazi\`eres soumis au statut national du personnel des industries \'electriques et gazi\`eres (sauf si elles ont exerc\'e leur option d'adh\'erer au r\'egime d'assurance ch\^omage avant leur assujettissement au statut national).
\end{itemize}

Les employeurs sus-cit\'es rel\`event du r\'egime d'auto-assurance dans lequel c'est l'employeur public lui m\^eme qui assure l'indemnisation de ses salari\'es en cas de perte d'emploi. Ces employeurs sont donc tenus de verser 1\% de la r\'emun\'eration de leurs salari\'es au titre de la contribution de solidarit\'e, sauf dans le cas o\`u leurs salari\'es sont plac\'es sous le r\'egime d'assurance ch\^omage classique tel que d\'efini \`a l'article L. 5422-13 du Code du travail. Leurs salari\'es sont assujettis au versement de cette contribution quelle que soit leur affectation, qu'ils soient en activit\'e ou qu'ils assurent des prestations donnant lieu \`a des r\'emun\'erations.\newline

Cependant, certains employeurs peuvent choisir, par option irr\'evocable, d'adh\'erer au r\'egime d'assurance ch\^omage et de payer les cotisations aff\'erentes. Ils sont \`a ce titre dispens\'es du versement de la contribution de 1\%. Les employeurs concern\'es par cette option sont les suivants:
\begin{itemize}
\item Les entreprises inscrites au r\'epertoire national des entreprises contr\^l\'ees majoritairement par l'\'etat;
\item Les \'etablissements publics \`a caract\`ere industriel et commercial des collectivit\'es territoriales;
\item Les soci\'et\'es d'\'economie mixte dans lesquelles ces collectivit\'es ont une participation majoritaire;
\item Les chambres de m\'etiers et de l'artisanat;
\item Les services \`a caract\`ere industriel et commercial g\'er\'es par les chambres de commerce et d'industrie, les chambres d'agriculture, les \'etablissements et services d'utilit\'e agricole de ces chambres.
\end{itemize}

De plus, les agents non titulaires des collectivit\'es territoriales, les agents non statutaires des \'etablissements publics administratifs autres que ceux de l'\'etat, les agents non statutaires des groupements d'int\'er\^et public, ainsi que les agents non statutaires ou non titulaires des \'etablissements publics d'enseignement sup\'erieur et des \'etablissements publics \`a caract\`ere scientifique et technologique peuvent \'egalement opter pour le r\'egime d'auto-assurance ou bien le r\'egime d'assurance ch\^omage.\\
Le r\'egime d'auto-assurance correspond au r\'egime de droit commun dans la fonction publique. Tous les agents sont alors assujettis \`a la contribution \`a 1\%, et, en cas de perte d'emploi, leurs allocations ch\^omage sont directement \`a la charge de l'employeur public, qui peut \'eventuellement souscrire une convention de gestion avec P\^ole emploi.\\
En cas d'adh\'esion au r\'egime d'assurance ch\^omage par décision des agents, ils sont toujours tenus de verser une contribution \`a hauteur de 1\% de leur r\'emun\'eration, et l'employeur a la charge de compl\'eter cette contribution jusqu'\`a atteindre le total des parts employeurs et salari\'es de la cotisation ch\^omage pr\'evue par ce r\'egime. Pour certains agents dont la r\'emun\'eration est inf\'erieure au seuil d'assujettissement \`a la contribution de solidarit\'e, l'employeur assume alors le paiement de la cotisation globale. Les employeurs ayant adh\'er\'e au r\'egime d'assurance ch\^omage ne sont alors plus tenus de verser la contribution de solidarit\'e.\newline

\textbf{Cas particuliers}: 
\begin{itemize}
\item D\'etachement: lorsque l'agent est d\'etach\'e pour servir dans un organisme de droit priv\'e, ou bien un organisme public ayant choisi volontairement d'adh\'erer au r\'egime d'assurance ch\^omage, il est alors soumis \`a la cotisation du r\'egime d'assurance ch\^omage, mais n'est pas redevable de la contribution de solidarit\'e.
\item La position Hors Cadres: si le fonctionnaire en position hors-cadre est employ\'e par un des employeurs list\'es au d\'ebut du paragraphe \textit{D\'etermination du champ d'application} et que celui-ci n'a pas opt\'e pour le r\'egime d'assurance ch\^omage, il est alors assujetti \`a la contribution de solidarit\'e. Si le fonctionnaire en position hors-cadre est employ\'e par un de ces m\^emes employeurs et que celui-ci a opt\'e pour le r\'egime d'assurance ch\^omage, ou bien s'il est employ\'e par un organisme de droit priv\'e, il est alors soumis au r\'egime d'assurance ch\^omage, et est alors dispens\'e du paiement de la contribution de solidarit\'e.
\item La disponibilit\'e: le fonctionnaire en disponibilit\'e est assujetti aux r\`egles de droits communs s'il est employ\'e dans un organisme relevant du r\'egime d'assurance ch\^omage, et n'est donc pas redevable de la contribution de solidarit\'e.
\item Activit\'e accessoire: la r\'emun\'eration principale est soumise \`a la contribution de solidarit\'e, mais la r\'emun\'eration au titre de l'activit\'e accessoire suit les m\^emes r\`egles que celles \'enonc\'ees ci-dessus.
\item La mise \`a disposition: de m\^eme, la r\'emun\'eration principale reste soumise \`a la contribution de solidarit\'e, mais la r\'emun\'eration compl\'ementaire \'eventuellement touch\'ee suit les m\^eme r\`egles que celles \'enonc\'ees ci-dessus.
\end{itemize}

\vspace{3mm}


\texttt{R\'ef\'erences l\'egislatives:}{Articles L.5424-1 et L.5424-2, L.5422-13, L.5423-26, L.5423-24, L.5423-32 et R 5424-1 du Code de travail; Circulaire interminist\'erielle \no FP7 2033 du 27 mai 2003; loi \no 82-939 du 4 novembre 1982 relative \`a la contribution exceptionnelle de solidarit\'e en faveur des travailleurs priv\'es d'emploi}

 

\subsection{D\'etermination de l'assiette et du mode de calcul}

 

Selon l'article L.5423-27 du Code du travail, "La contribution exceptionnelle de solidarit\'e est assise sur la r\'emun\'eration nette totale des salari\'es, y compris l'ensemble des \'el\'ements ayant le caract\`ere d'accessoire du traitement, de la solde ou du salaire, \`a l'exclusion des remboursements de frais professionnels, dans la limite du plafond mentionn\'e \`a l'article L. 5422-3."\newline
Elle doit \^etre vers\'ee d\`es lors que l'assiette prise en compte d\'epasse un certain seuil d'assujettissement.\\

Plus pr\'ecis\'ement, l'assiette correspond \`a la r\'emun\'eration mensuelle nette. Il convient donc d'inclure:
\begin{itemize}
\item La r\'emun\'eration de base mensuelle brute, qui comprend le traitement indiciaire, la bonification indiciaire, la nouvelle bonification indiciaire et toute r\'emun\'eration accessoire calcul\'ee proportionnellement au traitement indiciaire (primes et indemnit\'es de toute nature);
\item L'indemnit\'e de r\'esidence;
\item Le suppl\'ement familial de traitement;
\item Les indemnit\'es horaires ou forfaitaires pour travaux suppl\'ementaires;
\item Les indemnit\'es de caisse et de responsabilit\'e des comptables publics;
\item La CSG et la CRDS, tout comme les \'eventuelles cotisations vers\'ees \`a des r\'egimes de pr\'evoyance ou de retraite compl\'ementaires non obligatoire.
\end{itemize}

\vspace{3mm}

A l'inverse, sont \`a exclure:
\begin{itemize}
\item Les cotisations de s\'ecurit\'e sociale obligatoires;
\item Les pr\'el\`evements pour pension;
\item Le remboursement de frais correspondant \`a des d\'epenses r\'eelles;
\item Les indemnit\'es de licenciement ou de d\'epart \`a la retraite;
\item Les prestations familiales et remboursement de frais de garde;
\item Les avantages en nature;
\item La prise en charge partielle des frais de transports;
\item L'allocation vers\'ee aux parents d'enfants handicap\'es;
\item Le pr\'el\`evement au titre des r\'egimes de retraite compl\'ementaire obligatoire.
\end{itemize}

\vspace{3mm}

\textbf{Cas particuliers}:
\begin{itemize}
\item Pluralit\'e d'ordonnateurs de r\'emun\'eration: l'ensemble des r\'emun\'erations est soumise \`a la contribution de solidarit\'e d\`es lors que la r\'emun\'eration principale y est assujettie ou que la somme des r\'emun\'erations exc\`ede le seuil d'assujettissement;
\item Temps partiel: la r\'emun\'eration prise en compte est la r\'emun\'eration mensuelle effective \`a temps partiel, qui est compar\'ee au seuil d'assujettissement;
\item Indemnit\'es journali\`eres de S\'ecurit\'e sociale: elles sont prises en compte pour la d\'etermination de l'assujettissement et le calcul de la contribution seulement dans le cas o\`u elles sont vers\'ees directement par l'employeur dans le cadre d'accords de mensualisation ou d'un dispositif visant au maintien du salaire. Quand elles sont vers\'ees par la s\'ecurit\'e sociale directement, elles n'interviennent pas dans le calcul de l'assiette ou la d\'etermination de l'assujettissement;
\item Les \'elus: si les d\'eput\'es et s\'enateurs sont explicitement assujettis \`a la contribution de solidarit\'e, dans le cas des \'elus locaux, aucune loi ne fait mention d'un \'eventuel versement de la contribution de solidarit\'e. Ils en sont donc exempt\'es;
\item Les mandataires sociaux, n'ayant pas de contrat de travail, rel\`event du r\'egime des travailleurs non agricoles, et ne sont donc pas tenus de payer la contribution de solidarit\'e;
\item Les rappels de traitements: dans le cas o\`u la r\'emun\'eration habituelle de l'agent est inf\'erieure au seuil d'assujettissement, et que la prise en compte du rappel de traitements \'etal\'e au prorata des mois auxquels il se rapporte fait passer la r\'emun\'eration durablement au dessus du seuil, alors la contribution de solidarit\'e est pr\'elev\'ee sur la r\'emun\'eration compl\'et\'ee de la part du rappel de traitement correspondante;
\item Les treizi\`emes et quatorzi\`emes mois et les primes pr\'efix\'ees \`a l'embauche: puisqu'elles font partie int\'egrante du salaire, ces sommes doivent \^etre r\'eparties sur les douze mois de l'ann\'ee pour d\'eterminer l'assujettissement \`a la contribution de solidarit\'e. Si elles ont pour effet de faire passer durablement la r\'emun\'eration au-dessus du seuil, la contribution est pr\'elev\'ee sur la r\'emun\'eration effective;
\item Les indemnit\'es forfaitaires repr\'esentatives de frais sont exclues de l'assiette de calcul et d'assujettissement.
\end{itemize}

\vspace{3mm}

D'autre part, l'indemnit\'e compensant les jours de repos travaill\'es ne doit pas \^etre prise en compte dans la consid\'eration de l'assujettissement, mais elle doit en revanche \^etre r\'eint\'egr\'ee \`a l'assiette de calcul en cas de paiement de la contribution. Cependant, la cotisation suppl\'ementaire due au titre de la Retraite additionnelle de la fonction publique et la "surcotisation" pour pension civile des agents \`a temps partiel ne sont \`a int\'egrer \`a aucune des deux bases de calcul.

Le seuil d'assujettissement mensuel est r\'eguli\`erement r\'evis\'e. Il est \'egal, depuis le 01/01/2013, \`a 1 430,76\euro.\newline

La contribution est calcul\'ee sur l'assiette d\'ecrite ci-dessus, dans la limite d'un plafond \'equivalent \`a 4 fois le plafond de la S\'ecurit\'e sociale, et fix\'e \`a 12 872\euro ~ pour 2016.\newline

\texttt{R\'ef\'erences l\'egislatives:}{Articles L.5423-27, L.5422-33 du Code de travail; Circulaire interminist\'erielle \no FP7 2033 du 27 mai 2003; position du Conseil d’Administration du Fonds de Solidarit\'e en date du 21 juin 1983; article 7 de la loi \no 82-939 du 4 novembre 1982; circulaire \no 3-83 du 12/12/83 du Fonds de Solidarit\'e; circulaire du Fonds de Solidarit\'e \no 3-83 du 12/12/83}

\section{\label{section:sec2}Financement du dispositif du dispositif du Contrat de S\'ecurisation Professionnelle (CSP)}

 

Le Contrat de S\'ecurisation Professionnelle (CSP) est un dispositif offert aux licenci\'es \'economiques, qui comprend un accompagnement renforc\'e, une offre de formation, des incitations \`a la reprise d'une activit\'e et une indemnisation \'egale \`a 80\% (75\% depuis le 1\up{er} f\'evrier 2015) du salaire brut de r\'ef\'erence pour les licenci\'es \'economiques ayant une anciennet\'e dans l'entreprise de plus d'un an, ou \'egale \`a l'ARE pour les licenci\'es \'economiques ayant une anciennet\'e de moins d'un an.\newline

Le financement de ce dispositif repose \`a la fois sur l'employeur, l'\'Etat, et l'Un\'edic:
\begin{itemize}
\item L'employeur est tenu de verser \`a P\^ole emploi, pour les employ\'es ayant plus d'un an d'anciennet\'e et ayant accept\'e d'adh\'erer au CSP, une contribution \'egale au pr\'eavis que le salari\'e aurait per\c cu s'il n'avait pas accept\'e le CSP, dans la limite de trois mois de salaire, charges patronales et salariales comprises (la partie exc\'edant trois mois de salaire est vers\'ee au salari\'es). Pour l'ensemble des salari\'es ayant accept\'e d'adh\'erer au CSP, l'employeur doit \'egalement verser une participation au financement \'equivalente aux heures acquises par le salari\'e au titre du droit individuel \`a la formation et qui n'ont pas \'et\'e utilis\'ees. Cette participation se calcule en multipliant le nombre d'heures par le salaire horaire net du salari\'e divis\'e par deux.
\item Concernant les salari\'es b\'en\'eficiant de l'allocation de s\'ecurisation professionnelle (\'egale \`a 80\% du salaire brut de r\'ef\'erence), l'Un\'edic verse une participation \'egale au montant de l'ARE que ces salari\'es auraient touch\'e s'ils n'avaient pas accept\'e le CSP
\item Pour les salari\'es b\'en\'eficiant de l'ASP et t\'emoignant d'une anciennet\'e dans l'entreprise comprise entre 1 et 2 ans, l'\'Etat et l'Un\'edic financent la moiti\'e du surco\^ut li\'e \`a la diff\'erence entre ASP et ARE.
\end{itemize}

\texttt{R\'ef\'erences l\'egislatives}{Loi \no 2011-893 du 28 juillet 2011 pour le d\'eveloppement de l'alternance et la s\'ecurisation des parcours professionnels; Accord national interprofessionnel du 31 mai 2011 relatif au contrat de s\'ecurisation professionnelle; Accord national interprofessionnel du 8 d\'ecembre 2014 relatif au contrat de s\'ecurisation professionnelle; Convention du 19 juillet 2011 relative au contrat de s\'ecurisation professionnelle; Convention du 26 janvier 2015 relative au contrat de s\'ecurisation professionnelle}

\vspace{3mm}

Maintenant que les modalit\'es de financement ont \'et\'e d\'etaill\'ees, nous pouvons nous int\'eresser aux diff\'erentes allocations distribu\'ees aux ch\^omeurs selon leur situation professionnelle et familiale, en reprenant la m\^eme distinction entre allocations d'assurance et allocations d'assistance. En effet, ces deux types d'allocations ne r\'epondent pas aux m\^emes objectifs et ne s'adressent pas forc\'ement \`a la m\^eme population.
 

\ifx\isEmbedded\undefined
\newpage
\bibliography{../../Biblio/biblio-rapport}
\end{document}
\else \fi