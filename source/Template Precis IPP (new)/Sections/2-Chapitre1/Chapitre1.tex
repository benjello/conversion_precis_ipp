\ifx\isEmbedded\undefined
\input{../../Style/ipp-separate}
\setcounter{chapter}{0}
\chapter{\label{Chapitre1}Titre du chapitre 1}
\else \fi


%%%%%%%%%%%%%%%%%%%%%%%%%%%%%%%%%%%%%%%%%%%%%%%%%%%%%%%%%%%%%%%%%%%%%%%%%%%%%%%%%%%%%%%%%%

\lipsum[1] (sections \ref{section:sec1} et \ref{section:sec2}).

\section{\label{section:sec1}Titre de la section 1}

\lipsum[66]

%%%%%%%%%%%%%%%%%%%%%%%%%%%%%%%%%%%%%%%%%%%%%%%%%%%%%%%%%%%%%%%%%%%%
% Environnement FIG: insertion d'une figure dans le corps du texte %
%%%%%%%%%%%%%%%%%%%%%%%%%%%%%%%%%%%%%%%%%%%%%%%%%%%%%%%%%%%%%%%%%%%%

% Figure avec notes %

\begin{fig}
[15cm]                                                                          % Argument 1 (15 cm) : largeur du titre de la figure
{                                                                               % Argument 2 : titre de la figure
Exemple de figure avec notes ins�r�e dans le corps du texte.\label{fig:test1}
}
{                                                                               % Argument 3 : lien vers le fichier pdf de la figure (� stocker imp�rativement dans le dossier \Sections\3-Section1\Figures\)
\graphique{fig1.pdf}
}
{                                                                               % Argument 4 : notes de la figure
\textsc{Lecture:} bla bla bla bla bla bla bla bla bla bla
bla bla bla bla bla bla bla bla bla bla bla.\\
\textsc{Sources:} bla bla bla bla bla bla bla bla bla bla
bla bla bla bla bla bla bla bla bla bla bla.
}
\end{fig}


\lipsum[66]

\subsection{Titre de la sous-section 1.1}

\lipsum[1]

%%%%%%%%%%%%%%%%%%%%%%%%%%%%%%%%%%%%%%%%%%%%%%%%%%%%%%%%%%%%%%%%%%%%
% Environnement FIG: insertion d'une figure dans le corps du texte %
%%%%%%%%%%%%%%%%%%%%%%%%%%%%%%%%%%%%%%%%%%%%%%%%%%%%%%%%%%%%%%%%%%%%

% Figure sans notes %

\begin{fig}
[15cm]                                                                          % Argument 1 (15 cm) : largeur de la figure
{                                                                               % Argument 2 : titre de la figure
Exemple de figure sans notes ins�r�e dans le corps du texte.\label{fig:test2}
}
{                                                                               % Argument 3 : lien vers le fichier pdf de la figure (� stocker imp�rativement dans le dossier \Sections\3-Section1\Figures\)
\graphique{fig2.pdf}
}
{}                                                                              % Argument 4 : notes de la figure (vide)
\end{fig}

\lipsum[66]


\subsection{Titre de la sous-section 1.2}

\lipsum[1]

%%%%%%%%%%%%%%%%%%%%%%%%%%%%%%%%%%%%%%%%%%%%%%%%%%%%%%%%%%%%%%%%%%%
% Environnement FIGP: insertion d'une figure sur une page s�par�e %
%%%%%%%%%%%%%%%%%%%%%%%%%%%%%%%%%%%%%%%%%%%%%%%%%%%%%%%%%%%%%%%%%%%

% Figure avec notes %

\begin{figp}
[15cm]                                                                          % Argument 1 (15 cm) : largeur du titre de la figure
{                                                                               % Argument 2 : titre de la figure
Exemple de figure avec notes ins�r�e dans une page s�par�e.\label{fig:test3}
}
{                                                                               % Argument 3 : lien vers le fichier pdf de la figure (� stocker imp�rativement dans le dossier \Sections\3-Section1\Figures\)
\graphique{fig3.pdf}
}
{                                                                               % Argument 4 : notes de la figure
\textsc{Lecture:} bla bla bla bla bla bla bla bla bla bla
bla bla bla bla bla bla bla bla bla bla bla.\\
\textsc{Sources:} bla bla bla bla bla bla bla bla bla bla
bla bla bla bla bla bla bla bla bla bla bla.
}
\end{figp}


\section{\label{section:sec2}Titre de la section 2}

\lipsum[66]

%%%%%%%%%%%%%%%%%%%%%%%%%%%%%%%%%%%%%%%%%%%%%%%%%%%%%%%%%%%%%%%%%%%%
% Environnement TAB: insertion d'un tableau dans le corps du texte %
%%%%%%%%%%%%%%%%%%%%%%%%%%%%%%%%%%%%%%%%%%%%%%%%%%%%%%%%%%%%%%%%%%%%

% Tableau avec notes %

\begin{tab}
[8cm]                                                                           % Argument 1 [8 cm] : largeur du tableau (mettre 15 cm comme taille maximale)
{
Exemple de tableau (avec notes) ins�r� dans le corps du texte.\label{tab:test1} % Argument 2 : titre du tableau
}
{
\begin{tabular}{lccc} \toprule                                                  % Argument 3 : contenu du tableau
�chantillon             & Contr�le     & Traitement  & Diff�rence \\
                        & (1)          & (2)         & (3)          \\
\midrule
coeff1                  & 0,01         & 0,05        & 0,04        \\
                        & (0,001)      & (0,001)     & (0,001)     \\
coeff2                  & 0,01         & 0,05        & 0,04        \\
                        & (0,001)      & (0,001)     & (0,001)     \\
\\
Nombre d'observations   & 1~023        &  999        &             \\
\bottomrule
\end{tabular}
}
{
\textsc{Notes:} bla bla bla bla bla bla bla bla bla bla bla                     % Argument 4 : notes du tableau
bla bla bla bla bla bla bla bla bla bla
bla bla bla bla bla bla bla bla bla bla.\\
\textsc{Sources:} bla bla bla bla bla bla bla bla bla bla bla.
}
\end{tab}

\lipsum[66]

\subsection{Titre de la sous-section 2.1}

\lipsum[1]

%%%%%%%%%%%%%%%%%%%%%%%%%%%%%%%%%%%%%%%%%%%%%%%%%%%%%%%%%%%%%%%%%%%%
% Environnement TAB: insertion d'un tableau dans le corps du texte %
%%%%%%%%%%%%%%%%%%%%%%%%%%%%%%%%%%%%%%%%%%%%%%%%%%%%%%%%%%%%%%%%%%%%

% Tableau sans notes %

\begin{tab}
[8cm]                                                                           %  Argument 1 [8 cm] : largeur du tableau (mettre 15 cm comme taille maximale)
{
Exemple de tableau (sans notes) ins�r� dans le corps du texte.\label{tab:test2} % Argument 2 : titre du tableau
}
{
\begin{tabular}{lccc} \toprule                                                  % Argument 3 : contenu du tableau
�chantillon             & Contr�le     & Traitement  & Diff�rence \\
                        & (1)          & (2)         & (3)          \\
\midrule
coeff1                  & 0,01         & 0,05        & 0,04        \\
                        & (0,001)      & (0,001)     & (0,001)     \\
coeff2                  & 0,01         & 0,05        & 0,04        \\
                        & (0,001)      & (0,001)     & (0,001)     \\
\\
Nombre d'observations   & 1~023        &  999        &             \\
\bottomrule
\end{tabular}
}
{}                                                                              % Argument 4 : notes du tableau (vide)
\end{tab}

\lipsum[66]


\subsection{Titre de la sous-section 2.2}

\lipsum[1]

%%%%%%%%%%%%%%%%%%%%%%%%%%%%%%%%%%%%%%%%%%%%%%%%%%%%%%%%%%%%%%%%%%%
% Environnement TABP: insertion d'un tableau sur une page s�par�e %
%%%%%%%%%%%%%%%%%%%%%%%%%%%%%%%%%%%%%%%%%%%%%%%%%%%%%%%%%%%%%%%%%%%

% Tableau avec notes %

\begin{tabp}
[12cm]                                                                          % Argument 1 [12 cm] : largeur du tableau (mettre 15 cm comme taille maximale)
{
Exemple de tableau (avec notes) ins�r� dans une page s�par�e.\label{tab:test3}  % Argument 2 : titre du tableau
}
{
\begin{tabular}{lccc} \toprule                                                  % Argument 3 : contenu du tableau
�chantillon             & Contr�le     & Traitement  & Diff�rence \\
                        & (1)          & (2)         & (3)          \\
\midrule
coeff1                  & 0,01         & 0,05        & 0,04        \\
                        & (0,001)      & (0,001)     & (0,001)     \\
coeff2                  & 0,01         & 0,05        & 0,04        \\
                        & (0,001)      & (0,001)     & (0,001)     \\
\\
Nombre d'observations   & 1~023        &  999        &             \\
\bottomrule
\end{tabular}
}
{
\textsc{Notes:} bla bla bla bla bla bla bla bla bla bla bla                     % Argument 4 : notes du tableau
bla bla bla bla bla bla bla bla bla bla
bla bla bla bla bla bla bla bla bla bla.\\
\textsc{Sources:} bla bla bla bla bla bla bla bla bla bla bla.
}
\end{tabp}



\ifx\isEmbedded\undefined
\newpage
\bibliography{../../Biblio/biblio-rapport}
\end{document}
\else \fi
